\documentclass[12pt,a4paper,english]{article}
\usepackage{../../vpl}
\graphicspath{{../../images/}}

\begin{document}
\thispagestyle{empty}

\begin{center}
\begin{bfseries}

\begin{Large}
Erste schritte in der Robotik mit dem Thymio Roboter
und der Aseba/VPL Pragrammierumgebung
\bigskip

Aworten zu den Aufgaben 
\end{Large}

Version 1.3{\textasciitilde}pre1

\bigskip

\href{http://www.weizmann.ac.il/sci-tea/benari/}{Moti Ben-Ari} und andere Mitwirkende

\end{bfseries}
\end{center}

\bigskip

\copyright{}\  2013--14 Moti Ben-Ari und andere Mitwirkende.

Dieser Inhalt ist unter der Creative-Commons-Lizenz vom Typ Namensnennung -
Weitergabe unter gleichen Bedingungen 3.0 Unported lizenziert.
Um eine Kopie dieser Lizenz einzusehen, besuchen Sie \url{http://creativecommons.org/licenses/by-sa/3.0/} oder schreiben Sie einen Brief an Creative Commons, 444 Castro Street, Suite 900, Mountain View, California, 94041, USA.

\begin{center}
\hspace{6pt}\includegraphics[width=.2\textwidth]{../images/by-sa}
\end{center}

\section{Dein erstes Robotik Projekt}

Keine Aufgaben

\section{Sich ändernde Farben}

\textbf{\thesection.1}: 
Werden drei Farben des Thymio gemischt, kann jede Farbe eine Intensität zwischen 0 (aus) und 32 (max.) haben. Deshalb sind $33 \times 33 \times 33=35,937$ verschiedene Farbabstufungen möglich. Beachten wir nur die Extreme der Intensitäten aus und maximal sind die Farben: keine Farbe (alle Farben auf 0), weiss (alle Farben auf maximaler Intensität), rot, grün, blau (eine Farbe eingeschaltet), gelb (rot und grün ein-, blau ausgeschaltet), türkis (blau und grün ein-, rot ausgeschaltet) und purpur (rot und blau ein-, grün ausgeschaltet).


\section{Beweg dich}

\textbf{\thesection.1}: Der Roboter soll fähig sein zu stoppen. Die maximale Geschwindigkeit des Thymio ist 20 cm pro Sekunge. Da die Sensoren 10 mal pro Sekunde messen, kann der Roboter maximal mit 2 cm fahren, bevor er stoppt. Dies sollte genügen, um zu verhindern, dass der Roboter vom Tisch fällt, sie aber bereit den Roboter aufzufangen für den Fall, dass dies doch geschieht!



\section{Ein Roboterhaustier}

\textbf{\thesection.1}: Ersetze die ersten zwei Ereignis-Aktions Paare durch die Paare gezeit in der Abbildung~\ref{fig.answer1}.
Das erste Paar startet beide Motoren, wenn alle Sensoren \emph{kein} Hindernis entdecken.
Das zweite Paar stoppt die Motoren, wenn die Bodensensoren, das Ende einer Tischkante entdecken.


{\raggedleft \hfill Program file \bu{likes-and-stops.aesl}}

\begin{figure}[hbt]
\begin{center}
\gr{likes-and-stops}{0.4}
\caption{Stoppen an der Tischkante}\label{fig.answer1}
\end{center}
\end{figure}

\textbf{\thesection.2}: 
Ein Ereignis-Aktions Paar läuft ab, wenn ein Ereignis stattfindet. Ereignisse, wie das Drücken eines Kops finden statt, wenn ein externes Ereignis stattfindet. Andere Ereingisse, wie das Messen der Sensoren, finden in festgesetzten Zeitintervallen statt, wie 10 mal pro Sekunde. Wenn die Sensoren gemessen werden, finden alle Ereignisse zur "selben Zeit" statt. Der Computer im Roboter kann jedoch nicht alle Ereignis-Aktions Paare gleichzeitig bearbeiten, stattdessen werden die Ereignis-Aktions Paare eines nach dem anderen durchgeführt \emph{der Reihe} nach von oben nach unten, wie sie im Programm erscheinen.

Im Programm in Abbildung~\ref{fig.answer1},
entdecken die horizontalen Sensoren kein Hindernis und lassen die Motoren laufen; \emph{dann} entdeckt der Bodensensor das Klebeband und stoppt die Motoren.

Im Programm in der Abbildung~\ref{fig.change}, entdecken die Bodensensoren das Klebeband und versuchen die Motoren zu stoppen, aber das zweite Ereignis-Aktions Paar lässt die Motoren laufen, bevor diese stoppen konnten.

{\raggedleft \hfill Program file \bu{likes-changed-order.aesl}}

\begin{figure}[hbt]
\begin{center}
\gr{likes-change-order}{0.4}
\caption{Ändere die Anordnung der Ereignis-Aktions Paare}\label{fig.change}
\end{center}
\end{figure}

\textbf{\thesection.3}: 
\begin{itemize}
\item Benütze die Sensoren 1 und 3: Der Roboter ist weniger empfindlich für deine Hand von der Seite her. Du wirst deine Hand mehr zur Mitte bewegen müssen.
\item Benütze die beiden Sensoren 0 und 1, um den Roboter nach links und beide Sensoren 3 und 4, um den Roboter nach rechts drehen zu lassen: Der Roboter nimmt ein grösseres Gebiet vor sich wahr. Du wirst deine Hand nicht mehr so genau platzieren müssen.

\item Füge ein Ereignis-Aktions Paar für die hinteren Sensoren 5 und 6 hinzu: Du kannst den Roboter drehen lassen, indem du deine Hand hinter den Roboter platzierst.
\end{itemize}


\section{The Robot Finds Its Way by Itself}


\textbf{\thesection.1}: Das Ereignis-Aktions Paar \blk{gentle-left}, vollführt eine leichte Linksdrehung, wenn beide Sensoren viel Licht detektieren. Wenn du die Geschwindigkeit erhöhst, fährt der Roboter eventuell zu weit vom Klebeband weg  und findet dieses nach der Drehung nicht mehr. Wenn der Roboter zum Ende des Klebebandes kommt, wird er ebenfalls eine leichte Linksdrehung vollführen.

{\raggedleft \hfill Program file \bu{find-line.aesl}}

\textbf{\thesection.2}: Der Roboter wird natürlich vom Klebeband \emph{wegfahren}.


\textbf{\thesection.3}:
\begin{itemize}
\item Leichten Kurven kann einfacher gefolgt werden;
\item Enge Kurven sind schwerer zu folgen; 
\item Sind die Linien in einem Zickzack angeordnet, findet der Roboter eventuell nicht zum Klebeband zurück, bevor schon die nächste Kurve begonnen hat.
\item Breitere Linien führen dazu, dass die Wahrscheinlichkeit geringe ist, dass der Roboter von der Linie abkommt; 
\item Schmalere Linien führen dazu, dass die Wahrscheinlichkeit grösser ist, dass der Roboter von der Linie abkommt;
Deshalb kommt es zu häufigeren Drehungen und die Bewegung des Roboters wird ruckartig.
\end{itemize}


\textbf{\thesection.4}:
\begin{itemize}
\item Wenn die Bodensensormessungen häufig erfolgen, reagiert der Roboter schneller und bleibt eher auf dem Klebeband, werden die Messungen weniger häufig durchgeführt, kann der Roboter das Klebeband vollständig verlieren, bevor die nächste Messung erfolgt.
\item Wenn die Sensoren weiter auseinander liegen, wird ein breiteres Klebeband benötigt, aber es ist wahrscheinlicher, dass das Verlassen des Klebebandes eher erkannt wird, bevor dieses ganz verlassen wird. Das Gegenteil trifft zu, wenn die Sensoren näher beieinander liegen.
\item Wenn es mehrere Sensoren gibt, wird der Roboter genauere Bewegungen durchführen können: leichte Drehungen, wenn nur ein Sensor den Boden entdeckt und stärkere Drehungen, wenn mehr als ein Sensor den Boden entdeckt.
\end{itemize}

\section{Glocken und Pfeifen}

\textbf{\thesection.1}: Im Morsecode ist ein Strich drei mal länger als ein Punkt. Die Aktion in der Abbildung~\ref{fig.morse} benutzt drei weisse Kreise in der höchsten Tonlage für ein Strich und einer für ein Punkt. Weil jeweils genau sechs Töne gespielt werden müssen, füllen wir die Melodie mit kurzen Noten in der tiefsten Tonlage aus, welche nicht so gut höhrbar sind, wie die hohen Töne.

{\raggedleft \hfill Program file \bu{bells-morse.aesl}}

\begin{figure}
\begin{center}
\gr{morse}{.4}
\caption{Morsecode}\label{fig.morse}
\end{center}
\end{figure}


\textbf{\thesection.2}: 
Das Programm in der Abbildung~\ref{fig.clap-to-start} schaltet die Motoren ein, wenn ein Klatschereignis stattfindet und stellt die Motoren ab, wenn der mittlere Knopf gedrückt wird.


{\raggedleft \hfill Program file \bu{clap-start.aesl}}

Vertauscht man die Ereignisse, scheint es, dass das Programm in Abbildung~\ref{fig.clap-to-stop}  ebenfalls funktionieren sollte. Dies tut es jedoch nicht. Der Grund ist, dass das Ereignis, welche die Motoren stoppen sollte, nicht nur Klatschen, sondern \emph{jedes} laute Geräusch ist. Die Motoren sind so laut, dass sobald die Motoren laufen, das Lärmereignis stattfindet und die Motoren stoppen. 

Die Lektion, die wir dabei lernen ist, dass wir nicht zu viel Vertrauen in die Symbole der visuellen Blöcke haben sollten!

{\raggedleft \hfill Program file \bu{clap-stop.aesl}}

\begin{figure}
\begin{center}
\gr{clap-to-start}{.4}
\caption{Klatschen, um die Motoren zu starten}\label{fig.clap-to-start}
\end{center}
\end{figure}

\begin{figure}[hbt]
\begin{center}
\gr{clap-to-stop}{.4}
\caption{Klatschen, um die Motoren zu stoppen}\label{fig.clap-to-stop}
\end{center}
\end{figure}

\textbf{\thesection.3}:
Das Program hat zwei Ereignis-Aktions Paare (Abbildung~\ref{fig.bump}): Eines, um den Roboter fahren zu lassen, wenn ein Knop gedrückt wird und eines um den Roboter zu stoppen, wenn ein Klatschereignis stattfindet.
he program has two event-action pairs (Figure~\ref{fig.bump}): one to
move the robot when a button is touched and the other to stop it when a
tap event occurs.

{\raggedleft \hfill Program file \bu{bump.aesl}}

\begin{figure}[hbt]
\begin{center}
\gr{bump}{.4}
\caption{Stoppe den Motor, wenn der Roboter in eine Wand fährt}\label{fig.bump}
\end{center}
\end{figure}


\section{Angenehme Zeit}

\textbf{\thesection.1}:
Betrachte Abbildung~\ref{fig.three}. Wenn der vordere Knopf gedrückt wird, starten die Motoren und ein drei Sekunden Timer wird gestartet. Ist der Timer abgelaufen, ändern die Motoren die Drehrichtung der Räder und wenn der mittlere Knopf gedrückt wird, stoppen die Motoren.

{\raggedleft \hfill Program file \bu{run-three-seconds.aesl}}


\begin{figure}[hbt]
\begin{center}
\gr{run-three-seconds}{.4}
\caption{Fahre drei Sekunden und kehre dann um}\label{fig.three}
\end{center}
\end{figure}


\section{Zustände: Mach nicht immer dasselbe}

\textbf{\thesection.1}:
Es gibt zwei Zustände: Zustand \emph{links} \blksm{state-left} wenn der Roboter nach links dreht und Zustand \emph{rechts}  \blksm{state-right} wenn der Zustand nach rechts dreht. Wird am Anfang der Vorwärtsknopf gedrückt, wird  ein Timer von 1 Sekunde gesetzt und der Anfangszustand (aus, aus, aus, aus) in den Zustand \emph{links} (Figure~\ref{fig.dance-start}) geändert. Wenn der Timer agbelaufen ist, dreht der Roboter nach links, stellt den Timer auf zwei Sekunden ein und ändert in den Zustand \emph{rechts} (Figure~\ref{fig.dance-left}). 
Ist der Timer abgelaufen während dem Zustand \emph{rechts}, dreht der Roboter nach rechts, stellt den Timer wiederum auf 2 Sekunden ein und ändert in den Zustand \emph{links} (Figure~\ref{fig.dance-right}). Diese zwei Verhalten werden immer wieder wiederholt, so musst du auf stopp \blksm{stop} klicken, um den Roboter zu stoppen.


{\raggedleft \hfill Program file \bu{dance.aesl}}


\begin{figure}
\begin{center}
\gr{dance-start}{0.4}
\caption{Starte den Tanz}\label{fig.dance-start}
\end{center}
\end{figure}

\begin{figure}
\begin{center}
\gr{dance-left}{0.4}
\caption{Tanze zwei Sekunden nach links}\label{fig.dance-left}
\end{center}
\end{figure}

\begin{figure}
\begin{center}
\gr{dance-right}{0.4}
\caption{Tanze zwei Sekunden nach rechts}\label{fig.dance-right}
\end{center}
\end{figure}

\textbf{\thesection.2}:
Ich konnte dieses Problem nicht lösen. Ich habe vermutet, dass möglich sei das Ereignis zu erkennen, wenn \emph{ein Sensor} nicht länger das Klebeband erkennt, \emph{bevor} das Ereignis wenn \emph{beide Sensoren} das Klebeband nicht mehr erkennen. Dies würde dem Programm gestatten zu erkennen und zu erinnern von welcher Seite der Roboter das Klebeband verlassen hat. Dies war jedoch unmöglich auch bei geringer Geschwindigkeit.

Das beste, was ich tun konnte war, das das Programm sich merken konnte, auf welcher Seite er das letzte mal das Klebeband verlassen hatte, auch wenn dieses Ereignis viel früher eingetreten war.

{\raggedleft \hfill Program file \bu{follow-line-and-find.aesl}}

Der Roboter hat ein super Verhalten: Gelangt er zum Ende des Klebebandes, dreht er sich langsam vollständig um und fährt das Klebeband zurück! 

Der Zustand \emph{links} \blksm{state-left}  erinnert, dass der Roboter das Klebeband auf der linken Seite verlassen hat und Zustand \emph{rechts} \blksm{state-right} erinnert, dass der Roboter auf der rechten Seite das Klebeband verlassen hat. (Figure~\ref{fig.follow3}).

Wenn beide Sensoren das Klebeban nicht mehr entdecken können, dreht der Roboter in die Richtung, die durch den aktuellen Zustand vorgegeben wird (Figure~\ref{fig.follow1}).

Es ist auch möglich, dass der Roboter vom Klebeband fährt, ohne vorher je das Klebeband auf die eine oder andere Seite verlasse zu haben. Dabei ist der Anfangszustand: alle\emph{aus}; in diesem Fall wählen wir zufällig eine Richtung, hier rechts \blk{follow2}.

\begin{figure}
\begin{center}
\gr{follow3}{0.4}
\caption{Ändere den Zustand, um die Richtung zu merken}\label{fig.follow3}
\end{center}
\end{figure}

\begin{figure}
\begin{center}
\gr{follow1}{0.4}
\caption{Suche nach links oder rechts}\label{fig.follow1}
\end{center}
\end{figure}

\section{Zähler}

\textbf{\thesection.1}:
Du kannst bis vier zählen, weil es vier Viertel gibt, für jeden Zustand eines. Dieser kann auf  \textbf{ein} gestellt werden, um ein Objekt zu zählen. 

\textbf{\thesection.2}:
Das linke untere Viertel des Zustands Icon wird gebraucht, um die 4er Zahlen in einer binären Zahl zu repräsentieren. Falls alle Viertel ausgeschaltet sind, wird die Zahl 0 repräsentiert. Sind alle Zahlen eingeschaltet, wird die Zahl 4+2+1=7 repräsentiert. Acht Ereignis-Aktions Paare werden gebraucht, jedes für eine Umschaltung zwischen $n$ und $n+1$ (modulo8).

{\raggedleft \hfill Program file \bu{count-to-eight.aesl}}


\textbf{\thesection.3}:
Es gibt vier Viertel für die Zustände. Wir können jedes Viertel benutzen, um die 1er, 2er, 4er und 8er zu zählern. Deshalb können wir von 0 bis 8+4+2+1=15 zählen.

\textbf{\thesection.4}:

Dieses Programm ist ein Spiegelbild für das Programm Addieren. Zum Beispiel, wenn die binäre Zahl xyz1 ist, dann ist die Substraktion 1 gleich xyz0, was auch immer der Wert von xyz ist. Wenn die Zahl xyz0 ist, wirst du dir eine binäre Ziffer ausleihen müssen. So werden verschiedene ereignis-Aktions Paare benötigt, abhängig vom Wert von xyz. Schlussendlich, die Substraktion von 1 von 0000=0 gibt 1111=15 in einer zyklischen Arithmetik.


{\raggedleft \hfill Program file \bu{subtraction.aesl}}

\textbf{\thesection.5}:
Ein Ereignis-Aktions Paar wird gebraucht, um jeden Streifen des Klebebandes zu zählen. Hier ein Beispiel: das folgende Paar\blk{count3} ändert die Zählung von 2 zu 3, wenn das Klebeband entdeckt wird.

{\raggedleft \hfill Program file \bu{count-tapes-four.aesl}}

\end{document}
