\documentclass[12pt,a4paper,french]{article}
\usepackage{../../vpl}
\graphicspath{{../../images/}}

\begin{document}
\thispagestyle{empty}

\begin{center}
\begin{bfseries}

\begin{Large}
Premiers pas en robotique

avec le robot
Thymio-II

et l'environnement
Aseba/VPL

\bigskip

Réponses aux exercises

\end{Large}

Version 1.3{\textasciitilde}pre1

\bigskip

\href{http://www.weizmann.ac.il/sci-tea/benari/}{Moti Ben-Ari} et autres contributeurs

\end{bfseries}
\end{center}

\bigskip

\copyright{}\  2013--14 par \href{http://www.weizmann.ac.il/sci-tea/benari/}{Moti Ben-Ari} et autres contributeurs.

Cette œuvre est mise à disposition selon les termes de la Licence Creative Commons Attribution - Partage dans les Mêmes Conditions 3.0 non transposé.
Pour voir une copie, de la license, visitez \url{http://creativecommons.org/licenses/by-sa/3.0/deed.fr}.

\begin{center}
\hspace{6pt}\includegraphics[width=.2\textwidth]{../images/by-sa}
\end{center}


\section{Votre premier projet de robotique}

Il n'y a pas d'exercices dans ce chapitre.

\section{Changer les couleurs}

\textbf{\thesection.1}: 
Lorsque l'on mélange les trois couleurs de Thymio, chaque couleur peut être réglée à une intensité entre 0 (éteinte) et 32 (allumée au maximum). Donc, il y a $33 \times 33 \times 33=35 937$ couleurs différentes. Si l'on ne considère que les cas extrêmes, nous avons: éteint (toutes les couleurs éteintes), blanc (toutes les couleurs au maximum), jaune (rouge et vert allumés, bleu éteint), cyan (bleu et vert allumés, rouge éteint), magenta (rouge et bleu allumés, vert éteint) et bien sûr les cas où une seule LED est allumée.


\section{Thymio en mouvement}

\textbf{\thesection.1}: Le robot devrait pouvoir s'arrêter. La vitesse maximum de Thymio est d'environ 20 centimètres par seconde. Comme les capteurs de proximité sont actualisés 10 fois par seconde, Thymio peut faire au maximum 2 centimètres entre deux actualisations. Donc il parcours au pire 2 centimètres avant de s'arrêter, ce qui devrait suffir pour qu'il ne tombe pas de la table, mais soyez prêt à le rattraper au cas où !


\section{Un robot de compagnie}

\textbf{\thesection.1}: Remplacez les deux paires événement-action par celles montrées sur la \cref{fig.answer1}. La première paire fait avancer Thymio si tous ses capteurs de proximité avant ne détectent pas d'obstacle. La deuxième paire arrête Thymio si les capteurs de proximité du sol ne voient plus la table.

{\raggedleft \hfill Programme \bu{likes-and-stops.aesl}}

\begin{figure}[hbt]
\begin{center}
\gr{likes-and-stops}{0.4}
\caption{Thymio s'arrête au bord de la table}\label{fig.answer1}
\end{center}
\end{figure}

\textbf{\thesection.2}: Une paire événement-action est lancée lorsque l'événement survient. Les événements comme «\,toucher un bouton\,» s'activent grâce à une action externe. D'autres, comme l'actualisation des capteurs de proximité, surviennent à intervalles réguliers, 10 fois par seconde dans ce cas. Lorsque les capteurs sont actualisés, tous les événements sont censés être lancés «\,en même temps\,». En réalité, un ordinateur ne peut pas gérer plusieurs événements en même temps, au lieu de cela, les paires événement-action qui ont lieu en même temps sont traitées les unes après les autres dans l'ordre dans lequel elles apparaissent dans le programme.

Dans le programme de la \cref{fig.answer1}, les capteurs de proximité horizontaux ne détectent pas d'obstacles et les moteurs continuent à rouler, \emph{ensuite} les capteurs de sol détectent le ruban adhésif noir et arrêtent le robot.

Dans le programme de la \cref{fig.change}, les capteurs de sol détectent le ruban adhésif noir et essaient d'arrêter les moteurs, mais la deuxième paire événement-action garde les moteurs allumés avant qu'ils aient une chance d'être arrêtés.

{\raggedleft \hfill Programme \bu{likes-changed-order.aesl}}

\begin{figure}[hbt]
\begin{center}
\gr{likes-change-order}{0.4}
\caption{En changeant l'ordre des paires événement-action, le comportement de Thymio peut changer}\label{fig.change}
\end{center}
\end{figure}

\textbf{\thesection.3}: 
\begin{itemize}

\item Utiliser les capteurs 1 et 3: Thymio est moins sensible à la présence de votre main sur le côté. Vous devrez la bouger plus près du centre.

\item Utiliser les capteurs 0 et 1 pour faire tourner le robot à gauche et les capteurs 3 et 4 pour le faire tourner à droite: Le robot est sensible dans une plus grande zone devant lui et sur ses côtés. Vous n'avez plus besoin d'être très précis avec le placement de votre main.

\item Ajouter une paire événement-action pour les capteurs arrières 5 et 6: Vous pouvez maintenant faire tourner le robot en plaçant votre main près de lui, depuis l'arrière.

\end{itemize}


\section{Thymio est sur une piste}


\textbf{\thesection.1}: La paire événement-action de la \cref{fig.turn_left_gentle} fera tourner Thymio lentement sur lui-même lorsque les deux détecteurs de sol verront beaucoup de lumière, donc si Thymio n'est plus sur la ligne noire. Si vous augmentez la vitesse, Thymio risque de tourner trop vite et de ne pas pouvoir détecter la ligne. 

{\raggedleft \hfill Programme \bu{find-line.aesl}}

\begin{figure}[hbt]
\begin{center}
\gr{gentle-left}{0.4}
\caption{Thymio tourne lentement sur lui-même s'il n'est plus sur la ligne}\label{fig.turn_left_gentle}
\end{center}
\end{figure}

\textbf{\thesection.2}: Thymio va s'éloigner de la ligne.

\textbf{\thesection.3}:
\begin{itemize}
\item Des virages doux sont plus simples à suivre. 
\item Des virages serrés sont plus difficiles à suivre.
\item Si la ligne fait des zig zag, Thymio risque de ne pas réussir à se remettre complètement sur la ligne avant d'arriver au prochain virage.
\item Des lignes plus larges rendent moins probable le fait que Thymio quitte la ligne.
\item Des lignes fines augmenteront la probabilité que Thymio doive corriger sa trajectoire. Ses mouvements seront plus erratiques et saccadés.
\end{itemize}


\textbf{\thesection.4}:
\begin{itemize}
\item Si l'événement de mise à jour des capteurs de sol était plus fréquent, Thymio réagirait plus rapidement lorsqu'il quitte la ligne. S'ils étaient moins fréquents, il pourrait quitter complètement la ligne avant de se rendre compte qu'il a dévié.
\item Si les capteurs étaient plus loin l'un de l'autre, une ligne plus large serait nécessaire mais il serait plus probable que Thymio détecte qu'il commence à quitter la ligne avant d'être complètement hors de celle-ci. L'opposé est vrai si les capteurs étaient plus rapprochés l'un de l'autre.
\item S'il y avait plus de capteurs, Thymio pourrait être plus précis dans ses mouvements et pourrait corriger sa trajectoire à différents niveaux (tourner un peu si un seul capteur sortait de la ligne, tourner plus rapidement si plusieurs sortaient de la ligne\ldots)
\end{itemize}

\section{Sons et chocs}

\textbf{\thesection.1}:
En code Morse, un trait est trois fois plus long qu'un point. Les paires événement-action de la \cref{fig.morse} utilisent trois notes blancs très aigues pour le trait et une seule note très aigue pour le point. Vu qu'il faut absolument avoir six notes, nous avons rempli le reste de l'action avec des courtes notes noires au plus bas ton que l'on entendra moins bien que les notes aigues.

{\raggedleft \hfill Programme \bu{bells-morse.aesl}}

\begin{figure}
\begin{center}
\gr{morse}{.4}
\caption{Le code Morse}\label{fig.morse}
\end{center}
\end{figure}


\textbf{\thesection.2}: Le programme de la \cref{fig.clap-to-start} allume les moteurs si l'événement «\,clap\,» est lancé et arrête les moteurs lorsque l'on appuie sur le bouton central.

{\raggedleft \hfill Programme \bu{clap-start.aesl}}

Pour le comportement opposé, il semble évident que le programme dans la \cref{fig.clap-to-stop} devrait fonctionner mais ce n'est pas le cas. La raison est que l'événement qui arrête les moteurs n'est pas forcément seulement de taper dans ses mains, n'importe quel son assez fort déclenche l'événement. Les moteurs font du bruit lorsqu'ils s'allument, ce qui déclenche l'événement et les stoppent directement.

{\raggedleft \hfill Programme \bu{clap-stop.aesl}}

\begin{figure}
\begin{center}
\gr{clap-to-start}{.4}
\caption{Faites du bruit pour démarrer les moteurs}\label{fig.clap-to-start}
\end{center}
\end{figure}

\begin{figure}[hbt]
\begin{center}
\gr{clap-to-stop}{.4}
\caption{Faites du bruit pour arrêter les moteurs}\label{fig.clap-to-stop}
\end{center}
\end{figure}

\textbf{\thesection.3}:
Ce programme a deux paires événement-action comme le montre la \cref{fig.bump}. Une paire fait avancer Thymio lorsque le bouton central est pressé et une autre pour le faire s'arrêter lorsque l'événement «\,tape\,» se produit.

{\raggedleft \hfill Programme \bu{bump.aesl}}

\begin{figure}[hbt]
\begin{center}
\gr{bump}{.4}
\caption{Arrêter les moteurs lorsque Thymio rentre dans un mur}\label{fig.bump}
\end{center}
\end{figure}


\section{Aimer pour un temps}

\textbf{\thesection.1}:
Voyez la \cref{fig.three} pour la solution complète. Lorsque le bouton avant est pressé, les moteurs se mettent en route et un compte à rebours de 3 secondes est lancé. Lorsque le compte à rebours arrive au bout des trois secondes, Thymio recule. La dernière paire événement-action arrête Thymio lorsque le bouton central est pressé.

{\raggedleft \hfill Programme \bu{run-three-seconds.aesl}}


\begin{figure}[hbt]
\begin{center}
\gr{run-three-seconds}{.4}
\caption{Avance trois secondes puis recule}\label{fig.three}
\end{center}
\end{figure}


\section{États: Pour ne pas toujours faire la même chose (avancé)}

\textbf{\thesection.1}:
Il y aura deux états dans ce cas: l'état \emph{gauche} \blksm{state-left} lorsque Thymio tourne à gauche et l'état \emph{droite} \blksm{state-right} lorsqu'il tourne à droite. Initialement, toucher le bouton avant dans l'état initial (éteint, éteint, éteint, éteint) règle un compte à rebours d'une seconde et passe l'état à \emph{gauche} (\cref{fig.dance-start}). Lorsque le compte à rebours arrive à zéro dans l'état \emph{gauche}, Thymio tourne à gauche, règle un compte à rebours de deux secondes et passe en état \emph{droite} (\cref{fig.dance-left}). Lorsque le nouveau compte à rebours arrive à zéro en état \emph{droite}, Thymio tourne à droite, règle un nouveau compte à rebours de deux secondes et retourne dans l'état \emph{gauche} (\cref{fig.dance-right}). Ces deux comportements se répètent indéfiniment, vous devrez donc appuyer sur \blksm{stop} pour arrêter le robot ou ajouter une paire événement-action pour stopper Thymio, par exemple avec le bouton central.

{\raggedleft \hfill Programme \bu{dance.aesl}}


\begin{figure}
\begin{center}
\gr{dance-start}{0.4}
\caption{Commencer à danser}\label{fig.dance-start}
\end{center}
\end{figure}

\begin{figure}
\begin{center}
\gr{dance-left}{0.4}
\caption{Danse pendant deux secondes à gauche}\label{fig.dance-left}
\end{center}
\end{figure}

\begin{figure}
\begin{center}
\gr{dance-right}{0.4}
\caption{Danse pendant trois secondes à droite}\label{fig.dance-right}
\end{center}
\end{figure}

\textbf{\thesection.2}:
Je n'ai pas été capable de résoudre ce problème. J'ai pensé qu'il serait possible de détecter l'événement lorsqu'\emph{un capteur} ne détecte plus la ligne \emph{avant} que l'événement \emph{les deux capteurs} ne voient plus la ligne. Ce programme permettrait d'identifier et de se rappeler de quel côté Thymio a quitté la ligne. Ceci s'est prouvé impossible, même à vitesse lente.

Le mieux que j'ai pu faire était d'avoir un programme qui se souvienne de quel côté Thymio a quitté la ligne pour la dernière fois, même si ça s'est passé bien avant.

{\raggedleft \hfill Programme \bu{follow-line-and-find.aesl}}

Thymio a quand même un comportement intéressant: lorsqu'il arrive au bout de la ligne, il tourne lentement sur lui-même complètement jusqu'à se retrouver à nouveau sur la ligne dans l'autre direction.

L'état \emph{gauche} \blksm{state-left} se souvient que Thymio a quitté la ligne du côté gauche et l'état \emph{droite} \blksm{state-right} se souvient que Thymio a quitté lla ligne du côté droit (\cref{fig.follow3}).

Lorsque les deux capteurs ne détectent plus la ligne, Thymio tourne dans la direction indiquée pas l'état courant (\cref{fig.follow1}). 

Il est aussi possible que Thymio ne voit plus la ligne ligne sans qu'il n'ai jamais quitté la ligne ni à gauche, ni à droite. Dans ce cas, il est dans l'état complètement éteint et il tournera à droite : \blkc{follow2} Le choix de tourner à droite plutôt qu'à gauche est complètement arbitraire.

\begin{figure}
\begin{center}
\gr{follow3}{0.4}
\caption{Changer l'état pour se souvenir de la direction}\label{fig.follow3}
\end{center}
\end{figure}

\begin{figure}
\begin{center}
\gr{follow1}{0.4}
\caption{Cherche à gauche ou à droite}\label{fig.follow1}
\end{center}
\end{figure}

\section{Thymio apprend à compter}

\textbf{\thesection.1}:
Vous pouvez compter jusqu'à 4 puisqu'il y a 4 quartiers. Chacun d'eux peut être allumé pour compter un objet.

\textbf{\thesection.2}:
Le quartier d'en bas à gauche sera utilisé pour représenté le nombre de 4 dans le chiffre binaire. Si tous les quartiers sont éteints, le nombre représenté et 0. S'ils sont tous allumé, le nombre représenté est $4+2+1=7$. Huit paires événement-action sont nécessaires, une pour chaque transition entre $n$ et $n+1$ (modulo 8).

{\raggedleft \hfill Programme \bu{count-to-eight.aesl}}

\textbf{\thesection.3}:
Il y a 4 quartiers en tout. Nous pouvons utiliser un quartier pour représenter le nombre de 1, un autre pour les 2, un autre pour les 4 et le dernier pour les 8. Si tous les quartiers sont allumés, nous avons donc: $8+4+2+1 = 15$.


\textbf{\thesection.4}:
Ce programme est l'image miroir du programme d'addition. Par exemple, si le nombre binaire est $xyz1$, soustraire 1 reviens à écrire $xyz0$, quelle que soit la valeur de $x$, $y$ ou $z$. Si le nombre est $xyz0$, vous devrez affecter $z$, voire même $y$ et $x$ suivant leur valeur. Finalement, soustraire 1 depuis 0000 donne 1111 (donc quinze).

{\raggedleft \hfill Programme \bu{subtraction.aesl}}

\textbf{\thesection.5}:
Une paire événement-action est nécessaire pour compter chaque bande de ruban adhésif noir. Par exemple, la paire suivante: \blkc{count3} change le compte de 2 à 3 lorsque le ruban adhésif noir est détecté.

{\raggedleft \hfill Programme \bu{count-tapes-four.aesl}}

\end{document}
