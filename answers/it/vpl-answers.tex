\documentclass[12pt,a4paper,italian]{article}
\usepackage{../../vpl}
\graphicspath{{../../images/}}

\begin{document}
\thispagestyle{empty}

\begin{center}
\begin{bfseries}

\begin{Large}
Primi passi nella Robotica

con il Robot Thymio-II

e l'ambiente visuale di programmazione Aseba/VPL

\bigskip

Risposte agli Esercizi

\end{Large}

Version 1.3{\textasciitilde}pre1

\bigskip

\href{http://www.weizmann.ac.il/sci-tea/benari/}{Moti Ben-Ari} e altri

\end{bfseries}
\end{center}

\bigskip

\copyright{}\  2013--14 by Moti Ben-Ari e altri.

Questo lavoro è rilasciato con licenza Creative Commons
Attribution-ShareAlike 3.0 Unported. Per vedere una copia della
licenza, visitare
\url{http://creativecommons.org/licenses/by-sa/3.0/}
o inviare una lettera a Creative Commons, 444 Castro Street, Suite 900,
Mountain View, California, 94041, USA.

\begin{center}
\hspace{6pt}\includegraphics[width=.2\textwidth]{../images/by-sa}
\end{center}

\section{Il tuo primo Progetto di Robotica}

Nessun esercizio.

\section{Cambiare colore}

\textbf{\thesection.1}: 
Quando si mescolano i 3 colori del robot
Thymio-II, ciascun colore può avere una intensità tra 0 (spento) e
32 (acceso al massimo). Quindi ci sono $33 \times 33 \times 33=35,937$
differenti colori. Se si considerano solo gli estremi dell'intensità
spento o accesso al massimo, i colori sono: spento (tutti i colori
spenti), bianco (tutti i colori accesi), rosso, verde, blu (un solo
colore accceso alla volta), giallo (rosso e verde acceso, blu spento),
ciano (blu e verde accesi, rosso spento), magenta (rosso e blu acceso,
verde spento).

\section{Mettersi in movimento}

\textbf{\thesection.1}: Il robot dovrebbe essere in grado di fermarsi.
La velocità massima di Thymio-II è 20 centimetri per secondo. Poichè
i sensori sono campionati dieci volte al secondo, il robot si può
muovere al mssimo di 2 centimetri prima di fermarsi. Questo dovrebbe
essere sufficientemente veloce per impedire all robot di cadere dalla
tavola, ma siate pronti a prenderlo al volo, non si sa mai!

\section{Un robot da compagnia}

\textbf{\thesection.1}: Sostituire le prime due coppie evento-azione
con le coppie indicate in \cref{fig.answer1}. la prima coppia
accende entrambi i motori quando tutti i sensori \emph{non} rilevano un ostacolo;
la seconda coppia arresta i motori quando i sensori a terra rilevano il bordo
del tavolo.

{\raggedleft \hfill File di programma \bu{likes-and-stops.aesl}}

\begin{figure}[hbt]
\begin{center}
\gr{likes-and-stops}{0.4}
\caption{Fermarsi al bordo del tavolo}\label{fig.answer1}
\end{center}
\end{figure}

\textbf{\thesection.2}: Una coppia evento-azione viene eseguita quando il suo evento
si verifica. Eventi come toccare un pulsante si verificano quando questo evento esterno
ha luogo. Altri eventi, come campionare i sensori, si verificano a intervalli fissi
ad esempio 10 volte al secondo. Quando i sensori sono campionati, tutti gli eventi per
i sensori si verificano ``allo stesso tempo.'' Il computer nel robot non può
effettivamente eseguire tutte le coppie evento-azione simultaneamente; invece, le
coppie evento-azione vengono eseguite ad una ad una \emph{nell'ordine} in cui
appaiono dall'alto verso il basso nel programma.

Nel programma in \cref{fig.answer1}, i sensori orizzontali
non rilevano un ostacolo e mantengono i motori accesi; \emph{allora}  i
sensori del terreno rilevano il nastro e fermano il robot.

Nel programma in \cref{fig.change}, i sensori del terreno
rilevano il nastro e cercano di fermare i motori, ma la seconda coppia evento-azione
mantiene i motori accesi, prima che possano fermarsi.

{\raggedleft \hfill File di programma \bu{likes-changed-order.aesl}}

\begin{figure}[hbt]
\begin{center}
\gr{likes-change-order}{0.4}
\caption{Cambiare l'ordine delle coppie evento-azione}\label{fig.change}
\end{center}
\end{figure}

\textbf{\thesection.3}: 
\begin{itemize}
\item Usa i sensori 1 e 3: Il robot è meno sensibile alla presenza
della tua mano sul fianco. Si dovrà spostarlo più vicino al centro.
\item Usa entrambi i sensori 0 e 1 per far girare il robot a sinistra e entrambi i sensori 3
e 4 per far girare il robot a destra: Il robot rileva una zona più ampia di
fronte e lateralmente. Non c'è più bisogno di essere così attento nel mettere mano.
\item Aggiungi coppie evento-azione per i sensori posteriori 5 e 6: è ora possibile
far girare il robot mettendo la mano vicino alla parte posteriore del robot.
\end{itemize}


\section{Il robot trova la sua strada da solo}

\textbf{\thesection.1}: la coppia evento-azione \blkc{gentle-left}
provocherà una svolta a sinistra dolce quando entrambi i sensori del terreno rilevano molta
luce. Se si aumenta la velocità, il robot potrebbe correre troppo lontano dalla
linea e non rilevarla più quando si gira. Se il robot arriva alla fine della
linea farà anche una curva dolce a sinistra.

{\raggedleft \hfill File di programma \bu{find-line.aesl}}

\textbf{\thesection.2}: Il robot naturalmente si sposterà \emph{lontano}  dalla linea.

\textbf{\thesection.3}:
\begin {itemize}
\item Le curve dolci sono più facili da seguire;
\item Le curve strette sono più difficili da seguire;
\item Se le linee sono a zig-zag il robot potrebbe non tornare sulla linea prima
che si incontri la prossima curva.
\item linee più larghe rendono meno probabile che il robot abbandoni la linea;
\item linee strette  rendono più probabile che una piccola deviazione faccia sì che 
il robot abbanoni la linea. Pertanto, ci saranno curve frequenti
e il movimento sarà a scatti.
\end{itemize}

\textbf{\thesection.4}:
\begin{itemize}
\item Se gli eventi di rilevamento sensori sono più frequenti, il robot avrà
maggiori probabilità di rispondere in tempo prima di uscire fuori dalla linea; se sono meno
frequenti, potrebbe uscire completamente fuori dalla linea prima che avvenga il rilevamento.
\item Se i sensori sono più distanti tra loro sarà necessaria una linea più ampia, ma
è più probabile che l'allontanarsi dalla linea sia rilevato prima di abbanonarla completamente; è vero il contrario se i sensori sono più vicini.
\item Se ci sono più sensori, il robot può essere più preciso nei suoi
movimenti: curve dolci se un solo sensore rileva il pavimento e curve 
nette se più di un sensore rileva il pavimento.
\end{itemize}

\section{Sonagli e fischietti}

\textbf{\thesection.1}:
Nel codice Morse, una linea è tre volte più lunga di un punto. Le azioni in
\cref{fig.morse} utilizzano tre note bianche al tono più alto per la
linea e una per il punto. Dal momento che ci devono essere esattamente sei toni, riempiamo
la melodia con note brevi al tono più basso che non verrà sentito
così bene come i toni alti.

{\raggedleft \hfill File di programma \bu{bells-morse.aesl}}

\begin{figure}
\begin{center}
\gr{morse}{.4}
\caption{Codice morse}\label{fig.morse}
\end{center}
\end{figure}


\textbf{\thesection.2}: Il programma in \cref{fig.clap-to-start} accende i motori quando
l'evento applauso si verifica  e li spegne quando il tasto centrale viene
toccato.

{\raggedleft \hfill File di programma \bu{clap-start.aesl}}

Per il comportamento opposto, sembra ovvio che il programma in
\cref{fig.clap-to-stop} dovrebbe funzionare, ma non è così. Il motivo è
che l'evento che provoca l'arresto del motore non è specificamente
battere le mani; \emph{qualsiasi} rumore forte farà in modo che l'evento si verifichi. I motori fanno così tanto rumore che non appena partono, l'evento forte rumore si verifica e i motori si fermano.

La lezione da imparare è di non mettere
troppa fiducia in ciò che viene mostrato su un blocco visivo!

{\raggedleft \hfill File di programma \bu{clap-stop.aesl}}

\begin{figure}
\begin{center}
\gr{clap-to-start}{.4}
\caption{Battere le mani per avviare il motore}\label{fig.clap-to-start}
\end{center}
\end{figure}

\begin{figure}[hbt]
\begin{center}
\gr{clap-to-stop}{.4}
\caption{Battere le mani per fermare il motore}\label{fig.clap-to-stop}
\end{center}
\end{figure}

\textbf{\thesection.3}:
Il programma ha due coppie evento-azione (\cref{fig.bump}): una per
muovere il robot quando un pulsante viene toccato e l'altra per fermarlo quando si verifica un
evento urto.

{\raggedleft \hfill File di programma \bu{bump.aesl}}

\begin{figure}[hbt]
\begin{center}
\gr{bump}{.4}
\caption{Ferma il motore quando il robot sbatte in un muro}\label{fig.bump}
\end{center}
\end{figure}


\section{Un tempo per amare}

\textbf{\thesection.1}:
Guarda \cref{fig.three}. 
Quando il pulsante frontale
viene toccato i motori vengono accesi e avviato un timer impostato a tre secondi.
Allo scadere del timer, i motori sono invertiti. Infine, quando  si tocca il
pulsante centrale, i motori vengono fermati.

{\raggedleft \hfill File di programma \bu{run-three-seconds.aesl}}


\begin{figure}[hbt]
\begin{center}
\gr{run-three-seconds}{.4}
\caption{Corri tre secondi e torna indietro}\label{fig.three}
\end{center}
\end{figure}


\section{Stati: non fare sempre le stesse cose}

\textbf{\thesection.1}:
Ci saranno due stati: stato \emph{sinistra} \blksm{state-left} quando il
robot gira a sinistra e stato \emph{destra} \blksm{state-right} quando
il robot gira a destra. Inizialmente, toccando il pulsante in avanti nello
lo stato iniziale (spento, spento, spento, spento) si imposta un timer di un secondo e lo stato
\emph{sinistra} (\cref{fig.dance-start}). Quando il timer scatta nello
stato \emph{sinistra}, il robot gira a sinistra, (re) imposta il timer per
due secondi e va nello stato \emph{destra}
(\cref{fig.dance-left}). Quando il timer scatta nello stato di
\emph{destra}, ​​il robot gira a destra, (ri) imposta un timer per due
secondi e va nello stato \emph{sinistra} (\cref{fig.dance-right}).
Questi due comportamenti sono eseguiti ripetutamente, quindi dovrete cliccare
\blksm{stop} per fermare il robot.


{\raggedleft \hfill File di programma \bu{dance.aesl}}


\begin{figure}
\begin{center}
\gr{dance-start}{0.4}
\caption{Avvia la danza}\label{fig.dance-start}
\end{center}
\end{figure}

\begin{figure}
\begin{center}
\gr{dance-left}{0.4}
\caption{Danza 2 secondi a sinistra}\label{fig.dance-left}
\end{center}
\end{figure}

\begin{figure}
\begin{center}
\gr{dance-right}{0.4}
\caption{Danza 3 secondi a destra}\label{fig.dance-right}
\end{center}
\end{figure}

\textbf{\thesection.2}:
Non ero in grado di risolvere questo problema. Avevo pensato che sarebbe
possibile rilevare l'evento quando \emph{un sensore}  non rileva la
linea \emph{prima} dell'evento quando \emph{entrambi}  i sensori non rilevano
la linea. Ciò consentirebbe al programma di identificare e ricordare da
che parte il robot corse fuori dalla linea. Questo si è rivelato impossibile, anche a
basse velocità.

Il meglio che ho potuto fare è di fare sì che il programma ricordi su quale lato il
robot è uscito dalla linea, anche se ciò è avvenuto molto prima.

{\raggedleft \hfill File di Programma \bu{follow-line-and-find.aesl}}

Il robot ha ancora un comportamento interessante: quando arriva alla fine della
linea, lentamente gira completamente su se stesso e torna indietro
lungo la linea!

Lo stato \emph{sinistra} \blksm{state-left} ricorda che il robot ha lasciato il
lato sinistro della linea e lo stato \emph{destra} \blksm{state-right}
ricorda che il robot è uscito dal lato destro della linea.
(\cref{fig.follow3}).

Quando entrambi i sensori non riescono a riconoscere la linea, il robot gira nella
direzione indicata dallo stato attuale (\cref{fig.follow1}).

È anche possibile che il robot esca dalla linea, senza mai
essere uscito di lato; lo stato corrente quindi è quello iniziale con tutto \emph{spento};
in questo caso scegliamo arbitrariamente di girare a destra. \blkc{follow2}

\begin{figure}
\begin{center}
\gr{follow3}{0.4}
\caption{Cambia lo stato per ricordare la direzione}\label{fig.follow3}
\end{center}
\end{figure}

\begin{figure}
\begin{center}
\gr{follow1}{0.4}
\caption{Cerca a destra o a sinistra}\label{fig.follow1}
\end{center}
\end{figure}

\clearpage

\section{Contare}

\textbf{\thesection.1}:
Puoi contare fino a quattro perché ci sono quattro quarti dello stato
ciascuno dei quali può essere \textbf{acceso} per contare un oggetto.

\textbf{\thesection.2}:
L'elemento in basso a sinistra dell'icona di stato verrà utilizzato per rappresentare il
numero di 4 di in un numero binario. Se tutti gli elementi sono spenti, il numero rappresentato
è 0; se tutti gli elementi sono accesi, il numero rappresentato è 4 + 2 + 1 = 7.
Sono necessarie otto coppie evento-azione, una per ogni transizione tra $n$ e
$n+1$ (modulo 8).


{\raggedleft \hfill File di programma \bu{count-to-eight.aesl}}


\textbf{\thesection.3}:
Ci sono quattro elementi nello stato. Possiamo usare un elemento per contare
ciascuno dei numeri di 1, 2, 4 e 8. Pertanto, possiamo contare
da 0 a 8+4+2+1=15.


\textbf{\thesection.4}:
Questo programma è l'immagine speculare del programma per sommare.
Ad esempio, se il numero binario è XYZ1,
sottraendo uno si ha xyz0, qualunque siano i valori di XYZ.
Se il numero è xyz0, si dovrà prendere in prestito una cifra binaria,
e ci saranno diverse coppie evento-azione a seconda del valore
di xyz. Infine, sottraendo uno da 0000 = 0 dà 1111 = 15, nella aritmetica ciclica.

{\raggedleft \hfill File di programma \bu{subtraction.aesl}}

\textbf{\thesection.5}:
E' necessaria Una coppia evento-azione per contare ogni striscia di nastro adesivo.
Ad esempio, la seguente coppia \blkc{count3} modifica il conteggio da 2 a 3 quando viene rilevato un nastro.

{\raggedleft \hfill File di programma \bu{count-tapes-four.aesl}}

\end{document}
