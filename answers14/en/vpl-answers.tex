\documentclass[11pt,a4paper,english]{article}
\usepackage{../../vpl}
\graphicspath{{../../images14/}}

\begin{document}
\thispagestyle{empty}

\begin{center}
\begin{bfseries}

\begin{Large}
First Steps in Robotics

with the
Thymio Robot

and the
Aseba/VPL Environment

\bigskip

Answers to the Exercises

\end{Large}

Version 1.4{\textasciitilde}pre1

\bigskip

\href{http://www.weizmann.ac.il/sci-tea/benari/}{Moti Ben-Ari} and other contributors

\end{bfseries}
\end{center}

\bigskip

\copyright{}\  2013--14 by Moti Ben-Ari and other contributors.

This work is licensed under the Creative Commons
Attribution-ShareAlike 3.0 Unported License. To view a copy
of this license, visit
\url{http://creativecommons.org/licenses/by-sa/3.0/}
or send a letter to Creative Commons, 444 Castro Street, Suite 900,
Mountain View, California, 94041, USA.

\begin{center}
\hspace{6pt}\includegraphics[width=.2\textwidth]{../images/by-sa}
\end{center}

\section{Your First Robotics Project}

No exercises.

\section{Changing Colors}

\textbf{\thesection.1}: 
When mixing the three colors of the Thymio robot, each color can have
an intensity between 0 (off) and 32 (on maximum). Therefore, there are
$33 \times 33 \times 33=35,937$ different colors. If we consider only
the extremes of intensity off or on maximum, the colors are: off (all
colors off), white (all colors on), red, green, blue (one color on),
yellow (red and green on, blue off), cyan (blue and green on, red off),
magenta (red and blue on, green off).


\section{Let's Get Moving}

\textbf{\thesection.1}: The robot should be able to stop. The
maximum speed of the Thymio is 20 centimeters per second. Since the
sensors are sampled 10 times a second, the robot can move at most 2
centimeters before it stops. That should be sufficiently fast to prevent
the robot from falling off the table, but be prepared to catch it just
in case!

There are no ground sensors at the back of the Thymio robot, so that the
time that the sensors in the front detect the edge of the table, the
entire body of the robot has moved beyond the edge and fallen off.

\section{A Pet Robot}

\textbf{\thesection.1}: Replace the first two event-action pairs
with the pairs shown in \cref{fig.answer1}. The first pair turns
both motors on when all the sensors do \emph{not} detect an obstacle;
the second pair stops the motors when the ground sensors detect the edge
of the table.

{\raggedleft \hfill Program file \bu{likes-and-stops.aesl}}

\begin{figure}[hbt]
\begin{center}
\gr{likes-and-stops}{0.3}
\caption{Stop at the edge of the table}\label{fig.answer1}
\end{center}
\end{figure}

\textbf{\thesection.2}: An event-action pair is run when its event
occurs. Events like touching a button occur when this external event
occurs. Other events, like sampling sensors, occur at fixed intervals
such as 10 times a second. When the sensors are sampled, all events for
the sensors occur ``at the same time.'' The computer in the robot cannot
actually run all the event-action pairs simultaneously; instead, the
event-action pairs are run one-by-one \emph{in the order} that they
appear from top to bottom in the program.

In the program in \cref{fig.answer1}, the horizontal sensors
do not detect an obstacle and keep the motors running; \emph{then} the
ground sensors detect the tape and stop the robot.

In the program in \cref{fig.change}, the ground sensors
detect the tape and try to stop the motors, but the second event-action
pair keeps the motors on before they have a chance to stop.

{\raggedleft \hfill Program file \bu{likes-changed-order.aesl}}

\begin{figure}[hbt]
\begin{center}
\gr{likes-changed-order}{0.3}
\caption{Changing the order of the event-action pairs}\label{fig.change}
\end{center}
\end{figure}

\textbf{\thesection.3}: 
\begin{itemize}

\item Use sensors 1 and 3: The robot is less sensitive to the presence
of your hand on the side. You will have to move it closer to the center.

\item Use both sensors 0 and 1 to turn the robot left and both sensors 3
and 4 to turn the robot right: The robot senses a wider area in the
front and side. You don't have to be as careful in placing your hand.

\item Add event-action pairs for the rear sensors 5 and 6: You can now
cause the robot to turn by placing your hand near the back of the robot.

\end{itemize}


\section{The Robot Finds Its Way by Itself}


\textbf{\thesection.1}: The event-action pair \blkc{gentle-left}
will cause a gentle left turn when both ground sensors detect a lot of
light. If you increase the speed, the robot might run too far off the
line and not detect it when turning. If the robot gets to the end of the
line it will also make a gentle left turn.

{\raggedleft \hfill Program file \bu{find-line.aesl}}

\textbf{\thesection.2}: The robot will, of course, move \emph{away} from
the line.

\textbf{\thesection.3}:
\begin{itemize}
\item Gentle turns are easier to follow; 
\item Sharp turns are harder to follow; 
\item The robot might not return to the line before
the next turn is encountered;
\item Wider lines make it less likely that the robot will run off the line; 
\item Narrow lines make it more likely that a small deviation will cause
the robot to run off the line. Therefore, there will be frequent turns
and the movement will be jerky.
\end{itemize}


\textbf{\thesection.4}:
\begin{itemize}
\item If the ground sensing events are more frequent, the robot will be
more likely to respond to running off the line; if they are less
frequent, it might run completely off the line before detection occurs.
\item If the sensors are further apart a wider line will be needed but
it is more likely that leaving the line will be detected before running
off it completely; the opposite is true for sensors that are closer
together.
\item If there are more sensors, the robot can be more precise in its
movements: gentle turns if only one sensor detects the floor and sharper
turns if more than one sensor detects the floor.
\end{itemize}

\section{Bells and Whistles}

\textbf{\thesection.1}:
In Morse code, a dash is three times longer than a dot. The actions in
\cref{fig.morse} use three white notes at the highest tone for the
dash and one for the dot. Since there must be exactly six tones, we fill
out the tune with short notes at the lowest tone which will not be heard
as well as the highest tones.

{\raggedleft \hfill Program file \bu{bells-morse.aesl}}

\begin{figure}
\begin{center}
\gr{morse}{.3}
\caption{Morse code}\label{fig.morse}
\end{center}
\end{figure}


\textbf{\thesection.2}: The program in \cref{fig.clap-to-start} turns the motors on when
the clap event occurs and turns them off when the center button is
touched.

{\raggedleft \hfill Program file \bu{clap-start.aesl}}

For the opposite behavior, it seems obvious that the program in
\cref{fig.clap-to-stop} should work, but it doesn't. The reason is
that the event that causes the motor to stop is not specifically
clapping; \emph{any} loud noise will cause the event to occur. The
motors make so much noise that as soon as they start, the loud-noise
event occurs and the motors stop.

{\raggedleft \hfill Program file \bu{clap-stop.aesl}}

\begin{figure}
\begin{center}
\gr{clap-to-start}{.3}
\caption{Clap to start the motor}\label{fig.clap-to-start}
\end{center}
\end{figure}

\begin{figure}[hbt]
\begin{center}
\gr{clap-to-stop}{.3}
\caption{Clap to stop the motor}\label{fig.clap-to-stop}
\end{center}
\end{figure}

\textbf{\thesection.3}:
The program has two event-action pairs (\cref{fig.bump}): one to
move the robot when a button is touched and the other to stop it when a
tap event occurs.

{\raggedleft \hfill Program file \bu{bump.aesl}}

\begin{figure}[hbt]
\begin{center}
\gr{bump}{.3}
\caption{Stop the motor when the robot bumps into a wall}\label{fig.bump}
\end{center}
\end{figure}

\section{A Time to Like}

\textbf{\thesection.1}:
See \cref{fig.three}. When the front button
is touched the motors are turned on and a three-second timer is started.
When the timer expires, the motors are reversed. Finally, when the
center button is touched, the motors are stopped.

{\raggedleft \hfill Program file \bu{run-three-seconds.aesl}}


\begin{figure}[hbt]
\begin{center}
\gr{run-three-seconds}{.3}
\caption{Run three seconds and reverse}\label{fig.three}
\end{center}
\end{figure}


\section{States}

\textbf{\thesection.1}:
There will be two states: state \emph{left} \blksm{state-left} when the
robot turns to the left and state \emph{right} \blksm{state-right} when
the robot turns to the right. Initially, touching the forward button in
the initial state (off, off, off, off) sets a one-second timer and state
\emph{left} (\cref{fig.dance-start}). When the timer runs down in
state \emph{left}, the robot turns to the left, (re-)sets a timer for
two seconds and goes to state \emph{right}
(\cref{fig.dance-left}). When the timer runs down in state
\emph{right}, the robot turns to the right, (re-)sets a timer for two
seconds and goes to state \emph{left} (\cref{fig.dance-right}).
These two behaviors are run repeatedly, so you will have to click
\blksm{stop} to stop the robot.

{\raggedleft \hfill Program file \bu{dance.aesl}}


\begin{figure}
\begin{center}
\gr{dance-start}{0.4}
\caption{Starting the dance}\label{fig.dance-start}
\end{center}
\end{figure}

\begin{figure}
\begin{center}
\gr{dance-left}{0.4}
\caption{Dance two seconds to the left}\label{fig.dance-left}
\end{center}
\end{figure}

\begin{figure}
\begin{center}
\gr{dance-right}{0.4}
\caption{Dance three seconds to the right}\label{fig.dance-right}
\end{center}
\end{figure}

\newpage

\textbf{\thesection.2}:
I was not able to solve this problem. I had assumed that it would be
possible to detect the event when \emph{one sensor} no longer detect the
line \emph{before} the event when \emph{both sensors} no longer detect
the line. This would enable the program to identify and remember from
which side the robot ran off the line. This proved impossible even at
low speeds.

The best I could do is to have the program remember on which side the
robot last ran off the line, even if that occurred much earlier.

{\raggedleft \hfill Program file \bu{follow-line-and-find.aesl}}

The robot does have one cool behavior: When it gets to the end of the
line, it slowly turns completely around and heads back the other way
along the line!

State \emph{left} \blksm{state-left} remembers that the robot left the
left side of the line and state \emph{right} \blksm{state-right}
remembers that the robot ran off the right side of the line
(\cref{fig.follow3}).

When both sensors fail to detect the line, the robot turns in the
direction indicated by the current state (\cref{fig.follow1}).

It is also possible that the robot runs off the line without ever having
left one side so that the state is the initial state of all \emph{off};
in that case, we arbitrary choose to turn right (\cref{fig.follow2}).

\begin{figure}
\begin{center}
\gr{follow3}{0.4}
\caption{Change state to remember direction}\label{fig.follow3}
\end{center}
\end{figure}

\begin{figure}
\begin{center}
\gr{follow1}{0.4}
\caption{Search to the left or the right}\label{fig.follow1}
\end{center}
\end{figure}

\begin{figure}
\begin{center}
\gr{follow1}{0.4}
\caption{Turn to the right}\label{fig.follow2}
\end{center}
\end{figure}


%\clearpage

\section{Counting}

\textbf{\thesection.1}:
You can count to four because there are four quarters of the state
each of which can be set \textbf{on} to count an object.

\textbf{\thesection.2}:
The lower left quarter of the state icon will be used to represent the
number of 4's in a binary number. If all quarters are off, the number represented
is 0; if all quarters are on, the number represented is 4+2+1=7.
Eight event-action pairs are needed, one for each transition between $n$ and
$n+1$ (modulo 8).

{\raggedleft \hfill Program file \bu{count-to-eight.aesl}}


\textbf{\thesection.3}:
There are four quarters in the state. We can use one quarter to count
each of the number of 1's, 2's, 4's and 8's. Therefore, we can count
from 0 to 8+4+2+1=15.


\textbf{\thesection.4}:
This program is a mirror image of the program for adding.
For example, if the binary number is xyz1,
then subtracting one gives xyz0, whatever the values of xyz.
If the number is xyz0, you will have to borrow a binary digit,
and there will be different event-action pairs depending on the value
of xyz. Finally, subtracting one from 0000=0 gives 1111=15 in cyclic
arithmetic.

{\raggedleft \hfill Program file \bu{subtraction.aesl}}

\textbf{\thesection.5}:
An event-action pair is needed to count each strip of tape.
For example, the following pair \blkc{count3} changes the count from 2 to 3 when a tape is detected.

{\raggedleft \hfill Program file \bu{count-tapes-four.aesl}}


\section{Accelerometers}

\textbf{\thesection.1}: VPL does not allow identical events in more than
one event-actions pair, unless the parameters are different. For this
block, choosing a different angle for the red segment makes the events
different.

There are seven different positions of the red segment between the
vertical and horizonal on one side. Therefore, each segment represents
$90 / 6.5 \approx 14$ degrees. (The vertical segment is half within the
left quarter circle and half within the right quarter circle, so we
divide 90 by 6.5 and not 7.)

\end{document}
