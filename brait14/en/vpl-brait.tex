\documentclass[12pt,a4paper,english]{report}

\usepackage{../../vpl}
\usepackage{url}
\def\UrlFont{\sf\small}
\graphicspath{{../../images14/}}


%%%%%%%%%%%%%%%%%%%%%%%%%%%%%%%%%%%%%%%%%%%%%%%%%%%%%%%%%%%%%%%%%%%%%%
\begin{document}

\thispagestyle{empty}

\begin{center}
\begin{Huge}
\begin{bfseries}
Braitenberg Creatures in VPL
\end{bfseries}
\end{Huge}

\bigskip
\bigskip

\begin{Large}
Version 1.0
for \href{https://aseba.wikidot.com/en:downloadinstall}{Aseba 1.4}
\end{Large}


\bigskip
\bigskip
\bigskip
\bigskip
\bigskip

\begin{LARGE}
\href{http://www.weizmann.ac.il/sci-tea/benari/}{Moti Ben-Ari}
\end{LARGE}

\begin{Large}
\href{http://www.weizmann.ac.il/sci-tea/benari/}{http://www.weizmann.ac.il/sci-tea/benari/}
\end{Large}

\end{center}

\vfill

\begin{center}
\copyright{}\  2015 by \href{http://www.weizmann.ac.il/sci-tea/benari/}{Moti Ben-Ari}% and other contributors.
\end{center}

This work is licensed under the Creative Commons
Attribution-ShareAlike 3.0 Unported License. To view a copy
of this license, visit
\url{http://creativecommons.org/licenses/by-sa/3.0/}
or send a letter to Creative Commons, 444 Castro Street, Suite 900,
Mountain View, California, 94041, USA.

\begin{center}
\includegraphics[width=.2\textwidth]{by-sa}
\end{center}

\newpage
\thispagestyle{empty}
\setcounter{page}{1}

%%%%%%%%%%%%%%%%%%%%%%%%%%%%%%%%%%%%%%%%%%%%%%%%%%%%%%%%%%%%%%%%%%%%%%

\begin{center}
\Large \textbf{What are Braitenberg Creatures?}
\end{center}

\href{http://en.wikipedia.org/wiki/Valentino_Braitenberg}{Valentino
Braitenberg} was a neuroscientist who published a small book describing
the design of virtual vehicles, which exhibited surprisingly complex
behavior.\footnote{V. Braitenberg. \textit{Vehicles: Experiments in
Synthetic Psychology} (MIT Press, 1984).} Braitenberg's vehicles have
been widely used in educational robotics.

Researchers at the MIT Media Lab created hardware implementations of
Braintenberg's vehicles, which they called \emph{Braitenberg
creatures}.\footnote{David W. Hogg, Fred Martin, Mitchel Resnick.
\textit{Braitenberg Creatures}. MIT Media Laboratory, E\&L Memo 13,
1991. \url{http://cosmo.nyu.edu/hogg/lego/braitenberg_vehicles.pdf}.}
The vehicles were build from \emph{programmable bricks} that were the
forerunner of the LEGO Mindstorms robotics kits.

This document describes an implementation of the twelve Braitenberg
creatures from the MIT report adapted for the Thymio-II robot with the
Aseba/\textsc{VPL} programming environment. The MIT hardware used light
and touch sensors, while the Thymio robot relies primarily on
infrared proximity sensors. To enable comparison with the MIT report,
the names of the creatures used there have been retained, even though
they may not be appropriate for the Thymio implementations.

In the descriptions, the phrase ''detects an object'' is used. Unless
otherwise indicated, this means that an object is detected by the front
center sensor. The easiest way to do this is to place and move your hand
so that it is detected by a sensor.

The \textsc{VPL} source code is not given here, but is available in the
zip archive in which this document is delivered. The file names are the
same as the names of the creatures with the extension \texttt{\small
aesl}. For some creatures, additional behaviors are suggested in the
exercises and their implementations are given in file names with an
additional number.

Building the Braitenberg creatures is especially suited as an activity
for novices, because they require very short programs (nine of them
require only two event-actions pairs!), yet they require some thought to
implement correctly. Feel free to expand the implementations to include
other features, such as detecting the edge of a table, using buttons to
turn the robot on and off, and displaying the robot's current behavior
using different colors of the LEDS.

\textbf{Version note} The implementations were done using Version 1.4 of
Aseba/\textsc{VPL} that has not yet been released. Most of them can also
be implemented in Version 1.3 with multiple event-action pairs replacing
the multiple actions per event supported in 1.4.

\textbf{Advanced mode} When an implementation needs blocks from advanced
mode such as timers or states, the word \emph{advanced} appears.

\newpage

\begin{center}
\Large \textbf{The Braitenberg Creatures}
\end{center}

\begin{description}

\item[Timid] When the robot does not detect an object, it moves forwards.
When it detects an object, it stops.

\item[Indecisive] When the robot does not detect an object, it moves
forwards. When it detects an object, it moves backwards. At just the
right distance, the robot will \emph{oscillate}, that is, it will move
forwards and backwards in quick succession.

\item[Paranoid] When the robot detects an object, it moves forwards. When
it does not detect an object, it turns to the left.

\textbf{Exercise (Paranoid1)} When an object is detected by the center
sensor of the robot, it moves forwards. When an object is detected by
the right sensor (but not by the center sensor), the robot turns right.
When an object is detected by the left sensor (but not by the center
sensor), the robot turns left.

\textbf{Exercise (Paranoid2, advanced)} As in \textbf{Paranoid}, but the
robot alternates the direction of its turn every second. \textbf{Hint}
Use states to keep track of the direction and a timer to change states.

\item[Dogged] When the robot detects an object in front, it moves
backwards. When the robot detects an object in back, it moves forwards.

\textbf{Exercise (Dogged1)} As in \textbf{Dogged}, but also: When the
robot does not detect an object, it stops.

\textbf{Exercise (Dogged2, advanced)} The robot moves forwards until it
detects an object and then moves backwards. The robot moves backwards
until it detects an object and then moves forwards. \textbf{Hint} Use
states to keep track of the direction. Detecting an objects causes the
state to change and the state controls the motors.

\item[Insecure] If an object is not detected, turn the robot's right
motor on and the left motor off. If an object is detected, turn the
right motor off and the left motor on. The robot should follow a wall
placed to its left. \textbf{Hint} See Note~\ref{e.motor} below.

\item[Driven] If the robot detects an object on the left, it turns the
right motor on and the left motor off. If the robot detects an object on
the right, it turns the left motor on and the right motor off. The robot
should approach the object in a zigzag. \textbf{Hint} See
Note~\ref{e.motor} below.

\item[Persistent (advanced)] The robot moves forwards until it detects
an object. It then moves backwards for one second and reverses to move
forwards again.

\item[Attractive and repulsive] When an object approaches the robot from
behind, the robot runs away until it is out of range.

\item[Consistent (advanced)] The robot cycles through four states when
it is tapped: moving forwards, turning left, turning right, moving
backwards.

\item[Inhumane] The robot travels forwards until it detects an obstacle
and then it stops and never moves again.

\item[Frantic (advanced)] The top LED flashes red. \textbf{Hint} You the
sensor event block with all sensors gray as explained in
note~\ref{e.gray} below.

\textbf{Exercise (Franctic1, advanced)} Implement the flashing LED using
the button event block instead of the sensor event block. Is there a
difference in the behavior of the robot? If so, what causes it?

\item[Observant (advanced)] The robot turns the top LED green when the
right sensor detects an object. The robot turns the top LED red when the
left sensor detects an object. Once a LED is turned on, it waits three
seconds before turning off; during this period, the LED does not change,
no matter what is detected.

\end{description}

\begin{center}
\Large \textbf{Notes on the VPL Blocks}
\end{center}

\begin{enumerate}

\item\label{e.motor} When one motor rotates forwards at a certain speed
and the other motor rotates backwards at the same speed, the robot turns
in place. However, if only one motor is on, the robot goes both forwards
and in the opposite direction. Since only one motor is powered, you may
need to set a higher power to overcome ground resistance.

\item\label{e.gray} The small squares on the sensor event block are
initially colored light gray, meaning that the corresponding sensors to
not take part in determining when the event occurs. In most projects,
you click one or more of the sensors to indicate the they participate in
\emph{filtering} the events: red causes an event to occur if this sensor
detects and object and black causes an event if this sensors does not
detect an object. However, if \emph{all} the sensors remain in the
original gray, then no filtering is done, that is, the event will occur
no matter what values are read from the sensors. Such a block is a
convenient way of expressing: the event \emph{always} occurs, where
\emph{always} means many times per second.

The meaning of the button event block is similar. When \emph{all}
buttons remain in the original gray color, the button event occurs many
times per second.

\end{enumerate}

\end{document}
