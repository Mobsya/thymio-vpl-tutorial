\chap{Bells and Whistles}\label{ch.bells}

Let us take a time out from complicated tasks like line following and
have some fun with the robot. We show how you can:
\begin{itemize}
\item Compose music for the robot;
\item Have the robot respond to a sound;
\item Have the robot respond when it is tapped.
\end{itemize}

\sect{Playing music}

The Thymio-II robot contains a sound synthesizer and you can program it
to play simple tunes using the music action block \blk{action-music}.


{\raggedleft \hfill Program file \bu{bells.aesl}}

You won't become a new Beethoven---there are only six notes, each of
which can be one of five tones and two lengths---but you can compose a
tune that will make your robot stand out. Figure~\ref{fig.music} shows
two event-action pairs that respond with a tune when the front or back
button is touched. There is a different tune associated with each event.

\begin{figure}
\begin{center}
\gr{music}{.4}
\caption{Playing a tune}\label{fig.music}
\end{center}
\end{figure}

The small circles are the six notes. A white note is a long note
and a black note is a short note; to change from one length to the
other, click on the circle. There are five colored horizontal bars,
representing five tones. To move a circle to one of bars, click on the
\emph{bar} above or below the circle. (Don't try to drag and drop the
circle; it won't work.)

\sect{Exercise \thechapter.1}
Write a program that will enable you to send a message is Morse code.
Letters in Morse code are encoded in sequences of long tones
(\emph{dashes}) and short tones (\emph{dots}). For example, the letter
\emph{V} is encoded by three dots followed by one dash.


\sect{Controlling your robot by sound}

The Thymio-II has a microphone. The event \blk{event-clap} will occur
when the microphone senses a loud noise, for example, from clapping your
hands. The event-action pair \blk{clap} will turn on the bottom lights
when you clap your hands.

\centeredbox{In a noisy environment, you may not be able to use this
event, because the sound level will always be high and cause repeated
events.}

\sect{Exercise \thechapter.2}

Write a program that causes the robot to move when you clap
your hands and to stop when you touch a button.

Write a program that does the opposite: starts when you touch a button
and stops when you clap your hands.


\sect{Good job, robot}

Pets don't always do what we ask them to do. Sometimes they need a pat
on the head to encourage them. You can do the same with your robot. The
Thymio-II contains a tap sensor that causes the event \blk{event-tap} to
occur in response to a quick touch on the top of the robot. For example,
the event-action pair \blk{touch} causes the top lights to turn on when
you tap the top of the robot.

Construct a program from this event-action pair and the pair \blk{clap}
that turns on the bottoms lights when you clap your hands.

{\raggedleft \hfill Program file \bu{whistles.aesl}}

Can you turn on just the top lights? This is difficult to do: a tap
causes a sound that can be loud enough to cause the bottom lights to be
turned on as well. With a little practice I was able to tap the robot
gently enough so that the sound was not considered an event.

\sect{Exercise \thechapter.3}

Write a program that causes the robot to move forward until it hits a
wall.

\textbf{Warning!} Make sure that the robot moves slowly so that it
doesn't damage itself.

