\chap{Glocken und Pfeifen}
\label{ch.bells}

Lass uns eine Pause von den schwierigen Aufgaben,
wie dem folgen einer Linie machen
und etwas Spass mit dem Roboter haben.
Wir zeigen Dir wie:

\begin{itemize}
\item Komponiere Musik für den Roboter
\item Lass den Roboter auf Musik antworten
\item Lass den Roboter antworten, falls er berührt wird.
\end{itemize}

\sect{Spiele Musik}

Der Thymio-II Roboter enthält ein Synthesizer
und Du kannst diesen programmieren,
um einfache Melodien zu spielen,
indem du den Aktionsblock Musik benutzt \blk{action-music}.

{\raggedleft \hfill Program file \bu{bells.aesl}}

Du wirst kein neuer Beethoven werden
- da nur sechs Noten, in fünf Tonlagen und zwei Tonlängen zur Verfügung stehen -
aber Du kannst eine Melodie komponieren, die dein Roboter einzigartig macht. 
Die Figur Figure~\ref{fig.music} zeigt zwei Ereignis-Aktions Paare,
welche mit einer Melodie antworten,
falls der vordere oder der hintere Schalter berührt wird.
Jedem Ereignis ist eine andere Melodie zugeordnet.

\begin{figure}
\begin{center}
\gr{music}{.4}
\caption{Spiele eine Melodie}
\label{fig.music}
\end{center}
\end{figure}

Die kleinen Kreise sind die sechs Noten.
Die weissen Noten sind lange Tonlängen
und schwarze Noten sind kurze Tonlängen.
Klicke auf den Kreis, um die Tonlänge zu ändern.
Die fünf farbigen, horizontalen Balken stehen für die Tonhöhe.
Klicke auf den \emph{Balken} über oder unter dem Kreis, um den Kreis zu bewegen. 
(Versuche nicht die Kreise zu verschieben (drag and drop), es wird nicht funktionieren). 

\sect{Übung \thechapter.1}

Schreibe ein Programm,
das Dir erlaubt eine Morsebotschaft zu verschicken.
Eine Morsebotschaft wird mit langen und kurzen Tönen kodiert.
(\emph{Striche} für lange Töne und \emph{Punkte} für kurze Töne).
Zum Beispiel wird der Buchstabe \emph{V} mit drei Punkten und einem Strich kodiert.


\sect{Kontrolliere deinen Roboter durch Töne}

Der Thymio-II hat ein Mikrofon.
Das Ereignis \blk{event-clap} findet statt,
wenn ein lautes Geräusch aufgenommen wird, wie z.B. Händeklatschen.
Das Ereignis-Aktions Paar \blk{clap} wird die Bodenlichter einschalten,
wenn du in die Hände klatschst.

\centeredbox{In einer lauten Umgebung kann ein Geräusch eventuell nicht als Ereignis verwendet werden,
da durch den hohen Geräuschepegel dauernd Ereignisse ausgelöst werden.}

\sect{Übung \thechapter.2}

Schreibe ein Programm,
dass den Roboter losfahren lässt,
falls Du in die Hände klatschst und den Roboter stoppt,
falls Du auf den Schalter drückst.

Schreibe ein Programm, das umgekehrt funktioniert:
Der Roboter soll losfahren,
falls Du auf den Schalter drückst und stoppen,
falls Du in die Hände klatschst.

\sect{Gute Arbeit Roboter!}

Haustiere machen nicht immer das,
was wir von ihnen verlangen.
Manchmal brauchen sie einen freundschaftlichen Klaps um sie zu ermutigen.
Genau gleich funktioniert das mit deinem Roboter.
Der Thymio-II enthält ein Berührungssensor, welcher ein Ereignis auslöst \blk{event-tap}, falls dem Roboter kurz auf seine Oberseite geklopft wird. So bewirkt zum Beispiel das Ereignis-Aktions Paar \blk{touch}, dass die Lichter angehen, falls auf die Oberseite des Roboters geklopft wird.

Erstelle ein Programm für dieses Ereignis-Aktions Paar
und das Paar\blk{clap}, welches die Bodenlichter angehen lässt,
falls Du in die Hände klatschst.

{\raggedleft \hfill Program file \bu{whistles.aesl}}

Kannst Du nur die Oberlichter anstellen?
Dies ist schwierig, da ein Klaps immer auch ein Geräusch erzeugt,
welches laut genug sein kann, um ebenfalls die Bodenlichter anzustellen.
Mit ein bisschen Übung wird es Dir jedoch möglich sein,
dem Roboter ein so gefühlsvollen Klaps zu geben,
dass das Geräusch kein Ereignis auslöst.

\sect{Übung \thechapter.3}

Schreibe ein Programm, dass den Roboter vorwärts fahren lässt bis er die Wand berührt. 

\textbf{Warnung} Stell sicher, dass der Roboter langsam fährt und sich dadurch nicht selbst beschädigt.
