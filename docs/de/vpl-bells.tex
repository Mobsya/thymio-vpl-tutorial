\chap{Schnickschnack}\label{ch.bells}
Wir wollen etwas Spass haben mt Thymio. Betrachten wir, wie Thymio Musik macht, auf Geräusche und auf Berührung reagiert.

\sect{Musik spielen}

Der Roboter Thymio enthält einen Synthesizer den man programmieren kann,
um einfache Melodien zu spielen. Dazu verwenden Sie den Aktionsblock für Musik: 
\blkc{action-music}
{\raggedleft \hfill Beispielprogramm \bu{bells.aesl}}

Sie werden wohl kaum zu einem zweiten Beethoven werden---es stehen nur sechs Noten, in fünf Tonlagen und zwei Tonlängen zur Verfügung---aber Sie können eine Melodie komponieren, die Ihren Roboter einzigartig macht. Die \cref{fig.music} zeigt zwei Ereignis-Aktions-Paare, welche mit einer Melodie antworten, falls der vordere oder der hintere Knopf gedrückt wird. Jedem Ereignis ist eine andere Melodie zugeordnet.

\begin{figure}
\begin{center}
\gr{bells}{.3}
\caption{Melodie spielen}\label{fig.music}
\end{center}
\end{figure}

Die kleinen Kreise stellen Noten dar. Ein weisser Kreis steht für einen langen Ton, ein schwarzer für einen kurzen. Die Tonlänge ändern Sie, indem Sie auf den Kreis klicken. Die fünf grauen, horizontalen Balken stehen für die Tonhöhe. Klicken Sie auf den \emph{Balken} über oder unter dem Kreis, um den Kreis zu bewegen oder verschieben Sie ihn per drag and drop. 

\exercisebox{\thechapter.1}{
	Schreiben Sie ein Programm, das es Ihnen erlaubt eine \href{http://de.wikipedia.org/wiki/Morsezeichen}{Morsebotschaft} zu verschicken.
	Eine Morsebotschaft wird mit langen und kurzen Tönen kodiert. 	(\emph{Striche} für lange Töne und \emph{Punkte} für kurze Töne). Zum Beispiel wird der Buchstabe \emph{V} mit drei Punkten und einem Strich kodiert.}

\sect{Roboter durch Töne steuern}

Der Roboter Thymio hat ein Mikrofon. Das Ereignis \blksm{event-clap} tritt ein, wenn ein lautes Geräusch aufgenommen wird, wie z.B. in die Hände klatschen. Das Ereignis-Aktions-Paar: 
\blkc{clap-lights} 
wird die unteren Lichter in der Farbe Türkis einschalten, wenn man in die Hände klatscht.

\trickbox[Information]{In einer lauten Umgebung kann ein Geräusch eventuell nicht als Ereignis verwendet werden, da durch den hohen Geräuschpegel dauernd Ereignisse ausgelöst werden.}

\exercisebox{\thechapter.2}{
	Schreiben Sie ein Programm, dass den Roboter losfahren lässt, wenn man in die Hände klatscht und den Roboter stoppt, wenn man einen Schalter betätigt.
	\vspace{.5em}\\
	Schreiben Sie ein Programm, das umgekehrt funktioniert: Der Roboter soll losfahren, wenn man auf den Schalter drückt und stoppen, wenn man in die Hände klatschst.
}

\bigskip
\sect{Gute Arbeit Roboter!}

Haustiere machen nicht immer das, was wir von ihnen verlangen. Manchmal brauchen sie einen freundschaftlichen Klaps um sie zu ermutigen. Genau gleich funktioniert das mit Ihrem Roboter. Thymio enthält einen Erschütterungssensor, welcher ein Ereignis auslöst \blksm{event-tap}, falls dem Roboter kurz auf seine Oberseite geklopft wird. So bewirkt zum Beispiel das Ereignis-Aktions-Paar: 
\blkc{good-job}
dass die Lichter angehen, falls auf die Oberseite des Roboters geklopft wird.

Erstellen Sie ein Programm für dieses Ereignis-Aktions-Paar und eines für das folgende Paar: \blkc{clap-lights} welches die Bodenlichter in türkis angehen lässt, wenn Sie in die Hände klatschen.

{\raggedleft \hfill Beispielprogramm \bu{whistles.aesl}}

Können Sie nur die oberen Lichter einschalten? Das ist schwierig, da ein Klaps immer auch ein Geräusch erzeugt, welches laut genug sein kann, um ebenfalls die Bodenlichter einzuschalten. Mit ein bisschen Übung wird es Ihnen möglich sein, dem Roboter einen so gefühlsvollen Klaps zu geben, dass das Geräusch kein Ereignis auslöst.


\bigskip

\exercisebox{\thechapter.3}{
Schreiben Sie ein Programm, das den Roboter vorwärts fahren lässt bis er die Wand berührt (Erschütterung). 
\vspace{.5em}\\
\textbf{Achten Sie darauf}, dass der Roboter \textbf{langsam} fährt und durch den Aufprall nicht beschädigt wird.
}
