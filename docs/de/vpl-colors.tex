\chap{Zeige Farben an}

Konstruiere ein VPL Programm, das zwei verschiedene Farben oben auf dem Roboter
anzeigt, wenn der Vorwärts- oder der Rückwärts-Knopf berührt wird. Wenn die
Links- und Rechts-Knöpfe berührt werden, sollen unten am Roboter zwei andere
Farben angezeigt werden.

{\raggedleft \hfill Beispielprogramm \bu{colors.aesl}}

Dazu brauchen wir vier Ereignis-Aktions-Paare, da es vier verschiedene
Ereignisse (das Berühren der vier Knöpfe) gibt, auf die der Roboter reagieren
soll. Sei vorsichtig: es gibt zwei verschiedene Aktions-Blöcke, um die obere
und untere Farbe des Roboters zu ändern (\blk{action-colors-up} und
\blk{action-colors-down}). Im Block für die unteren Lichter sind zusätzlich die
schwarzen Räder abgebildet.

Das fertige Programm ist im Bild~\ref{fig.colors} dargestellt.

Welche Farben werden angezeigt? In den ersten drei Ereignis-Aktions-Paaren ist
jeweils eine Grundfarbe alleine gewählt. Die weissen Quadrate für die anderen
beiden Farben sind jeweils ganz links. Diese drei Aktionen werden also reines
Rot, Blau oder Grün anzeigen. Beim letzten Ereignis-Aktions-Paar werden aber
zwei Farben gemischt: die weissen Quadrate sind bei Rot und Grün ganz rechts.
Es wird also Rot und Grün gemischt. Welche Farbe entsteht daraus?

\sect{Aufgabe \thechapter.1}

Spiele mit dem Farbbalken und finde heraus welche Farben du anzeigen kannst.
Wie kannst du Orange erstellen?

\sect{Schalte die Lichter aus}

Lass uns nun das Programm so verändern, dass alle Lichter ausgehen, wenn der
mittlere Knopf berührt wird. Wir brauchen also zwei neue
Ereignis-Aktions-Paare. Eines um die oberen und ein zweites,
um die unteren Lichter aus zu schalten. Damit die Lichter ausgehen, müssen wir alle drei
Farbbalken ganz links einstellen.

Das Ereignis ist jetzt für beide der neuen Ereignis-Aktions-Paare dasselbe,
nämlich der mittlere Knopf.

\centeredbox{
\begin{center}
\textbf{Mehrere Ereignis-Aktions-Paare}
\end{center}
\begin{itemize}
\item Wenn ein Programm ausgeführt wird, werden alle Ereignis-Aktions-Paare
    gleichzeitig ausgeführt.
\item Man kann mehrere Ereignis-Aktions-Paare mit dem gleichen Ereignis
    konstruieren, solange sie verschiedene Aktionen haben.
\item Sind zwei Ereignis-Aktions-Paare gleich (gleiches Ereignis und gleiche
    Aktion), zeigt VPL eine Fehlermeldung an.
\end{itemize}
}

\begin{figure}[htb]
\begin{center}
\gr{colors1}{0.4}
\caption{Die Farben verändern sich, wenn ein Knopf berührt wird.}\label{fig.colors}
\end{center}
\end{figure}

\begin{figure}[htb]
\begin{center}
\gr{colors2}{0.4}
\caption{Lösche die Lichter mit dem mittleren Knopf.}\label{fig.colors-off}
\end{center}
\end{figure}
