% !TeX root = vpl.tex

\chap{Ändern der Farben}\label{ch.colors}

\sect{Farben anzeigen}

Erstellen Sie ein VPL-Programm, das zwei verschiedene Farben oben auf dem Roboter anzeigt, wenn der Vorwärts- bzw. der Rückwärts-Knopf gedrückt wird. Wenn die Links- und Rechts-Knöpfe gedrückt werden, sollen unten am Roboter zwei andere Farben angezeigt werden.

{\raggedleft \hfill Beispielprogramm \bu{colors.aesl}}

Wir brauchen vier Ereignis-Aktions-Paare, da es vier verschiedene
Ereignisse gibt für das Drücken der vier Knöpfe. Mit jedem Ereignis ist eine Farbaktion verbunden. Beachten Sie den Unterschiede der Aktions-Blöcke für die obere Farben (\blksm{action-colors-up}) und die untere Farben \blksm{action-colors-down}) des Roboters. Der erste Block ändert die Farben der Lichter, die auf der Oberseite des Roboters liegen, während der zweite die Farben der Lichter der Unterseite des Roboters ändert. Im Block für die unteren Lichter sind die Räder als schwarze Balken abgebildet sowie mit einem weissen Punkt für die Ausbuchtung vorne unten (\cref{fig.bottom}).

Das fertige Programm ist im \cref{fig.colors} dargestellt.

\begin{figure}
	\gr{colors1}{.35}
	\caption{Ändern der Farben durch Betätigung der Knöpfe}\label{fig.colors}
\end{figure}

Welche Farben werden angezeigt? In den ersten drei Ereignis-Aktions-Paaren ist jeweils eine Grundfarbe alleine gewählt. Die weissen Quadrate für die anderen beiden Farben sind jeweils ganz links. Diese drei Aktionen werden also reines Rot, Blau oder Grün anzeigen. Beim letzten Ereignis-Aktions-Paar werden aber zwei Farben gemischt: die weissen Quadrate sind bei Rot und Grün ganz rechts. Es wird also Rot und Grün gemischt, also Gelb dargestellt. Man erkennt, dass der Hintergrund des Kastens jeweils angepasst wird, wenn man den Regler bewegt. 

Starten Sie das Programm (\blksm{run}) und testen Sie die 4 Pfeiltasten aus. \cref{fig.front} zeigt Thymio mit roter Farbe oben, \cref{fig.bottom} zeigt ihn mit grüner Farbe unten.

\exercisebox{\thechapter.1}{ Experimentieren Sie mit den Farbreglern um herauszufinden, welche Farben möglich sind. }

\trickbox[Hinweis]{Durch Kombination der Farben Rot, Grün und Blau lassen sich alle Farben erzeugen (\cref{fig.cube})!}

\begin{figure}
\gr{color-cubes}{.85}
\caption{Der Rot-Grün-Blau-Farbwürfel (RGB)}\label{fig.cube}
\end{figure}


\sect{Mehrere Aktionen nach einem Ereignis}

Verändern wir das Programm so, dass alle Lichter ausgelöscht werden, wenn man den mittleren Knopf betätigt. Wir benötigen dazu \emph{zwei} Aktionen für ein Ereignis in einem Ereignis-Aktions-Paar. Nachdem wir das Ereignis und die erste Aktion eingegeben haben, erscheint rechts von der Aktion ein mit einer grauen gestrichelten Linie umrahmtes Quadrat: 

\blkc{multiple-outline}

Sie können nun in dieses Quadrat die zweite Aktion hineinziehen, womit wir ein Paar mit einem Ereignis und zwei Aktionen erhalten:\label{p.multiple}

\blkc{colors-multiple}

{\raggedleft \hfill Beispielprogramm \bu{colors-multiple.aesl}}

Vergessen Sie nicht, \blksmpure{run} anzuklicken, um das Programm zu laden und zu starten. In Zukunft werden wir an diesen Schritt nicht mehr erinnern..

\bigskip

\importantbox[Regeln für Paare mit mehreren Aktionen]{
\begin{itemize}[noitemsep,nosep,leftmargin=*]
	\item Wenn ein Programm ausgeführt wird, werden \emph{alle} Ereignis-Aktions-Paare ausgeführt.
	\item Man kann mehrere Ereignis-Aktions-Paare mit dem gleichen Ereignis
	konstruieren, solange sie verschiedene Parameter haben. Man kann mehrere Ereignisse mit Knöpfen haben, wenn dabei jeweils unterschiedliche Knöpfe betätigt werden müssen. 
	\item Werden zwei oder mehrere Ereignis-Aktions-Paare eingegeben mit demselben Ereignis (und denselben Parametern), zeigt VPL eine Fehlermeldung an. (Feld 3 in \cref{fig.vplgui}).
	Es ist nicht möglich ein Programm laufenzulassen solange Fehlermeldungen angezeigt werden.
\end{itemize}
}

\sect{Feedback zur Laufzeit}\label{p.feedback}

Jedes Mal, wenn eine Taste berührt wird, wird ein Ereignis erzeugt, und das Ereignis-Aktions-Paar, dem dieses Ereignis zugeordnet ist, wird ausgeführt. VPL bietet ein dynamisches Feedback zur Laufzeit, so dass Sie genau sehen können, welches Paar ausgeführt wird, nämlich dasjenige mit einem gelben Rahmen und einem gelben Pfeil auf der linken Seite:

\blkc{feedback}

Das Feedback wird nur kurz angezeigt, sobald das Ereignis eintritt. Wenn Sie zum Beispiel eine Taste berühren, wird ein Ereignis generiert, sobald Sie die Taste berühren; falls Sie die Taste weiterhin berühren, werden keine zusätzlichen Ereignisse generiert --- das Feedback wird entfernt. Es erscheint nur dann erneut, wenn Sie die Taste loslassen und dann wieder berühren.
