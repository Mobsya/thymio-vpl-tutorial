\part{Anhänge}

\chap{Die VPL Benutzer-Schnittstelle}\label{a.toolbar}

Zuoberst im VPL-Fenster hat es eine Werkzeugleiste (Toolbar):

\begin{center}
\gr{toolbar}{1}
\end{center}

\bigskip

\textbf{Neu} \blksm{new}: Löscht den bisher programmierten Code und stellt eine leere Programmierumgebung dar.

\bigskip

\textbf{Öffnen} \blksm{open}: Klicken Sie, um ein bestehendes VPL-Programm zu öffnen. Ein Fenster wird erscheinen und Sie können durch die Ordner navigieren zum Verzeichnis, wo Sie Ihre Programmdateien gespeichert haben (Datei-Erweiterung \p{aesl}).

\bigskip

\textbf{Speichern} \blksm{save}: Speichert das aktuelle Programm. Es ist eine gute Idee, diesen Knopf regelmässig zu betätigen, damit Sie Ihre Arbeit nicht verlieren, wenn ein Fehler auftritt. 

\bigskip

\textbf{Speichern als} \blksm{saveas}: Speicher das aktuelle Programm unter einem \emph{anderen Namen}. Benutzen Sie ''Speichern als'' wenn Sie in einem Programm etwas ausprobieren wollen und das bisherige Programm nicht verlieren wollen. 

\bigskip

\textbf{Zurücksetzen} \blksm{undo} (undo): Mit ''Zurücksetzen'' oder auch ''rückgängig machen'' werden getätigte Eingaben rückgängig gemacht (wie z.B. das Löschen eines Ereignis-Aktions-Paares)\label{p.undo}

\bigskip

\textbf{Vorwärts machen} \blksm{redo} (redo): Mit ''Vorwärts machen'' oder ''Wiederholen'' wird eine Eingabe, die mit ''Zurücksetzen'' rückgängig gemacht wurde, erneut ausgeführt. 

\bigskip

\textbf{Laden und ausführen} \blksm{run} (run): Kompiliert das Programm, lädt es auf den Roboter und führt es aus. Die Übertragung geschieht nur, das Programm korrekt kompiliert werden konnte. Wenn Sie das Programm geändert haben, blinkt der Knopf grün um Sie daran zu erinnern, dass Sie das geänderte Programm erneut kompilieren und übertragen müssen. 

\bigskip

\textbf{Halten} \blksm{stop}(stopp): Stoppt das laufende Programm und lässt den Roboter anhalten. Verwenden Sie diesem Knopf um den Roboter anzuhalten, wenn es kein Ereignis-Aktions-Paar gibt, das den Roboter anhalten lässt. 

\bigskip

\textbf{Fortgeschrittener Modus} \blksm{advanced}: Der fortgeschrittene Modus aktiviert zusätzliche Elemente und Optionen wie Zustände, Timer, Beschleunigungssensoren und Sensor-Grenzwerte. Der Knopf ändert die Farbe in orange.

\textbf{Standard-Modus}: Der selbe Knopf \blksm{basic} wird betätigt, um wieder zurück in den Standard-Modus zu wechseln (die Farbe wechselt wieder zurück in blau). 

\textbf{Hilfe} \blksm{info1}: Stellt in einem Internet-Browser-Fenster die VPL-Dokumentation unter folgendem Link dar: \href{https://www.thymio.org/en:thymiovpl}{https://www.thymio.org/en:thymiovpl}. Eine Internetverbindung ist nötig!

\bigskip

\textbf{Bildschirmfoto} \blksm{export}: \label{p.export} Exportiert ein Bild vom aktuellen VPL-Programm. Sie können dieses Bild dann in ein beliebiges Dokument integrieren. Verschiedene Grafikformate stehen zur Verfügung; \textsc{svg} ergibt die höchste Qualität, aber \textsc{png} ist wohl am weitesten verbreitet.

%\section*{Feedback}

%(To be written)

%\label{p.feedback}
