\chap{Dein erstes Robotikprojekt}

\sect{Den Roboter verbinden und VPL starten}

Verbinde deinen Thymio-II Roboter mit einem USB Kabel mit dem Computer. Der
Roboter spielt eine kleine Tonabfolge ab und  ein grünen Licht leuchtet auf dem
Roboter. Falls der Roboter ausgeschaltet ist, schalte ihn an, indem du fünf
Sekunden den mittleren Knopf berührst. Starte nun VPL; mache dazu einen
Doppelklick auf das VPL Symbol \blksm{thymiovpl}. VPL wird gestartet und das
Fenster in Bild~\ref{fig.connect} wird nun angezeigt. Wähle die Box
\bu{Serial}, klicke auf \bu{Thymio-II Robot \ldots} darunter, wähle eine
Sprache und klicke dann auf \bu{Connect}.\footnote{Abhängig von der
Konfiguration deines Computers und des Betriebssystems, das du verwendest, kann es
sein, dass das Fenster etwas anders ausschaut.}


\begin{figure}
\begin{center}
\gr{connect}{.3}
\caption{Den Roboter und die VPL Umgebung verbinden}\label{fig.connect}
\end{center}
\end{figure} 

\sect{Thymio-II kennenlernen}

Das Bild~\ref{fig.front} zeigt den Roboter von vorne. Oben sieht man einen Knopf in
der Mitte und die vier Richtungsknöpfe. Hinter den Knöpfen wird der
Batteriestand in grün angezeigt. Dahinter sieht man die zwei oberen Lichter, die auf
rot eingestellt sind. Der Roboter hat unten weitere solche Lichter, die auf
grün eingestellt sind (Bild~\ref{fig.bottom}). Die schwarzen Rechtecke vorne
sind Sensoren, die du im Kapitel~\ref{ch.pet} kennenlernen wirst.

\begin{figure}
\begin{center}
\gr{front}{.5}
\caption{Der Thymio-II Roboter von vorne}\label{fig.front}
\end{center}
\end{figure} 

\clearpage

\sect{Die VPL Bedienungsoberfläche}

Die Benutzeroberfläche von VPL (Bild~\ref{fig.gui}) hat vier Bereiche:

\begin{itemize}
\item Oben hat sie eine Symbolleiste mit Symbolen, um Programme zu öffnen, zu
    speichern und auszuführen.
\item Unterhalb der Symbolleiste befindet sich er Bereich, in dem man Programme 
    für den Roboter konstruieren kann.
\item Links und rechts des Programmbereichs sind Bereiche mit Aktions- und
    Ereignisblöcken.
\item Ganz rechts wird das konstruierte Programm in ein Textprogramm für den
    Roboter übersetzt.
\end{itemize}

\begin{figure}[hbt]
\gr{gui}{1.}
\caption{VPL Benutzeroberfläche}\label{fig.gui}
\end{figure}

%\clearpage

\sect{Konstruiere ein neues Programm}

Wenn du VPL öffnest, ist der Programmierungsbereich leer. Um einen neues leeres
Programm zu starten klicke auf \blksm{new}.

Ein VPL Programm besteht aus mehreren (oder auch nur einem) Paar von Ereignis
und Aktionsblöcken. Ein Beispiel: Das Paar \blkc{e-a-pair} verursacht, dass das
obere Licht auf dem Roboter rot leuchtet, sobald der Vorwärts-Knopf des
Roboters berührt wird.

\importantbox{
Die Bedeutung eines Ereignis-Aktions-Paars ist:\\
\textbf{\textit{Wenn das Ereignis eintritt, wird die Aktion durchgeführt.}}
}

Lass uns nun ein Ereignis-Aktions-Paar erstellen. Im Programmierbereich siehst
du eine Vorlage, um ein Paar zu erstellen: \blkc{event-action-pair-empty} Das
linke, hellblaue Quadrat ist für das Ereignis und das rechte, hell-violette
Quadrat für die Aktion. Du kannst nun das Knopf-Ereignis \blksm{event-buttons}
von der linken Ereignis-Auswahl in das linke Quadrat ziehen. Wähle nun den
Aktionsblock für die oberen Farben \blksm{action-colors-up} aus der rechten
Auswahl und ziehe ihn in das rechte Quadrat. Du hast nun dein erstes
Ereignis-Aktions-Paar erstellt.

Jetzt können wir das Ereignis und die Aktion so verändern, dass sie machen was
wir wollen. Beim Ereignis kannst du zum Beispiel auf den Vorwärts-Knopf
klicken; er wird dann rot.

Das bedeutet, dass \textbf{ein Ereignis stattfindet, wenn der Vorwärts-Knopf
berührt wird}.

Der Farb-Aktions-Block hat drei Balken mit den Grundfarben Rot, Grün und Blau.
Jeder dieser Balken hat ganz links ein weisses Quadrat, mit dem du diese Farbe
einstellen kannst. Wenn du die weissen Quadrate verstellst, verändert sich die
Hintergrundfarbe des Blocks und zeigt so die Farbe an, die der Roboter anzeigt.
Schiebe das Quadrat im roten Balken ganz nach rechts \blkc{red} der Roboter
leuchtet nun rot, wenn du den Vorwärts-Knopf berührst. 

\sect{Speichere das Programm}

Bevor du dein erstes Programm ausführen kannst, musst du es speichern. Klicke
auf das Symbol \blksm{save} in der Symbolleiste. Du musst dem Programm nun
einen Namen geben; wähle einen Namen, der dir später hilft, dich zu erinnern
was das Programm macht (zum Beispiel: \bu{rot-leuchten}).

\sect{Führe das Programm aus}

Um das Programm auszuführen, musst du auf das Symbol \blksm{run} klicken. Jetzt
kannst du den Vorwärts-Knopf auf den Roboter berühren und dann sollte der
Roboter rot leuchten.

\sect{Schalte den Roboter aus}

Wenn du nicht mehr mit dem Thymio-II Roboter spielen möchtest, kannst du ihn
ausschalten. Drücke dazu den mittleren Knopf für fünf Sekunden bis du einige
Töne hörst. Die Batterie des Roboters wird weiter aufgeladen, solange das Kabel
an den Computer angeschlossen ist (und er Computer läuft). Die Batterie lädt,
wenn das kleine Licht neben dem USB Stecker rot ist. Wenn es blau ist, ist die
Batterie vollständig aufgeladen (Bild~\ref{fig.back}).

\begin{figure}
\begin{center}
\gr{back}{.6}
\caption{Die Rückseite des Thymio-II Roboters mit dem USB Kabel und der
Ladeanzeige (blaues Licht)}\label{fig.back}
\end{center}
\end{figure}

\sect{Verändere ein Programm}

\begin{itemize}
\item Um ein Ereignis-Aktions-Paar zu löschen, klicke auf \blksm{x}, das neben
    jedem Paar angezeigt wird.
\item Um ein weiteres Ereignis-Aktions-Paar hinzuzufügen, klicke auf
    \blksm{plus}, das zwischen den Paaren und ganz unten angezeigt wird.
\item Um ein Ereignis-Aktions-Paar zu verschieben, kannst du es einfach an die
    neue Position ziehen.
\end{itemize}

\sect{Öffne ein Programm}

Angenommen du hast dein Programm gespeichert und den Roboter ausgeschaltet,
möchtest aber später wieder an deinem Programm weiter arbeiten. Verbinde den
Roboter wie beschrieben und klicke dann auf öffnen \blksm{open} und wähle das
Programm das du öffnen möchtest (zum Beispiel \bu{rot-leuchten}). Das Programm
wird jetzt im Programmierbereich angezeigt und du kannst es verändern.

\sect{Weitere Funktionen der VPL Benutzeroberfläche}

\begin{itemize}
\item \textbf{Speichern unter} \blksm{saveas}: Klicke auf dieses Symbol, um einem
    Programm einen neuen Namen zu geben. Das ist nützlich, wenn du ein Programm
    kopieren möchtest, um etwas auszuprobieren.
\item \textbf{Stopp} \blksm{stop}: Dieses Symbol stoppt das Programm das gerade
    auf dem Roboter ausgeführt wird. Das kannst du zum Beispiel brauchen, wenn
    der Roboter nicht mehr selbst stoppt.
\item \textbf{Farbpalette ändern} \blksm{scheme}: Hier kannst du auswählen,
    welche Farbe dir besser gefällt.
\item \textbf{Fortgeschrittener Modus} \blksm{advanced}: Im fortgeschrittene
    Modus kannst du Zustandvariablen brauchen. Diese lernst du im
    Kapitel~\ref{ch.states} kennen.
\item \textbf{Hilfe} \blksm{info1}: Zeigt die VPL Dokumentation in deinem Browser
    an. Die Dokumentation findest du hier: 
    \url{https://aseba.wikidot.com/de:thymiovpl}.
\end{itemize}
