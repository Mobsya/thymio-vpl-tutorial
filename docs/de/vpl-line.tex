\chap{Der Roboter findet seinen Weg selbst}
\label{ch.line}

In diesem Kapitel programmieren wir den Roboter so,
wie Roboter auch zur Lösung echter Probleme benutzt werden:
\emph{Der Roboter folgt der Linie}.
Stell Dir ein Lagerhaus mit Roboterwagen vor,
welche Gegenstände von einer zentralen Verteilstelle herbringen.
Dazu werden Linien auf den Boden des Lagerhauses gezeichnet
und die Roboter erhalten die Anweisung bestimmten Linien zu folgen,
bis diese den Lagerplatz für den gewünschten Gegenstand erreichen.

Schreibe ein Programm, welches den Roboter einer Linie auf dem Boden folgen lässt.

{\raggedleft \hfill Program file \bu{follow-line.aesl}}

Die Aufgabe "Der Roboter folgt der Linie" bringt
alle Unsicherheiten der Konstruktion von Robotern in der echten Welt zutage.
Die Linie mag nicht perfekt gerade sein,
Staub kann auf der Linie liegen und diese abdecken,
Dreck führt eventuell dazu, dass ein Rad weniger schnell ist wie das Andere.
Damit der Roboter der Linie folgt, benötigt der Roboter eine
\emph{Steuereinheit} die entscheidet wie viel Leistung jeder Motor benötigt,
abhängig von den Daten, die von den Sensoren geliefert werden.

\sect{Vorbereitung}

Falls der Boden von heller Farbe ist,
wird der Sensor viel reflektiertes Licht wahrnehmen und das Ereignis  \blk{lots-of-light} wird ausgelöst.
Somit brauchen wir eine Linie, die eine Ereignis auslöst, 
falls wenig Licht reflektiert wird \blk{little-light}.
Das lässt sich einfach mit dem Anbringen eines schwarzen
Isolierklebebandes auf den Boden bewerkstelligen.
Das Klebeband muss breit genug sein,
dass beide Bodensensoren schwarz erkennen,
solange der Roboter erfolgreich dem Klebeband folgt.
Ein Klebeband mit der Breite von 5 Zentimetern genügt dem Roboter,
um dem Klebeband folgen zu können sogar mit geringen Abweichungen.

\begin{figure}
\begin{center}
\gr{blacktape}{.6}
\caption{Der Thymio-II Roboter folgt der Klebebandlinie}
\label{fig.tape}
\end{center}
\end{figure}

Versichere dich,
dass das USB Kabel lange genug ist (ca. die 2m),
damit der Thymio-II auch in Bewegung mit dem Computer verbunden bleibt.

\sect{Starte und Stoppe den Roboter}

Der Roboter muss sich vorwärts bewegen,
falls \emph{beide} Sensoren das schwarze Klebeband erkennen
und stoppen, falls \emph{beide} Sensoren den Boden erkennen.
Die Ereignis-Aktions Paare sind in der Folgenden Figure~\ref{fig.start-stop} gezeigt.
Das erste Paar stoppt die Motoren, falls beide Sensoren den Boden erkennen,
während das zweite Paar den Roboter veranlasst sich vorwärts zu bewegen,
falls beide Sensoren das Klebeband erkennen.

\begin{figure}
\begin{center}
\gr{line-forward}{.4}
\caption{Starte und stoppe den Roboter}
\label{fig.start-stop}
\end{center}
\end{figure}

Der nächste Schritt besteht im Programmieren der Steuerungseinheit,
die den Roboter der Linie folgen lässt.

\begin{itemize}

\item Falls der Roboter nach der \emph{linken} Seite vom Klebeband abkommt,
wird der \emph{linke} Sensor den Boden erkennen,
während der \emph{rechte} Sensor immer noch das Klebeband erkennt.
In diesem Fall soll der Roboter ein wenig nach \emph{rechts} abbiegen

\item Falls der Roboter nach der \emph{rechten} Seite vom Klebeband abkommt,
wird der \emph{rechte} Sensor den Boden erkennen,
während der \emph{linke} Sensor immer noch das Klebeband erkennt.
In diesem Fall soll der Roboter ein wenig nach \emph{links} abbiegen (Figure~\ref{fig.one-off}).
\end{itemize}


\begin{figure}
\begin{center}
\begin{picture}(200,100)
%\put(0,0){\framebox(200,100)}
\thicklines
\put(0,20){\line(1,0){200}}
\put(0,70){\line(1,0){200}}
\thinlines
\put(80,0){
\put(0,0){\line(1,1){50}}
\put(0,0){\line(-1,1){50}}
\put(50,50){\line(-1,1){50}}
\put(-50,50){\line(1,1){50}}
\put(20,55){\framebox(10,10){}}
\put(5,72){\framebox(10,10){}}
\put(-5,95){\line(1,0){10}}
\put(45,45){\line(0,1){10}}
\put(80, 0){\makebox(40,10)[l]{\bu{Floor}}}
\put(80, 40){\makebox(40,10)[l]{\bu{Tape}}}
\put(80, 80){\makebox(40,10)[l]{\bu{Floor}}}
\put(25,75){\vector(1,1){20}}
}
\end{picture}
\caption{Der linke Sensor ist nicht auf dem Klebeband, der rechte Sensor ist auf dem Klebeband}
\label{fig.one-off}
\end{center}
\end{figure}

Zwei Ereignis-Aktions Paare werden gebraucht  (Figure~\ref{fig.follow-line}).

\sect{Setzen der Parameter}

Es ist einfach zu erkennen,
dass der Roboter, falls er von der linken Seite des Klebebands abkommt,
wie in Figur (Figure~\ref{fig.follow-line}) nach rechts drehen muss,
aber wie weit nach rechts? Falls die Drehung zu gering ausfällt,
wird der rechte Sensor eventuell \emph{auch} vom Klebeband abkommen,
bevor der Roboter auf die Spur zurückgelangt.
Falls der Roboter jedoch zu stark nach rechts dreht,
riskiert der Roboter die Klebeband Spur auf der anderen Seite zu verlieren.
In jedem fall sind starke Drehungen gefährlich für den Roboter und jegliches Gut,
welches er transportiert.

\centeredbox{\begin{center} \textbf{Parameters} \end{center}
\emph{Parameter} sind Werte,
welche geändert werden können
ohne die Gesamtstruktur der Ereignis-Aktions Paare zu verändern,
die das Programm bilden.}

\begin{figure}
\begin{center}
\gr{line-controller}{.4}
\caption{Folge der Linie}
\label{fig.follow-line}
\end{center}
\end{figure}

Die Parameter, welche in diesem Programm verändert werden können,
sind die Geschwindigkeiten des linken und des rechten Motors auf jedem Aktionsblock der Motoren.
Experimentiere solange mit diesen Werten,
bis der Roboter \emph{zuverlässig} läuft.
Zuverlässig heisst hier, dass der Roboter mit den selben Programmeinstellungen mehrmals erfolgreich der Linie folgt.
Platziere bei jedem Versuch den Roboter ein wenig anders in Bezug auf die Richtung und Position zum Klebeband.
Das Programm muss mehrmals getestet werden, um die Zuverlässigkeit zu überprüfen.

Überraschenderweise ist die vorwärtsgerichtete Geschwindigkeit entlang des Klebebands ebenfalls ein wichtiger Parameter. 
Ist der Roboter zu schnell,
ist dieser schon vom Klebeband weggefahren,
bevor die Drehbewegung die Richtung des Roboters verändern konnte.
Ist der Roboter jedoch zu langsam, wird Dir natürlich  niemand deinen Roboter abkaufen
und in einem Lagerhaus einsetzen.

Es gibt verschiedene Arten,
wie die Parameter der Aktionsblöcke der Motoren gesetzt werden können,
um die  Drehung zu kontrollieren.
Du kannst einen Motor etwas schneller als den Anderen laufen lassen
oder Du kannst den einen Motor vorwärts und den Anderen rückwärts laufen lassen.

\sect{Übung \thechapter.1}

Der Roboter stoppt,
falls beide Sensoren erkennen,
dass sie vom Klebeband abgekommen sind.
Verändere das Programm so,
dass der Roboter eine geringe Drehung nach Links vollführt mit der Absicht das Klebeband wiederzufinden.
Versuche es auf einem Klebeband mit einer Linkskurve,
wie in Figure~\ref{fig.tape}.  angezeigt.
Versuche die Vorwärtsgeschwindigkeit des Roboters zu erhöhen.
Was passiert, wann der Roboter das Ende des Klebebandes erreicht hat?

\sect{Übung \thechapter.2}

Verändere das Programm aus der vorangehenden Übung so,
dass der Roboter nach rechts dreht,
falls er vom Klebeband abkommt.
Was passiert?

Es wäre wünschenswert,
wenn der Roboter  \emph{speichern} könnte,
welchen Sensor als letzter den Kontakt zum Klebeband verloren hat,
um den Roboter in die korrekte Richtung zu führen,
um das Klebeband wieder zubinden.
In Kapitel ~\ref{ch.states} werden wir lernen,
wie Informationen gespeichert werden können.

\sect{Übung \thechapter.3}

Experimentiere mit unterschiedlichen Anordnungen des Klebebandes:

\begin{itemize}
\item weiten Kurven
\item engen Kurven
\item zickzack Linien
\item breiteren Linien (benutze dazu die doppelte Klebebandbreite)
\item schmaleren Linien ( schneide dazu das Klebeband in zwei Hälften)
\end{itemize}

Veranstalte Roboterrennen mit deinen Freunden:
Welcher Roboter folgt erfolgreich den meisten Linien?
Welcher Roboter fährt am schnellsten eine Linie ab?

\sect{Übung \thechapter.4}

Bespreche den Effekt der folgenden Veränderungen  auf den Thymio-II Roboter und dessen Fähigkeit der Linie zu folgen.

\begin{itemize}
\item Die Messereignisse der Bodensensoren erfolgen öfters bzw. weniger oft als 10 mal pro Sekunde.
\item Die Sensoren sind weiter auseinander bzw. näher beisammen.
\item Es gibt mehr als zwei Bodensensoren am Boden des Roboters.
\end{itemize}
