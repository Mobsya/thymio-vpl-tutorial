\chap{Los, Thymio-II bewege dich}\label{c.moving}

\sect{Vorwärts und Rückwärts fahren}

Der Thymio-II Roboter hat zwei Motoren mit denen er seine zwei Räder antreiben
kann. Beide Motoren können vorwärts und rückwärts drehen. Dadurch kann der
Roboter vorwärts und rückwärts fahren. Lass uns ein kleines Projekt machen, mit
dem du lernen kannst, wie du die Motoren steuern kannst.

Der Aktions-Block für die Motoren \blk{action-motors} zeigt ein kleines Bild
des Roboters in der Mitte und zwei Balken. Mit den beiden Balken kannst du die
Geschwindigkeit der beiden Motoren einstellen. Mit dem linken Balken die des
linken Motors und mit dem rechten Balken die des rechten Motors. Wenn das
weisse Quadrat in der Mitte ist, dreht der Motor nicht. Wenn du das Quadrat
nach oben schiebst, dreht der Motor immer schneller vorwärts. Schiebst du es
nach unten, dreht der Motor rückwärts.

Konstruiere ein Programm, um den Roboter vorwärts fahren zu lassen, wenn der
Vorwärts-Kopf berührt wird und rückwärts, wenn der Rückwärts-Knopf berührt
wird.

{\raggedleft \hfill Beispielprogramm \bu{moving.aesl}}

Wir brauchen zwei Ereignis-Aktions-Paare (Bild~\ref{fig.nostop}). Erstelle die
beiden Paare und stelle bei den Motoren die gleiche Geschwindigkeit ein.

Führe das Programm aus, indem du auf \blksm{run} klickst. Jetzt kannst du den
Vorwärts- und Rückwärts-Knopf berühren. Was macht der Roboter?

\begin{figure}
\begin{center}
\gr{no-stop-motors}{.4}
\caption{Vorwärts und rückwärts fahren}\label{fig.nostop}
\end{center}
\end{figure}

\sect{Stoppe den Roboter}

\textbf{Hilfe!} Ich kann den Roboter nicht mehr stoppen!

Klicke auf das Symbol \blksm{stop}, um den Roboter zu stoppen.

Wie können wir dieses Problem beheben? Lass uns ein neues Ereignis-Aktions-Paar
hinzufügen, das beide Motoren stoppt, wenn du den mittleren Knopf berührst.
Hier musst du beim Aktions-Block für die Motoren die Balken in der Mitte
lassen.

\sect{Falle nicht vom Tisch}

Wenn der Roboter auf dem Boden fährt, kann er im schlimmsten Fall in eine Wand
fahren, oder sein USB Kabel ausziehen. Aber wenn der Roboter auf einem Tisch
fährt, kann er auf den Boden fallen und kaputt gehen! Lass uns den Roboter so
programmieren, dass das nicht passieren kann.

\centeredbox{\textbf{Achtung! Wenn der Roboter auf einem Tisch fährt, musst du
bereit sein, um ihn aufzufangen falls er runter fällt.}}

Drehe deinen Thymio-II auf den Rücken. Nun siehst du, dass er unten zwei
kleine, schwarze Rechtecke mit optischen Elementen hat (Bild~\ref{fig.bottom}).
Das sind \emph{Bodensensoren}, die Licht aussenden, und erkennen können, ob
viel oder wenig Licht zurückkommt. Wenn etwas unter dem Roboter ist (zum
Beispiel der Tisch) wird viel Licht reflektiert. Wenn aber die Spitze des
Roboters über den Tisch hinausragt, geht das Licht weit weg und nur sehr wenig
kommt zurück. Wenn das passiert, möchten wir den Roboter stoppen.

\centeredbox{\textbf{Hier nehmen wir an, das dein Tisch eine helle Farbe hat.
Wenn du einen sehr dunklen Tisch hast, musst du das Programm ändern.}}

Ziehe das Bodensensor-Ereignis \blk{event-ground} in dein Programm. Oben hat
dieser Block zwei kleine Quadrate. Wenn du sie anklickst, werden sie weiss, rot
und dann wieder grau. Die Farben haben verschiedene Bedeutungen:

\begin{itemize}
\item \textbf{Grau}: Der Sensor wird nicht gebraucht.
\item \textbf{Rot}: Ein Ereignis passiert, wenn viel Licht reflektiert wird.
\item \textbf{Weiss}: Ein Ereignis passiert, wenn wenig Licht reflektiert wird. 
\end{itemize}

Klicke auf die Quadrate bis beide weiss sind, um den Roboter zu stoppen, wenn
wenig Licht reflektiert wird. Erstelle so das Ereignis-Aktions-Paar
\blk{dont-fall}.

\begin{figure}
\begin{center}
\gr{bottom}{0.6}
\caption{Unten hat der Thymio-II Roboter zwei Bodensensoren.}\label{fig.bottom}
\end{center}
\end{figure}

Stelle den Roboter nahe an die Tischkante und berühre den Vorwärts-Knopf. Der
Roboter sollte bis zur Tischkante fahren und dann stoppen.

Als ich das Programm ausprobiert habe, ist der Roboter heruntergefallen. Das ist
passiert, weil mein Tisch eine runde Kante hat. Als der Roboter die Kante
richtig bemerkt hat, war es schon zu spät und er ist runter gefallen. Ich habe
ein Stück schwarzes Klebeband an die Tischkante geklebt, damit hat es dann
funktioniert.

\sect{Aufgabe \thechapter.1}

Probiere verschiedene Geschwindigkeiten mit dem Roboter aus. Kann der Roboter
auch bei der schnellsten Geschwindigkeit noch rechtzeitig stoppen, um nicht vom
Tisch zu fallen? Falls nicht, wie schnell kannst du ihn einstellen, damit er
noch rechtzeitig stoppen kann?

