\chap{Los, beweg dich}\label{c.moving}

\sect{Vorwärts- und rückwärtsfahren}

Der Thymio Roboter hat zwei Motoren mit denen er seine zwei Räder unabhängig antreiben
kann. Beide Motoren können vorwärts und rückwärts drehen. Dadurch kann der
Roboter vorwärts und rückwärts fahren. Lass uns mit einem kleineen Projekt starten bei dem du mehr über die Motoren lernst.

Der Aktions-Block für die Motoren \blksm{action-motors} zeigt ein kleines Bild
des Roboters in der Mitte mit zwei Schieberegler links und rechts. Mit den beiden Balken kannst du die
Geschwindigkeit der beiden Motoren einstellen, mit dem linken Balken die des
linken Motors und mit dem rechten Balken die des rechten Motors. Wenn das
weisse Quadrat in der Mitte ist, dreht der Motor nicht. 
Du kannst die Geschwindigkeit ändern indem du das weisse Quadrat verschiebst.
Wenn du das Quadrat
nach oben schiebst, dreht der Motor immer schneller vorwärts. Schiebst du es
nach unten, dreht der Motor rückwärts.

Erstelle ein Programm, um den Roboter vorwärts fahren zu lassen, wenn der
Vorwärts-Kopf gedrückt und rückwärts, wenn der Rückwärts-Knopf gedrückt
wird.

{\raggedleft \hfill Beispielprogramm \bu{moving.aesl}}

Wir brauchen zwei Ereignis-Aktions-Paare (Bild~\ref{fig.nostop}). 
Ziehe die Ereignis- und Aktionsblöcke in die Programmierumgebung und stelle für beiden Motoren die Schieberegler auf halbe Geschwindigkeit ein. Dies indem du die Quadrate für Vorwärtsfahren halb nach oben und für Rückwärtsfahren halb nach unten verschiebst.

\begin{figure}
\begin{center}
\gr{no-stop-motors}{.4}
\caption{Vorwärts- und rückwärtsfahren}\label{fig.nostop}
\end{center}
\end{figure}

Führe das Programm aus und drücke die Knöpfe, um den Roboter vorwärts- und rückwärts fahren zu lassen. 

\newpage

\sect{Stoppe den Roboter}

\textbf{Hilfe!} Ich kann die Motoren des Roboters nicht mehr stoppen!

Klicke auf das Symbol \blksm{stop}, um den Roboter zu stoppen.

Lass uns diese Problem beheben, indem wir ein neues Ereignis-Aktions-Paar
hinzufügen: \blkc{stop-motors}Dieses soll die Motoren stoppen, wenn der mittlere Knopf gedrückt wird. Wenn du den Motoraktionsblock in die Programmierumgebung ziehst, sind die Schieberegler in der Mitte, was die Motoren abstellen lässt.

\sect{Falle nicht vom Tisch}

Wenn der Roboter auf dem Boden fährt, kann er im schlimmsten Fall in eine Wand
fahren oder sein USB Kabel herausziehen. Aber wenn der Roboter auf einem Tisch
fährt, kann er auf den Boden fallen und kaputt gehen! Lass uns den Roboter so
programmieren, dass der Roboter stoppt sobald er an die Tischkante gelangt.

\warningbox{Wenn der Roboter auf einem Tisch fährt, musst du
bereit sein, um ihn aufzufangen, falls er herunterfällt.}

Drehe deinen Thymio auf den Rücken. Nun siehst du, dass er unten zwei
kleine, schwarze Rechtecke mit optischen Elementen hat (Bild~\ref{fig.bottom}).
Das sind \emph{Bodensensoren}. Diese senden Infrarotlichtimpulse aus und messen wieviel Licht relektiert wird. 
Auf einem hellen Tisch wird viel Licht reflektiert. Fährt der Roboter über die Tischkante wird wenig Licht reflektiert. Tritt dies ein, möchten wir, dass der Roboter stoppt.

\trickbox{Benutze einen hellen Tisch aber keinen aus Glas, da von diesem kein Licht reflektiert wird. Thymio kann dann nicht erkennen, ob er auf einem Tisch ist oder nicht.}

Ziehe das Bodensensor-Ereignis \blksm{event-ground} in dein Programm. Oben hat
dieser Block zwei kleine Quadrate. Wenn du sie anklickst, werden sie weiss, rot
und dann wieder grau. Die Farben haben verschiedene Bedeutungen:

\begin{itemize}
\item \textbf{Grau}: Der Sensor wird nicht gebraucht.
\item \textbf{Rot}: Ein Ereignis passiert, wenn viel Licht reflektiert wird.
\item \textbf{Weiss}: Ein Ereignis passiert, wenn wenig Licht reflektiert wird. 
\end{itemize}

\trickbox[Information]{Grau, rot und weiss, die für diesen Aktionsblock verwendet werden, wurden zufällig ausgewählt und könnten auch durch andere ersetzt werden.}

Klicke auf die Quadrate bis beide weiss sind, um den Roboter zu stoppen, wenn
wenig Licht reflektiert wird. Erstelle das Ereignis-Aktions-Paar
\blk{dont-fall}.

\begin{figure}
\begin{center}
\gr{bottom}{0.6}
\caption{Unterseite des Thymio Roboters mit zwei Bodensensoren in der vorderen Hälfte.}\label{fig.bottom}
\end{center}
\end{figure}

Stelle den Roboter mit der Vorderseite nahe an die Tischkante und drücke den Vorwärtsknopf. Der
Roboter sollte bis zur Tischkante fahren und dann stoppen.

\exercisebox{\thechapter.1}{
Probiere verschiedene Geschwindigkeiten mit dem Roboter aus. Kann der Roboter
auch bei der schnellsten Geschwindigkeit noch rechtzeitig stoppen, um nicht vom
Tisch zu fallen? Falls nicht, wie schnell kannst du ihn einstellen, damit er
noch rechtzeitig stoppen kann? Kannst du ein Herunterfallen verhindern, indem du die Räder rückwärtsdrehen lässt, anstatt die Motoren nur zu stoppen?
}

\trickbox{
Als ich das Programm laufen liess, \emph{fiel} der Roboter vom Tisch. Der Grund war, dass mein Tisch runde Kanten hatte; als der Roboter eine geringe Reflektion detektierte, war es schon zu spät und der Roboter rutschte vom Tisch. Meine Lösung war, dass ich ein schwarzes Klebeband an den Tischrand anbrachte.
}