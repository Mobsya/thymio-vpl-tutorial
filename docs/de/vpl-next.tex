\chap{Wie weiter?}


Dieses Tutorial hat den den Thymio Roboter und die Aseba/VPL Umgebung eingeführt. Die VPL Umgebung mit ihrer simplen visuellen Programmierung ist für Anfänger gedacht. Du wirst jedoch die Aseba Studio Umgebung kennenlernen wollen, um fortgeschrittene Programme für deinen Roboter entwickeln zu können. (Figure~\ref{fig.studio}).

\begin{figure}[hbt]
\begin{center}
\gr{studio}{.8}
\caption{Aseba Studio Umgebung}
\label{fig.studio}
\end{center}
\end{figure}

Programmieren im Aseba Studio ist auch auf dem Konzept der Ereignis-Aktions Paare aufgebaut, aber dir stehen viele weitere Möglichkeiten zur Verfügung:

\begin{itemize}
\item Du kannst genau bestimmen, wann ein Ereignis stattfindet, z.B. abhängig vom Licht, welches vom Boden zu einem Bodensensor reflektiert wird, oder die Distanz zu einem horizontalen Sensor.
\item  Du kannst einzelne Aktionen aus mehreren verschiedenen Aufgaben zusammensetzen, wie Kontrolle der Motoren, Änderung der Einstellungen, setzen von Grenzwerten, Ein- und Ausschalten der Lichter.
\item  Du hast die Flexibilität einer vollwertigen Programmiersprache mit Variablen, Befehlen und Kontrollfunktionen.

\end{itemize}

Aseba Studio gibt dir Zugang zu den folgenden Eigenschaften des Thymio, welche nicht in VPL zugänglich sind:

\begin{itemize}
\item Du kannst alle Lichter kontrollieren, wie den Lichtkreis, der die Schalter umgibt.
\item Du bist flexibler bei der Komposition von Melodien.
\item Es gibt einen Temperatursensor.
\item Anstelle des Erkennens eines Schlages, können die Beschleunigungsmesser die Gravitation und Geschwindigkeitsänderungen in allen drei Dimensionen erkennen. 
\item Eine Fernsteuerung kann mit dem Roboter benützt werden.
\end{itemize}

Falls Du mit Aseba Studio arbeitest, kannst Du VPL öffnen, indem Du das Feld \bu{Launch VPL} unten links drückst. Du kannst VPL ins Aseba Studio importieren, einfach durch öffnen des Files.

Um Aseba Studio zu benutzen, starte von der \emph{Thymio Programmier}seite auf :\\
\url{https://aseba.wikidot.com/en:thymioprogram}\\
und folge dem Link \emph{Text Programming Environment}.

Du findest viele interessante Projekte unter der folgenden Seite:\\\url{https://aseba.wikidot.com/en:thymioexamples}.

\vspace{4em}

\infobox{Viel Spass und lerne viel! }{Danke, dass du das Tutorial gelesen hast!}{yellow!10}{\bcetoile}
