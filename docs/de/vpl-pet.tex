\chap{Ein Roboterhaustier}
\label{ch.pet}

In diesem Kapitel machen wir den Thymio-II zu eimen \emph{autonomen Roboter}.
Wir programmieren ihn so, dass er unabhängige Verhalten hat, wie zum Beispiel
Katzen oder Hunde. Diese Verhalten können wir mittels \textit{Rückkkoppelung}
erreichen: Der Roboter nimmt ein Ereignis in der Umwelt mit seinen Sensoren
wahr und verändert seine Aktionen entsprechend.

\sect{Der Roboter gehorcht dir}

Zuerst lehren wir den Roboter, uns zu gehorchen. 
Und zwar so, dass er sich normalerweise nicht bewegt; 
wenn er aber deine Hand vor sich wahrnimmt, soll er
zu deiner Hand fahren.

Der Roboter vorne hat fünf horizontale Distanzsensoren\footnote{Der richtige
 (technische) Ausdruck ist \emph{Näherungssensor}, aber wir verwenden hier
 das einfachere Wort \emph{Distanzsensor} oder nur Sensor. Führe deine Hand langsam näher an
 den Roboter. Wenn deine Hand nahe genug ist, leuchtet ein kleines, rotes
Licht neben dem Sensor auf und zeigt somit an, dass die Hand erkannt wurde.}
und an seiner Hinterseite zwei. 

\begin{figure}
\begin{center}
\gr{detect}{.6}
\caption{Die Vorderseite des Thymio-II. Zwei Distanzsensoren haben die Finger
erkannt.}
\label{fig.detect}
\end{center}
\end{figure}

Der Block \blk{event-prox} wird gebraucht, 
um zu erfahren, ob etwas nahe am Sensor ist oder nicht. 
Beide Fälle werden als Ereignis gewertet. 
Die schmalen, grauen Quadrate (fünf vorne und zwei hinten) können wahrnehmen,
 wann ein Ereignis stattfindet. 
 Indem auf ein Quadrat gedrückt wird, 
 ändert sich dieses von Grau zu Weiss, 
 von Weiss zu Rot und zurück auf Grau.

\begin{itemize}
\item \textbf{Grau}: Der Sensor beeinflusst das Programm nicht.
\item \textbf{Rot}: Eine Aktion wird ausgelöst, falls ein Objekt im Bereich des Sensors angegeben wird.
\item \textbf{Weiss}: Eine Aktion wird gestartet, falls \emph{kein} Objekt im Bereich des Sensors angegeben wird.
\end{itemize}

Damit  ein bestimmtes Verhalten des Roboters ausgelöst wird, 
brauchen wir die beiden Ereignis-Aktions Paare, 
wie in der Figure~\ref{fig.follow-hand} gezeigt. 
Das erste Paar besteht aus dem zentralen vorderen Sensor mit der Einstellung weiss. 
Die dazugehörige Aktion ist, dass die Motoren ausgeschaltet werden. 
Somit wird der Roboter stehen bleiben, falls er schon still gestanden ist oder anhalten, 
falls er in Bewegung war. 
Das zweite Paar besteht aus dem zentralen vorderen Sensor mit der Einstellung rot. 
Die dazugehörige Aktion ist, dass beide Motoren schnell laufen. 
Falls Du nun deine Hand vor den Roboter hältst, wird eine Aktion ausgelöst, die bewirkt, 
dass sich der Roboter auf die Hand zubewegt.

\begin{figure}
\begin{center}
\gr{likes-forward}{.4}
\caption{Bewegung zu der Hand hin}
\label{fig.follow-hand}
\end{center}
\end{figure}

\sect{Steuere den Thymio-II Roboter}

Der Thymio-II Roboter hat kein Steuerrad wie ein Auto oder ein Lenker wie ein Fahrrad.
Wie kann der Roboter nun gelenkt werden? Der Roboter benutzt ein \emph{Differentialgetriebe},
welches ähnlich funktioniert wie bei Raupenfahrzeuge z.B. einem Bulldozer (Figure~\ref{fig.bull}).
Die gewünschte Richtung wird anstelle eines Steuerrads mit \emph{unterschiedlichen} Geschwindigkeiten
des linken und des rechten Rades bzw. der Raupe erreicht. Dreht das rechte Rad schneller als das linke,
biegt das Fahrzeug nach links ab und dreht das linke Rad schneller als das rechte, biegt das Fahrzeug rechts ab.

Je grösser der Unterschied in den Geschwindigkeiten,
desto enger  wird die Kurve. Indem die Räder in unterschiedliche Richtungen drehen,
wird ein möglichst grosser Unterschied der Geschwindigkeiten erreicht. 
Tatsächlich dreht sich das Fahrzeug auf der Stelle, 
falls sich die Räder mit der genau \emph{gleichen} 
Geschwindigkeit in entgegengesetzter Richtung bewegen.

\begin{figure}
\begin{center}
\gr{bulldozer}{0.35}
\caption{Ein Bulldozer mit Raupenantrieb} 
\label{fig.bull}
\end{center}
\end{figure}

Mit VPL kann das Differentialgetriebe des Thymio-II Roboters ausgeführt werden,
indem die Regler für den linken und den rechten Motor im Aktionsblock auf verschiedene Werte eingestellt werden.
Im Aktionsblock \blk{differential} wird der linke Regler auf schnelle Geschwindigkeit rückwärts 
und der rechte Regler auf schnelle Geschwindigkeit vorwärts eingestellt. 
Als Resultat wird der Roboter eine enge Linkskurve vollführen, 
wie dies auf dem kleinen Bild des Roboter bzw. des Motoraktionsblocks dargestellt ist.

Experimentiere mit dem Ereignis-Aktions Paar \blk{turning}.

Stelle den linken und den rechten Regler ein.
Lass das Programm laufen und berühre den zentralen Schalter.
Um den Roboter zu stoppen, klicke auf \blksm{stop}.
Nun kannst Du die Regler ändern und es erneut ausprobieren.

\sect{Der Roboter mag dich}

Ein echtes Haustier folgt dir.
Damit der Roboter deiner Hand folgt,
musst du zwei weitere Ereignis-Aktions Paare hinzufügen.
Falls der Roboter ein Objekt vor seinem
ganz links platzierten Distanzsensor wahrnimmt,
soll er nach links und falls er ein Objekt
vor seinem ganz rechts platzierten Distanzsensor wahrnimmt, soll er nach rechts abbiegen.

{\raggedleft \hfill Program file \bu{likes.aesl}}

Das Programm für "der Roboter mag dich" besteht aus zwei Ereignis-Aktions Paaren
(Figure~\ref{fig.likes}).
Probiere die Regler an jedem Motoraktionsblock aus!

\begin{figure}
\begin{center}
\gr{likes-turns}{0.4}
\caption{Der Roboter mag dich}
\label{fig.likes}
\end{center}
\end{figure}

\sect{Übung \thechapter.1}

Modifiziere den Haustierroboter so,
dass er vorwärts fährt, falls das Programm am laufen ist
und dass er stoppt, falls er das Ende des Tisches (oder ein Klebeband) wahrnimmt.

\sect{Übung \thechapter.2}

Was passiert, falls Du die Reihenfolge des Ereignis-Aktions Paares änderst, 
welches Du in der vorangegangenen Übung verwendet hast?

\centeredbox{
\begin{center}
\textbf{Horizontale Sensoren und Bodensensoren}

\end{center}

Pass auf, dass du das Verhalten
der horizontalen Sensoren und der Bodensensoren nicht verwechselst.

\begin{itemize}

\item Bei den horizontalen Sensoren bedeutet das weisse Quadrat, 
dass ein Ereignis stattfinden wird, falls \emph{Nichts} in der Nähe ist, 
während das rote Quadrat bedeutet, dass ein Ereignis stattfinden wird, 
falls \emph{Etwas} in der Nähe ist.

\item Für die Bodensensoren, bedeutet das weisse Quadrat, dass ein Ereignis stattfinden wird, 
falls \emph{nur ein wenig Licht, von der Oberfläche reflektiert wird}, 
während das rote Quadrat bedeutet, dass ein Ereignis stattfindet, 
falls \emph{viel Licht von der Oberfläche reflektiert wird}.

\end{itemize}
}

Wie im Kapitel~\ref{c.moving} erklärt,
wird viel Licht von einer weissen Oberfläche
und wenig Licht von einer schwarzen Oberfläche reflektiert.
Je nach Boden oder Tisch musst du entscheiden,
wann du auf das weisse oder das rote Quadrat drückst,
abhängig vom Boden oder Tisch auf den du den Roboter setztst.

\newpage

\sect{Der Roboter mag dich nicht}

Manchmal mag dein Roboterhaustier in schlechter Laune sein und von deiner Hand zurückweichen. Erstelle ein Programm, welches dieses Verhalten auslöst.
Sometimes your pet may be in a bad mood and turn away from your hand.
Write a program that causes this behavior in the robot.

{\raggedleft \hfill Program file \bu{does-not-like.aesl}}

Öffne das Programm für den Haustierroboter, 
der Dich mag und ändere die Zusammenhänge für die Ereignisse mit den Aktionen. 
Das Erkennen eines Hindernisses beim linken Sensor bewirkt, 
dass der Roboter nach rechts abbiegt, 
während das Erkennen eines Hindernisses beim rechten Sensor bewirkt, 
dass der Roboter nach links abbiegt (Figure~\ref{fig.hates}).

\begin{figure}[htb]
\begin{center}
\gr{hates}{0.4}
\caption{Der Roboter mag Dich nicht}
\label{fig.hates}
\end{center}
\end{figure}

\sect{Übung \thechapter.3}

Experimentiere mit der Auswahl der Sensoren.
Verwende andere Sensoren anstelle von den Sensoren 0 und 4.\footnote{Die horizontalen Sensoren werden vorne von links nach rechts wie folgt nummeriert 0,1,2,3,4.
Die hinteren Sensoren sind Sensor Nummer 5 auf der linken und Sensor Nummer 6 auf der rechten Seite.}

\begin{itemize}

\item Benutze Sensor 1 bzw. Sensor 3, um den Roboter nach links bzw. nach rechts abbiegen zu lassen.
\item Benutze Sensoren 0 und 1 bzw. 3 und 4, um den Roboter nach links bzw. nach rechts abbiegen zu lassen.
\item Füge Ereignis-Aktions Paare für die hinteren Sensoren 5 und 6 hinzu.
\end{itemize}

\sect{Stelle die Regler genau ein (fortgeschritten)}

Es ist schwierig die Regler so präzise zu setzten, dass die Motoren z.B. mit exakt derselben Geschwindigkeit laufen. Indem die Übersetzung der Ereignis-Aktions Paare im Texteditor betrachtet werden, kann die Genauigkeit erhöht werden. Der Text ist auf der Rechten Seite des VPL Fensters angezeigt. Hier die Übersetzung des Programms, welches bewirkt, dass der Haustierroboter dich mag und Dir folgt. 

\begin{small}
\begin{verbatim}
onevent prox
  if prox.horizontal[2] < 400 then
    motor.left.target = 0
    motor.right.target = 0
  end
  if prox.horizontal[2] > 500 then
    motor.left.target = 300
    motor.right.target = 300
  end
  if prox.horizontal[0] > 500 then
    motor.left.target = -300
    motor.right.target = 300
  end
  if prox.horizontal[4] > 500 then
    motor.left.target = 300
    motor.right.target = -300
  end
\end{verbatim}
\end{small}

Die Zeile \p{onevent prox} bedeutet:
Immer wenn das Ereignis "Messen der horizontalen Distanz" des Sensors
(des \emph{Näherungssensors} stattfindet (dies findet 10 mal pro Sekunde statt),
die folgenden Zeilen des Programms ausgeführt werden.
Die gemessenen Werte werden mit einem tiefen Wert von 400
und einem hohen Wert von 500 verglichen. Die Geschwindigkeit des Motors wird
nun anhand des Resultats dieses Vergleiches angepasst.
Falls der zentrale Sensor 2 z.B. einen Wert höher als 500 misst
(dies bedeutet, dass Etwas in der Nähe ist),
werden die Geschwindigkeiten für beide Motoren links und rechts auf 300 eingestellt.

Experimentiere mit den Einstellungen
der Regler des Aktionsblocks Motor. Du wirst sehen,
dass die Geschwindigkeit der Motoren in 50iger Schritten von $-$500
bis 500 verändert werden kann.
Indem die Regler vorsichtig eingestellt werden,
kannst Du sehen, dass Du die Geschwindigkeit
beliebig auf diese Werte einstellen kannst.
