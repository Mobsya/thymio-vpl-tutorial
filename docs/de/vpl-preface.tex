\chapter*{Vorwort}

\sect{Was ist ein Roboter?}

Du fährst dein Fahrrad und plötzlich siehst du, dass die Strasse vor dir
aufwärts geht.  Du trittst kräftiger in die Pedale, um den Rädern mehr Energie
zuzuführen. Dadurch wirst du auch aufwärts nicht langsamer. Nachdem du oben
angekommen bist, geht es wieder runter. Du betätigst die Bremsen deines
Fahrrads damit die Bremsblöcke gegen die Räder gedrückt werden. Dadurch wird
dein Fahrrad nicht zu schnell.  Wenn du Fahrrad fährst sind deine Augen
\emph{Sensoren}, die deine Umgebung \emph{wahrnehmen}. Wenn diese Sensoren ---
deine Augen --- ein \emph{Ereignis} erfassen (zum Beispiel eine Kurve der
Strasse), führst du eine \emph{Aktion} aus (zum Beispiel indem du den
Fahrradlenker nach links oder rechts steuerst).

In einem Auto sind Sensoren eingebaut, die \emph{messen} was in der Umgebung
passiert. Der Tachometer misst wie schnell das Auto fährt. Wenn du siehst, dass
das Auto schneller als die Tempolimite fährt, sagst du dem Fahrer, er fahre zu
schnell. Darauf kann der Fahrer eine Aktion ausführen, nämlich die Bremse
betätigen, damit das Auto langsamer wird. Die Tankanzeige misst wie viel Benzin
noch im Tank des Autos ist. Wenn du siehst, dass die Anzeige zu tief ist,
kannst du der Fahrerin sagen, dass sie eine Tankstelle suchen muss. Sie kann
dann eine Aktion ausführen: Sie kann den Blinker einschalten und das Steuerrad
nach rechts drehen, um zur Tankstelle zu fahren.

Jeder Fahrradfahrer und Autofahrer erhält Informationen von den Sensoren,
entscheidet dann welche Aktion nötig ist und verursacht diese Aktionen.
Ein \emph{Roboter} ist ein System, in welchem dieser Prozess --- Informationen
erhalten, Entscheidungen treffen, Aktionen ausführen --- von einem Computer
ausgeführt wird. Normalerweise macht der Roboter das ohne die Hilfe von
Menschen.

\sect{Der Thymio-II Roboter und die Aseba VPL Umgebung}

Der Thymio-II ist ein kleiner Roboter, der für Ausbildungszwecke vorgesehen ist
(Bild~\ref{fig.front}). Die Sensoren des Roboters messen Licht, Töne und
Distanzen und detektieren ob Knöpfe gedrückt werden oder ob an den Roboter
geklopft wird.

Aseba ist eine Programmierumgebung für den Thymio-II Roboter. VPL ist ein Teil
vom Aseba und dient der \emph{visuellen Programmierung} (visual programming).
Mit VPL ist es ganz einfach, Programme mit farbigen Blöcken zu konstruieren.

Mehr Informationen zum Thymio-II und Aseba gibt es hier: \url{https://aseba.wikidot.com/de:start}.

