\chap{Reading Barcodes}\label{ch.barcode}

Barcodes are universally used in supermarkets and elsewhere to identify
objects. The identification is a number or a sequence of symbols that is
different for each type of object. The identification is used to access
a database containing information about an object, such as its price.
Let us build a barcode reader from the Thymio robot.

\textbf{Specification}

\begin{enumerate}
\item Carefully measure the distance between two front horizontal
sensors and the width of a single sensor.
Using a piece of flexible cardboard, black tape and strips of aluminum
foil, construct an object with various arrangements of the strips of foil:

\begin{center}
\gr{barcode}{.6}
\end{center}

\item Each configuration of the three center horizontal sensors will
represent a different code. (How many codes can there be?) For some or
all of these codes, implement event-actions pairs that display a top
color depending on the code identified.

\end{enumerate}

\textbf{Guidance}:

We will only use the three center sensors, so the squares for the
outer sensors will be gray. For the center sensors, squares where the
reflecting foil must appear in the code are white, while squares where
the foil must not appear are black. For example, the following
event-actions pair displays yellow for the code \texttt{on-off-on}:

\blkc{barcode1-3}

The program in the archive identifies barcodes with foil opposite any
two of the three center sensors, as well as the code with no foil.

{\raggedleft \hfill \textbf{Program file}: \bu{barcode.aesl}}
