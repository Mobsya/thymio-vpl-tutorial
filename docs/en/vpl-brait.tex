% !TeX root = vpl.tex

\part{Projects}

\chap{Braitenberg Creatures}\label{ch.brait}

\sect{What are Braitenberg creatures?}

\href{http://en.wikipedia.org/wiki/Valentino_Braitenberg}{Valentino
Braitenberg} was a neuroscientist who wrote a book describing
the design of virtual vehicles which exhibited surprisingly complex
behavior.\footnote{V. Braitenberg. \textit{Vehicles: Experiments in
Synthetic Psychology} (MIT Press, 1984).} Braitenberg's vehicles have
been widely used in educational robotics. Researchers at the MIT Media
Lab created hardware implementations of the vehicles called
\emph{Braitenberg creatures}.\footnote{David W. Hogg, Fred Martin,
Mitchel Resnick. \textit{Braitenberg Creatures}. MIT Media Laboratory,
E\&L Memo 13, 1991.
\href{http://cosmo.nyu.edu/hogg/lego/braitenberg_vehicles.pdf}{http://cosmo.nyu.edu/hogg/lego/braitenberg\_vehicles.pdf}.} The
vehicles were build from \emph{programmable bricks} that were the
forerunner of the LEGO Mindstorms robotics kits.

This chapter describes an implementation of most of the Braitenberg
creatures from the MIT report adapted for the Thymio robot with VPL. The
MIT hardware used light and touch sensors, while the Thymio robot relies
primarily on infrared proximity sensors. To enable comparison with the
MIT report, the names of the creatures used there have been retained,
even though they may not be appropriate for the Thymio implementations.
The order of presentation from the report has also been retained,
although this does not correspond to the difficulty of implementation
in VPL.

In the descriptions, the phrase ``detects an object'' is used. Unless
otherwise indicated, this means that an object is detected by the front
center sensor. The easiest way to do this is to place your hand
so that it is detected by a sensor.

The \textsc{VPL} source code is available in the archive. The file names
are the same as the names of the creatures with the extension
\texttt{\small aesl}. For some creatures, additional behaviors are
suggested as exercises and their implementations also appear in the
archive.

\sect{Specification of the creatures}

\begin{description}

\item[Timid] When the robot does not detect an object, it moves forwards.
When it detects an object, it stops.

\item[Indecisive] When the robot does not detect an object, it moves
forwards. When it detects an object, it moves backwards. At just the
right distance, the robot will \emph{oscillate}, that is, it will move
forwards and backwards in quick succession.

\item[Paranoid] When the robot detects an object, it moves forwards. When
it does not detect an object, it turns to the left.

\textbf{Exercise (Paranoid1)} When an object is detected by the center
sensor of the robot, it moves forwards. When an object is detected by
the right sensor (but not by the center sensor), the robot turns right.
When an object is detected by the left sensor (but not by the center
sensor), the robot turns left.

\textbf{Exercise (Paranoid2, advanced mode)} As in \textbf{Paranoid}, but the
robot alternates the direction of its turn every second. \textbf{Hint}:
Use states to keep track of the direction and a timer to change states.

\item[Dogged] When the robot detects an object in front, it moves
backwards. When the robot detects an object in back, it moves forwards.

\textbf{Exercise (Dogged1)} As in \textbf{Dogged}, but when an
object is not detected, the robot stops.

\item[Insecure] If an object is not detected by the left sensor, turn the robot's right
motor on and the left motor off. If an object is detected by the left sensor, turn the
right motor off and the left motor on. The robot should follow a wall
placed to its left. \textbf{Hint}: See the note about turning the robot in \cref{a.blocks}.

\item[Driven] If an object is detected by the left sensor, the robot
turns the right motor on and the left motor off. If an object is
detected by the right sensor, the robot turns the left motor on and the
right motor off. The robot should approach the object in a zigzag.

\item[Persistent (advanced mode)] The robot moves forwards until it detects
an object. It then moves backwards for one second and reverses to move
forwards again.

\item[Attractive and repulsive] When an object approaches the robot from
behind, the robot runs away until it is out of range.

\item[Consistent (advanced mode)] The robot cycles through four states when
it is tapped: moving forwards, turning left, turning right, moving
backwards.

\item[Frantic (advanced mode)] The top light flashes red. \textbf{Hint}:
You can use the sensor event block with all sensors gray as explained in
\cref{a.blocks}.

\textbf{Exercise (Franctic1, advanced mode)} Implement the flashing light using
the button event block instead of the sensor event block. Is there a
difference in the behavior of the robot? If so, what causes it?

\item[Observant (advanced mode)] The robot turns the top light green when the
right sensor detects an object. The robot turns the top light red when the
left sensor detects an object. Once a light is turned on, it waits three
seconds before turning off; during this period, the light does not change.

\end{description}
