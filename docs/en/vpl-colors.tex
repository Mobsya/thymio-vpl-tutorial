\chap{Changing Colors}

\sect{Display colors}

Write a VPL program that causes two different colors to be displayed on
the top of the Thymio-II robot when the front and back buttons are
touched, and two other colors to be displayed on the bottom of the robot
when the left and right buttons are touched.

{\raggedleft \hfill Program file \bu{colors}.}

We need four event-action pairs. There are four events---touching the
four buttons---and an action is associated with each event. Note the
difference between the action blocks \blk{action-colors-up} and
\blk{action-colors-down}. The first block changes the color
displayed on the top of the robot, while the second changes the color on
the bottom of the robot. The block for the bottom has two black marks
that represent the wheels.

The program is shown in Figure~\ref{fig.colors}.

What colors are displayed? In the first three actions, the slider for
one color is moved to the right edge and displayed, while the sliders
for the other two colors are moved to the left edge and are not mixed
in. Therefore, these actions display pure red, blue and green,
respectively. The action associated with the left button mixes red and
green giving yellow. Run the program and check that these are colors are
displayed. Figure~\ref{fig.front} shows the Thymio-II displaying red on
the top and Figure~\ref{fig.bottom} shows it displaying green on the
bottom.

\sect{Exercise \thechapter.1}

Experiment with the sliders to see which colors can be displayed.

\sect{Turn the lights off}

Let us modify the program so that the lights are turned off when the
center button is touched. We need two event-action pairs, one to turn
off the top light and the other to turn off the bottom light
(Figure~\ref{fig.colors-off}). By moving all three sliders in the color
action block to the left, no colors are displayed and the light is
turned off.

The event is the same in both pairs---touching the center button---but
the actions are different---turning off the top or bottom light. 

\centeredbox{
\begin{center}
\textbf{Multiple event-action pairs}
\end{center}
\begin{itemize}
\item When a program is run, all the event-action pairs in the program
are run.
\item It is possible for several pairs to have the same event as
long as they have different actions.
\item If one event-action pair is the same as another one, VPL will
display an error message.
\end{itemize}
}

\begin{figure}[htb]
\begin{center}
\gr{colors1}{0.4}
\caption{Changing colors when a button is touched}\label{fig.colors}
\end{center}
\end{figure}

\begin{figure}[htb]
\begin{center}
\gr{colors2}{0.4}
\caption{Turn the colored lights off}\label{fig.colors-off}
\end{center}
\end{figure}
