\chap{Finite Automata}\label{ch.fa}

A \emph{finite automaton (FA)}\footnote{The plural is \emph{finite
automata} and its acronym is also FA.} is an abstract machine that is
capable of performing computations. FA are extremely important in many
areas of computer science.\footnote{FA are formally defined and
developed in textbooks such as: J.E. Hopcroft, R. Motwani, J.D. Ullman.
\textit{Introduction to Automata Theory, Languages, and Computation},
Pearson, 2013.} Consider finite strings composed of two symbols $a$ and
$b$: \begin{quote} aabbbababbaba \end{quote}

The task is to read such strings and to decide if the number of $a$'s is
odd or even. A FA to solve this problem has two states: it is in state 0
if the number of $a$'s read so far is even and it is in state 1 if the
number of $a$'s read so far is odd. The FA is represented in the
following diagram. It consists of two states, 0 and 1, and transitions
from each state that are labeled $a$ and $b$.

\begin{center}
\begin{picture}(175,45)
%\put(0,0){\framebox(175,45){}}
\put(35,25){\circle{30}}
\put(140,25){\circle{30}}
\put(3,25){\vector(1,0){15}}
\put(20,10){\makebox(30,30){0}}
\put(125,10){\makebox(30,30){1}}
\put(50,20){\vector(1,0){75}}
\put(50,10){\makebox(75,10){a}}
\put(125,30){\vector(-1,0){75}}
\put(50,30){\makebox(75,10){a}}
\put(20,10){
   \put(0,0){\oval(20,18)[b]}
   \put(0,0){\oval(20,18)[tl]}
   \put(-1,9){\vector(1,0){1}}
   \put(-20,-5){\makebox(10,10){b}}
}
\put(145,10){
    \put(10,0){\oval(20,18)[b]}
    \put(10,0){\oval(20,18)[tr]}
    \put(11,9){\vector(-1,0){1}}
    \put(20,-5){\makebox(10,10){b}}
}
\end{picture}
\end{center}

When the FA is in a state and reads a symbol from the string, it changes
its state as indicated by the transitions. If the number of $a$'s read
so far is even (state 0) and an $a$ is read, the FA takes the transition
to state 1, and conversely if the number of $a$'s read so far is odd
(state 1), the transition to state 0 is taken. If a $b$ is read, the
state does not change, because the number of $a$'s read does not change.

The FA starts in state 0 since the initial number of $a$'s read is 0
which is even. This initial state is indicated by the small arrow.

\textbf{Specification}\footnote{The specification and implementation
were inspired by the \href{https://aseba.wikidot.com/en:barcodelightpainting}{light-painting project}.}

Print out the file \p{fa-path-alternate.pdf} containing the
following image:\footnote{The path files can be found in the directory
\bu{image14} in this archive.}

\gr{fa-path-parity}{.8}

The shaded line is used to ensure that the robot moves forward and does
not move off to the right or the left. The string is encoded by squares,
where $a$ is represented by a black square and $b$ is represented by a
white square. This image represents the string $babababab$.

Place the robot at the left of the image before the first square, facing
right, with the right ground sensor in the middle of the shaded line.
The behavior of the robot is as follows:

\begin{enumerate}

\item Touch the forward button to start the robot. The top light is
turned off, the state is initialized (see below) and a timer is started.

\item Touch the center button to stop the robot.

\item The robot moves to the right. It uses the \emph{right} ground
sensor to detect if it starts to turn left or right. If the robot turns
right, the sensor will detect a lower (blacker) light level and the
robot turns left; if the robot turns left, the sensor will detect a
higher (whiter) light level and the robot turns right.

\item When the timer expires, the value of the \emph{left} ground sensor
is checked. If the robot is over a white square (reading a $b$), the top
light is turned red and the timer reset. If the robot is over a black
square (reading a $a$), the top light is turned green, the timer reset
and the state changed from even to odd or from odd to even.

\end{enumerate}


\textbf{Guidance}

The robot will use three of the four quarters in the state events and
actions:

\begin{itemize}
\item The upper-left quarter is used to indicate if the program is
running or not.
\item The upper-right quarter is used to indicate if the color (black or
white) of a square should be detected.
\item The lower-left quarter keeps track of whether the
number of $a$'s read is even or odd.
\end{itemize}

Here is an example of one event-actions pair:

\blkc{parity-0-1}

\begin{quote}
if the left sensor detects little light (a black square) \emph{and}\\
\hspace*{1em} the program is running \emph{and}
the color of the square should be detected \emph{and}\\
\hspace*{1em} an even number of $a$'s has been read so far, then\\
\hspace*{2em} turn on the top red light\\
\hspace*{2em} change the state: the number of $a$'s
read is odd \emph{and}\\
\hspace*{4em} the color of the square should not be detected\\
\hspace*{2em} reset the timer
\end{quote}

{\raggedleft \hfill \textbf{Program file}: \bu{fa.aesl}}

You will have to experiment with the speed of the motors and the
duration of the timers to reliably detect the black and white squares.

\textbf{Exercise} Print out the file \p{fa-path-blank.pdf} which
contains an image where all the squares are white. Use a marker pen to
blacken a different set of squares and test the program. In particular,
check if the program works where two or more adjacent squares are
blackened, representing a string with more than one consecutive $a$.

\textbf{Exercise} Modify the program so that it detects the remainder
$0$, $1$ or $2$ of the number of $a$'s read divided by $3$.

\bigskip

\textbf{Modifying the layout}

\warningbox{This section explains how to modify the path (the shaded
line and the squares), but you have to be familiar with the \LaTeX{}
document processing system to do so.}

The files \p{\footnotesize fa-path-*.tex} in the archive contain the
\LaTeX{} source code.

The following instruction draws a shaded line 2 cm wide by 23 cm
long:\footnote{These instructions are from the Ti\textit{k}Z
graphics library.}
\begin{footnotesize}
\begin{verbatim}
\shade[left color=black,right color=white] (0,0) rectangle +(2,23);
\end{verbatim}
\end{footnotesize}

The black and white squares are drawn using the following instructions: 
\begin{footnotesize}
\begin{verbatim}
\foreach \a in {1, 3, 5, 7}
  \filldraw[color=black] (\offset,\height*\a) rectangle +(\width,\height);

\foreach \a in {0, 2, 4, 6, 8}
  \draw (\offset,\height*\a) rectangle +(\width,\height);
\end{verbatim}
\end{footnotesize}
You can change the list of numbers in the \p{foreach} instructions
to specify which squares are black and which are white.

The \p{filldraw} and \p{draw} instructions use length parameters that
can be easily changed:
\begin{footnotesize}
\begin{verbatim}
\setlength{\height}{2.4cm}  % Height of a square
\setlength{\width}{3cm}     % Width of a square
\setlength{\offset}{2.3cm}  % Offset of the squares from the shaded line
\end{verbatim}
\end{footnotesize}

Format the file with \p{pdflatex} and print the page using software
such as \p{Adobe Reader} or \p{SumatraPDF}. The image is created in
portrait layout, but you can fix the page to a table in any orientation.
Two or more images can be taped to a table in sequence to obtain a
representation of a longer string.
