\part{Appendices}

\chap{The VPL User Interface}\label{a.toolbar}

At the top of the VPL window is a toolbar:

\begin{center}
\gr{toolbar}{1}
\end{center}

\bigskip

\textbf{New} \blksm{new}: Clears the current program and displays an
empty program area.

\bigskip

\textbf{Open} \blksm{open}: Click to open an existing program in VPL. A
window will pop up and you can navigate to the directory where the
program file (extension \p{aesl}) exists.

\bigskip

\textbf{Save} \blksm{save}: Saves the current program. It is a good idea
to click this button frequently so you don't lose your work if an fault
occurs.

\bigskip

\textbf{Save as} \blksm{saveas}: Saves the current program with a
\emph{different name}. Use it when you have a program and you want to
try something new without changing the existing program.

\bigskip

\textbf{Undo} \blksm{undo}: Undo previous actions such as deleting
an event-actions pair.\label{p.undo}

\bigskip

\textbf{Redo} \blksm{redo}: Redo an action that has been undone.

\bigskip

\textbf{Run} \blksm{run}: Runs the current program. This button is only
active if the compilation was successful. If you have changed the
program after a previous run, the button will flash to remind you that
you must click it to load the modified program into the Thymio robot.

\bigskip

\textbf{Stop} \blksm{stop}: Stops the program that is running and sets
the speed of the motors to zero. Use it when the program asks the robot
to move but does not include an event-actions pair that can stop the
motor.

\textbf{Advanced mode} \blksm{advanced}: The advanced mode enables
additional features: states, timers, accelerometers, setting sensor
thresholds.

\textbf{Basic mode} The above icon changes to \blksm{basic} in advanced mode.
Click it to return to basic mode.

\newpage

\textbf{Help} \blksm{info}: Displays the VPL documentation in your
browser. An Internet connection is required. The documentation can also
be viewed at \href{https://aseba.wikidot.com/en:thymiovpl}{https://aseba.wikidot.com/en:thymiovpl}.

\bigskip

\textbf{Export} \blksm{export}: \label{p.export} Exports a graphical
image of the program to a file. You can then import the graphics file
into a document such as a textbook or worksheet. Several formats are
available. \textsc{svg} will give the best quality, but \textsc{png} is
more widely supported.

%\section*{Feedback}

%(To be written)

%\label{p.feedback}
