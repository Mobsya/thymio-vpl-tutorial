\chap{Your First Robotics Project}

\sect{Connect the robot and run VPL}

Connect your Thymio-II robot to the computer with a USB cable. The robot
will play a sequence of tones and green lights on top of the robot will
flash. If the robot is turned off, turn it on by touching the center
button for five seconds. Run VPL by double-clicking on the icon
\blk{thymiovpl}. The window shown in Figure~\ref{fig.connect} will be
displayed. Check the box next to \bu{Serial}, click on \bu{Thymio-II
Robot \ldots} below it, select a language, and then click
\bu{Connect}.\footnote{Depending on the configuration of your computer
and the operating system that you are using, there may be several
entries in this table and the data following \bu{Thymio-II} may be
different from what is shown in the Figure.}

\begin{figure}
\begin{center}
\gr{connect}{.3}
\caption{Connecting the Thymio-II to the VPL environment}\label{fig.connect}
\end{center}
\end{figure} 

\sect{Getting to know your Thymio-II}

Figure~\ref{fig.front} shows the front and top of the Thymio-II. On the
top you can see the center circular button and four directional buttons.
Behind the buttons, the green light shows how much charge remains in the
battery. At the back are the top lights which have been set to red.
There are similar lights on the bottom which have been set to green
(Figure~\ref{fig.bottom}). The small black rectangles are sensors which
you will learn about in Chapter~\ref{ch.pet}. You can ignore the small
red lights for now.

\begin{figure}
\begin{center}
\gr{front}{.5}
\caption{The top and front of the Thymio-II robot}\label{fig.front}
\end{center}
\end{figure} 

\clearpage

\sect{The VPL user interface}

The user interface of VPL is shown in Figure~\ref{fig.gui}
There are four areas in the interface:

\begin{itemize}
\item At the top is a toolbar with icons for opening, saving and running
a program. 
\item Below the toolbar is an area where programs for controlling
the robot are written.
\item To the left and right of the program area are columns of blocks
for events and actions, respectively.
\item The area at the right shows the translation of the visual
program into a textual program. You can ignore this for now.
\end{itemize}

\begin{figure}[hbt]
\gr{gui}{.8}
\caption{VPL user interface}\label{fig.gui}
\end{figure}

%\clearpage

\sect{Write a new program}

When you open VPL, a blank programming area is displayed; should you
ever wish to obtain a new blank programming area, click \blksm{new}.

A program in VPL consists of one or more event-action pairs constructed
from an event block and an action block. For example, the pair
\blk{e-a-pair} causes the top light of the robot to display red when the
front button on the robot is touched.

\centeredbox{The meaning of an event-action pair is:\\ \centering
\textbf{\textit{When the event occurs, run the action.}}}

Let us construct an event-action pair. In the program area you will see
a template for constructing an event-action pair:
\blk{event-action-pair-empty}. The left, light blue, square is for the
event; the right, light purple square is for the action.\footnote{The
actual colors are \textit{cyan} (a mixture of blue and green) and
\textit{magenta} (a mixture of red and blue).} Start by dragging the
button event block \blk{event-buttons} from the left column and dropping
it in the blue square. Now, drag the top color action block
\blk{action-colors-up} from the right column and drop it in the purple
square. You have constructed an event-action pair.

Next we have to modify the event and the action to do what we want. For
the event, click on the front button (the top triangle); it will turn
red: \blk{forward}.

This specifies that \textbf{an event occurs when the front button is touched}.

The color action block contains three bars with the primary colors red,
green, blue; at the left end of each bar is a white square. The colored
bar with its white square is called a \emph{slider}. Drag a white square
to the right and then back to the left, and you will see that the
background color of the block changes. Move the red slider until the
square is at the far right, and move the green and blue sliders until
they are at the far left. The color will be all red and no blue or
green: \blk{red}.

\sect{Save the program}

Before running the program, save it. Click on the icon \blksm{save}
in the toolbar. You will be asked to give the program a name; choose a
name that will help you remember what the program does, perhaps,
\bu{display-red}.

\sect{Run the program}

To run the program, click on \blksm{run} in the toolbar. Touch
the front button on the robot; the light on top of the robot should
change to red.

\sect{Turn the robot off}

When you have finished working with the Thymio-II robot, you can turn it
off by touching and holding the center button for four seconds until
you hear a sequence of descending tones. The battery in the robot will
continue charging as long as it is connected to a working computer. A
red light next to the USB cable connector means that the robot is
charging; it turns blue when the charging is completed. You can
disconnect the cable when you will be away for a long time
(Figure~\ref{fig.back}).

\begin{figure}
\begin{center}
\gr{back}{.6}
\caption{The back of the Thymio-II showing the USB cable and the
 charging light}\label{fig.back}
\end{center}
\end{figure}

Should the USB cable become disconnect, VPL will wait for the connection
to be made again. Check both ends of the cable, reconnect and see if VPL
is working. If you have a problem, you can always close VPL, reconnect
the robot and open VPL again.

\sect{Modify a program}

\begin{itemize}

\item To delete an event-action pair, click \blksm{x} which
appears next to each pair.

\item To add an event-action pair, click \blksm{plus}
which appears between every two pairs and at the end of the program.

\item To move an event-action pair to another position in the
program, drag and drop it.

\end{itemize}

\sect{Open an existing program}

Suppose that you have saved your program, and turned off the robot and
the computer, but later you wish to continue to work on the program.
Connect the robot and run VPL as described above. Click on the icon
\blk{open} and select the program you want to open, for example,
\bu{display-red}. The event-action pairs of the program will be displayed in
the program area and you can modify the program.

\sect{Additional operations in the VPL user interface}

\begin{itemize}

\item \textbf{Save as} \blk{saveas}: Click on this icon to save the
current program with a \emph{different name}. Use it when you have a
working program and you want to try something new without changing the
existing program.

\item \textbf{Stop} \blksm{stop}: This stops the program that is running
and sets the speed of the motors to zero. Use it when the program asks
the robot to move but does not include an event-action pair that can
stop the motor.

\item \textbf{Change color scheme} \blk{scheme}: You can select a
different pair of colors for the event and action blocks.

\item \textbf{Advanced mode} \blk{advanced}: The advanced mode enables
the use of state variables as described in Chapter~\ref{ch.states}.

\item \textbf{Help} \blk{info1}: Displays the VPL documentation in
your browser.

The documentation is located at
\url{https://aseba.wikidot.com/en:thymiovpl}.

\end{itemize}
