% !TeX root = vpl.tex

\chap{Your First Robotics Project}\label{ch.intro}

\sect{Getting to know your Thymio}

\Cref{fig.front} shows the front and top of Thymio II. On the
top you can see the center circular button (\textcolor{blue}{A}) and four directional buttons (\textcolor{blue}{B}).
Behind the buttons, the green light (\textcolor{blue}{C}) shows how much charge remains in the
battery. At the back are the top lights (\textcolor{blue}{D}), which have been set to red in this picture.
There are similar lights on the bottom (set to green in \cref{fig.bottom}).
The small black rectangles (\textcolor{blue}{E}) are sensors which
you will learn about in \cref{ch.pet}. You can ignore the small
red lights for now.

\begin{figure}[h]
\begin{center}
\gr{front.jpg}{.8}
\caption{The top and front of the Thymio robot}\label{fig.front}
\end{center}
\end{figure}

\pagebreak

\sect{Connect the robot and run VPL}

Connect your Thymio robot to your computer with a USB cable; the robot will play a sequence of tones.
If the robot is turned off, turn it on by touching the center
button for five seconds. Run VPL by double-clicking on the icon
\blksm{thymiovpl}.

\importantbox{When a small icon appears in the text,
a larger image is displayed in the margin.}

VPL may connect automatically to your robot.
If not, the window shown in \cref{fig.connect} will be
displayed. Check the box next to \bu{Serial}, click on \bu{Thymio
Robot} below it, select a language, and then click
\bu{Connect}.
Depending on the configuration of your computer
and the operating system that you are using, there may be several
entries in this table and the data following \bu{Thymio} may be
different from what is shown in the Figure.

\trickbox{It is also possible to access VPL from Aseba Studio, the text-based programming environment, through the VPL plugin found in the \textit{Tool} area at the bottom left of the screen.}

\begin{figure}
\begin{center}
\gr{connect}{.4}
\caption{Connect to Thymio, through serial port (USB)}\label{fig.connect}
\end{center}
\end{figure}

\vfill

\pagebreak

\sect{The VPL user interface}

The user interface of VPL is shown below.
There are six areas in the interface:
\begin{enumerate}[noitemsep,nosep]
\item A toolbar with icons for opening, saving, running a program, etc.
\item A program area where programs for controlling the robot are constructed.
\item An indication whether the program you are creating is well-formed or not.
\item A column with available event blocks to construct your program.
\item A column with available action blocks to construct your program.
\item The text translation of the program (see at the bottom of this page).
\end{enumerate}
The event and action blocks will be described in the course of this document.

\plainfloat
\begin{figure}[h]
\gr{gui}{1}
\vskip -1em
\caption{The VPL window}\label{fig.vplgui}
\end{figure}
\framedfloat

\vfill

\trickbox[To go further]{
When you construct a program using VPL, the text program that will be loaded into the robot appears in the right part of the window.
If you are curious and wish to understand this language, you can read the \href{https://aseba.wikidot.com/en:thymiotutoriel}{text mode tutorial} (\url{http://aseba.wikidot.com/en:thymiotutoriel}).
}

\pagebreak

\sect{Write a program}

When you start VPL, a blank program area is displayed; if, after having built a piece of program, you wish to clear the content of the program area, click \blksm{new}.

A program in VPL consists of one or more \emph{event-action pairs},
each constructed
from an event block and an action block. For example, the pair:
\blkc{e-a-pair} causes the top light of the robot to display red when one touches the front button on the robot.

\importantbox{
The meaning of an event-action pair is:\\
\textbf{When the event occurs, the robot runs the action.}
}

Let us construct an event-action pair.
In the program area you see an outline of a pair:
\blkc{event-action-pair-empty} 
%It is a template for constructing the event-action pair.
The left, light blue, square is the space for the
event; the right, pink, square is the space for the action. %\footnote{The actual colors are \textit{cyan} (a mixture of blue and green) and \textit{magenta} (a mixture of red and blue).}
To bring a block to the program area from the sides (areas 4 and 5 of \cref{fig.vplgui}), you can click on it or drag it into the corresponding square by holding the left mouse button pressed and releasing it when the block is
in its place.

Start by bringing the button event block \blksm{event-buttons} into the blue square.
Now, bring the top color action block \blksm{action-colors-up} into the pink square.
You have constructed an event-action pair!

Next we have to modify the event and the action to do what we want. For
the event, click on the front button (the top triangle); it will turn
red: \blkc{forward}
This specifies that \textbf{an event will occur when the front button of Thymio is touched}.

The color action block contains three bars with the primary colors red,
green, blue; at the left end of each bar is a white square. The colored
bar with its white square is called a \emph{slider}. Drag a white square
to the right and then back to the left, and you will see that the
background color of the block changes.
All colors can be made by mixing these three primary colors: red, green and blue.
Now move the red slider until the
square is at the far right, and move the green and blue sliders until
they are at the far left. The color will be all red with no blue nor
green: \blkc{red}

\sect{Save the program}

Before running the program, save it on your computer.
Click on the icon \blksm{save}
in the toolbar. You will be asked to give the program a name; choose a
name that will help you remember what the program does, perhaps,
\bu{display-red}.
Choose the location where you want to save your program, on the desktop for instance, and click on Save.

\sect{Run the program}

To run the program, click on \blksm{run} in the toolbar. Touch
the front button on the robot; the light on top of the robot should
change to red.

\infobox{Congratulations!}{
You have created and executed your first program. Its behavior is:\\
\textbf{When you touch the forward button of the Thymio, it becomes red.}
}{yellow!10}{\bcetoile}

\sect{Turn the robot off}

When you have finished working with the Thymio robot, you can turn it
off by touching and holding the center button for four seconds until
you hear a sequence of descending tones. The battery in the robot will
continue charging as long as it is connected to a working computer. A
red light next to the USB cable connector means that the robot is
charging; it turns blue when the charging is completed (\cref{fig.back}).
You can disconnect the cable when you do not use the robot.

\trickbox{
You can recharge the robot faster by using a mobile phone charger with a micro-USB connector.}

\begin{figure}
\begin{center}
\gr{back}{.6}
\caption{The back of the Thymio showing the USB cable and the
 charging light}\label{fig.back}
\end{center}
\end{figure}

Should the USB cable disconnect during programming, VPL will wait for the connection to be made again.
Check both ends of the cable, reconnect and see if VPL is working.
If you have a problem, you can always close VPL, reconnect the robot and open VPL again.

\sect{Modify a program}

\begin{itemize}

\item To delete an event-action pair, click \blksm{x} at the top-right of the pair.
\item To add an event-action pair, click \blksm{plus} available below every pair.
\item To move an event-action pair to another position in the program, drag it and drop it at the desired location.

\end{itemize}

\sect{Open an existing program}

Suppose that you have saved your program and turned off the robot and
the computer, but later you wish to continue to work on the program.
Connect the robot and run VPL as described previously. Click on the icon
\blksm{open} and select the program you want to open, for example,
\bu{display-red}. The event-action pairs of the program will be displayed in
the program area, and you can continue modifying it.

\sect{Additional operations in the VPL user interface}

In the toolbar, you will find further features:

\begin{itemize}

\item \textbf{Save as} \blksm{saveas}:
Click on this icon to save the current program with a \emph{different name}.
Use it when you have a working program and you want to try something new without changing the existing program.

\item \textbf{Stop} \blksm{stop}: This stops the program that is running
and sets the speed of the motors to zero. Use it when the program asks
the robot to move but does not include an event-action pair that can
stop the motor.

\item \textbf{Change color scheme} \blksm{scheme}: You can select a
different pair of colors for the background of the event and action blocks.

\item \textbf{Advanced mode} \blksm{advanced}: The advanced mode enables
the use of state variables as described in \cref{ch.states}.

\item \textbf{Help} \blksm{info1}: Displays the VPL documentation in
your browser (an Internet connection is required).
This documentation is located at
\url{https://aseba.wikidot.com/en:thymiovpl}.

\end{itemize}


