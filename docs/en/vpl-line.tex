\chap{The Robot Finds Its Way by Itself}\label{ch.line}

In this chapter we program the robot to solve a real-world problem:
\emph{line following}. Consider a warehouse with robotic carts that
bring objects to a central dispatching area. There are lines painted on
the floor of the warehouse and the robot receives instructions to follow
certain lines until it reaches the storage bin of the desired object.

Write a program that causes the robot to follow a line on the floor.

{\raggedleft \hfill Program file \bu{follow-line.aesl}}

The line-following task brings out all the uncertainty of constructing
robots in the real world. The line might not be perfectly straight, dust
may obscure part of the line, dirt may cause one wheel to move more
slowly than the other one. To follow a line, the robot must use a
\emph{controller} that decides how much power to apply to each motor
depending on the data received from the sensors.

\sect{Setup}

If the floor is light-colored, the sensor will detect a lot of reflected
light and the event \blk{lots-of-light} will occur. We need a line
that will cause an event to occur when there is little reflected light
\blk{little-light}. This is easy to do by placing black
electrician's tape on the floor (Figure~\ref{fig.tape}). The tape must
be wide enough so that both ground sensors will sense black when the
robot is successfully following the tape. A tape of width of 5
centimeters is sufficient for the robot to follow the line even if there
are small deviations.

\begin{figure}
\begin{center}
\gr{blacktape}{.6}
\caption{The Thymio-II robot following a line of tape}\label{fig.tape}
\end{center}
\end{figure}

Make sure that you use a USB cable that is long enough (say, two
meters), so that the Thymio-II can stay connected to the computer even
as it moves.

\sect{Start and stop the robot}

The robot must move forward whenever \emph{both} sensors detect the
black tape and stop whenever \emph{both} sensors detect the floor. The
event-pairs are shown in Figure~\ref{fig.start-stop}. The first pair
stops the motor when both ground sensors detect the floor, while the
second pair causes the robot to move forwards when both sensors detect
the tape.

\begin{figure}
\begin{center}
\gr{line-forward}{.4}
\caption{Start and stop the robot}\label{fig.start-stop}
\end{center}
\end{figure}

The next step is to program the controller that follows the line:
\begin{itemize}

\item If the robot moves off the tape to the \emph{left}, the
\emph{left} sensor will detect the floor while the \emph{right} sensor
is still detecting the tape; in that case the robot should turn slightly
to the \emph{right} (Figure~\ref{fig.one-off}).

\item If the robot moves off the tape to the \emph{right}, the
\emph{right} ground sensor will detect the floor while the \emph{left}
sensor is still detecting the tape; in that case the robot should turn
slightly to the \emph{left}.

\end{itemize}


\begin{figure}
\begin{center}
\begin{picture}(200,100)
%\put(0,0){\framebox(200,100)}
\thicklines
\put(0,20){\line(1,0){200}}
\put(0,70){\line(1,0){200}}
\thinlines
\put(80,0){
\put(0,0){\line(1,1){50}}
\put(0,0){\line(-1,1){50}}
\put(50,50){\line(-1,1){50}}
\put(-50,50){\line(1,1){50}}
\put(20,55){\framebox(10,10){}}
\put(5,72){\framebox(10,10){}}
\put(-5,95){\line(1,0){10}}
\put(45,45){\line(0,1){10}}
\put(80, 0){\makebox(40,10)[l]{\bu{Floor}}}
\put(80, 40){\makebox(40,10)[l]{\bu{Tape}}}
\put(80, 80){\makebox(40,10)[l]{\bu{Floor}}}
\put(25,75){\vector(1,1){20}}
}
\end{picture}
\caption{The left sensor is off the tape and the right sensor is on the tape}
\label{fig.one-off}
\end{center}
\end{figure}

Two event-action pairs are needed (Figure~\ref{fig.follow-line}).

\sect{Setting the parameters}

It is easy to see that if the robot runs off the left edge of the tape
as shown in Figure~\ref{fig.one-off}, it has to turn to the right, but
how tight should the turn be? If the turn is too gentle, the right
sensor might \emph{also} run off the tape before the robot turns back;
if the turn is too aggressive, it might cause the robot to run off the
other end of the tape. In any case, aggressive turns can be dangerous to
the robot and whatever it is carrying.

\centeredbox{\begin{center} \textbf{Parameters} \end{center}
\emph{Parameters} are values that can be changed without affecting the
overall structure of the event-action pairs that form the program.}

\begin{figure}
\begin{center}
\gr{line-controller}{.4}
\caption{Following the line}\label{fig.follow-line}
\end{center}
\end{figure}

The parameters that you can set in this program are the speeds of the
left and right motors in each motor action block. You will need to
experiment with these values until the robot runs \emph{reliably}. Here,
reliably means that the robot can successfully follow the line if you
run the program several times. Since each time you place the robot on
the line you might place it at a slightly different position and
pointing in a slightly different direction, it is not enough to test the
program only once.

Surprisingly, the forward speed of the robot on the tape is also an
important parameter. If it is too fast, the robot can run off the tape
before the turning actions can affect its direction. However, if the
speed is too slow, no one will buy your robot to use in a warehouse.

There are many ways to set the parameters of the motor action blocks
that control the turns. You can make one motor go slightly faster than
the other, or you can have one motor go forward and one go backwards.

\sect{Exercise \thechapter.1}

The robot stops when both ground sensors detect that they are off the
tape. Modify the program so that the robot makes a gentle left turn in
an attempt to find the tape again. Try it on a tape with a left turn
like the one shown in Figure~\ref{fig.tape}. Try increasing the forward
speed of the robot. What happens when the robot gets to the end of the
tape?

\sect{Exercise \thechapter.2}

Modify the program from the previous exercise so that
the robot turns right when it runs off the tape. What happens?

It would be nice if we could \emph{remember} which sensor was the last
one to lose contact with the tape in order to cause the robot turn in
the correct direction to find the tape again. In Chapter~\ref{ch.states}
we will learn how to remember information.

\sect{Exercise \thechapter.3}

Experiment with different arrangements of the lines of tapes:
\begin{itemize}
\item Gentle turns;
\item Sharp turns;
\item Zigzagging lines;
\item Wider lines (use an overlapping double line of tape);
\item Narrow lines (cut the tape lengthwise before placing it on the floor).
\end{itemize}

Run competitions with your friends: Whose robot successfully follows the
most lines? For each line, whose robot follows it in the shortest time?

\sect{Exercise \thechapter.4}

Discuss what effect the following modifications to the Thymio-II would
have on the ability of the robot to follow a line:

\begin{itemize}
\item Ground sensing events occur more often or less often than 10 times a second;
\item The sensors are further apart or closer together;
\item There are more than two ground sensors on the bottom of the robot.
\end{itemize}
