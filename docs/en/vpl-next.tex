% !TeX root = vpl.tex

\chap{What Next?}\label{ch.next}

This tutorial has introduced the Thymio-II robot and the Aseba/VPL
environment. The VPL environment with its simple visual programming is
intended for beginners. To develop more advanced programs for the robot,
you will want to learn how to use the Aseba Studio environment (\cref{fig.studio}).

\begin{figure}[hbt]
\begin{center}
\gr{studio}{.8}
\caption{Aseba Studio environment}\label{fig.studio}
\end{center}
\end{figure}

Programming in Aseba Studio is also based upon the concepts of events
and actions.
All features you discovered while using VPL are available.
But you have a lot more freedom because:
\begin{itemize}
\item You can control precisely when an event causes an action, depending, for
example, on the amount of reflected light measured from a ground sensor or
the distance from a horizontal sensor.
\item You can specify that a single action consists of several different
operations: controlling the motors, changing the state, setting
thresholds, turning lights on and off, etc.
\item You have the flexibility of a full programming language with
variables, expressions, and control statements.
\end{itemize}

Aseba Studio gives you access to features of the Thymio-II that are not
available in VPL:

\begin{itemize}
\item You can control all the lights such as the circle of lights
surrounding the buttons.
\item You have more flexibility in synthesizing sound.
\item There is a temperature sensor.
\item Instead of just sensing the shock of a tap, accelerometers
can sense gravity and changes of speed in three dimensions.
\item A remote control device can be used with the robot.
\end{itemize}
When you are working with Aseba Studio, you can open VPL by clicking on
the button \bu{Launch VPL} in the \emph{Tools} tab at the bottom left of the window.
You can import VPL programs into Aseba Studio simply by opening the file.

Tu use Aseba Studio, start from the \emph{Programmation} page of the Thymio web site : \url{https://aseba.wikidot.com/en:thymioprogram} and follow the links corresponding to the \emph{text programming}.
If you are out of inspiration, you will find many challenges on the example page :  \url{https://aseba.wikidot.com/en:thymioexamples}.

\vspace{4em}

\infobox{Have fun and learn a lot!}{Thank you for reading this tutorial!}{yellow!10}{\bcetoile}

