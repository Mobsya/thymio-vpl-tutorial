\part{Parsons Puzzles}

\chap{Parsons Puzzles for VPL}\label{ch.parsons}

\newcommand*{\eblock}{\framebox[40pt]{\rule[-11pt]{0pt}{32pt}}\ }

\sect{What are Parsons puzzles?}

\emph{Parsons puzzles} are a form of exercise that can help students
learn how to program.\footnote{Parsons, D. and Haden, P. Parson's
programming puzzles: A fun and effective learning tool for first
programming courses. \textit{Proceedings of the 8th Australian
Conference on Computing Education}, Darlinghurst, Australia, 2006,
157–163.} A Parsons puzzle consists of a specification of a program
together with a set of statements in a programming language. Your task
is to place the statements in the correct order so that they the form a
program that implements what is required. A Parsons puzzle may also
include \emph{distractors}, which are incorrect statements or extra
statements that are not needed in the solution.
The advantage of Parsons puzzles is that all the statements needed for the
solution are visible to the student and have the correct syntax.

In VPL, there is almost no meaning to the order of the set of
event-actions pairs in a program. Therefore, the puzzles will be
programs where one or more pairs are missing the event block, the action
block or both. To the right of each event-actions pair will appear two
or more blocks; select the correct block and draw an arrow from it to
the empty block.

\textbf{Example}
When the forwards button is touched, the top green light is turned on.

\bigskip\bigskip

\begin{center}
\begin{tabular}{l@{\hspace{5em}}lll}
\blk{forward} $\rightarrow$ \eblock  &  \blk{red} & \blk{green}\\
\end{tabular}
\begin{picture}(250,20)
\put(230,60){\line(0,1){20}}
\put(230,80){\line(-1,0){155}}
\put(75,80){\vector(0,-1){20}}
\end{picture}
\end{center}

\vspace*{-8ex}

\sect{The puzzles}


\begin{enumerate}

\item When the right button is touched the bottom red light is turned on.

\bigskip

\begin{tabular}{l@{\hspace{5em}}lll}
\blk{right-button} $\rightarrow$ \eblock  &  \blk{red-bottom} & \blk{red}\\
\end{tabular}

\bigskip

\item When the right button is touched the top red light is turned on.

\bigskip

\begin{tabular}{l@{\hspace{5em}}lll}
\eblock $\rightarrow$ \blk{red} & \blk{left-button} &
 \blk{right-button}\\
\end{tabular}

\bigskip

\item When the left button is touched the bottom green light is turned on.

\bigskip

\begin{tabular}{l@{\hspace{5em}}lllll}
\eblock $\rightarrow$ \eblock  &  \blk{right-button} & \blk{left-button}
 & \blk{green} & \blk{green-bottom}\\
\end{tabular}

\bigskip

\item When the left button \textbf{or} the right button is touched, the
top green light is turned on.

\bigskip

\begin{tabular}{l@{\hspace{5em}}lll}
\blk{left-button} $\rightarrow$ \eblock  &  \blk{green} &
  \blk{green-bottom}\\
\\
\eblock $\rightarrow$ \blk{green}  &  \blk{right-button} &
 \blk{left-button}\\
\end{tabular}

\bigskip

\item When \textbf{both} the left button \textbf{and} the right button
are touched, the top red light is turned on.
Select one of the following two programs:

\begin{center}
\begin{tabular}{c@{\hspace{5em}}c@{\hspace{5em}}c}
\blk{left-right-button} $\rightarrow$ \blk{red} & \textbf{or}&
\blk{left-button} $\rightarrow$ \blk{red}\\
&&\blk{right-button} $\rightarrow$ \blk{red}
\end{tabular}
\end{center}

\vspace{-2ex}

\bigskip

\item If an object is detected \textbf{only} by the leftmost sensor, turn left.

\bigskip

\begin{tabular}{l@{\hspace{5em}}lllll}
\eblock $\rightarrow$ \blk{left-turn} & \blk{sensor-and-prox} &
\blk{right-prox} & \blk{center-prox} & \blk{left-prox} \\
\end{tabular}

\bigskip

\item Stop the robot when the end of the table has been reached.

\bigskip

\begin{tabular}{l@{\hspace{5em}}llll}
\eblock $\rightarrow$ \blk{action-motors} & \blk{event-prox-ground} &
 \blk{ground2} & \blk{ground1}\\
\end{tabular}

\bigskip

\item When the robot detects a wall, the top red light is turned on.

\bigskip

\begin{tabular}{l@{\hspace{5em}}lll}
\eblock $\rightarrow$ \blk{red} & \blk{center-prox} & \blk{ground1}\\
\end{tabular}

\bigskip

\item When the robot hits the wall, the motors are turned off.
\bigskip

\begin{tabular}{l@{\hspace{5em}}llll}
\blk{event-tap} $\rightarrow$ \eblock & \blk{full} & \blk{back-full} & \blk{action-motors}\\
\end{tabular}

\bigskip

\item The robot turns to the left if there is an object in front of the center sensor.

\bigskip

\begin{tabular}{l@{\hspace{5em}}llll}
\blk{center-prox} $\rightarrow$ \eblock & \blk{left-turn} & \blk{full} & \blk{right-turn}\\
\end{tabular}

\bigskip

\item The robot turns to the right if there is \textbf{no} object in front of the center sensor.

\bigskip

\begin{tabular}{l@{\hspace{5em}}llll}
\eblock $\rightarrow$ \blk{right-turn} & \blk{center-prox} & \blk{no-detect-forward} &
\blk{neither-prox}\\
\end{tabular}

\bigskip

\item The motors are turned off when the left button is touched
\textbf{or} if the robot is tapped.

\bigskip

\begin{tabular}{l@{\hspace{5em}}lllll}
\eblock $\rightarrow$ \blk{action-motors} & \blk{event-buttons} &
\blk{left-right-button} & \blk{left-button} & \blk{right-button}\\
\\
\eblock $\rightarrow$ \blk{action-motors} & \blk{event-tap} &
\blk{event-clap}
\end{tabular}

\bigskip

\item When the forwards button is touched, the robot moves forward
for three seconds and then moves backwards.

\bigskip

\begin{tabular}{l@{\hspace{5em}}llll}
\blk{forward} $\rightarrow$ \blk{full}\\
\\
\blk{forward} $\rightarrow$ \eblock & \blk{event-timer} & \blk{three-seconds}\\
\\
\eblock       $\rightarrow$ \blk{back-full} & \blk{event-timer} &  \blk{three-seconds}\\
\end{tabular}

\bigskip

\item The robot moves towards an object that is detected by its left,
right or center sensor.

\bigskip

\begin{tabular}{l@{\hspace{5em}}llll}
\blk{center-prox} $\rightarrow$ \blk{full}\\
\\
\blk{left-prox} $\rightarrow$ \eblock & \blk{right-turn} & \blk{full} &
 \blk{left-turn} & \blk{action-motors}\\
\\
\eblock       $\rightarrow$ \eblock & \blk{right-turn} & \blk{left-turn} &
 \blk{left-prox} & \blk{right-prox}\\
\end{tabular}

\bigskip

\item The robot is following a line on the floor. It turns left if it no
longer detects the line in its right sensor and it turns right if it no
longer detects the line in its left sensor,

\bigskip

\begin{tabular}{l@{\hspace{5em}}llll}
\eblock $\rightarrow$ \blk{right-turn} & \blk{bottom-right} & \blk{bottom-left} & \blk{left-prox} & \blk{right-prox}\\
\\
\eblock $\rightarrow$ \eblock & \blk{bottom-right} & \blk{bottom-left} & \blk{right-turn} & \blk{left-turn}\\
\\
\end{tabular}

\item The robot counts 0,1,2,3,0,1,2,3, \ldots, whenever it
detects a clap event.

\bigskip

\begin{tabular}{l@{\hspace{3em}}llll}

\blk{event-clap} \blk{state-0} $\rightarrow$ \eblock &
\blk{state-0} & \blk{state-1} & \blk{state-2} & \blk{state-3}\\ 
\\
\blk{event-clap} \eblock $\rightarrow$ \blk{state-2} &
\blk{state-event-0} & \blk{state-event-1} & \blk{state-event-2} & \blk{state-event-3}\\
\\
\blk{event-clap} \eblock $\rightarrow$ \blk{state-3} &
\blk{state-event-0} & \blk{state-event-1} & \blk{state-event-2} & \blk{state-event-3}\\
\\
\blk{event-clap} \eblock $\rightarrow$ \eblock &
\blk{state-event-0} & \blk{state-event-3} & \blk{state-0} & \blk{state-3}\\ 
\\
\end{tabular}

\newpage

\item When the center button is touched, the right front and left front circle
lights turn on and off alternately at one-second intervals.

\bigskip

\begin{tabular}{l@{\hspace{3em}}llll}

\blk{center-button} \blk{event-state} $\rightarrow$ \eblock \blk{one-second} &
\blk{action-states} & \blk{state-0} & \blk{state-1} & \blk{state-2}\\ 
\\
\blk{event-timer} \blk{state-1} $\rightarrow$ \blk{state-2} \eblock &
\blk{event-timer} & \blk{action-timer} & \blk{one-second} & \blk{three-seconds}\\ 
\\
\eblock \blk{state-2} $\rightarrow$ \eblock \blk{one-second} &
\blk{event-timer} & \blk{action-timer} & \blk{state-0} & \blk{state-1}\\ 
\\
\end{tabular}

\bigskip

\item The bottom light of the robot turns green when it detects an object 
far away from it and the top light of the robot turns red when it
detects an object close to it.

\bigskip

\begin{tabular}{l@{\hspace{3em}}llll}

\eblock \blk{event-state} $\rightarrow$ \blk{bottom-green} &
\blk{far} & \blk{close} & \blk{far-no} & \blk{close-no}\\ 
\\

\eblock \blk{event-state} $\rightarrow$ \blk{red} &
\blk{far} & \blk{close} & \blk{far-no} & \blk{close-no}\\ 
\\
\end{tabular}

\bigskip


\item Tilt the robot on its left side; the top light turns blue and the bottom
light is turned off. Tilt the robot on its back; the top light is turned
off and the bottom light turns yellow.

\bigskip

\begin{tabular}{l@{\hspace{3em}}llll}

\eblock \blk{event-state} $\rightarrow$ \blk{blue} \blk{action-colors-down} &
\blk{tilt-left} & \blk{tilt-right} & \blk{tilt-front} & \blk{tilt-back}\\ 
\\

\eblock \blk{event-state} $\rightarrow$ \blk{action-colors-up} \blk{yellow-bottom} &
\blk{tilt-left} & \blk{tilt-right} & \blk{tilt-front} & \blk{tilt-back}\\ 
\\
\end{tabular}

\end{enumerate}
