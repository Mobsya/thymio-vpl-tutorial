\chap{A Pet Robot}\label{ch.pet}

The robots we build in this chapter are called \emph{autonomous robots}.
They display independent behavior that is normally associated with
living things like cats and dogs. The behavior is achieved by
\textit{feedback}: the robot will sense that something occurs in the
world and modify its behavior accordingly.

\sect{The robot obeys you}

First, we will train the robot to obey. Normally, the robot stays in
place without moving; when it detects your hand in front of it, it will
move towards your hand.

There are five horizontal distance sensors on the front of the Thymio-II
robot and two on the rear of the robot.\footnote{The technical term is
\emph{proximity sensor}, but we will use the simpler term
\emph{horizontal distance sensor}.} Bring your hand slowly towards the
sensors; when it gets close, a red light will appear around the sensors
that detect your hands (Figure~\ref{fig.detect}).

\begin{figure}
\begin{center}
\gr{detect}{.6}
\caption{The front of the Thymio-II. Two sensors detect the fingers.}\label{fig.detect}
\end{center}
\end{figure}

The block \blk{event-prox} is used to sense if something is close to the
sensor or not. In either case it causes an event to occur. The small
gray squares (five on the front and two on the rear) are used to specify
when an event occurs. Clicking on a square will change it from gray to
white to red and back to gray.

\begin{itemize}
\item \textbf{Gray}: The value of the sensor does not influence the
program.
\item \textbf{Red}: An event occurs when the sensor detects an object
close to it.
\item \textbf{White}: An event occurs when the sensor detects that there
is \emph{no} object close to it.
\end{itemize}

To implement the behavior, we need the two event-action pairs shown in
Figure~\ref{fig.follow-hand}. In the first pair, the center front sensor
is white and the associated action is that the motors are off.
Therefore, when the robot sits on the table it will not move, and it
will stop if it had been moving. In the second pair, the center front
sensor is red and the associated action is that both motors run quite
fast. Therefore, when you bring your hand near the front of the robot,
an event occurs that causes the robot to move forward.

\begin{figure}
\begin{center}
\gr{likes-forward}{.4}
\caption{Moving towards your hand}\label{fig.follow-hand}
\end{center}
\end{figure}


\sect{Steering the Thymio-II robot}

The Thymio-II robot does not have a steering wheel like a car or a
handlebar like a bicycle. So how can it turn? The robot uses
\emph{differential drive}, which is familiar from tracked vehicles like
bulldozers (Figure~\ref{fig.bull}). Instead of turning a handlebar the
desired direction, the left and right tracks or wheels are driven by
individual motors at \emph{different} speeds. If the right track moves
faster than the left one, the vehicle turns left, and if the left track
moves faster than the right one, the vehicle turns right.

The greater the difference between the speeds, the tighter the turn. To
achieve a large difference of speeds, you can drive one track forward
and one track backwards. In fact, if one track moves forward at a
certain speed, while the other track moves backwards at the \emph{same}
speed, the vehicle turns in place.

\begin{figure}
\begin{center}
\gr{bulldozer}{0.35}
\caption{A bulldozer with tracks for differential drive}\label{fig.bull}
\end{center}
\end{figure}

In VPL you can implement differential drive on the Thymio-II robot by
setting the left and right sliders of a motor action block are set to
different values. In the block \blk{differential}, the left slider has
been set for fast speed backwards, while the right slider has been set
for fast speed forwards. The result is that the robot will make a tight
left turn, as indicated by the small image of the robot within the motor
action block.

Experiment with an event-action pair such as \blk{turning}.

Set the left and right sliders, run the program
and touch the center button; to stop the robot click on \blksm{stop}.
Now you can change the sliders and try again.


\sect{The robot likes you}

A real pet follows you around. To make the robot follow your hand, add
two additional event-action pairs: if the robot detects an object in
front of its left-most sensor, it turns to the left, while if it detects
an object in front of its right-most sensor, it turns to the right.

{\raggedleft \hfill Program file \bu{likes.aesl}}

The program for the robot that likes you consists of two event-action
pairs (Figure~\ref{fig.likes}). Experiment with the sliders on each
motor action block.

\begin{figure}
\begin{center}
\gr{likes-turns}{0.4}
\caption{The robot likes you}\label{fig.likes}
\end{center}
\end{figure}

\sect{Exercise \thechapter.1}

Modify the pet robot so that it starts moving forward when the program
is run and stops when it detects the edge of a table (or a strip of
tape).

\sect{Exercise \thechapter.2}

What happens if you change the order of
the event-action pairs that you used in the previous exercise?


\centeredbox{
\begin{center}
\textbf{Ground sensors and horizontal sensors}
\end{center}
Be careful not to confuse the behavior of the horizontal
sensors with the behavior of the ground sensors.
\begin{itemize}

\item For the horizontal sensors, the white square specifies that an
event will occur if there is \emph{nothing nearby}, while the red square
specifies that an event will occur if there is \emph{something nearby}.

\item For the ground sensors, the white square specifies that an event
will occur if \emph{only a little light is reflected from the surface},
while the red square specifies that an event will occur \emph{if a lot
of light is reflected from the surface}.

\end{itemize}
}

As explained in Chapter~\ref{c.moving}, a lot of light will be reflected
from a white surface, while very little light will be reflected from a
black surface. You will have to experiment to determine when to click on
a white square and when on a red square, depending on the floor or
table where you place the robot.

\newpage

\sect{The robot doesn't like you}

Sometimes your pet may be in a bad mood and turn away from your hand.
Write a program that causes this behavior in the robot.

{\raggedleft \hfill Program file \bu{does-not-like.aesl}}

Open the program for the pet that likes you and exchange the association
of the events with the actions. Detection of an obstacle by the left
sensor causes the robot to turn right, while detection of an obstacle by
the right sensor causes the robot to turn left (Figure~\ref{fig.hates}).

\begin{figure}[htb]
\begin{center}
\gr{hates}{0.4}
\caption{The robot doesn't you}\label{fig.hates}
\end{center}
\end{figure}

\sect{Exercise \thechapter.3}

Experiment with the sensors. Instead of using sensors 0 and
4:\footnote{The front horizontal sensors are numbered 0, 1, 2, 3, 4 from
the left of the robot to its right. The rear sensors are numbered 5 for
the left one and 6 for the right one.}

\begin{itemize}
\item Use sensors 1 and 3 to turn the robot left and right,
respectively.
\item Use both sensors 0 and 1 to turn the robot left and both sensors 3
and 4 to turn the robot right.
\item Add event-action pairs for the rear sensors 5 and 6.
\end{itemize}

\sect{Setting the slides precisely (advanced)}

It is difficult to set the sliders precisely so that, for example, both
motors run at the same speed. By looking at the translation of the
event-action pairs into a textual program you can improve the precision.
The text is shown at the right of the VPL window.
Here is the translation of the program where the
pet likes you and follows you around.

\begin{small}
\begin{verbatim}
onevent prox
  if prox.horizontal[2] < 400 then
    motor.left.target = 0
    motor.right.target = 0
  end
  if prox.horizontal[2] > 500 then
    motor.left.target = 300
    motor.right.target = 300
  end
  if prox.horizontal[0] > 500 then
    motor.left.target = -300
    motor.right.target = 300
  end
  if prox.horizontal[4] > 500 then
    motor.left.target = 300
    motor.right.target = -300
  end
\end{verbatim}
\end{small}

The line \p{onevent prox} means: whenever the event of sampling the
horizontal distance sensors (the \emph{proximity} sensors) occurs (it
occurs 10 times a second), the lines that follow will be run. Values are
read from the sensors and compared with a low value of 400 and a high
value of 500. According to the results of the comparisons, the motor
speeds are changed. For example, if the center sensor 2 senses a value
higher than 500 (meaning that something is near it), the target speeds
of both the left and right motors will be set to 300.

Experiment with moving the sliders on the motor action blocks. You will
see that the target speeds of the motors jump by 50 in the range $-$500
to 500. By moving the sliders carefully, you can set the speeds to any
of these values.
