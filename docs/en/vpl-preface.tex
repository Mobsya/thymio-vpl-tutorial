% !TeX root = vpl.tex

\chapter*{Preface}

\sect{What is a robot?}

You are riding your bicycle and suddenly you see that the street starts
to go uphill. You pedal faster to supply more power to the wheels so
that the bicycle won't slow down. When you reach the top of the hill and
start to go downhill, you squeeze the brake lever. This causes a rubber
pad to be pressed against the wheel and the bicycle slows down.
When you ride a bicycle, your eyes are \textit{sensors} that sense what is
going on in the world. When these sensors---your eyes---detect an
\textit{event} such as a curve in the street, you perform an
\textit{action}, such as moving the handlebar left or right.

In a car, there are sensors that \textit{measure} what is going on in the
world. The speedometer measures how fast the car is going; if you see it
measuring a speed higher than the limit, you might tell the driver that
he is going too fast. In response, he can perform an action, such as
stepping on the brake pedal to slow the car down. The fuel meter
measures how much fuel remains in the car; if you see that it is too
low, you can tell the driver to find a gas station. In response, she can
perform an action: raise the turn-signal lever to indicate a right turn
and turn the steering wheel in order to drive into the station.

The rider of the bicycle and the driver of the car receive data from
the sensors, decide what actions to take and cause the actions to be
performed. A \textit{robot} is a system where this process---receive data,
decide upon an action, perform the action---is carried out by a
computerized system, usually without the participation of a human.

\sect{The Thymio-II robot and the Aseba VPL environment}

The Thymio II is a small robot intended for educational purposes
(\cref{fig.front}). The robot includes sensors that can measure
light, sound and distance, and can detect when buttons are touched and
when the robot's body is tapped. The most important action that it can
perform is to move using two wheels, each powered by its own motor.
Other actions include generating sound and turning lights on and off.

In the rest of this document, the Thymio~II robot will be called simply Thymio.
It will always refer to the version II of the robot.

Aseba is a programming environment for small mobile robots such as the Thymio.
VPL is a component of Aseba for \textit{visual programming} that was designed to program Thymio in an easy way through event and action blocks.
This tutorial assumes that you have Aseba installed on your computer; if it is not the case, go to \url{https://aseba.wikidot.com/en:downloadinstall}, select your operating system, download and install.


