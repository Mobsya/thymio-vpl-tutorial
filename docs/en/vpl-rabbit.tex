% !TeX root = vpl.tex

\chap{Project: The Rabbit and the Fox}\label{ch.rabbit}

This chapter contains the specification of a large project (my
implementation uses 17 event-action pairs). You should have enough
experience designing and implementing VPL programs in order to write it
yourself. We will give the specification of the behavior of the robot as
a list of tasks and suggest that you develop the program by implementing
each task in turn. Check the complete behavior of the robot each time
you add some event-action pairs in order to ensure that that additional
new pairs haven't caused a problem with the existing ones.


\textbf{Story}:\footnote{The story is loosely inspired by a joke
well-known to PhD students.\\ See:
\url{http://www.cs.hmc.edu/~fleck/parable.html}.} The robot is a rabbit,
walking in the forest. A fox chases the the rabbit to catch
it from behind. The rabbit senses the fox, turns around and catches the
fox.


\textbf{Specification}

For each event, we specify a color that should be shown on the top LEDs
when the event occurs.

\begin{enumerate}
\item Touch the forwards button: the robot moves forwards (blue).
\item Touch the backwards button: the robot stops (off).
\item If the robot detects the edge of the table it stops (off).
\item If the left rear sensor detects an object, the robot quickly turns
left (counterclockwise) until the object is detected by the front center
sensor (red).
\item If the right rear sensor detects an object, the robot quickly turns
right (clockwise) until the object is detected by the front center
sensor (green).
\item When the object is detected by the front center sensor, the robot
moves forward quickly for one second and then stops (yellow).
\end{enumerate}

\textbf{Guidance}: You will have to use states from advanced mode to
ensure that the robots stops after moving forward even if the front
center sensor still detects the object.

\bigskip

\textbf{Program file}: \bu{rabbit-fox.aesl}
