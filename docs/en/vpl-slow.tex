\chap{Multiple sensor thresholds}\label{ch.slow}

As explained in \cref{a.tech}, in advanced mode sensors events can be
specified in three different ways: an event occurs when the reflected
light is below a threshold (black), an event occurs when the reflected
light is above a threshold (white), and an event occurs when the
reflected light falls between two thresholds (dark gray):

\begin{center}
\begin{tabular}{ccc}
\blk{slow-low}&\blk{slow-mid}&\blk{slow-high}\\
\end{tabular}
\end{center}

\textbf{Specification}

Construct a program that causes the robot to approach an object,
starting at a high speed, slowing down as it gets closer, and finally
stopping when the robot is very close to the object.

\textbf{Guidance}

\begin{itemize}
\item Use three event-actions pairs, one with each type of sensor
event.

\item Carefully adjust the sliders (see \cref{a.tech}) so that the
high value of one threshold is the same as the low value of the next
threshold. 

\item Add a color block to each pair so that you can see the robot's
speed setting.

\item Use reflector tape to extend the range of
the sensors as explained in \cref{a.blocks}. 
\end{itemize}

\bigskip

{\raggedleft \hfill \textbf{Program file}: \bu{slow.aesl}}
