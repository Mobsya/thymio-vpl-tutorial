\chap{Measuring Speed}\label{ch.speed}

\textbf{Specification}

Measure the speed of the Thymio robot for different settings of the
motors. Place a strip of black tape on a light-colored surface as you
did for the line-following program (\cref{ch.line}). Put the robot just
before the one end of the tape. Implement the following behavior:

\begin{itemize}

\item The robot starts moving forward when the center button is touched.

\item When the start of the tape is detected by the ground sensors,
start a one-second timer.

\item When the timer expires, change the top color and reset the timer
to one second.

\item When the end of the tape is detected, turn the motors off.

\end{itemize} 

Run the program and count the number of times the color changes. This is
the number of seconds that the robot took to move over the tape. Divide
the length of the tape by the number of seconds to get the speed. For
example, if the length of the tape is 30 centimeters and the color
changes 6 times, the speed of the robot is 30/6=5 centimeters per
second.

Experiment with different motor settings and lengths of the tape.

\textbf{Guidance}

Write down a list of colors, say, 1=red, 2=blue, 3=green, 4=yellow, etc.
Use the list to translate a color of the robot to a number of seconds.

Use states to keep track of the current and next colors. For example, in
state 3, the color is green; when the timer expires \emph{and} the state
is 3, change the state to 4, change the color to yellow and reset the
timer. There are three actions for each timer event.

\bigskip

{\raggedleft \hfill \textbf{Program file}: \bu{measure-speed.aesl}}
