\chap{States: Don't Always Do the Same Thing (Advanced)}\label{ch.states}

A program in VPL is a list of event-action pairs. All the events are
checked one after another and the appropriate actions are taken. Then
the check begins again. We want is a way to specify that certain
event-action pairs are active at a certain time, while others are not.
For example, in Chapter~\ref{ch.line}, when the robot ran off the tape,
we wanted it to turn left or right in order to search for the tape with
the direction depending on which side it ran off.

States are supported in the \emph{advanced} mode of VPL. Click on
\blk{advanced} before beginning to work on the following projects.

\sect{Tap, tap}

In many programs, we used one button to start the robot's behavior and
another to stop it. Consider, though, the power switch on my computer
\blksm{power-button}: The same switch is used to turn the light on and
off; the switch remembers whether it is in the state \bu{on} or the
state \bu{off}. The switch includes a small green light which indicates
its current state.


Write a program that turns the robot's lights on when it is tapped and
turns them off when tapped again.

{\raggedleft \hfill Program file \bu{tap-on-off.aesl}}

It is convenient to display the required behavior in a state diagram:

\begin{center}
\begin{picture}(240,45)
%\put(0,0){\framebox(240,40){}}
\put(20,20){\circle{40}}
\put(0,0){\makebox(40,40){\textsf{off}}}
\put(220,20){\circle{40}}
\put(200,0){\makebox(40,40){\textsf{on}}}
\put(40,10){\vector(1,0){160}}
\put(0,10){\makebox(240,10){\textsf{tap $\rightarrow$ turn on}}}
\put(200,30){\vector(-1,0){160}}
\put(0,30){\makebox(240,10){\textsf{tap $\rightarrow$ turn off}}}
\end{picture}
\end{center}

In the diagram there are two states indicated by circles labeled with
the names of the states \bu{off} and \bu{on}. From state \bu{off} the
robot can go to state \bu{on} and back, but only by following the
instructions on the arrows. They are event-action pairs and mean:

\begin{itemize}

\item \emph{When} in state \bu{off} \textbf{\textit{and}} \emph{tap}
occurs $\rightarrow$ turn the lights \emph{on} \textbf{\textit{and}} go
to state \bu{on}.

\item \emph{When} in state \bu{on} \textbf{\textit{and}} \emph{tap}
occurs $\rightarrow$ turn the lights \emph{off} \textbf{\textit{and}} go
to state \bu{off}.

\end{itemize} The event of an event-action pair occurs only if the robot
is in its associated state and the action can include changing the
state. The same event \emph{tap} appears in two event-action pairs, but
only one occurs at any time, depending on the state of the robot.

The program is shown in Figures~\ref{fig.turn-on-off1}
and~\ref{fig.turn-on-off2}. Let us look at the event-action pairs
one-by-one.

\begin{figure}
\begin{center}
\gr{tap-on-off1}{0.4}
\caption{Tap to turn on and off}\label{fig.turn-on-off1}
\end{center}
\end{figure}

In the event-action pair \blk{tap-turn-on}, the event is composed of the
tap block together with an indication of the state.

A state is indicated by four quarters of a circle, each of which can be
either on (orange) or off (white). We will use the left-front quarter to
indicate whether the lights are off or on. In this pair, the left-front
quarter is colored white, meaning that it is off. The meaning of
\blk{tap-turn-on} is: if the robot is tapped and the light is off, turn
it on.

Similarly, the event-action pair \blk{tap-turn-off} means: if the robot
is tapped and the light is on, turn it off. The quarter is colored
orange, meaning that it is on.

If you look again at the description of the behavior of the robot, you
will see that only half the job is done. When turning the light on or
off, we also have to change the state of the robot from \bu{off} to
\bu{on} or from \bu{on} to \bu{off}. For that we write two
additional event-action pairs using the state action block
\blk{action-states} (Figure~\ref{fig.turn-on-off2}).

\begin{figure}
\begin{center}
\gr{tap-on-off2}{0.4}
\caption{Tap to change the state}\label{fig.turn-on-off2}
\end{center}
\end{figure}

The meaning of \blk{tap-state-on} is: \emph{when} the robot is tapped
\emph{and} the state is \bu{off}, change the state to \bu{on}.
Similarly, the meaning of \blk{tap-state-off} is \emph{when} the robot
is tapped \emph{and} the state is \bu{on}, change the state to \bu{off}.

Referring the complete program consisting of the four pairs in
Figures~\ref{fig.turn-on-off1} and~\ref{fig.turn-on-off2}, we see that
each event causes both an action on the light and a change of the state
of the robot. Both the action and the change of state depend on the
current state.

\sect{How many states can the robot be in?}

The state is indicated by a circle divided into four quarters. Each state can be:
\begin{itemize}
\item Gray: the part is not taken into account;
\item White: the quarter is \emph{off};
\item Orange: the quarter is \emph{on}.
\end{itemize}

In \blksm{states}, the left-front and right-rear quarters are on, the
right-front one is off and the left-rear one is not taken into account,
meaning that if \blksm{states} is associated with an event block, the
event will occur if the state is either \blk{states1} or \blk{states2}.

Since each of the four quarters can be either on or off, there are 16
possible states:
\begin{quote}
\bu{(off, off, off, off)\\(off, off, off, on)\\(off, off, on, off)\\
\mbox{}\hspace{3em}\ldots\\
(on, on, on, off)\\
(on, on, on, on)}.
\end{quote}

When the state of the robot is set using the block \blk{action-states},
the state is displayed in the ring of lights on top of the robot.

Figure~\ref{fig.state-leds} shows the robot in the state \bu{(on, on,
on, on)}.

\centeredbox{When a program is run, the initial state is
\bu{(off, off, off, off)}: \blksm{state-all-off}.}

\begin{figure}
\begin{center}
\gr{state-leds}{0.4}
\caption{Light indicating the state}\label{fig.state-leds}
\end{center}
\end{figure}

\sect{Get the mouse}

Write a program that causes the robot to turn from right to left,
searching for a mouse. If it detects a mouse with its leftmost sensor it
continues the search until the mouse is detected by its rightmost
sensor. Then, it positions itself facing the mouse
(Figure~\ref{fig.cat-mouse}).

{\raggedleft \hfill Program file \bu{mouse.aesl}}


\begin{figure}
\begin{center}
\gr{cat-mouse}{0.4}
\caption{The cat has found the mouse}\label{fig.cat-mouse}
\end{center}
\end{figure}

The event-action pair \blk{mouse1} causes the robot to turn left. This
only occurs when the left-front quarter is off, but initially all
quarters of the state are off.

The first event-action pair in Figure~\ref{fig.mouse2} waits until the
mouse is detected at the leftmost sensor. Note that the small square
next to it is colored white so that the event occurs only when the
rightmost sensor alone detects the mouse. The second event-action pair
in the Figure changes the state.

\begin{figure}
\begin{center}
\gr{mouse2}{0.4}
\caption{Searching for the mouse at the rightmost sensor}\label{fig.mouse2}
\end{center}
\end{figure}

The final event-action pair in the program is \blk{mouse3} which halts
the robot when the mouse is directly facing the center
sensor.\footnote{You will have to experiment with the position of the
mouse. If it is too close to the robot, the sensors on either side of
the central sensor will also detect the mouse, while the event requires
that they \emph{not} detect it.}

Why does the event in this pair need to depend on the state? The reason
is that the central sensor will also detect the mouse in its initial
scan from right to left. We want the robot to perform a full scan
before returning to the position of the mouse, so it is necessary that
this first detection of the mouse be ignored. This is achieved by
stopping the scan only when the state is \bu{on} and this is set only
when the full scan has been completed.

\sect{Exercise \thechapter.1}

Write a program that causes the robot to dance: it turns left in
place for two seconds and then turns right in place for three seconds.
These movements are repeated indefinitely.

\sect{Exercise \thechapter.2 (Difficult)}

Modify the line-following program from Chapter~\ref{ch.line} so that the
robot turns left when it leaves the right-hand side of the line and
turns right when it leaves the left-hand side of the line.
