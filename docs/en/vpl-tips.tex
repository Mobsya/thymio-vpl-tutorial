% !TeX root = vpl.tex

\appendix

\chap{Tips for Programming with VPL}\label{a.tips}

\sect{Exploring and experimenting}
\
\begin{description}
\item[Understand each event and action block] For each event block and each action block,
spend some time experimenting until you understand exactly how it works.
To explore the behavior of an action block, construct a pair whose
event is touching a button. This is a simple and robust event
that will let you concentrate on learning what the action block does.
Similarly, to explore the behavior of an event block, construct a pair whose
action is changing the top colour of the robot. This is a simple and robust action
that will let you concentrate on learning what the even block does.

\item[Experiment with the sensor event blocks]
The small red lights next to each sensor which show when
that sensor detects an obstacle.
Move your fingers in front of the sensors and see which lights
are turned on. 
Now, construct an event-action pair consisting of the sensor event
and the top colour action. Experiment with the settings of the
small squares in the event block
(gray, red, white) until you understand how they work.

Experiment separately with the front/back sensors and the ground sensors
because the meaning of an event occurrence is slightly different in each case,
as explained in the box on page~\pageref{page.sensors}.

\item[Experiment with the motors]
Experiment with the settings of the motors
so that you get a general impression how fast the robot moves
for a particular setting.
It is very important to experiment with different settings
for the two motors in order to learn how the robot turns.
\end{description}


\sect{Constructing a program}

\begin{description}
\item[Plan your program]
Constructing a program is not the same as experimenting with blocks
and event-action pairs.
Write down a description of how the program is supposed to work:
a sentence for each event-action pair.

\item[Construct one event-action pair at a time]
When you understand how each
event-action pair works, you can put them together in a program.

\item[Test each addition to the program]
Test your program each time you add a new event-action pair.
That way, you will know that any problem
is due to the interaction of the new pair and the previous pairs.

\item[Use ``Save as'' for each modification]
Before you change your program, \emph{always} use the icon
\blksm{saveas} to save the program under a different name.
If the change makes things worse,
it is easy to go back to the previous version.

\item[Display what happens]
Use colours or sounds to display what the program is doing.

For example, if one sensor causes the robot to turn left and another to
turn right, use the same events in additional event-action pairs
to display different top colours.
You will be able to see if a problem is caused by a sensor or
if the motors are not responding correctly to a sensor event.

\end{description}


\sect{Troubleshooting}

\begin{description}

\item[Use a smooth surface]
Make sure that the surface on which the robot moves---a table or the floor---is
very clean and smooth. If there is a lot of friction from dirt,
the motors may not be able to move the robot, or turns may be uneven.

\item[Use a long cable]
Make sure that your cable is long enough.
If the robot moves too far, the cable can cause the robot to slow down
or stop, or the robot can pull the cable out of its connector.

\item[Sensor events may not be detected]
Sensor events happen 10 times per second.
If the robot is moving very fast,
an event might not be detected.

For example, if the robot is supposed to detect the edge of the table and stop,
and if the robot is moving very fast, it might be unstable and fall off the table
before the sensor event can stop the motor.
When you start to run a program, run it at a low speed
and gradually increase the speed.
The speed must be fast enough to overcome friction, but not so fast so that
events are missed.

For another example, consider a line-following program in Chapter~\ref{ch.line}.
Its algorithm depends on being able to detect when one
ground sensor detects the line and the other doesn't.
If the robot is moving too fast, \emph{both} sensors
might move from a position where the both sensors detect the line to
to a position where neither sensor detects the line,
and the intermediate position where only one sensor
detects the line is not detected.

\item[Event-action pairs are run sequentially]
In theory, the event-action pairs are run \emph{concurrently}---at the same
time; in practice, they are run one after the other in the order they appear
in the program. As shown in Exercise~4.2, this can cause a problem, because
the second action might conflict with what was done by the first action.
Try to avoid constructing programs where the \emph{same} action block
appears in pairs with two different event blocks that can occur at the same time.

\item[Problems with the clap event]
Do \emph{not} use the clap event \blksm{event-clap}
when the motors are running. The motors make a lot of noise and
can cause unexpected clap events.

Similarly, do \emph{not} use the tap event \blksm{event-tap}
in the same program with the clap event. Tapping on the robot causes
noise that can be unexpectedly interpreted as a clap.


\end{description}