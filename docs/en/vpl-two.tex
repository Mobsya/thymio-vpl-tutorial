\chap{Multiple Thymios}\label{ch.two}

You can run two or more Thymio robots at the same time.

\textbf{Specification}

Place two robots T1 and T2 facing each other. T1 chases T2; when
T1 detects that it is close to T2 it will stop. If T2 detects that T1 is
close to it, T2 retreats until it no longer detects T1. 

\textbf{Guidance}

\begin{itemize}

\item The programs for T1 and T2 have two event-action pairs: one whose
event is the detection of an object by the center horizontal sensor, and
another whose event is the non-detection of an object. However, the
actions for T1 and T2 are different.

\item Connect two Thymios T1 and T2 to the computer and turn them on.
Run two copies of VPL. In the target-selection window
(Figure~\ref{fig.connect}), both T1 and T2 should appear; select T1 in
one copy of VPL and T2 in the other. Open and run program \p{chase} in
T1 and program \p{retreat} in T2.
 
\end{itemize}

\textbf{Experiments}

\begin{itemize}

\item What happens if you exchange the programs: T1 runs
\p{retreat} and T2 runs \p{chase}? Explain.

\item In advanced mode, experiment with different settings of the sensor
thresholds.

\end{itemize}

%\bigskip

{\raggedleft \hfill Program file \bu{chase.aesl}, \bu{retreat.aesl}}

\bigskip

\informationbox{Communications between robots}{Multiple Thymio robots
can send messages to each other. This capability is supported in the
AESL language and Studio environment.}
