\chap{Les accéléromètres (Mode avancé)}
\label{ch.angles}

Tout le monde est familier avec le concept d'\emph{accélération}, par exemple lorsqu'une voiture met les gaz ou freine.
Un \emph{accéléromètre} est un outil qui permet de mesurer
l'accélération.
L'airbag dans les voitures utilise un accéléromètre pour
détecter un ralentissement trop rapide à cause d'un accident,
ce qui permet de déclancher l'airbag.

Thymio a trois accéléromètres, un pour chacune des directions:
avant/arrière, gauche/droite, haut/bas.

Il est difficle d'obtenir des acélérations assez importantes
pour être mesurées, sauf dans le cas de la \emph{gravité}, l'accélération qui nous attire vers le centre de la Terre.
Dans ce projet, nous allons mesurer l'angle d'inclinaison
du Thymio.

Il y a deux événements qui permettent de détecter l'angle
du robot par rapport à la Terre: \label{p.accel}

\begin{itemize}

\item \blksm{event-roll}: un événement est déclanché lorsque l'angle gauche/droite du robot est à l'intérieur de l'angle blanc
du demi-cercle.

\item \blksm{event-pitch}: un événement est déclanché 
lorsque l'angle avant/arrière est à l'intérieur
de l'angle blanc du demi-cercle.

\end{itemize}

Au début, l'angle blanc de ces blocs est placé verticalement
au-dessus du Thymio, de sorte qu'un événement se produit
lorsque le Thymio est à plat, sur le sol ou sur une table.
En glissant cet angle avec la souris, on peut choisir 
d'autres angles; par exemple, le bloc suivant
déclanchera un événement lorsque le robot est penché vers la
gauche, plus ou moins à moitié entre la verticale et
l'horizontale:
\blkc{roll-left}

\begin{quote}
\textbf{Programme}\\
Tenez le robot face à vous et penchez le vers la gauche
et vers la droite.
La lumière du dessus du robot doit changer de couleur
entre chaque intervalle possible de l'angle blanc.
\end{quote}

{\raggedleft \hfill \textbf{Programme}: \bu{measure-angle.aesl}}

Construisez un ensemble de paires événement-actions
dans lesquels chaque événement est un événement accéléromètre
et l'action associée change la couleur du haut du robot:
\blkc{measure-angles}
Créez une liste qui donne les équivalences entre angle
et couleur pour que vous puissiez traduire chaque couleur
en son angle correspondant.

Les quartiers du bloc événement état sont gris pour que 
l'événement déclanche l'action quelque soit l'état.

\bigskip

\exercisebox{\thechapter.1}{Est-il possible que deux
événements utilisent le même angle blanc?\\
Combien d'événements avec des angles différents pouvez-vous 
construire?}
