\chap{Lire des codes-barres}\label{ch.barcode}

Les codes-barres sont largement utilisés dans les supermarchés et ailleurs pour identifier
des objets. L'identifiant est un nombre ou une suite de symboles unique à chaque type d'objet.
Cet identifiant est utilisé pour retrouver dans la base de données des informations sur
cet objet comme le prix par exemple.
Construisons maintenant un lecteur de code-barres grâce au Thymio.

\textbf{Instructions}

\begin{enumerate}
\item Mesurez soigneusement la distance entre deux capteurs frontaux ainsi que la largeur d'un de ces
    capteurs.
    Utilisez une feuille de carton pliable, du ruban adhésif noir et des bandes de papier d'aluminium
    pour créer un bricollage qui contient différentes combinaisons de bandes noires ou blanches.

\begin{center}
\gr{barcode}{.6}
\end{center}

\item Chacune des configurations pour les trois capteurs horizontaux centraux
    représentent un code différent. (Combien de codes différents peut-on créer?)
    Implémentez des paires événement-actions pour certains voire tous ces différents codes
    et créez différentes couleurs sur le haut du Thymio selon quel code est identifié.

\end{enumerate}

\textbf{Conseils}:

Seuls les trois capteurs centraux seront utilisés; mettez les carrés des capteurs horizontaux
en mode gris.
Les carrés des capteurs centraux doivent être blancs quand l'on souhaite détecter une bande d'aluminium blanche
et ils doivent être noirs quand l'on souhaite détecter le fond noir du ruban adhésif.
Ainsi, la paire événement-actions suivante affiche du jaune pour le code \texttt{allumé-éteint-allumé}:

\blkc{barcode1-3}

La solution dans l'archive identifie tous les codes avec deux bandes d'aluminium ainsi que 
le code sans aucune barre d'aluminium.

{\raggedleft \hfill Programme: \bu{barcode.aesl}}
