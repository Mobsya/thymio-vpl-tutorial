\chap{Une petite pause}\label{ch.bells}

Faisons une pause avec les tâches compliquées et jouons un peu avec Thymio. Dans ce chapitre, nous vous montrerons que Thymio peut jouer de la musique, répondre à un son et répondre lorsqu'on lui donne une tape amicale!

\sect{Thymio mélomane}

Thymio possède un synthétiseur de son, il permet de jouer des notes de musique! Vous pouvez le programmer simplement en utilisant le bloc action \blk{action-music}.

{\raggedleft \hfill Programme \bu{bells.aesl}}

Thymio peut jouer six notes différentes, chaque note peut-être placée sur une barre de couleur représentant un ton. Il suffit de cliquer avec votre souris sur la barre de couleur de votre choix au niveau de la note que vous souhaitez modifier et voilà! Ensuite, vous pouvez choisir entre jouer une noir, un temps, et une blanche, deux temps, en cliquant sur la note que vous souhaitez modifier.

\Cref{fig.music} illustre deux chansons jouées par Thymio lorsque vous appuyez sur le bouton avant ou arrière.

\begin{figure}
\begin{center}
\gr{music}{.4}
\caption{Jouer une chanson}\label{fig.music}
\end{center}
\end{figure}

\begin{bclogo}[couleur = pink!30, arrondi = 0.1, logo = \bccrayon, ombre = true]{Challenge!}Écrivez un programme en utilisant VPL qui joue un thème connu!
\end{bclogo}

\sect{Thymio! Par ici!}

Thymio a un microphone, il peut donc réagir à un son! L'événement \blk{event-clap} se déclenche si Thymio entend un son fort, comme quelqu'un qui tape dans ses mains. 

En utilisant une paire événement-action avec l'événement \blk{event-clap} et le bloc d'action moteur, vous pouvez sans problème réaliser un programme qui fait avancer Thymio lorsque vous tapez dans vos mains.

\begin{bclogo}[couleur = blue!30, arrondi = 0.1, logo = \bcinfo, ombre = true]{Truc et astuce!}Si vous vous trouvez dans un environnement bruyant, il sera difficile pour Thymio de vous entendre. Essayez cette fonctionnalité dans un environnement calme.
\end{bclogo}

\sect{C'est bien Thymio!}

Il est important de récompenser votre animal de compagnie quand il est gentil, c'est pareil avec Thymio! Il peut vous détecter si vous lui donner une petite tape sur la tête grâce à  l'événement: \blk{event-tap}. 

\begin{bclogo}[couleur = pink!30, arrondi = 0.1, logo = \bccrayon, ombre = true]{Challenge!}Écrivez un programme en utilisant VPL qui fait approcher Thymio quand vous taper dans vos mains et qui s'arrête et devient rose si vous lui donner une petite tape sur la tête. Avec un seul microphone, Thymio n'arrivera pas à repérer l'origine du son, alignez-le donc dans votre direction.
\end{bclogo}

{\raggedleft \hfill Programme \bu{aux-pieds.aesl}}