\chap{Changer les couleurs}\label{ch.colors}

\sect{Colorer Thymio}

Créons un programme qui affiche deux couleurs différentes sur le dessus de Thymio lorsque les boutons avant ou arrière sont touchés, et deux autres couleurs sur le dessus et les côtés de Thymio lorsque les boutons gauche ou droite sont touchés.

{\raggedleft \hfill Programme: \bu{colors.aesl}}

Nous avons besoin de quatre paires événement-actions.
Il y a quatre événements --- toucher l'un des quatre boutons --- et une action couleur est associée avec chaque événement.
Notez la différence entre le bloc \blksm{action-colors-up} et le bloc \blksm{action-colors-down}.
Le premier des deux change la couleur affichée sur le dessus de Thymio alors que le deuxième change la couleur affichée sur le dessous et sur les côtés.
Le deuxième bloc a deux marques noires qui représentent les roues du robot et un point blanc qui représente le point d'appui à l'avant du robot (\cref{fig.bottom}).

Ce programme est illustré sur la \cref{fig.colors}.

\begin{figure}
\gr{colors1}{.35}
\caption{Changer de couleur quand un bouton est touché}\label{fig.colors}
\end{figure}

Quelles couleurs seront affichées?
Pour les premières trois actions, le \textit{slider} d'une couleur a été glissé tout à droite alors que les autres sont restés à gauche.
Ces actions affichent donc respectivement purement du rouge, du bleu et du vert.
L'action associée avec le bouton gauche mixe du rouge et du vert, ce qui produit donc du jaune.
Vous pouvez voir que le fond du bloc change de couleur quand on déplace les \textit{sliders}, ce qui vous montre de quelle couleur sera Thymio!

Lancez le programme \blksm{run} et touchez les boutons pour changer la couleur du robot.
La \cref{fig.front} montre Thymio allumé en rouge sur le dessus et la \cref{fig.bottom} montre Thymio allumé en vert sur le dessous.

\exercisebox{\thechapter.1}{
Expérimentez avec les \textit{sliders} pour voir quelles couleurs peuvent être affichées.
}

\trickbox[Information]{
En mixant ensemble du rouge, du vert et du bleu, vous pouvez créer n'importe quelle couleur (\cref{fig.cube}) !
}

\begin{figure}
\gr{color-cubes}{.85}
\caption{Le cube des couleurs rouge-vert-bleu (RVB)}\label{fig.cube}
\end{figure}

%\sect{Éteindre les lumières}
\sect{Plusieurs actions associées à un même événement}

Modifions maintenant le programme pour que les lumières s'éteignent lorsque le bouton central est touché.
Deux actions doivent avoir lieu lors d'un même événement --- le bouton du milieu est touché.
Il est possible d'associer \emph{deux} actions à un événement
dans une paire événement-actions.
Après avoir inséré l'événement et la première action (l'action couleur du haut), un nouveau cadre gris apparaîtra à droite de l'action:

\blkc{multiple-outline}

Vous pouvez maintenant glisser-déposer l'action couleur du bas 
dans ce cadre.
Vous obtiendrez une paire avec un événement et deux actions:\label{p.mulitple}

\blkc{colors-multiple}

{\raggedleft \hfill Programme \bu{colors-multiple.aesl}}

%\importantbox[De multiples paires événement-actions]{
N'oubliez pas de lancer le programmer en cliquant sur l'icône \blksmpure{run}.
À l'avenir, il sera implicite qu'il vous faudra lancer chaque programme en cliquant sur cet icône, nous ne vous le dirons plus.

\bigskip

\importantbox[Les règles des paires événement-actions]{
\begin{itemize}[noitemsep,nosep,leftmargin=*]
\item Lorsqu'un programme est lancé, toutes les paires événement-actions sont actives.
\item Il est possible d'avoir plusieurs paires événement-actions avec le même événement mais il faut que les paramètres associés soient différents.
Vous pouvez par exemple avoir plusieurs paires avec l'événement bouton à condition que les combinaisons de boutons soient différentes entre les différents blocs événement.
\item Si les événements sont absoluement identiques dans plusieurs paires, VPL vous indiquera qu'il y a une erreur (zone 3 dans la \cref{fig.vplgui}).
Vous ne pourrez pas lancer le programme tant qu'il y a des erreurs.
\end{itemize}
}
