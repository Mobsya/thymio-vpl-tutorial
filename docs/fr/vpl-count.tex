% !TeX root = vpl.tex

\chap{Thymio apprend à compter (avancé)}\label{ch.counting}

Dans ce chapitre, nous allons montrer comment les états de Thymio peuvent être utilisés pour compter et même pour faire un peu d'arithmétique. 

L'implémentation détaillée des projets ne sera pas donné ici. Nous pensons que vous avez désormais assez d'expérience pour les développer par vous-mêmes. Les codes source des différents programmes de ce chapitre se trouvent dans l'archive, essayez de ne pas les regarder tout de suite mais plutôt de faire un maximum de chemin de votre côté.

Les projets suivants utilisent l'événement \blksm{event-clap} pour changer l'état de Thymio et il est représenté par le cercle de LEDs.

\importantbox{L'état du robot est affiché grâce au cercle de LEDs en dessus du robot.
La \cref{fig.state-leds} montre Thymio dans l'état \bu{(on, on, on, on)}.}

Sentez-vous libre de changer n'importe lequel de ces comportements.

\sect{Pair et impair}

\begin{quote}
\textbf{Programme}\\Choisissez l'un des quartiers du cercle de LEDs. Il sera \bu{éteint} (blanc) si vous tapez dans vos mains un nombre pair de fois et \bu{allumé} (orange) si vous tapez dans vos mains un nombre impair de fois. Presser le bouton central de Thymio le fera retourner au mode pair (puisque zéro est un nombre pair).
\end{quote}

{\raggedleft \hfill Programme \bu{count-to-two.aesl}}

La méthode de comptage illustrée ici montre le concept de \emph{modulo 2}. Nous comptons de 0 à 1 puis de nouveau à 0. Le terme \emph{modulo} est proche de \emph{retenue}: si vous tapez dans vos mains 7 fois, et que nous divisons 7 par 2, nous obtenons 3 avec une retenue de 1. En \emph{modulo 2}, nous ne gardons que la retenue de 1.

Dans cette arithmétique, 0 et 1 sont souvent appelés \emph{pair et impair}. Nous appelons aussi ce concept \emph{l'arithmétique cyclique}. Au lieu de compter de 0 à 1, puis de 1 à 2, nous \emph{retournons} à 0: 0, 1, 0, 1 \ldots

Ce concept est en fait très familier puisqu'il est utilisé dans les horloges. Les secondes et les minutes sont comptées en \emph{modulo 60} et les heures en \emph{modulo 12 ou 24}. Donc, la seconde après 59 n'est pas 60 mais 0. Dans la même logique, l'heure après 23 n'est pas 24 mais 0. S'il est 23:00 et que nous nous mettons d'accord pour nous rencontrer 3 heures après, l'heure du rendez-vous est 26 modulo 24, donc 2 heures du matin.

\sect{Compter en unaire}

\begin{quote}
Modifiez le programme pour compter en modulo 4. Il y a quatre états possibles: 0, 1, 2, 3. Choisissez trois LEDs du cercle, elles représenteront les états 1, 2 et 3; l'état 0 sera représenté en éteignant toutes les LEDs.
\end{quote}

Cette méthode de représentation de nombre est appelée \emph{unaire} car différents éléments d'un état représentent des nombres différents. Nous utilisons souvent cette représentation pour compter des points par exemple:
\begin{picture}(35,10)
\multiput(5,0)(5,0){4}{\put(0,0){\line(0,1){10}}}
\put(0,0){\line(3,1){25}}
\put(32,0){\line(0,1){10}}
\end{picture}
représente 6.

{\raggedleft \hfill Programme \bu{count-to-four.aesl}}

\exercisebox{\thechapter.1}{Jusqu'à combien pouvons-nous compter sur le Thymio en utilisant la représentation unaire?}

\sect{Compter en binaire}

Nous sommes familiarisés avec \emph{les représentations en base}, en particulier avec la base 10 (décimal). Les symboles 256 en base 10 ne représentent pas trois objets différents mais bien un seul. Le 6 représente le nombre de 1, le 5 représente le nombre de 10, le 2 représente le nombre de 10 x 10 (donc de 100). Si nous additionnons ces nombres nous obtenons deux cents cinquante-six. Grâce à la base 10, nous pouvons représenter facilement de grands nombres. De plus, l'arithmétique que nous avons apprise à l'école nous aide à manipuler de grands chiffres.

Nous utilisons la base 10 parce que nous avons 10 doigts, ce qui facilite l'apprentissage de cette base. Les ordinateurs n'ont que deux ''doigts'' (\bu{off} et \bu{on}), c'est donc cette base qui est utilisée par les ordinateurs. Au début, la base 2 peut paraître étrange mais il est assez facile de s'y faire. Nous utilisons les même symboles qu'en base 10, le 0 et le 1. Au lieu de passer de 1 à 2, nous retournons à 0. 

Voici comment un ordinateur compte jusqu'à dix:

\begin{displaymath}
0000, 0001, 0010, 0011, 0100, 0101, 0110, 0111, 1000, 1001, 1010
\end{displaymath}

Prenons le nombre en base 2 suivant: 1101. Nous calculons sa valeur de droite à gauche, comme avec la base 10. Le chiffre le plus à droite représente le nombre de 1, celui d'à côté le nombre de 2, le suivante le nombre de 2 x 2 (donc de 4) et le dernier (le plus à gauche) le nombre de 2 x 2 x 2 (donc de 8). Comme nous avons le nombre en base 2: 1101, nous avons 8 + 4 + 0 + 1, ce qui fait treize, représenté en base 10 par 13.


\begin{quote}
\textbf{Programme}\\
Modifiez le programme pour que Thymio compte en modulo 4 en utilisant la représentation binaire (base 2).
\end{quote}

{\raggedleft \hfill Programme \bu{count-to-four-binary.aesl}}

Nous n'avons besoin que de deux LEDs du cercle pour représenter les nombres 0 à 3 en base 2. Prenons le quartier en haut à droite pour représenter le nombre de 1 (\bu{off} (blanc) pour aucun et \bu{on} (orange) pour un) et le quartier d'en haut à gauche pour représenter le nombre de 2. Par exemple, \blksm{state-right} représente le chiffre 1 et \blksm{state-left} représente le chiffre 2. Si les deux quartiers sont blanc, l'état représente le chiffre 0 et s'ils sont les deux oranges, le chiffre 3.

Il y a quatre transitions $0\rightarrow 1, 1\rightarrow 2, 2
\rightarrow 3, 3\rightarrow 0$, donc quatre paire événements-action sont nécessaires en addition à une paire événement-action pour remettre le compteur à zéro lorsque l'on appuie sur le bouton central.


\medskip

\trickbox{Les deux quartiers du bas ne sont pas utilisés. Ils sont donc laissés en gris et sont ignorés par le programme.}

\medskip

\exercisebox{\thechapter.2}{Développez le programme pour qu'il puisse compter en modulo 8. Le quartier en bas à gauche représentera le nombre de 4.}

\exercisebox{\thechapter.3}{Jusqu'à combien pouvons-nous compter avec Thymio en utilisant la représentation binaire (base 2)?}

\sect{Additionner et soustraire}

Écrire un programme pour compter jusqu'à 8 est plutôt long puisqu'il faut créer 8 paires événement-action, une pour chaque transition de \emph{n} à \emph{n + 1} (modulo 8). Au lieu de faire ça de cette façon, nous utilisons des méthodes pour additionner les nombres chiffre à chiffre. En base 10, nous faisons ainsi:

\begin{displaymath}
\begin{array}{r}
387\\
+426\\
\rule[1pt]{1.5em}{1pt}\\
813\\
\end{array}
\end{displaymath}

Et en base 2:

\begin{displaymath}
\begin{array}{r}
0011\\
+1011\\
\rule[1pt]{2em}{1pt}\\
1110\\
\end{array}
\end{displaymath}

En base 2, lorsque nous ajoutons 1 à 1, au lieu de 2, nous obtenons 10. Le 0 est écrit dans la même colonne que le 1 et à côté, à gauche, le nouveau 1 est reporté (c'est la retenue). L'exemple du dessus montre l'addition de 3 (0011) avec 11 (1011), ce qui donne 14 (1110).

\begin{quote}
\textbf{Programme}\\
Écrivez un programme qui commence par représenter 0 et qui ajoute 1 à chaque fois que vous tapez dans vos mains. L'addition doit être faite en modulo 16, donc ajouter 1 à 15 retournera à 0.
\end{quote}

Conseils:

\begin{itemize}
\item Le quartier d'en bas à droite sera utilisé pour représenter les 8.
\item Si le quartier d'en haut à droite (représentant le nombre de 1) est éteint (off, blanc) et que Thymio entend un 'clap', changez-le simplement à allumé (on, orange) quel que soit l'état des autres quartiers.
\item Si le quartier d'en haut à droite (représentant le nombre de 1) est allumé (on, orange) et que Thymio entend un 'clap', changez-le à éteint (off, blanc) ensuite, reportez le 1 en retenue.
\item Si tous les quartiers sont allumés (on, orange), la valeur 15 est représentée. Ajouter un à cet état dépasse le modulo et renvoie la valeur à 0. Il vous faut simplement éteindre toutes les LEDs.
\end{itemize}


{\raggedleft \hfill Programme \bu{addition.aesl}}

\exercisebox{\thechapter.4}{Modifiez le programme pour qu'il commence à 15 et soustraie 1 à chaque 'clap'. Lorsqu'il arrivera à 0 et qu'il entendra encore un 'clap', il devra retourner à 15.}


\exercisebox{\thechapter.5}{Coller des petites bandes de scotch noir (ou dessinez simplement) à intervalles réguliers sur une surface blanche. Écrivez ensuite un programme qui fait que Thymio avance et s'arrête lorsqu'il passe sur le quatrième scotch.}

Cet exercice n'est pas facile. Les bandes de scotch doivent être assez longue pour que Thymio les détecte mais pas trop pour ne pas qu'il lance plusieurs événements par bande. Il vous faudra faire des essais avec ça et avec la vitesse de Thymio.

