\chap{Automates finis}\label{ch.fa}

Un \emph{automate fini} (\emph{finite automaton} (FA) en anglais)  est une machine abstraite capable d'exécuter des calculs.
Ces automates sont extrêmement importants dans de nombreux domaines de l'informatique.
\footnote{Une définition formelle des automates finis peut se trouver par exemple dans
J.E. Hopcroft, R. Motwani, J.D. Ullman. \textit{Introduction to Automata Theory, Languages and Computation}, Pearson, 2013.}
Considérons une chaîne finie de caractères $a$ ou $b$:
\begin{quote} aabbbababbaba \end{quote}

L'objectif de ce problème est de lire une chaîne de caractères comme celle-ci,
puis de déterminer si le nombre de $a$ est pair ou impair.
Un FA pour résoudre ce problème contient deux états:
l'état 0 si le nombre de $a$ lus jusqu'ici est pair,
l'état 1 dans le cas contraire.
Ce FA est représenté dans le diagramme suivant.
Il contient deux états, 0 et 1, et des transitions d'un état à l'autre,
notées $a$ et $b$.

\begin{center}
\begin{picture}(175,45)
%\put(0,0){\framebox(175,45){}}
\put(35,25){\circle{30}}
\put(140,25){\circle{30}}
\put(3,25){\vector(1,0){15}}
\put(20,10){\makebox(30,30){0}}
\put(125,10){\makebox(30,30){1}}
\put(50,20){\vector(1,0){75}}
\put(50,10){\makebox(75,10){a}}
\put(125,30){\vector(-1,0){75}}
\put(50,30){\makebox(75,10){a}}
\put(20,10){
   \put(0,0){\oval(20,18)[b]}
    \put(0,0){\oval(20,18)[tl]}
    \put(-1,9){\vector(1,0){1}}
    \put(-20,-5){\makebox(10,10){b}}
}
\put(145,10){
    \put(10,0){\oval(20,18)[b]}
    \put(10,0){\oval(20,18)[tr]}
    \put(11,9){\vector(-1,0){1}}
    \put(20,-5){\makebox(10,10){b}}
}
\end{picture}
\end{center}

Lorsque notre FA dans un certain état lit un caractère de la chaîne,
il change d'état suivant les transistions du schéma.
Si le nombre de $a$ lus jusqu'ici est pair (état 0) et que le FA lit un $a$,
alors l'état devient 1.
Et inversément, si le nombre de $a$ lus jusqu'ici est impair (état 1) et que le FA lit un $a$,
le FA passe à l'état 0.
Si un $b$ est lu par contre, l'état ne change pas puisque le nombre de $a$ n'a pas changé.

Le FA démarre dans l'état 0 puisqu'au début zéro $a$ ont été lus, ce qui est un nombre pair.
L'état initial est indiqué sur le schéma par une petite flèche.

\textbf{Instructions}\footnote{Ces instructions ainsi que la solution ont été inspirés par
le \href{https://www.thymio.org/fr:barcodelightpainting}{projet de \textit{Light Painting}}.}

Imprimez le fichier \p{fa-path-alternate.pdf}.
Il contient l'image suivante:\footnote{Les fichiers tracé se trouvent dans le dossier \bu{image14} de cette archive.}

\gr{fa-path-parity}{.8}

La ligne dégradée permet au robot d'avancer droit et de ne pas sortir du tracé à gauche ou à droite.
La chaîne de caractères est représentée ici par des carrés noirs pour les $a$ et blancs pour les $b$.
Cette image représente la chaîne de caractères $babababab$.

Placez le robot à gauche de l'image, avant le premier carré.
Il doit être tourné vers la droite et le capteur du sol droit doit être au milieu de la ligne dégradée.
Voici le comportement du robot:

\begin{enumerate}

\item Appuyer sur le bouton avant démarre le robot. La lumière du haut doit être éteinte,
l'état est initialisé (voir ci-dessous) et un minuteur est démarré.

\item Appuyer sur le bouton central arrête le robot.

\item Le robot tourne vers la droite.
Il utilise le capteur du sol \emph{droit} pour détecter si il commence à dévier vers la droite ou
vers la gauche.
Si le robot tourne vers la droite, le capteur détectera un niveau de lumière plus bas (plus noir);
le robot tourne alors vers la gauche.
Si le robot tourne vers la gauche, le capteur détectera plus de lumière (plus blanc);
le robot tourne alors vers la droite.

\item Lorsque le minuteur est écoulé, la valeur du capteur du sol \emph{gauche} est relevée.
Si le robot est alors sur un carré blanc (il lit un $b$), la lumière du haut s'allume en rouge
et le minuteur est relancé.
Si le robot est alors sur un carré noir (il lit un $a$), la lumière du haut s'allume en vert,
le minuteur est relancé et l'état change de pair à impair, ou d'impair à pair.

\end{enumerate}

\textbf{Conseils}

Le Thymio utilisera trois des quatre quartiers dans les événements et actions d'état:

\begin{itemize}
\item Le quartier supérieur gauche indique si le programme a démarré ou non.
\item Le quartier supérieur droit indique si oui ou non il faut relever la couleur (noir ou blanc) 
du carré sur lequel se trouve le robot.
\item Le quartier inférieur gauche retient si le nombre de $a$ lus est pair ou impair.
\end{itemize}

Voici un exemple d'une paire événement-actions.

\blkc{parity-0-1}

\begin{quote}
si le capteur gauche détecte peu de lumière (un carré noir) \emph{et}\\
\hspace*{1em} le programme a déjà démarré \emph{et}
la couleur du carré doit être relevée \emph{et}\\
\hspace*{1em} jusqu'ici le nombre de $a$ lus est pair, alors\\
\hspace*{2em} allumer le haut en rouge\\
\hspace*{2em} changer d'état: le nombre de $a$ lus est impair \emph{et}\\
\hspace*{4em} ne plus relever la couleur du carré\\
\hspace*{2em} relancer le minuteur
\end{quote}

{\raggedleft \hfill Programme: \bu{fa.aesl}}

Vous devrez essayer différentes vitesses de moteur et longueurs du minuteur
pour pouvoir détecter fiablement les carrés noirs et blancs.

\textbf{Exercice}
Imprimez le fichier \p{fa-path-blank.pdf} qui contient une image avec des carrés tous blancs.
Utilisez un marqueur noir pour colorier certains carrés et essayer une nouvelle chaîne de caractères.
Verifiez en particulier que votre programme fonctionne lorsque plusieurs carrés de suite sont noirs,
représentant une chaîne de caractères avec plusieurs $a$ à la suite.

\textbf{Exercice}
Modifiez votre programme pour qu'il détermine le reste ($0$, $1$ ou $2$)
du nombre de $a$ divisé par $3$.

\bigskip

\textbf{Créer ses propres tracés}

\warningbox{Cette section montre comment modifier le tracé (la ligne dégradée et les carrés),
mais pour cela il faut que vous maîtrisiez le langage de mise en page de documents \LaTeX{}.}

Les fichiers \p{\footnotesize fa-path-*.tex} dans l'archive contiennent le code source \LaTeX{}.

Les instructions suivantes créent une ligne dégradée de 2 cm de largeur et de 23 cm de longueur:
\footnote{Ces instructions utilisent la bibliothèque graphique Ti\textit{k}Z.}
\begin{footnotesize}
\begin{verbatim}
\shade[left color=black,right color=white] (0,0) rectangle +(2,23);
\end{verbatim}
\end{footnotesize}

On peut dessiner les carrés noirs et blancs avec les instructions suivantes:
\begin{footnotesize}
\begin{verbatim}
\foreach \a in {1, 3, 5, 7}
  \filldraw[color=black] (\offset,\height*\a) rectangle +(\width,\height);

\foreach \a in {0, 2, 4, 6, 8}
  \draw (\offset,\height*\a) rectangle +(\width,\height);
\end{verbatim}
\end{footnotesize}

Vous pouvez changer la liste de nombres dans les instructions \p{foreach}
pour indiquer quels carrés doivent être blancs et lesquels doivent être noirs.

Les instructions \p{filldraw} et \p{draw} utilisent pour les dimensions des paramètres
qui peuvent être modifiés facilement:
\begin{footnotesize}
\begin{verbatim}
\setlength{\height}{2.4cm}  % Hauteur d'un carré
\setlength{\width}{3cm}     % Largeur d'un carré
\setlength{\offset}{2.3cm}  % Séparation entre carrés et ligne dégradée
\end{verbatim}
\end{footnotesize}

Compilez le fichier avec \p{pdflatex} et imprimez le en utilisant \p{Adobe Reader} ou \p{SumatraPDF}.
L'image est créée en format portrait mais vous pouvez fixer la feuille sur la table dans les deux sens.
On peut accrocher plusieurs pages ensemble sur une table pour représenter des chaînes de caractères plus longues.
