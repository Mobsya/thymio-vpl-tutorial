\part{Annexes}

\chap{L'interface d'utilisateur VPL}\label{a.toolbar}

Tout en haut de la fenêtre VPL se trouve une barre d'outils:

\begin{center}
\gr{toolbar}{1}
\end{center}

\bigskip

\textbf{Nouveau} \blksm{new}: Supprime le programme actuel et vide la zone de programmation.

\bigskip

\textbf{Ouvrir} \blksm{open}: Ouvre un projet existant dans VPL.
Une fenêtre va s'ouvrir pour que vous choisissiez le fichier (extension \p{aesl}) à ouvrir.

\bigskip

\textbf{Sauvegarder} \blksm{save}: Sauvegarde le programme actuel.
Il est conseillé de cliquer souvent sur ce bouton pour ne pas perdre votre travail
si un plantage devait survenir.

\bigskip

\textbf{Sauvegarder sous} \blksm{saveas}: Sauvegarde le programme actuel sous un 
\emph{nom différent}. Utilisez ce bouton lorsque vous avez un programme et que vous
souhaitez pouvoir le modifier sans perdre la version actuelle du programme.

\bigskip

\textbf{Annuler} \blksm{undo}: Annule la dernière action comme par exemple la suppression
d'une paire événement-actions.\label{p.undo}

\bigskip

\textbf{Rétablir} \blksm{redo}: Rétablit la dernière action qui a été annulée.

\bigskip

\textbf{Lancer} \blksm{run}: Lance le programme actuel.
Ce bouton n'est actif que si la compilation a réussi.
Si vous modifiez le programme après l'avoir déjà lancé une fois,
le bouton clignotera pour vous rappeler que vous devez cliquer dessus
pour charger dans le Thymio la version modifiée du programme.

\bigskip

\textbf{Arrêter} \blksm{stop}: Arrête le programme qui est en train d'être exécuté et
arrête les moteurs. Utilisez ce bouton quand votre programme a mis les moteurs du robot
en mouvement mais ne possède pas de paire événement-actions pour les arrêter.

\textbf{Mode avancé} \blksm{advanced}: Le mode avancé offre des fonctionnalités supplémentaires:
les états, les minuteurs, les accéléromètres et la possibilité de changer les seuils des capteurs.

\textbf{Mode débutant}: L'icône ci-dessus devient \blksm{basic} en mode avancé.
Cliquez sur ce bouton pour revenir en mode débutant.

\newpage

\textbf{Aide} \blksm{info1}: Affiche la documentation VPL dans votre navigateur internet.
Une connexion internet est nécessaire. Vous trouverez aussi la documentation sous
\href{https://www.thymio.org/en:thymiovpl}{https://www.thymio.org/en:thymiovpl}.

\bigskip

\textbf{Exporter} \blksm{export}: \label{p.export} Exporte une image graphique du programme vers
un fichier. Vous pouvez ensuite importer cette image dans un document comme un livre de cours
ou une fiche de travail.
Plusieurs formats sont disponibles. Les \textsc{svg} offrent la meilleure qualité, mais le format
\textsc{png} est plus souvent reconnu.
