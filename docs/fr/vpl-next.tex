\chap{Et après?}\label{c.next}

Ce tutoriel vous a présenté le robot Thymio et l'environement Aseba/VPL.
Cet environnement de programmation par image est très pratique et simple d'emploi mais il a tout de même des limitations.
Il n'est pas conçu pour programmer des comportements complexes.
Pour cela, il vous faudra vous familiariser avec l'environnement d'Aseba Studio, comme illustré sur la \cref{fig.studio}.

\begin{figure}[hbt]
\begin{center}
\gr{studio}{.8}
\caption{L'environnement d'Aseba Studio}\label{fig.studio}
\end{center}
\end{figure}

La programmation en utilisant Aseba Studio est aussi basée sur les concepts d'événements et d'actions.
On retrouve toutes les fonctionnalités que vous avez découvertes en utilisant le VPL.
Mais vous avez beaucoup plus de liberté parce que :
\begin{itemize}
\item Vous pouvez contrôler précisément quand un événement cause une action, en fonction, par exemple, de la quantité de lumière réfléchie mesurée par un capteur de sol ou la distance par un capteur horizontal.
\item Vous pouvez spécifier qu'une seule action consiste en plusieurs opérations différentes : contrôler les moteurs, changer l'état, définir des seuils, allumer ou éteindre les lumières, etc.
\item Vous avez la flexibilité d'un langage de programmation complet avec des variables, des expressions et des structures de contrôle.
\end{itemize}

Aseba Studio vous donne accès aux fonctionnalités de Thymio qui ne sont pas disponibles dans VPL :
\begin{itemize}
\item Vous pouvez contrôler toutes les lumières, pas seulement celles de dessus et de dessous.
\item Vous avez plus de flexibilité dans le contrôle du synthétiseur de son.
\item Vous avez accès à un capteur de température.
\item Vous pouvez voir précisément les valeurs des accéléromètres sur les trois dimensions et plus seulement détecter un choc.
\item Vous pouvez utiliser une télécommande pour contrôler Thymio.
\end{itemize}
Lorsque vous travaillez dans Aseba Studio, vous pouvez ouvrir VPL en cliquant sur le bouton \bu{Lancer VPL} dans l'onglet \emph{Outils} en bas à gauche de la fenêtre.
Vous pouvez importer dans Aseba Studio tous les programmes VPL en ouvrant les fichiers correspondants.

Pour utiliser Aseba Studio, commencer depuis la page \emph{Programmation} du site web du Thymio : \url{https://aseba.wikidot.com/fr:thymioprogram} et suivez les liens correspondants à la \emph{programmation texte}.
Si vous manquez d'inspiration, vous trouverez de nombreux nouveaux défis sur la page des exemples : \url{https://aseba.wikidot.com/fr:thymioexamples}.

\vspace{4em}

\infobox{Nous vous souhaitons beaucoup de plaisir à apprendre !}{Merci d'avoir lu ce tutoriel :-)}{yellow!10}{\bcetoile}