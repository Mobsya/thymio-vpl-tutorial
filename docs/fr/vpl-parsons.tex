\part{Les casse-têtes de Parson}

\chap{Les casse-têtes de Parson}\label{ch.parsons}

\newcommand*{\eblock}{\framebox[40pt]{\rule[-11pt]{0pt}{32pt}}\ }

\sect{Que sont les casse-têtes de Parson?}

Les \emph{casse-têtes de Parson} sont un type d'exercices
qui peuvent aider les étudiants à apprendre à programmer.
\footnote{Parsons, D. and Haden, P. Parson's programming puzzles: A fun and effective learning tool for first programming courses. \textit{Proceedings of the 8th Australian Conference
on Computing Education}, Darlinghurst, Australia, 2006, 157-163.}
Un casse-tête de Parson consiste en la donnée d'un objectif
pour un programme et d'une série d'instructions dans un langage de programmation.
L'exercice consiste alors à placer les instructions
dans le bon ordre pour créer un programme
qui remplit l'objectif donné.
Un casse-tête de Parson peut aussi contenir des \emph{pièges},
comme des instructions incorrectes ou inutiles.
L'avantage de ces casse-têtes est que toutes les instructions
dont on a besoin sont à disposition des étudiants et ont une
syntaxe correcte.

Dans VPL, l'ordre des paires événement-actions ne joue presque
aucun rôle.
C'est pourquoi les casse-têtes ici seront des programmes
avec des paires incomplètes,
où le bloc événement, action ou les deux manquent.
À droite de chaque paire événement-actions,
il y aura deux blocs ou plus; sélectionnez-en un 
et tracez une flèche de ce bloc vers la case vide.

\textbf{Exemple}
Lorsque le bouton avant est touché, la lumière verte est allumée.

\bigskip\bigskip

\begin{center}
\begin{tabular}{l@{\hspace{5em}}lll}
\blk{forward} $\rightarrow$ \eblock  &  \blk{red} & \blk{green}\\
\end{tabular}
\begin{picture}(250,20)
\put(230,60){\line(0,1){20}}
\put(230,80){\line(-1,0){155}}
\put(75,80){\vector(0,-1){20}}
\end{picture}
\end{center}

\vspace*{-8ex}

\sect{Les casse-têtes}


\begin{enumerate}

\item Lorsque le bouton droit est touché,
    la lumière rouge du bas est allumée.

\bigskip

\begin{tabular}{l@{\hspace{5em}}lll}
\blk{right-button} $\rightarrow$ \eblock  &  \blk{red-bottom} & \blk{red}\\
\end{tabular}

\bigskip

\item Lorsque le bouton droit est touché, la lumière rouge du haut est allumée.

\bigskip

\begin{tabular}{l@{\hspace{5em}}lll}
\eblock $\rightarrow$ \blk{red} & \blk{left-button} &
 \blk{right-button}\\
\end{tabular}

\bigskip

\item Lorsque le bouton gauche est touché,
    la lumière verte du bas est allumée.

\bigskip

\begin{tabular}{l@{\hspace{5em}}lllll}
\eblock $\rightarrow$ \eblock  &  \blk{right-button} & \blk{left-button}
 & \blk{green} & \blk{green-bottom}\\
\end{tabular}

\bigskip

\item Lorsque le bouton gauche \textbf{ou} le bouton droit est touché, 
    la lumière verte du haut est allumée.

\bigskip

\begin{tabular}{l@{\hspace{5em}}lll}
\blk{left-button} $\rightarrow$ \eblock  &  \blk{green} &
  \blk{green-bottom}\\
\\
\eblock $\rightarrow$ \blk{green}  &  \blk{right-button} &
 \blk{left-button}\\
\end{tabular}

\bigskip

\item Lorsque \textbf{à la fois} le bouton gauche \textbf{et}
    le bouton droit sont touchés,
    la lumière rouge du haut est allumée.
    Choisissez l'un des deux programmes suivants:

\begin{center}
\begin{tabular}{c@{\hspace{5em}}c@{\hspace{5em}}c}
\blk{left-right-button} $\rightarrow$ \blk{red} & \textbf{ou}&
\blk{left-button} $\rightarrow$ \blk{red}\\
&&\blk{right-button} $\rightarrow$ \blk{red}
\end{tabular}
\end{center}

\vspace{-2ex}

\bigskip

\item Si un objet est détecté par le capteur tout à gauche
    \textbf{seulement}, tournez à gauche.

\bigskip

\begin{tabular}{l@{\hspace{5em}}lllll}
\eblock $\rightarrow$ \blk{left-turn} & \blk{sensor-and-prox} &
\blk{right-prox} & \blk{center-prox} & \blk{left-prox} \\
\end{tabular}

\bigskip

\item Arrêtez le robot lorsqu'il atteint le bord de la table.

\bigskip

\begin{tabular}{l@{\hspace{5em}}llll}
\eblock $\rightarrow$ \blk{action-motors} & \blk{event-prox-ground} &
 \blk{ground2} & \blk{ground1}\\
\end{tabular}

\bigskip

\item Lorsque le robot détecte un mur, la lumière rouge
    du haut est allumée.

\bigskip

\begin{tabular}{l@{\hspace{5em}}lll}
\eblock $\rightarrow$ \blk{red} & \blk{center-prox} & \blk{ground1}\\
\end{tabular}

\bigskip

\item Lorsque le robot atteint un mur, les moteurs s'arrêtent.
\bigskip

\begin{tabular}{l@{\hspace{5em}}llll}
\blk{event-tap} $\rightarrow$ \eblock & \blk{full} & \blk{back-full} & \blk{action-motors}\\
\end{tabular}

\bigskip

\item Le robot tourne à gauche s'il y a un objet devant le
    capteur central.

\bigskip

\begin{tabular}{l@{\hspace{5em}}llll}
\blk{center-prox} $\rightarrow$ \eblock & \blk{left-turn} & \blk{full} & \blk{right-turn}\\
\end{tabular}

\bigskip

\item Le robot tourne à droite s'il n'y a \textbf{pas} d'objet devant le capteur central.

\bigskip

\begin{tabular}{l@{\hspace{5em}}llll}
\eblock $\rightarrow$ \blk{right-turn} & \blk{center-prox} & \blk{no-detect-forward} &
\blk{neither-prox}\\
\end{tabular}

\bigskip

\item Les moteurs sont arrêtés dès que le bouton gauche est touché 
\textbf{ou} si on donne une tape au robot.

\bigskip

\begin{tabular}{l@{\hspace{5em}}lllll}
\eblock $\rightarrow$ \blk{action-motors} & \blk{event-buttons} &
\blk{left-right-button} & \blk{left-button} & \blk{right-button}\\
\\
\eblock $\rightarrow$ \blk{action-motors} & \blk{event-tap} &
\blk{event-clap}
\end{tabular}

\bigskip

\item Lorsque le bouton avant est touché,
    le robot avance pendant trois secondes puis recule.

\bigskip

\begin{tabular}{l@{\hspace{5em}}llll}
\blk{forward} $\rightarrow$ \blk{full}\\
\\
\blk{forward} $\rightarrow$ \eblock & \blk{event-timer} & \blk{three-seconds}\\
\\
\eblock       $\rightarrow$ \blk{back-full} & \blk{event-timer} &  \blk{three-seconds}\\
\end{tabular}

\bigskip

\item Le robot s'approche d'un objet lorsqu'il le détecte
    par son capteur gauche, droite ou central.

\bigskip

\begin{tabular}{l@{\hspace{5em}}llll}
\blk{center-prox} $\rightarrow$ \blk{full}\\
\\
\blk{left-prox} $\rightarrow$ \eblock & \blk{right-turn} & \blk{full} &
 \blk{left-turn} & \blk{action-motors}\\
\\
\eblock       $\rightarrow$ \eblock & \blk{right-turn} & \blk{left-turn} &
 \blk{left-prox} & \blk{right-prox}\\
\end{tabular}

\bigskip

\item Le robot suit une ligne sur le sol.
    Il tourne à gauche s'il ne détecte plus la ligne sous son capteur droit et
    tourne à droite s'il ne détecte plus la ligne sous son capteur gauche.

\bigskip

\begin{tabular}{l@{\hspace{5em}}llll}
\eblock $\rightarrow$ \blk{right-turn} & \blk{bottom-right} & \blk{bottom-left} & \blk{left-prox} & \blk{right-prox}\\
\\
\eblock $\rightarrow$ \eblock & \blk{bottom-right} & \blk{bottom-left} & \blk{right-turn} & \blk{left-turn}\\
\\
\end{tabular}

\item Le robot compte dans l'ordre 0,1,2,3,0,1,2,3, \ldots
    chaque fois qu'il détecte un événement frappe les mains.

\bigskip

\begin{tabular}{l@{\hspace{3em}}llll}

\blk{event-clap} \blk{state-0} $\rightarrow$ \eblock &
\blk{state-0} & \blk{state-1} & \blk{state-2} & \blk{state-3}\\ 
\\
\blk{event-clap} \eblock $\rightarrow$ \blk{state-2} &
\blk{state-event-0} & \blk{state-event-1} & \blk{state-event-2} & \blk{state-event-3}\\
\\
\blk{event-clap} \eblock $\rightarrow$ \blk{state-3} &
\blk{state-event-0} & \blk{state-event-1} & \blk{state-event-2} & \blk{state-event-3}\\
\\
\blk{event-clap} \eblock $\rightarrow$ \eblock &
\blk{state-event-0} & \blk{state-event-3} & \blk{state-0} & \blk{state-3}\\ 
\\
\end{tabular}

\newpage

\item Lorsque le bouton central est touché,
    les arcs de cercles avants droit et gauche
    s'allument et s'éteignent alternativement chaque seconde.

\bigskip

\begin{tabular}{l@{\hspace{3em}}llll}

\blk{center-button} \blk{event-state} $\rightarrow$ \eblock \blk{one-second} &
\blk{action-states} & \blk{state-0} & \blk{state-1} & \blk{state-2}\\ 
\\
\blk{event-timer} \blk{state-1} $\rightarrow$ \blk{state-2} \eblock &
\blk{event-timer} & \blk{action-timer} & \blk{one-second} & \blk{three-seconds}\\ 
\\
\eblock \blk{state-2} $\rightarrow$ \eblock \blk{one-second} &
\blk{event-timer} & \blk{action-timer} & \blk{state-0} & \blk{state-1}\\ 
\\
\end{tabular}

\bigskip

\item La lumière du bas du robot devient verte lorsqu'il
    détecte un objet loin de lui et devient rouge
    lorsqu'il détecte un objet proche de lui.

\bigskip

\begin{tabular}{l@{\hspace{3em}}llll}

\eblock \blk{event-state} $\rightarrow$ \blk{bottom-green} &
\blk{far} & \blk{close} & \blk{far-no} & \blk{close-no}\\ 
\\

\eblock \blk{event-state} $\rightarrow$ \blk{red} &
\blk{far} & \blk{close} & \blk{far-no} & \blk{close-no}\\ 
\\
\end{tabular}

\bigskip


\item Penchez le robot vers la gauche; la lumière du haut
    devient bleue et la lumière du bas s'éteint.
    Penchez le robot sur son dos; la lumière du haut s'éteint
    et la lumière du bas devient jaune.

\bigskip

\begin{tabular}{l@{\hspace{3em}}llll}

\eblock \blk{event-state} $\rightarrow$ \blk{blue} \blk{action-colors-down} &
\blk{tilt-left} & \blk{tilt-right} & \blk{tilt-front} & \blk{tilt-back}\\ 
\\

\eblock \blk{event-state} $\rightarrow$ \blk{action-colors-up} \blk{yellow-bottom} &
\blk{tilt-left} & \blk{tilt-right} & \blk{tilt-front} & \blk{tilt-back}\\ 
\\
\end{tabular}

\end{enumerate}
