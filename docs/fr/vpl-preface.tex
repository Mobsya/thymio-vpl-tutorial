\thispagestyle{empty}
\chapter*{Preface}

\sect{Qu'est-ce qu'un robot?}

Imaginez-vous sur votre vélo, pédalant sur la route, lorsque vous vous trouvez face à une colline. Vous décidez donc de pédaler plus vite pour ne pas perdre trop de vitesse. Une fois en haut, vous vous trouvez face à une descente plutôt raide. Vous allez donc freiner pour éviter de prendre trop de vitesse et de perdre le contrôle de votre vélo. Lorsque vous êtes sur votre vélo, vos yeux sont vos plus importants \textit{capteurs}. Ils vous permettent de vous rendre compte de ce qui se passe autour de vous et de prendre des décisions adéquates à chaque situation, comme pédaler plus vite en montée ou freiner en descente.

Dans une voiture, nous pouvons trouver de nombreux \textit{capteurs}. Le \textit{tachymètre}, ce cadran à aiguille ou digital derrière le volant, vous permet de savoir à quelle vitesse votre voiture avance. Si vous remarquez que vous allez trop vite par rapport à la limitation, vous allez freiner et, au contraire, vous accélérerez si vous vous trouvez en dessous de la limite. Un autre cadran vous indique la quantité d'essence se trouvant dans votre réservoir. Si vous voyez qu'il n'y en a presque plus, vous déciderez d'entreprendre l'\textit{action} d'aller remplir votre réservoir.
 
Autant sur votre vélo que dans votre voiture, vous recevez des informations, des \textit{données}, grâce à certains \textit{capteurs} (vos yeux, votre jauge à essence\ldots) et vous décidez d'entreprendre des \textit{actions} vis-à-vis de ces données. Un robot est un système dans lequel le processus de \textit{Réception des données $\rightarrow$ Décision d'entreprendre une action $\rightarrow$ Réalisation de l'action} est entrepris par un système informatique, en général sans la participation d'un être humain.

\sect{Le robot Thymio II et l'environnement Aseba VPL}

Thymio II est un petit robot conçu à but éducatif. Il comprend différents capteurs qui peuvent mesurer la lumière, les distances, la température\ldots  Il possède également des boutons tactiles, il peut se rendre compte qu'on lui donne une petite tape et peut entendre quelqu'un qui frappe dans ses mains. Il possède deux roues, chacune reliée à un moteur, lui permettant de se déplacer sur une table, ou sur le sol. Il peut encore s'illuminer de toutes les couleurs et jouer de la musique!

Aseba Studio est un environnement de programmation pour Thymio, entre autre. Le VPL est une facette d'Aseba qui permet à toutes et tous de programmer \textit{visuellement} Thymio en utilisant des blocs d'événement et d'action très simples d'emploi.

Des informations plus détaillée et de nombreux exemples d'utilisation de Thymio se trouvent sur \url{https://aseba.wikidot.com/}.

\tableofcontents