\thispagestyle{empty}
\chapter*{Préface}

\sect{Qu'est-ce qu'un robot?}

Imaginez-vous sur votre vélo, pédalant sur la route, lorsque vous vous trouvez face à une colline. Vous décidez donc de pédaler plus vite pour ne pas perdre trop de vitesse. Une fois en haut, vous vous trouvez face à une descente plutôt raide. Vous allez donc freiner pour éviter de prendre trop de vitesse et de perdre le contrôle de votre vélo.
Lorsque vous êtes sur votre vélo, vos yeux sont vos \textit{capteurs} qui mesure ce qu'il se passe dans le monde.
Lorsque ces capteurs --- vos yeux -- détectent un \textit{événement} tel qu'une courbe de la route, vous effectuez une \textit{action}, telle que bouger le guidon à gauche ou à droite.

Dans une voiture, nous pouvons trouver de nombreux \textit{capteurs}. Le tachymètre, ce cadran à aiguille ou digital derrière le volant, vous permet de savoir à quelle vitesse votre voiture avance. Si vous remarquez que vous allez trop vite par rapport à la limitation, vous allez freiner et, au contraire, vous accélérerez si vous vous trouvez en dessous de la limite. Un autre cadran vous indique la quantité d'essence se trouvant dans votre réservoir. Si vous voyez qu'il n'y en a presque plus, vous déciderez d'entreprendre l'action d'aller remplir votre réservoir.
 
Autant sur votre vélo que dans votre voiture, vous recevez des informations, des données, depuis les capteurs et vous décidez d'entreprendre des actions vis-à-vis de ces données. Un \textit{robot} est un système dans lequel le processus --- recevoir des données, décider d'une action, réaliser l'action -- est effectué par un système informatique, en général sans la participation d'un être humain.

\sect{Le robot Thymio II et l'environnement Aseba VPL}

Thymio II est un petit robot conçu à but éducatif (\cref{fig.front}).
Il comprend différents capteurs qui peuvent mesurer la lumière, les distances, la température, etc.
Il possède également des boutons tactiles, il peut se rendre compte qu'on lui donne une petite tape et peut entendre quelqu'un qui frappe dans ses mains.
Il possède deux roues, chacune reliée à un moteur, lui permettant de se déplacer sur une table, ou sur le sol.
Il peut encore s'illuminer de toutes les couleurs et jouer de la musique !
Dans le reste de ce document, le robot Thymio~II sera souvent nommé simplement Thymio.
Il s'agira toujours de la version II du robot.

Aseba est un environnement de programmation pour petits robots comme le Thymio.
VPL est un composant d'Aseba qui permet à toutes et tous de programmer \textit{graphiquement} Thymio en utilisant des blocs d'événement et d'action très simples d'emploi.
Ce tutoriel suppose qu'Aseba est installé sur votre ordinateur ; si ce n'est pas le cas, allez sur \url{https://aseba.wikidot.com/fr:downloadinstall}, sélectionnez votre système d'exploitation, téléchargez et installez.
