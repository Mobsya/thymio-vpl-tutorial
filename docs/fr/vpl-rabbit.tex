\chap{Le lièvre et le renard}\label{ch.rabbit}

Ce chapitre donne les instructions pour un gros projet (mon programme
contient 7 paires événement-actions, chacune d'elles avec 2--3
actions).
Vous devriez entre temps avoir acquis assez d'expérience dans VPL
pour réaliser ce projet vous-même. Nous vous donnerons toutes les
caractéristiques du comportement à développer sous la forme d'une
liste d'instructions. Nous vous suggérons de construire le programme en
implémentant chaque contrainte une à une.

\textbf{Le problème}\footnote{Cette histoire est inspirée d'une \href{
http://www.cs.hmc.edu/~fleck/parable.html}{blague (en anglais)} bien connue
des doctorants}
Le robot est un lièvre qui marche dans la forêt.
Un renard le chasse et essaie de l'attraper depuis derrière.
Le lièvre remarque le renard, se retourne et attrape le renard.


\textbf{Instructions}

Pour chaque événement, nous donnons la couleur qui doit apparaître
sur le haut du robot lorsque l'événement a lieu.

\begin{enumerate}
\item Le bouton avant est touché: le robot avance (bleu).
\item Le bouton arrière est touché: le robot s'arrête (éteint).
\item Si le robot détecte le bord d'une table, il s'arrête (éteint).
\item Si le capteur arrière gauche détecte un objet, le robot tourne rapidement
vers la gauche (dans le sens contraire des aiguilles d'une montre) jusqu'à ce que
l'objet soit détecté par le capteur avant central (rouge).
\item Si le capteur arrière droit détecte un objet, le robot tourne rapidement
vers la droite (dans le sens des aiguilles d'une montre)
jusqu'à ce que l'objet soit détecté par le capteur avant central (vert).
\item Lorsque l'objet est détecté par le capteur avant central, le
robot avance rapidement pendant une seconde (jaune) et s'arrête (éteint).
\end{enumerate}

{\raggedleft \hfill Programme: \bu{rabbit-fox.aesl}}
