\chap{Attrapez les chauffards}\label{ch.radar}

\textbf{Instructions}

Aidez la police à arrêter les conducteurs qui roulent trop vite.
Calculez la vitesse en mesurant la vitesse que la voiture parcoure en un temps donné.

Le robot détecte les objets qui se déplacent de sa gauche vers sa droite grâce à ces capteurs
avants.
Allumez une couleur différente selon la distance qu'a parcourue un objet pendant la seconde
qui a suivi la détection de l'objet par le capteur tout à gauche.

\textbf{Conseils}

\begin{itemize}

\item Dans l'état initial, la détection d'un objet par le capteur avant tout à gauche lance un
    minuteur d'une seconde.

\item Quand le minuteur a exprié, passez à un nouvel état que l'on va appeler l'état \emph{mesurer}.

\item Construisez quatre paires événement-actions, une pour chacun des quatre autres capteurs avants;
    ces événements n'auront lieu que lorsque le robot sera dans le mode \emph{mesurer}.
    Lorsque l'un des capteurs détecte un objet, il affiche en haut une couleur associée à ce capteur.

\item Assurez-vous qu'une paire événement-actions n'a lieu que lorsqu'un capteur détecte un objet
    et qu'aucun autre capteur ne détecte d'objet.

\end{itemize}

\bigskip

{\raggedleft \hfill \textbf{Programme}: \bu{speeders.aesl}}
