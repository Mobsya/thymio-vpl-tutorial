\chap{Des capteurs avec plusieurs seuils}

Le mode avancé permet de définir les événements capteurs de trois manières différentes,
comme décrit dans l'\cref{a.tech}:
l'événement est déclanché lorsque la lumière réflechie est en-dessous d'un certain seuil (noir),
l'événement est déclanché lorsque la lumière réflechie est au-dessus d'un certain seuil (blanc),
et l'événement est déclanché lorsque la lumière réflechie est entre deux seuils (gris foncé):

\begin{center}
\begin{tabular}{ccc}
\blk{slow-low}&\blk{slow-mid}&\blk{slow-high}\\
\end{tabular}
\end{center}

\textbf{Consignes}

Créez un programme dans lequel le robot s'approche d'un objet,
d'abord à grande vitesse, puis il ralentit au fur et à mesure qu'il s'approche de l'objet
jusqu'à s'arrêter complètement lorsqu'il en est très proche.

\textbf{Conseils}

\begin{itemize}
\item Utilisez trois paires événement-actions, une pour chacun des types d'événement capteurs.

\item Glissez soigneusement les \emph{sliders} (voir \cref{a.tech}) de sorte à ce que
le seuil supérieur d'un intervalle corresponde au seuil inférieur du prochain intervalle.

\item Ajoutez un bloc couleur à chacune des paires pour pouvoir remarquer les changements de vitesse.

\item Utilisez du ruban adhésif réfléchissant pour augmenter la portée des capteurs,
comme expliqué dans l'\cref{a.blocks}.
\end{itemize}

\bigskip

{\raggedleft \hfill Programme: \bu{slow.aesl}}
