\chap{Mesurer sa vitesse}\label{ch.speed}

\textbf{Instructions}

Mesurez la vitesse du Thymio dans différentes configurations des moteurs des roues.
Placez une bande de ruban adhésif noire sur une surface claire comme lorsque le Thymio
suivait la ligne (\cref{ch.line}).
Placez le robot à l'un des bouts de la bande. Implémentez le comportement suivant:

\begin{itemize}

\item Le robot commence à avancer lorsque l'on appuie sur le bouton central.

\item Lorsque le robot détecte le début de la bande avec ces capteurs du bas,
    lancez un minuteur d'une seconde.

\item Quand le minuteur s'est écoulé, changez la couleur du haut et relancez le minuteur d'une seconde.

\item Quand le Thymio arrive au bout de la bande, éteignez les moteurs.

    \end{itemize}

Lancez le programme et comptez combien de fois la couleur change.
Ce sont le nombre de secondes qu'il a fallu au robot pour traverser la bande.
Divisez la longueur de la bande par le nombre de secondes obtenu pour obtenir la vitesse.
Si par exemple la bande mesure 30 centimètres et la couleur change 6 fois, alors la vitesse
du robot est 30/6=5 centimètres par seconde.

Essayez de modifier la configuration des moteurs et la longueur de la bande.

\textbf{Conseils}

Faites une liste de couleurs, par exemple 1=rouge, 2=bleu, 3=vert, 4=jaune, etc
et utilisez cette liste pour déduire le nombre de secondes à partir de la couleur du robot.

Utilisez des états pour mémoriser la couleur actuelle et les prochaines couleurs.
Par exemple, dans l'état 3, la couleur est vert; quand le minuteur est écoulé \emph{et}
l'état est 3, passez à l'état 4, affichez la couleur jaune et relancez le minuteur.
Pour chaque événement minuteur, il y a trois actions.

\bigskip

{\raggedleft \hfill Programme \bu{measure-speed.aesl}}
