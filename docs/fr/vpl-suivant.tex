\chap{Et après?}

Ce tutoriel vous a introduit Thymio, Aseba et le VPL. Cet environnement de programmation par image est très pratique, simple d'emploi et bien pensé mais il a tout de même des limitations. Il n'est pas conçu pour programmer des comportements complexes. Pour cela, il vous faudra vous familiariser avec l'environnement d'Aseba Studio, comme illustré sur \cref{fig.Aseba_general}.

\begin{figure}[hbt]
\begin{center}
\gr{Aseba_general}{1}
\caption{L'environnement d'Aseba Studio}\label{fig.Aseba_general}
\end{center}
\end{figure}

La programmation en utilisant Aseba Studio est aussi basée sur les événements. On retrouve toutes les fonctionnalités que vous avez découvertes en utilisant le VPL. La vraie différence vient des la liberté quasi totale que vous aurez en programmant sur Aseba Studio.

Vous pourrez ajuster tous les tests conditionnels, par exemple sur le taux de lumière réfléchie sur un capteur. Les restrictions, comme "le capteur ne voit pas de lumière" ou "le capteur voit de la lumière", sautent. Vous pourrez tout régler très précisément.

Vous obtiendrez la liberté de travailler avec des variables, d'avoir de la mémoire, d'utiliser toute une batterie d'expressions, de tests\ldots

De plus, vous pourrez contrôler toutes les lumières, pas seulement le dessus et le dessous de Thymio. Vous aurez plus de flexibilité pour le contrôle du synthétiseur de son. Vous aurez accès à un capteur de température. Vous pourrez voir plus précisément les valeurs des accéléromètre sur les trois dimensions et plus seulement détecter un choc. Finalement, vous pourrez utiliser une télécommande pour contrôler Thymio!

Vous pouvez vous rendre sur le site web d'Aseba pour avoir plus d'information et pour vous lancer dans la programmation \textit{écrite} et plus seulement \textit{visuelle}: \url{https://aseba.wikidot.com/fr:thymio}

Il est possible d'importer dans Aseba Studio tous les programmes qui proviennent du VPL en les ouvrant simplement dans Asbea Studio.

Il ne vous reste plus qu'à tester Thymio sous toutes les coutures! Si vous manquez d'inspiration, allez faire un tour sur la page des exemples du site web d'Aseba, vous y trouverez de nombreux nouveaux challenges!

\begin{figure}[h]
\begin{center}
\gr{Thymio}{1}
\end{center}
\end{figure}