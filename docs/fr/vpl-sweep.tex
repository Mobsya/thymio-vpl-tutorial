\chap{Nettoyer le sol}\label{ch.sweep}

Marre de nettoyer votre maison?
Il existe maintenant les \emph{robots aspirateurs} qui font le travail à votre place!
Le robot traverse systématiquement votre appartement pour aspirer les saletés
tout en évitant les obstacles.

\textbf{Instructions}

Lorsque l'on appuie sur le bouton avant, le robot traverse la chambre d'un bout à l'autre, 
tourne, avance un petit bout puis retourne vers le côté original:

\begin{center}
\begin{picture}(200,30)
%\put(0,0){\framebox(200,30){}}
\put(0,30){\vector(1,0){200}}
\put(200,30){\vector(0,-1){30}}
\put(200,0){\vector(-1,0){200}}
\end{picture}
\end{center}

\textbf{Conseils}

Le robot doit implémenter trois tâches distinctes: (1) traverser la chambre (vers la droite ou vers la gauche), (2) tourner à droite, (3) avancer un peu.
Ces tâches sont exécutées dans l'ordre suivant:

\begin{center}
\begin{picture}(380,20)
%\put(0,0){\framebox(380,20){}}
\put(30,10){\oval(60,20)}
\put(110,10){\oval(60,20)}
\put(190,10){\oval(60,20)}
\put(270,10){\oval(60,20)}
\put(350,10){\oval(60,20)}
\put(0,0){ \makebox(60,20){traverser}}
\put(80,0){\makebox(60,20){tourner}}
\put(160,0){\makebox(60,20){avancer peu}}
\put(240,0){\makebox(60,20){tourner}}
\put(320,0){\makebox(60,20){traverser}}
\put( 60,10){\vector(1,0){20}}
\put(140,10){\vector(1,0){20}}
\put(220,10){\vector(1,0){20}}
\put(300,10){\vector(1,0){20}}
\end{picture}
\end{center}

Il faudra utiliser des états différents pour mémoriser quelle tâche est en train d'être exécutée.
Dans chaque tâche, la direction et la quantité de mouvement sont déterminées par la vitesse
des moteurs droite et gauche et par la durée pendant laquelle ils sont allumés.
C'est pourquoi chaque tâche sera implémentée grâce à une paire événement-actions où
l'événement sera la fin du minuteur de la tâche précédente et les actions contrôleront
le changement d'état. Il faudra pour cela: (1) changer l'état; (2) changer les vitesses des
moteurs gauche et droite; (3) lancer un nouveau minuteur. 
Pour lancer le programme, on utilisera un événement bouton.

Il vous faudra ajuster les vitesses et la longueur des minuteurs pour que le robot
suive la trajectoire rectangulaire désirée.

\bigskip

{\raggedleft \hfill Programme: \bu{sweep.aesl}}

Vous pouvez encore vous amuser à ajouter des couleurs différentes pour les différents états:
vert quand le Thymio va tout droit, jaune quand il tourne et rouge quand il est arrêté.

\bigskip

{\raggedleft \hfill Programme: \bu{sweep1.aesl}}
