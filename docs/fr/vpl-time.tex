\chap{Aimer pour un temps}

Dans le \cref{ch.pet}, nous avons programmé un robot de compagnie qui nous aimait ou nous fuyait.
Considérons un comportement plus avancé : un compagnon timide qui n'arrive pas à se décider s'il nous aime ou pas.
Initiallement, le robot s'approchera de notre main tendue, puis il s'en éloignera.
Après un moment, il changera d'avis et reviendra en direction de notre main.

{\raggedleft \hfill Programme \bu{shy.aesl}}

Le comportement du robot est le suivant.
Lorsque le bouton droite est touché, le robot tourne à droite.
Quand il détecte votre main, il tourne à gauche mais après un moment il regrette sa décision et se retourne.
Nous savons comment construire les paires événement-action pour le premier mouvement :
\blkc{start-turn} et pour se détourner quand votre main est détectée : \blkc{turn-away}

Le comportement de se retourner «\,après un moment\,» peut être décomposé en deux paires événement-action :
\begin{itemize}

\item \emph{Quand} le robot commence à se détourner $\rightarrow$ \emph{démarrer un minuteur} pour deux secondes.

\item \emph{Quand} le temps du minuteur est écoulé $\rightarrow$ \emph{tourner} à droite.

\end{itemize}

Nous avons besoin d'une nouvelle \emph{action} pour la première partie du comportement et d'un nouvel \emph{événement} pour la seconde partie.

L'action démarre un \emph{minuteur}, qui ressemble à un réveil \blksm{action-timer}.
Ce minuteur est, en fait, un compte à rebours qui peut aller jusqu'à 4 secondes.
Pour régler la durée du compte à rebours, vous pouvez cliquer n'importe où dans le cadran de l'alarme.
Une petite animation vous montre combien de temps le compte à rebours va durer.

\newpage

La paire événement-action pour cette première partie du comportement est : \blkc{turn-clock}

Le compte à rebours est réglé sur deux secondes.
Lorsque l'événement détectant votre main se produit, il y aura donc deux actions : tourner le robot à gauche et lancer le compte à rebours.

La seconde partie du comportement nécessite un événement se produisant lorsque le compte à rebours arrive à zéro, c'est le bloc d'événement \emph{temps écoulé} \blksm{event-timer} qui montre un réveil sonnant.

La paire événement-action pour faire tourner à nouveau le robot vers la droite est donc : 
\blkc{turn-back}

\exercisebox{\thechapter.1}{
Écrivez un programme qui fasse avancer le robot à vitesse maximale pour trois seconde lorsque que le bouton avant est touché ; puis qui fasse reculer le robot.
Ajoutez une paire événement-action qui arrête le mouvement en touchant le bouton central.
}

\trickbox[Information]{En robotique, ce genre de compte à rebours s'appelle des \textit{timers}.
Ils sont extrêmement utiles dans de nombreuses situations et vous vous en apercevrez très rapidement en créant vos propres comportements pour Thymio.}
