% !TeX root = vpl.tex

\chap{Sonagli e fischietti}\label{ch.bells}

Prendiamoci qualche momento di svago dai compiti complessi come seguire una linea e
divertiamoci un po' con il nostro robot Thymio.
In questo capitolo ti mostriamo come il Thymio possa riprodurre musica, rispondere a un suono o reagire quando viene toccato.

\sect{Suonare musica}

Il robot Thymio contiene un sintetizzatore di suoni e si può programmare
per riprodurre melodie semplici utilizzando il blocco di azione musica: \blkc{action-music}

{\raggedleft \hfill File di programma \bu{bells.aesl}}

Non diventerai un nuovo Beethoven---si può suonare solo una sequenza di note alla volta,
su cinque toni e due lunghezze differenti---ma si può comporre una
melodia che farà distinguere il vostro robot da tutti gli altri. La \Cref{fig.music} mostra
due coppie evento-azione che rispondono con una melodia quando il pulsante anteriore o posteriore
vengono toccati. C'è una melodia diversa associato ad ogni evento.

\begin{figure}
\begin{center}
\gr{music}{.4}
\caption{Esegui una melodia}\label{fig.music}
\end{center}
\end{figure}

Il piccoli cerchi sono le sei note.
Una nota nera è una nota breve e una nota bianca è una nota lunga; per passare da una lunghezza all'altra, cliccare sul cerchio.
Ci sono cinque righi orizzontali colorati, che rappresentano cinque toni.
Per spostare un cerchio su un rigo, fare clic sul \emph{rigo} sopra o sotto il cerchio.
Non cercare di trascinare e rilasciare una nota; non funzionerà.

\exercisebox{\thechapter.1}{
Scrivere un programma che vi permetterà di inviare un messaggio in \href{http://en.wikipedia.org/wiki/Morse_code}{codice Morse}.
Le lettere in codice Morse sono codificate in sequenze di toni lunghi
(\emph{linee}) e toni brevi (\emph{punti}). Ad esempio, la lettera
\emph{V} è codificato da tre punti seguiti da una linea.}


\sect{Controllare il tuo robot con i suoni}

Il Thymio ha un microfono. L'evento \blksm{event-clap} si verificherà
quando il microfono avverte un forte rumore, per esempio un applauso con le tue
mani. la seguente coppia evento-azione accende le luci della parte inferiore
quando battete le mani: \blkc{clap}

\trickbox[Informazioni] {In un ambiente rumoroso potresti non essere in grado di utilizzare questo
evento, perché il livello sonoro sarà sempre alto causando ripetuti
eventi.}

\exercisebox{\thechapter.2}{
Scrivi un programma che fa muovere il robot quando batti
le mani e lo fa fermare quando si tocca un tasto.
\vspace{.5em}\\
Quindi scrivi un programma che fa il contrario: si avvia quando si tocca un tasto
e si ferma quando batti le mani.
}


\sect{Ottimo lavoro, robot!}

Gli animali non sempre fanno quello che chiediamo loro di fare. A volte hanno bisogno di una pacca
sulla testa per incoraggiarli. Si può fare lo stesso con il vostro robot. Il
Thymio contiene un sensore di urto che provoca l'evento \blksm{event-tap} in risposta ad un rapido tocco sulla parte superiore del robot. Per esempio,
la seguente coppia evento-azione fa sì che le luci superiori si accendano quando
si tocca la parte superiore del robot: \blkc{touch}

Costruisci un programma da questa coppia evento-azione e la seguente coppia
che accende le luci inferiori quando batti le mani: \blkc{clap}

{\raggedleft \hfill File di programma {bu whistles.aesl}}

Riesci ad attivare solo la parte superiore delle luci? Questo è difficile da fare: un colpetto
provoca un suono che può essere abbastanza forte da far accendere pure le luci inferiori. Con un po' di pratica sono riuscito a toccare il robot
abbastanza delicatamente in modo che il suono emesso dal colpetto non è stato considerato un evento.

\exercisebox{\thechapter.3}{
Scrivere un programma che faccia andare avanti il robot fino a quando non colpisce una
parete.
\vspace{.5em}  \\
\textbf{Assicurarsi} che il robot si \textbf{muova lentamente} in modo che
non si danneggi.
}

