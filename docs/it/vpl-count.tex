% !TeX root = vpl.tex

\chap{Contare (Avanzato)}\label{ch.counting}

In questo capitolo mostreremo come gli stati del robot possono essere usati per contare i numeri e anche effettuare semplici calcoli aritmetici.

Il disegno e l'implementazione dei progetti di questo capitolo non saranno presentati in dettaglio.
Assumiamo che tu abbia abbastanza esperienza ora per svilupparli da te.
Il codice sorgente dei programmi funzionanti è incluso nell'archivio, ma non consultarlo a meno che tu non abbia veramente
difficoltà a risolvere un problema.

Questi progetti usano l'evento \emph{clap} (battito di mani) per cambiare gli stati e il comportamento di base del cerchio di LED 
per mostrare lo stato:

\importantbox{Lo stato corrente del robot è mostrato nel cerchi di LED della parte superiore del robot.
La \cref{fig.state-leds} mostra il robot nello stato \bu{(on, on, on, on)}.}

Sentiti libero di modificare qualsiasi di questi comportamenti.

\sect{Pari e Dispari}

\begin{quote}
\textbf{Programma}\\Scegli uno degli elementi dello stato del robot.
Sarà \bu{spento} (bianco) se il numero di battiti di mani sarà pari, e 
 \bu{acceso} (arancione) se il numero di appalusi è dispari.
Toccando il bottone centrale si ripristina a zero cioè pari. 
(poichè zero è un numero pari per definizione).
\end{quote}

{\raggedleft \hfill File di programma \bu{count-to-two.aesl}}

Queso metodo di conteggio dimostra il concetto di
\emph{aritmetica modulo 2}.
Contiamo a partire da 0 a 1 e poi di nuovo a 0.
Il termine \emph{modulo} è simile al termine \emph{resto}:
se ci sono stati 7 battiti di mani, dividendo 7 per 2 da 3 e resto 1.
Teniamo solo il resto 1.

In aritmetica modulo 2, 0 e 1 sono spesso chiamati pari e dispari,
rispettivamente.

Un altro termine per lo stesso concetto è \emph{aritmetica ciclica}.
Invece di contare da 0 a 1 e poi da 1 a 2,
noi \emph{cicliamo} dall'inizio:
0, 1, 0, 1, \ldots.

Questi concetti sono molto familiari perché sono usati negli orologi.
Minuti e secondi vengono contati modulo 60 e le ore e sono contati modulo
12 o 24. Pertanto, il secondo dopo 59 non è 60; invece, si cicla e si
inizia di nuovo il conteggio da 0. Allo stesso modo, l'ora dopo le 23 non è 24,
ma 0.  Se sono le 23:00 e siamo d'accordo di incontrarci dopo 3 ore,
l'orario fissato per la riunione è 26 modulo 24, che equivale alle 2:00 di
mattina del giorno dopo.

\sect{Contare in unario}

\begin{quote}
Modifica il programma per contare modulo 4.
Ci sono quattro possibili resti 0, 1, 2, 3.
Scegli tre elementi dello stato, uno per rappresentare ciascuno dei valori 1, 2 e 3; il valore 0 verrà rappresentato
impostando ogni elemento dello stato a \bu{spento}.
\end{quote}

Questo metodo di rappresentazione dei numeri è chiamato
\emph{rappresentazione unaria} 
poiché ogni elemento dello stato rappresenta un numero diverso.
Spesso usiamo la rappresentazione unaria per tenere traccia del conteggio di
alcuni oggetti; per esempio,
\begin{picture}(35,10)
\multiput(5,0)(5,0){4}{\put(0,0){\line(0,1){10}}}
\put(0,0){\line(3,1){25}}
\put(32,0){\line(0,1){10}}
\end{picture}
rappresenta 6.

{\raggedleft \hfill File di programma \bu{count-to-four.aesl}}

\exercisebox{\thechapter.1}{Fino a che numero possiamo contare con Thymio usando la rappresentazione unaria?}

\sect{Contare in binario}

Noi abbiamo molta familiarità con la \emph{rappresentazione in base} dei numeri,
in particolare con la rappresentazione in base 10 (decimale).
Il numero 256 nella rappresentazione in base 10  non rappresenta tre
diversi oggetti (2, 5, 6).
In realtà il 6 rappresenta il numero di 1 (unità), il 5 rappresenta il
numero di 10$\times$1=10 (decine), e il 2 rappresenta il numero di
10$\times$10$\times$1=100 (centinaia). La somma di questi fattori dà il numero
duecentocinquantasei. Utilizzando la rappresentazione in base 10
possiamo scrivere numeri molto grandi in una rappresentazione compatta.
Inoltre, l'aritmetica sui grandi numeri
è relativamente facile con i metodi che abbiamo imparato a scuola.

Noi usiamo la rappresentazione in base 10  perché abbiamo 10 dita
quindi per noi è più facile imparare ad usare questa rappresentazione.
I computer invece hanno due ``dita''(\bu{spento} e \bu{acceso})
così si usa aritmetica in base 2 per fare i calcoli. L'aritmetica in base 2 sembra 
strana all'inizio: utilizziamo i simboli familiari 0 e 1
usati anche in base 10, ma le regole per il conteggio sono di ciclare a 2
anziché ciclare a 10:
\begin{displaymath}
0, 1, 10, 11, 100, 101, 110, 111, 1000, \ldots
\end{displaymath}

Dato un numero in base 2, ad esempio 1101, si calcola il suo valore da destra
a sinistra come nella rappresentazione base 10.
La cifra più a destra rappresenta il numero di uno, la cifra successiva
rappresenta il numero 1$\times$2=2, la terza cifra rappresenta
il numero 1$\times$2$\times$2=4, e la cifra più a sinistra
rappresenta il
numero 1$\times$2$\times$2$\times$2=8.
Pertanto, 1101 rappresenta 1 + 0 + 4 + 8, che è tredici,
rappresentato in base 10 come 13.

\newpage

\begin{quote}
\textbf{Programma}\\
Modificare il programma per il conteggio modulo 4 per usare la rappresentazione binaria.
\end{quote}

{\raggedleft \hfill File di programma \bu{count-to-four-binary.aesl}}

Abbiamo solo bisogno di due elementi dello stato per rappresentare i numeri
0--3 in base 2.
Fai in modo che l'elemento in alto a destra rappresenti il numero di 1,
\bu{spento} (bianco) per zero e \bu{acceso} (arancione) per uno,
e che l'elemento superiore sinistro rappresenti il numero di 2.
Ad esempio, \blksm{state-right} rappresenta il numero 1 e
\blksm{state-left} rappresenta il numero 2.
Se i due elementi sono entrambi di colore bianco, lo stato rappresenta 0, e se entrambi
gli elementi sono di colore arancione, lo stato rappresenta il 3.

Per contare sono necessarie quattro transizioni 0 $\rightarrow$ 1, 1 $\rightarrow$ 2, 2
$\rightarrow$ 3, 3 $\rightarrow$ 0, quindi sono necessarie quattro coppie evento-azione, oltre ad una coppia per ripristinare il programma
quando si tocca il pulsante centrale.

\medskip

\trickbox{I due elementi inferiori dello stato non vengono utilizzati, quindi sono lasciati in grigio
e vengono ignorati dal programma.}

\medskip

\exercisebox{\thechapter.2}{Estendere il programma in modo che conti
modulo 8. L'elemento dello stato in basso a sinistra rappresenterà il numero di 4.}

\exercisebox{\thechapter.3}{Fino a che numero possiamo contare con Thymio utilizzando
la rappresentazione binaria?}

\sect{Aggiungere e sottrarre}

Scrivere il programma per contare fino a 8 è piuttosto noioso perchè devi programmare 8 coppie evento-azione, una per ciascuna transizione da $n$ a $n+1$ (modulo 8).Certamente non è così che si conta, 
invece, normalmente utilizziamo metodi per eseguire le addizioni sommando le cifre in ciascun posto e facendo il riporto verso la cella successiva.
Nella rappresentazione in base 10:
\begin{displaymath}
\begin{array}{r}
387\\
+426\\
\rule[1pt]{1.5em}{1pt}\\
813\\
\end{array}
\end{displaymath}
e similmente nella rappresentazione in base 2:
\begin{displaymath}
\begin{array}{r}
0011\\
+1011\\
\rule[1pt]{2em}{1pt}\\
1110\\
\end{array}
\end{displaymath}
Quindi aggiungendo 1 a 1, invece di 2, otteniamo 10.
Lo zero è scritto nella stessa colonna e riportiamo un 1 nella successiva colonna a sinistra. 
L'esempio mostra la somma di  3 (=0011) e di 11 (=1011)
per ottenere 14 (=1110).

\begin{quote}
\textbf{Programma}\\
Scrivere un programma che inizia con una rappresentazione di 0.
Ciascun battito di mani aggiunge 1 al numero.
L'addizione è modulo 16, quindi aggiungendo 1 a 15 il risultato sarà 0.
\end{quote}

Linee guida:

\begin{itemize}
\item L'elemento in basso a destra verrà utilizzato per rappresentare il numero 8.
\item Se l'elemento in alto a destra che rappresenta il numero
1 mostra 0 (bianco),
semplicemente cambiarlo a 1 (arancione). Fate questo indipendentemente da ciò che gli altri
elementi mostrano.
\item Se l'elemento in alto a destra che rappresenta il numero di 1
è a 1 (arancione),
modificarlo a 0 (bianco) e poi riportare 1.
Ci saranno tre coppie evento-azione,
a seconda della posizione del \emph{prossimo} elemento che è a 0
(bianco).
\item Se tutti gli elementi sono a 1 (arancione), è rappresentato il valore 15.
Aggiungendo 1 a 15 modulo 16 si ottiene 0, rappresentato da tutti gli elementi
che sono a 0 (bianchi).
\end{itemize}


{\raggedleft \hfill File di programma \bu{addition.aesl}}

\exercisebox{\thechapter.4}{Modificare il programma in modo che parta dal valore 15 e sottragga uno per ciascun battito di mani,
e quindi ciclicamante torni a 15.}


\exercisebox{\thechapter.5}{Mettere una sequenza di corti segmenti di nastro nero su una superfice chiara (o nastro bianco su una superfice scura).
Scrivere un programma che faccia andare in avanti il Thymio e fermare quando viene riconosciuto il quarto nastro.}

L'esercizio non è facile: le strisce devo essere sufficientemente ampie in modo che il robot le riconosca, ma non così
ampie da far avvenire più di un evento per striscia.
Dovrete anche fare esperimenti con la velocità del robot.

