% !TeX root = vpl.tex

\chap{Il tuo primo Progetto di Robotica}\label{ch.intro}

\sect{Conoscere il tuo Thymio}

La \Cref{fig.front} mostra la parte anteriore e superiore del Thymio II. Sulla parte superiore
è possibile vedere il pulsante centrale circolare (\textcolor{blue}{A}) e quattro tasti direzionali (\textcolor{blue}{B}).
Dietro i pulsanti, la luce verde (\textcolor{blue}{C}) indica quanta carica rimane nella
batteria. Sul retro ci sono le luci superiori (\textcolor{blue}{D}), che sono state impostate su rosso in questa immagine.
Ci sono luci simili nella parte inferiore (impostate in verde in \cref{fig.bottom}).
I piccoli rettangoli neri (\textcolor{blue}{E}) sono sensori che
imparerai a conoscere in \cref{ch.pet}. È possibile ignorare le piccole
luci rosse per ora.

\begin{figure}[h]
\begin{center}
\gr{front.jpg}{.8}
\caption{Thymio robot visto da davanti}\label{fig.front}
\end{center}
\end{figure}

\pagebreak

\sect{Connettere il robot ed eseguire il VPL}

Collegate il vostro robot Thymio al computer con un cavo USB; il robot suonerà una sequenza di toni.
Se il robot è spento, accenderlo toccando il tasto centrale
per cinque secondi. Eseguite VPL facendo doppio clic sull'icona
\blksm{thymiovpl}.

\importantbox{Quando una piccola icona appare nel testo,
una immagine più grande viene visualizzata nel margine.}


VPL solitamente è in grado di connettersi automaticamente al vostro robot.
In caso contrario, sarà
visualizzata la finestra mostrata in \cref{fig.connect}. Selezionate la casella accanto a \bu{Seriale}, cliccate su \bu{Thymio
Robot} sotto di essa, selezionare una lingua, quindi fare clic su
\bu{Connetti}.
A seconda della configurazione del computer
e il sistema operativo che si sta utilizzando, ci possono essere diverse
voci in questa tabella e i dati di \bu{Thymio} possono essere
diversi da quanto mostrato in figura.

\trickbox{E 'anche possibile accedere a VPL da ASEBA Studio, l'ambiente di programmazione basato su testo, attraverso il plugin VPL che si trova nella zona degli \textit{Strumenti} in basso a sinistra dello schermo.}

\begin{figure}
\begin{center}
\gr{connect}{.4}
\caption{Connessione a Thymio, attraverso la porta seriale (USB)}\label{fig.connect}
\end{center}
\end{figure}

\vfill

\pagebreak

\sect{L'interfaccia utente del VPL}

L'interfaccia utente del VPL è mostrato sotto.
Ci sono sei aree nell'interfaccia:
\begin{enumerate}[noitemsep,nosep]
\item Una barra degli strumenti con le icone per l'apertura, il salvataggio, l'esecuzione di un programma, ecc...
\item Una zona programma in cui vengono costruiti i programmi per il controllo del robot.
\item Indica se il programma che si sta creando è ben formato oppure no.
\item Una colonna con blocchi di eventi disponibili per costruire il vostro programma.
\item Una colonna con blocchi di azione a disposizione per costruire il vostro programma.
\item La traduzione in testo del programma (vedere in fondo a questa pagina).
\end{enumerate}
I blocchi di evento e azione saranno descritti nel corso di questo documento.

\plainfloat
\begin{figure}[h]
\gr{gui-it}{1}
\vskip -1em
\caption{La schermata del VPL}\label{fig.vplgui}
\end{figure}
\framedfloat

\vfill

\trickbox[Per andare oltre]{
Quando si costruisce un programma utilizzando VPL, il programma di testo che verrà caricato nel robot appare nella parte destra della finestra.
Se siete curiosi e volete capire questo linguaggio, è possibile leggere il \href{https://aseba.wikidot.com/it:thymiotutoriel}{tutorial della modalità testuale} (\url{http://aseba.wikidot.com/it:thymiotutoriel}).
}

\pagebreak

\sect{Scrivere un programma}

Quando si avvia VPL, viene visualizzata un'area vuota del programma; se, dopo aver costruito un pezzo di programma, si desidera cancellare il contenuto dell'area di programma, fare clic su \blksm{new}.

Un programma in VPL è costituito da uno o più \emph{coppie evento-azione},
ciascuna fatta
da un blocco di evento e da un blocco di azione. Ad esempio, la coppia:
\blkc{e-a-pair} fa in modo che la luce superiore del robot sia rossa quando si tocca il pulsante frontale sul robot.

\importantbox{
Il significato di una coppia evento-azione:\\
\textbf{Quando si verifica l'evento, il robot esegue l'azione.}
}

Cerchiamo di costruire una coppia evento-azione.
Nella zona interessata dal programma si vede un contorno di una coppia:
\blkc{event-action-pair-empty}
% Si tratta di un modello per costruire la coppia evento-azione.
Sulla sinistra il quadrato azzurro  è lo spazio per l'evento; sulla destra il quadrato rosa è lo spazio per l'azione. %\footnote{I colori reali sono \textit{ciano} (una miscela di blu e verde) e \textit{magenta} (una miscela di rosso e blu).}
Per portare un blocco da lato all'area di programma  (zone 4 e 5 di \cref{fig.vplgui}), è possibile fare clic su di esso o trascinarlo nella casella corrispondente tenendo premuto il tasto sinistro del mouse e rilasciandolo quando il blocco è
al suo posto.

Inizia portando il blocco per l'evento "Bottoni"  \blksm{event-buttons} nel quadrato blu.
Ora, porta il blocco di azione "Colore Superiore" \blksm{action-colors-up} nel quadrato rosa.
Avete costruito una coppia evento-azione!

Ora dobbiamo modificare l'evento e l'azione per fare quello che vogliamo. per
l'evento, fare clic sul pulsante anteriore (il triangolo in alto); si colorerà di
rosso: \blkc{forward}
Questo specifica che \textbf{l'evento si verifica quando il pulsante anteriore del Thymio viene toccato}.

Il blocco di azione "Colore Superiore" contiene tre barre con i colori primari rosso,
verde e blu; all'estremità sinistra di ogni barra vi è un quadrato bianco. La barra del colore
con il suo quadrato bianco si chiama \emph{cursore}. Trascinate il quadrato bianco
a destra e poi di nuovo a sinistra, e vedrete i cambiamenti del
colore di sfondo del blocco.
Mescolando questi tre colori primari rosso, verde e blu si possono creare moltissimi altri colori secondari.
Ora sposta il cursore del rosso fino a quando il
quadrato è all'estrema destra, e sposta i cursori verdi e blu fino a quando
sono all'estrema sinistra. Il colore del blocco sarà tutto rosso senza né blu senza né
verde: \blkc{red}

\sect{Salvare il programma}

Prima di eseguire il programma, salvalo sul tuo computer.
Fare clic sull'icona \blksm{save}
nella barra degli strumenti. Ti verrà chiesto di dare un nome al programma; scegliere un
nome che ti aiuterà a ricordare ciò che fa il programma, ad esempio
\bu{display-red}.
Scegliere la posizione in cui si desidera salvare il programma, sul desktop, per esempio, e fare clic su Salva.


\sect{Eseguire il programma}

Per eseguire il programma, clicca su \blksm{run} nella barra degli strumenti. Tocca
il pulsante frontale sul robot; la luce sulla parte superiore del robot deve
diventare di colore rosso.

\infobox{Congratulazioni!}{
Hai creato ed eseguito il tuo primo programma. Il suo comportamento è: \\
\textbf{Quando si tocca il pulsante avanti il Thymio diventa rosso.}
}{yellow!10}{\bcetoile}

\sect{Spegnere il robot}

Quando avete finito di lavorare con il robot Thymio, è possibile spegnerlo
toccando il pulsante centrale per quattro secondi fino a quando
si sente una sequenza di toni discendenti. La batteria del robot
continuare la carica fintanto che è collegato a un computer. La
luce rossa accanto al connettore del cavo USB significa che il robot è in
carica; diventa blu quando la carica è completata (\cref{fig.back}).
È possibile scollegare il cavo quando non si utilizza il robot.

\trickbox{
È possibile ricaricare il robot più velocemente usando un caricabatterie del cellulare con un connettore micro-USB.}

\begin{figure}
\begin{center}
\gr{back}{.6}
\caption{Il retro di Thymio che mostra il cavo USB e la luce di ricarica in corso}\label{fig.back}
\end{center}
\end{figure}

Se il cavo USB si scollegasse durante la programmazione, VPL attenderà che il collegamento venga effettuato di nuovo.
Controllare entrambe le estremità del cavo, ricollegare e verificare se VPL funziona.
Se hai un problema, si può sempre chiudere VPL, ricollegare il robot e aprire di nuovo VPL.

\sect{Modificare un programma}

\begin{itemize}
\item Per eliminare una coppia evento-azione, fare clic su \blksm{x} in alto a destra della coppia.
\item Per aggiungere una coppia evento-azione, fare clic su \blksm{plus} presente sotto ogni coppia.
\item Per spostare una coppia evento-azione in un'altra posizione nel programma, trascinarla e rilasciarla nella posizione desiderata.
\end{itemize}

\sect{Aprire un programma esistente}

Supponiamo di aver salvato il programma e spento il robot e
il computer, ma in seguito si desidera continuare a lavorare sul programma.
Collegate il robot ed eseguite VPL come descritto in precedenza. Fate clic sull'icona
\blksm{open} e selezionate il programma che si desidera aprire, per esempio,
\bu{display-red}. Le coppie evento-azione del programma verranno visualizzate
nell'area programma, e si può continuare a modificarlo.

\sect{Altre operazioni nell'interfaccia VPL}

Nella barra degli strumenti, troverete altre funzionalità:

\begin{itemize}

\item \textbf{Salva come} \blksm{saveas}:
Fare clic su questa icona per salvare il programma corrente con un \emph{nome diverso} .
Usalo quando si ha un programma di lavoro e si vuole provare qualcosa di nuovo senza modificare il programma esistente.

\item \textbf{Stop} \blksm{stop}: Questo interrompe il programma in esecuzione
e porta la velocità dei motori a zero. Usatelo quando il programma ordina
al robot di muoversi, ma non include una coppia evento-azione che possa
arrestare il motore.

\item \textbf{Cambio schema colori} \blksm{scheme}: È possibile selezionare una
diversa coppia di colori per lo sfondo dei blocchi di evento e di azione.

\item \textbf{Modalità avanzata} \blksm{advanced}: La modalità avanzata consente
l'uso di variabili di stato come descritto in \cref{ch.states}.

\item \textbf{Aiuto} \blksm{info1}: Visualizza la documentazione VPL nel
browser (è necessaria una connessione a Internet).
Questa documentazione si trova a questo link:
\url{https://aseba.wikidot.com/it:thymiovpl}.

\end{itemize}


