% !TeX root = vpl.tex

\chap{Mettersi in movimento}\label{ch.moving}

\sect{Muoversi avanti e indietro}

Il robot Thymio ha due motori, uno collegato a ciascuna ruota. I
motori possono ruotare in avanti e indietro, facendo in modo che il robot si muova
avanti e indietro in linea retta o curva. Cominciamo con un semplice
progetto per conoscere i motori.

Il blocco azione Motori \blksm{action-motors} mostra una piccola immagine del
robot nel centro insieme a due cursori. I cursori controllano la
velocità dei motori, uno per il motore sinistro e uno per il motore destro. Quando il quadrato bianco è centrato nel cursore, il
corrispondente motore è spento. È possibile trascinare un quadrato verso l'alto per aumentare la
velocità in avanti e verso il basso per aumentare la velocità indietro.
Scriviamo un programma per spostare il robot in avanti quando il pulsante anteriore
viene toccato e indietro quando il pulsante indietro viene toccato.

{\raggedleft \hfill File di programma \bu{moving.aesl}}

Abbiamo bisogno di due coppie evento-azione (\cref{fig.nostop}). Trascina e rilascia
i blocchi degli eventi e delle azioni e imposta i cursori allineati per il motore sinistro e destro, a metà corsa verso l'alto per andare avanti e a metà strada verso il basso per andare indietro.

\begin{figure}
\begin{center}
\gr{no-stop-motors}{.4}
\caption{Muoversi avanti o indietro}\label{fig.nostop}
\end{center}
\end{figure}


Esegui il programma e tocca i pulsanti per
far andare il robot avanti e indietro.

\newpage

\sect{Fermare il robot}

\textbf{Aiuto!} Non riesco a fermare i motori del robot!

Fai clic sull'icona \blksm{stop} per fermare il robot.

Risolviamo questo problema aggiungendo una coppia evento-azione: \blkc{stop-motors}
Questa fermerà i motori quando si tocca il pulsante centrale. Quando si
trascina il blocco di azione motori nella zona interessata dal programma, è già impostato
con i cursori al centro corrispondente a spegnere i motori.

\sect{Non cadere dal tavolo}

Se il robot si muove sul pavimento, nel peggiore dei casi, potrebbe colpire un muro o scollegare il
suo cavo, ma se si posiziona il robot su un tavolo, potrebbe cadere
fuori, urtare e rompersi! Cerchiamo di fare in modo che il robot si fermi quando si
raggiunge la fine di un tavolo.

\warningbox{Ogni volta che il robot si muove su un tavolo, stai pronto a prenderlo nel caso in cui dovesse cadere dal bordo.}

Gira il Thymio sulla sua schiena. Vedrai sulla parte anteriore due piccoli
rettangoli neri con elementi ottici all'interno (\cref{fig.bottom}).
Questi sono i \emph{sensori del terreno}.
Essi mandano un impulso di luce infrarossa e misurano la quantità di luce che viene riflessa.
Su un tavolo di colore chiaro, c'è molta luce riflessa, però quando la parte anteriore del robot supera l'estremità del tavolo, ci sarà molta meno luce riflessa. Quando viene
rilevata questa situazione vogliamo che il robot si fermi.

\trickbox{
Utilizza un tavolo colorato con un colore chiaro, però non utilizzare un tavolo di vetro, in quanto è probabile che, non riflettendo la luce, Thymio creda di non essere su un tavolo!}

Trascina il blocco evento sensori terreno \blksm{event-ground} nel programma. Ci
sono due piccoli quadrati nella parte superiore dell'icona. Cliccando i quadrati
il colore cambia dal grigio al bianco al rosso ed infine di nuovo al grigio.
Per questo blocco, i significati di questi colori sono:

\begin{itemize}
\item \textbf{Grigio}: Il sensore non è considerato.
\item \textbf{Rosso}: l'evento si verifica se c'è molta luce riflessa.
\item \textbf{Bianco}: l'evento si verifica se c'è poca luce riflessa.
\end{itemize}

\trickbox[Informazione]{I colori grigio, rosso e bianco usati in un blocco sono
arbitrari e se ne sarebbero potuti scegliere altri.}

Per fare in modo che il robot si fermi al bordo del tavolo quando c'è poca luce riflessa, fai clic sui quadrati fino a che non sono bianchi e crea la seguente coppia evento-azione:
\blkc{dont-fall}

\begin{figure}
\begin{center}
\gr{bottom}{0.6}
\caption{La parte inferiore del Thymio con i due sensori del terreno}\label{fig.bottom}
\end{center}
\end{figure}

Posiziona il robot in prossimità del bordo del tavolo, di fronte al bordo e tocca il pulsante frontale.
Il robot deve andare avanti e fermarsi prima di cadere fuori dal tavolo.

\exercisebox{\thechapter.1}{
Esperimenta con la velocità del robot.
Alla massima velocità, il robot è ancora in grado di fermarsi e non cadere fuori dal tavolo?
Se no, a partire da che velocità il robot inizia a cadere?
Si può evitare che il robot cada quando sta andando indietro?
}

\trickbox{
Quando ho eseguito il programma, il robot \emph{è} caduto. Il motivo è stato
che la mia bella scrivania ha un bordo arrotondato; prima che il robot
rilevasse un basso livello di luce, non era già più stabile e si è ribaltato. La
soluzione è stata quella di mettere una striscia di nastro nero vicino al
bordo della scrivania.
}
