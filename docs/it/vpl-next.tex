% !TeX root = vpl.tex

\chap{Ed ora?}\label{ch.next}

Questo tutorial ha introdotto il robot Thymio e l'ambiente Aseba/VPL. L'ambiente VPL con la sua semplice programmazione visuale è stato pensato per i principianti. Per sviluppare programmi più avanzati per il robot dovrai imparare ad usare l'ambiente Aseba Studio (\cref{fig.studio}).

\begin{figure}[hbt]
\begin{center}
\gr{studio}{.8}
\caption{l'ambiente Aseba Studio}\label{fig.studio}
\end{center}
\end{figure}

La Programmazione in Aseba Studio è sempre basata sui concetti di eventi ed azioni.
poichè i programmi VPL sono tradotti in programmi testuali,
tutto quello che hai imparato in questo tutorial è disponibile anche in Studio
dove sono disponibili  molte altre funzionalità di programmazione:
\begin{itemize}
\item Puoi controllare esattamente quando un evento causa un'azione, in dipendenza, ad esempio, dall'ammontare di luce riflessa misurata da un sensore del terreno o della distanza da un sensore orizzontale.
\item Puoi specificare che una singola azione è composta da diverse operazioni differenti: 
controllare i motori, cambiare lo stato, definire soglie, accendere e spegnere luci, ecc...
\item Hai la flessibilità  di un linguaggio di programmazione completo con variabili, espressioni, direttive di controllo.
\end{itemize}

Aseba Studio vi dà accesso a funzionalità del robot Thymio che non sono disponibili in VPL:

\begin{itemize}
\item Puoi controllare tutte le luci come ad esempio il cerchio di luci che contorna i bottoni.
\item Hai più flessibilità nel sintetizzare suoni.
\item C'è un sensore di temperatura.
\item Invece di rilevare semplicemente gli urti, gli accellerometri possono sentire la forza di gravità e ogni
variazione di velocità nelle tre dimensioni.
\item Si può anche usare un telecomando per impartire comandi al robot.
\end{itemize}
Quando stai lavorando con Aseba Studio puoi aprire il VPL cliccando sul bottone \bu{Lancia VPL} nel tab \emph{Strumenti} nella parte a sinistra in basso della finestra.
Puoi importare programmi VPL direttamente in Aseba Studio semplicemente aprendo il file.

Per usare Aseba Studio, parti dalla pagina \emph{Programmare Thymio II} al link:\\
\url{https://aseba.wikidot.com/it:thymioprogram}\\
e segui il link \emph{Ambiente di programmazione testuale}.

puoi trovare tanti progetti interessanti con questo link:\\ \url{https://aseba.wikidot.com/it:thymioexamples}.

\vspace{4em}

\infobox{Divertiti e impara tante cose!}{Grazie per aver letto questo tutorial!}{yellow!10}{\bcetoile}

