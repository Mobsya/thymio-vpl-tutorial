% !TeX root = vpl.tex

\chapter*{Prefazione}

\sect{Che cosa è un Robot?}

Stai guidando la bicicletta e improvvisamente vedi che la strada inizia
ad andare in salita. Pedali più veloce per fornire più potenza alle ruote in modo
che la bicicletta non rallenti. Quando raggiungi la cima della collina e
cominci ad andare in discesa, stringi la leva del freno. Questo fa sì che due pezzi di gomma
(i pattini) vengano premuti contro la ruota e la bicicletta rallenti.
Quando guidi una bicicletta, i tuoi occhi sono i \textit{sensori}  che percepiscono ciò che
succede nel mondo. Quando questi sensori---i tuoi occhi---rilevano un
\textit{evento} come una curva della strada, esegui un'\textit{azione}, come spostare il manubrio a sinistra o a destra.

In una macchina, ci sono sensori che \textit{misurano} ciò che sta accadendo nel
mondo. Il tachimetro misura  quanto velocemente sta andando la vettura; se si rileva una velocità superiore al limite, si potrebbe dire al conducente che
sta andando troppo veloce. In risposta, egli può eseguire un'azione, come ad esempio
premere sul pedale del freno per rallentare la macchina. L'indicatore del livello carburante
misura quanto carburante rimane in macchina; se si vede che il livello è troppo
basso, si può dire al conducente di trovare un distributore di benzina. In risposta, egli può
eseguire un'azione: sollevare la leva dell'indicatore di direzione per indicare una svolta a destra
e girare il volante per andare nella stazione di servizio.

Il pilota della bicicletta e il conducente della vettura ricevono i dati dai sensori, decidono quali azioni intraprendere e fanno in modo che vengano eseguite. Un \textit{robot} è un sistema in cui questo processo---riceve i dati,
decidere un'azione, eseguire l'azione---è svolta da un
elaboratore, di solito senza la partecipazione di un essere umano.

\sect{Il robot Thymio-II e l'ambiente Aseba VPL}

Il Thymio II è un piccolo robot destinato a fini didattici
(\cref{fig.front}). Il robot include sensori in grado di misurare
la luce, il suono e la distanza, ed è in grado di rilevare quando i pulsanti vengono toccati e
quando il corpo del robot viene urtato. L'azione più importante che si può
effettuare è quella di spostarsi utilizzando due ruote, ciascuna alimentata da un proprio motore.
Altre azioni includono la generazione di suoni e accendere e spegnere luci.

Nel resto di questo documento, il robot Thymio~II sarà chiamato semplicemente Thymio.
Si farà sempre riferimento alla versione II del robot.

ASEBA è un ambiente di programmazione per i piccoli robot mobili come Thymio.
VPL è un componente di ASEBA di \textit{programmazione visuale} che è stato progettato per programmare Thymio in modo semplice attraverso blocchi di eventi e di azioni.
Questo tutorial presuppone che il programma ASEBA sia stato installato sul vostro computer; se non è così, andate su \url{https://aseba.wikidot.com/it:downloadinstall}, selezionare il sistema operativo, scaricate il programma ed installatelo.

\sect{Schede di Riferimento}
Troverete utile stampare le schede di riferimento, che sono nello stesso file zip con questo documento o disponibili online\footnote{Il link può essere trovato sulla pagina \textit{Programmare Thymio II}: \\
\url{https://aseba.wikidot.com/it:thymioprogram}.}:
\begin{itemize}
\item Una singola pagina che riassume eventi e azioni (\href{https://aseba.wdfiles.com/local--files/it:thymioprogram/thymio-vpl-ref-card-it.pdf}{online}).
\item Una doppia pagina che può essere piegata in tre per creare una comoda carta da tenere a portata di mano.
Riassume l'interfaccia del VPL, i blocchi eventi e azioni, e include esempi di programmi (\href{https://aseba.wdfiles.com/local--files/it:thymioprogram/thymio-vpl-folding-ref-card-it.pdf}{online})\\
\end{itemize}

\sect{Sommario dei capitoli}

Ecco una panoramica dei capitoli di questo tutorial.
Per ogni capitolo, diamo l'argomento principale coperto, nonché le liste
dei blocchi di eventi e azioni che vengono introdotti.
È possibile utilizzare la panoramica per decidere quali capitoli usare
e in quale ordine, seguendo le linee guida riportate sotto.

{\centering \textbf{Capitolo \ref{ch.intro}}\\}
\textbf{Argomenti}: Il Robot Thymio, ambiente di programmazione.

\textbf{Eventi}: Bottoni \hfill \textbf{Azioni}: Colori della parte superiore

\blkmed{event-buttons} \hfill \blkmed{action-colors-up-white}

\bigskip

{\centering \textbf{Capitolo \ref{ch.colors}}\\}
\textbf{Argomenti}: Coppie evento-azione.

\textbf{Eventi}: Bottoni \hfill \textbf{Azioni}: Colori della parte superiore, colori della parte inferiore.

\blkmed{event-buttons} \hfill \blkmed{action-colors-up-white} \quad \blkmed{action-colors-down-white}

\bigskip

{\centering \textbf{Capitolo \ref{ch.moving}}\\}
\textbf{Argomenti}: Muoversi, sentire.

\textbf{Eventi}: Bottoni, sensori del terreno \hfill \textbf{Azioni}: Motori

\blkmed{event-buttons} \quad \blkmed{event-ground} \hfill  \blkmed{action-motors}

\bigskip

{\centering \textbf{Capitolo \ref{ch.pet}}\\}
\textbf{Argomenti}: Controllo a retroazione (feedback), velocità dei motori.

\textbf{Eventi}: Sensori frontali \hfill \textbf{Azioni}: Motori

\blkmed{event-prox} \hfill \blkmed{action-motors}

\bigskip

{\centering \textbf{Capitolo \ref{ch.line}}\\}
\textbf{Argomenti}: Seguire una linea.

\textbf{Eventi}: Sensori del terreno \hfill \textbf{Azioni}: Motori

\blkmed{event-ground} \hfill \blkmed{action-motors}

\bigskip

{\centering \textbf{Capitolo \ref{ch.bells}}\\}
\textbf{Argomenti}: Suoni, urti.

\textbf{Eventi}: Urto, Rumore intenso \hfill \textbf{Azioni}: Musica, colori della parte superiore, colori della parte inferiore

\blkmed{event-tap} \quad \blkmed{event-clap} \hfill \blkmed{action-music} \quad \blkmed{action-colors-up-white} \quad \blkmed{action-colors-down-white}

% Tap \blkmed{event-tap}, clap \blkmed{event-clap}.\hfill\
% Music \blkmed{action-music},
% Top colours \blkmed{action-colors-up-white},
% bottom colours \blkmed{action-colors-down-white}.
\bigskip

{\centering \textbf{Capitolo \ref{ch.time}}\\}
\textbf{Argomenti}: Timer.

\textbf{Eventi}: Timer scaduto \hfill \textbf{Azioni}: Predisponi timer

\blkmed{event-timer} \hfill \blkmed{action-timer}

\bigskip

{\centering \textbf{Capitolo \ref{ch.states}}\\}
\textbf{Argomenti}: Stati.

\textbf{Eventi}: Stato associato ad un evento \hfill \textbf{Azioni}: Cambio stato

\blkmed{tap-turn-on-state-only} \hfill \blkmed{action-states}

\bigskip

{\centering \textbf{Capitolo \ref{ch.counting}}\\}
\textbf{Argomenti}: Contare, aritmetica binaria.

\textbf{Eventi}: Stato associato ad un evento  \hfill \textbf{Azioni}: Cambio stato (avanzato)

\blkmed{tap-turn-on-state-only} \hfill \blkmed{action-states}

\bigskip

{\centering \textbf{Capitolo \ref{ch.next}}\\}
\textbf{Argomenti}: ambiente Aseba studio.

\bigskip

%{\centering \textbf{Appendice \ref{a.tips}}\\}
%\textbf{Argomenti}: Esplorare e sperimentare,
%costruire un programma, risoluzione dei problemi.


\sect{Linee Guida}

\textbf{I Capitoli~\ref{ch.intro}--\ref{ch.colors}} sono un'introduzione essenziale
al robot, all'ambiente e al principale costrutto di programmazione---la
coppia evento-azione.

\textbf{I Capitoli~\ref{ch.moving}--\ref{ch.line}} presentano gli eventi, le azioni e gli algoritmi
per la costruzione di robot mobili autonomi e dovrebbero essere al centro di ogni
attività che preveda l'utilizzo del sistema.

\textbf{Il Capitolo~\ref{ch.bells}} descrive le caratteristiche del robot che possono
essere divertenti da usare, ma non sono essenziali. È possibile saltare il capitolo o lo si può
utilizzare subito dopo i capitoli introduttivi
per un'introduzione più dolce alla robotica.

\textbf{Il Capitolo~\ref{ch.time}} mostra come utilizzare eventi a tempo.
Questo è un argomento importante, ma è indipendente dagli altri
e può essere saltato se il tempo manca.

\textbf{Il Capitolo~\ref{ch.states}} utilizza funzioni disponibili nella modalità avanzata del VPL.
Le macchine a stati sono un costrutto fondamentale utilizzato in robotica e questo materiale
deve essere utilizzato in attività con studenti che siano sufficientemente maturi per capire. Il capitolo~\ref{ch.counting} è un capitolo opzionale che mostra
come utilizzare gli stati per implementare conti aritmetici.

\textbf{Il Capitolo~\ref{ch.next}} punta al passo successivo: l'utilizzo
dell'ambiente testuale Studio, che offre significativamente più supporto per
lo sviluppo di robot di quanto non faccia l'ambiente VPL.

%\textbf{L'Appendice~\ref{a.tips}} fornisce una guida per insegnanti e tutor
%di studenti che stanno imparando la Robotica con Thymio e VPL.
%Inizia incoraggiando l'esplorazione e la sperimentazione
%per acquisire familiarità con i blocchi evento e azione del VPL.
%La sessione succesiva si concentra sulle buone pratiche di programmazione;
%anche se VPL è un ambiente visivo semplice,
%gli studenti potranno già beneficiare molto dal lavorare con
%pratiche professionali.
%L'ultima sezione elenca alcuni problemi che si possono incontrare
%e dà suggerimenti su come superarli.
