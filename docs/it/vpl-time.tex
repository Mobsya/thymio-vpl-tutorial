% !TeX root = vpl.tex
\chap{Un tempo per amare}\label{ch.time}

Nel \cref{ch.pet}  abbiamo programmato un robot da compagnia a cui piaciamo o non piaciamo.
Consideriamo un comportamento più avanzato: un animale timido che
non può decidersi se gli piaciamo o no. Inizialmente,
l'animale si girerà verso la nostra mano cercando di raggiungerlla, ma poi si terrà a distanza. Dopo un po'
cambierà idea e tornerà indietro in direzione
della nostra mano.

{\raggedleft \hfill File di programma \bu{shy.aesl}}

Il comportamento del robot è il seguente. Quando il tasto destro viene toccato il robot gira a destra. Quando rileva la tua mano,
gira a sinistra, ma dopo un po' si rammarica per la sua decisione e gira
indietro. Noi sappiamo come costruire coppie evento-azione per la curva iniziale:
\blkc{start-turn} e per allontanarsi quando viene rilevata la mano:
\blkc{turn-away}

Il comportamento di tornare indietro "dopo un po'" può essere suddiviso in
due parti:

\begin{itemize}

\item \emph{Quando}  il robot inizia ad allontanarsi $\rightarrow$
\emph{avviare un timer} per due secondi.

\item \emph{Quando}  il timer scende a zero $\rightarrow$ \emph{girare}
a destra.

\end{itemize}

Abbiamo bisogno di una nuova \emph{azione} per la prima parte e un nuovo
\emph{evento} per la seconda parte.

L'azione è quella di impostare un \emph{timer}, che è come una sveglia
\blksm{action-timer}. Normalmente, si imposta una sveglia per un tempo assoluto,
ma quando ho impostato la sveglia nel mio smartphone per un tempo assoluto come le ore
7:00, mi dice il tempo relativo: "allarme impostato tra 11 ore e 23
minuti da oggi". Il blocco del timer funziona allo stesso modo: si imposta il
timer per un certo numero di secondi da quando si verifica l'evento e poi l'azione accade. Il timer può essere impostato per un massimo di quattro secondi. fare clic su
qualsiasi punto all'interno del cerchio nero che rappresenta il quadrante dell'orologio (ma non
sul cerchio nero stesso). Ci sarà una breve animazione e poi
la quantità di tempo fino a quando l'allarme scatta sarà mostrata di colore blu.

\newpage
La coppia evento-azione per questa prima parte del comportamento è:
\blkc{turn-clock}

Il timer è impostato per due secondi. Quando l'evento che rileva la presenza della mano
si verifica, ci saranno due azioni: ruotare il robot di fianco
e l'impostazione del timer.

La seconda parte del comportamento utilizza un evento che si verifica quando l'allarme si spegne,
cioè, quando la quantità di tempo impostato sul timer scende a zero. il blocco dell'evento
 \blksm{event-timer} mostra una sveglia che squilla.

Ecco la coppia evento-azione per fare girare il robot di nuovo a
destra quando il timer scende: \blkc{turn-back}

\exercisebox{\thechapter.1}{
Scrivi un programma che faccia muovere il robot in avanti alla massima velocità per
tre secondi quando il pulsante in avanti viene toccato; quindi corra
all'indietro. Aggiungere una coppia evento-azione per arrestare il robot toccando il
pulsante centrale.
}
