\chap{Let's Get Moving}

\sect{Move forwards and backwards}

The Thymio-II robot has two motors, one connected to each wheel. The
motors can be run forwards and backwards, causing the robot to move
forwards and backwards, and to make turns. Let us start with a simple
project to learn about the motors.

The motor action block \blk{action-motors} displays a small image of the
robot in the center together with two sliders. The sliders control the
speed of the motors, one slider for the left motor and one for the right
motor. When the white square is centered in the slider, the
corresponding motor is off. You can drag a square up to increase the
forward speed and down to increase the backwards speed.

Write a program to move the robot forward when the front button
is touched and backwards when the back button is touched.

{\raggedleft \hfill Program file \bu{moving}.}

We need two event-action pairs (Figure~\ref{fig.nostop}). Drag and drop
the event and action blocks and set the sliders equally for the left and
right motors, half-way up for forward and half-way down for backwards.

Run the program by clicking \blksm{run} and touch the buttons to
make the robot go forwards and backwards.

\begin{figure}
\begin{center}
\gr{no-stop-motors}{.4}
\caption{Moving forward and backwards}\label{fig.nostop}
\end{center}
\end{figure}

\sect{Stop the robot}

\textbf{Help!} I can't stop the robot's motors!

Click on the icon \blksm{stop} to stop the robot.

Let us fix this problem by adding an event-action pair \blk{stop-motors}
that will stop the motors when the center button is touched. When you
drag the motor action block into the program area, it is already set
with the sliders in the middle to turn off the motors.

\sect{Don't fall off the table}

If your robot moves on the floor, at worst, it might hit a wall or pull
its cable out, but if you place your robot on a table, it might fall
off, crash and break! Let us arrange for the robot to stop when it
reaches the end of a table.

\centeredbox{\textbf{Warning! Whenever the robot moves on a table, be
ready to catch it in case it does fall off.}}

We assume that the table is colored with a light color (my desk is light
gray). If your table is colored with a dark color, you will have to
modify the program.

Turn the Thymio-II on its back. You will see at the front two small
black rectangles with optical elements inside (Figure~\ref{fig.bottom}).
These are the \emph{ground sensors} that can detect if the amount of
reflected light is large or small. On a light-colored table, there is a
lot of reflected light, but when the front of the robot goes past the
end of the table, there will be much less reflected light. When this is
detected we want the robot to stop.

Drag the ground sensor event \blk{event-ground} into the program. There
are two small squares at the top of the icon. Clicking the squares
changes them from gray to white to red and finally back to gray. These
meanings of the colors are:

\begin{itemize}
\item \textbf{Gray}: The sensor is not used.
\item \textbf{Red}: An event occurs when there is a lot of reflected light.
\item \textbf{White}: An event occurs when there is little reflected light.
\end{itemize}

To cause the robot to stop when there is little reflected light, click
the squares until they are white and create the event-action pair
\blk{dont-fall}.

\begin{figure}
\begin{center}
\gr{bottom}{0.6}
\caption{The bottom of the Thymio-II with two ground sensors at the
 front}\label{fig.bottom}
\end{center}
\end{figure}

Place the robot not far from the edge of the table and touch the front
button. The robot should stop before falling off the table.

When I ran the program, the robot \emph{did} fall off. The reason was
that my nice desk has a rounded edge; by the time that the robot
detected a low light level, it was no longer stable and tipped over. The
solution was to place a strip of black electrician's tape close to the
edge of the desk.

\sect{Exercise \thechapter.1}

Experiment with the speed of the robot. At maximum speed, is the robot
still able to stop and not fall off the table?
