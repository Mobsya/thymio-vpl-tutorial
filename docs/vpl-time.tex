\chap{A Time to Like}

In Chapter~\ref{ch.pet} we programmed a pet robot who either did or did
not like us. Let us consider a more advanced behavior: a shy pet who
can't make up its mind whether it likes us or dislikes us. Initially,
the pet will turn towards our outreached hand, but then it will turn
away. After a while, it will reconsider and turn back in the direction
of our hand.

{\raggedleft \hfill Program file \bu{shy}.}

The behavior of the robot is as follows. When the right button is
touched the robot turns to the right. When it detects your hand, it
turns to the left but after a while it regrets its decision and turns
back. We know how to construct event-action pairs for the initial turn
\blk{start-turn} and for turning away when your hand is detected
\blk{turn-away}.

The behavior of turning back ``after a while'' can be broken down into
two event-action pairs:

\begin{itemize}

\item \emph{When} the robot starts to turn away $\rightarrow$
\emph{start a timer} for two seconds.

\item \emph{When} the timer runs down to zero $\rightarrow$ \emph{turn}
to the right.

\end{itemize}

We need a new \emph{action} for the first behavior and a new
\emph{event} for the second behavior.

The action is to set an \emph{timer}, which is like an alarm clock
\blk{action-timer}. Normally, we set an alarm clock to an absolute time,
but when I set the alarm clock in my smartphone to an absolute time like
07:00, it tells me the relative time: ``Alarm set for 11 hours, and 23
minutes from now.'' The timer block works the same way: you set the
timer for a certain number of seconds from when the event occurs and the
action happens. The timer can be set for up to four seconds. Click
anywhere within the black circle showing the face of the clock (but not
on the black circle itself). There will be a short animation and then
the amount of time until the alarm will be colored blue.

The event-action pair for the first behavior above is
\blk{turn-clock}.

The timer is set for two seconds. When the event of detecting your
hand occurs, there will be two actions: turning the robot to the left
and setting the timer.

The second behavior needs an event that occurs when the alarm goes off,
that is, when the amount of time set on the timer runs down to zero. The
event block \blk{event-timer} shows a ringing alarm clock.

The event-action pair is \blk{turn-back}, where the robot turns back to
the right when the timer runs down.

\sect{Exercise \thechapter.1}

Write a program that causes the robot to move forward at top speed for
three seconds when the forward button is touched; then it runs
backwards. Add an event-action pair to stop the run by touching the
center button.
