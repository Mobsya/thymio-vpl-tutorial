%
%    Aseba/VPL Reference Card
%
%  Copyright 2013 by Mordechai (Moti) Ben-Ari.
%  This work is licensed under the Creative Commons
%  Attribution-ShareAlike 3.0 License. 
%  To view a copy of this license, 
%  visit http://creativecommons.org/licenses/by-sa/3.0/; 
%  or, (b) send a letter to Creative Commons, 
%  543 Howard Street, 5th Floor, San Francisco, California, 94105, USA.
%
\documentclass[a4paper]{article}
\usepackage[a4paper,margin=.9cm]{geometry}
%\usepackage{mathpazo}
\usepackage[utf8x]{inputenc}
\usepackage[T1]{fontenc}
\usepackage{graphicx,url}
\usepackage{tabularx}
\usepackage[usenames,dvipsnames]{xcolor}
\usepackage{colortbl}
\usepackage[colorlinks=true,linkcolor=Blue,urlcolor=Blue,citecolor=Blue]{hyperref}
\usepackage{libertine}

\graphicspath{{../../images/}}

\renewcommand*\familydefault{\sfdefault}

\renewcommand{\baselinestretch}{1.1}
\setlength{\parskip}{0.3\baselineskip plus 1pt minus 1pt}
\parindent=0pt

\newcommand*{\blk}[1]{\raisebox{-40pt}%
{\includegraphics[height=50pt,keepaspectratio=true]{#1}}}

\newcommand*{\blkbig}[1]{\raisebox{-50pt}%
{\includegraphics[height=60pt,keepaspectratio=true]{#1}}}

\newcolumntype{L}{>{\raggedright\arraybackslash}X}

\colorlet{advancedmode}{orange!20}

\begin{document}
\thispagestyle{empty}

\fontsize{15pt}{18pt}\selectfont

\begin{center}
{\Huge \textbf{VPL Referenzkarte}}
\end{center}

\bigskip

\begin{tabularx}{\textwidth}{lL@{\hspace{1cm}}lL}

\multicolumn{2}{l}{\textbf{Ereignisse}} & \multicolumn{2}{l}{\textbf{Aktionen}} \\[.4cm]

\blk{event-buttons}  & \textbf{Gedrückte Knöpfe}

grau: beachte den Knopf nicht, rot: drücke den Knopf &

\blk{action-motors} & \textbf{Motorengeschwindigkeit}

Geschwindigkeit des linken und rechten Motors.%
%
\\[.6cm]

\blk{event-prox} & \textbf{Hinderniserkennung}

grau: ignorieren, rot: Nahes Objekt, weiss: Entferntes Obj. &

\blk{action-colors-up} & \textbf{Farbauswahl Oberseite}

Farbmischung aus rot, grün, blau für die Oberseite%
%
\\[.6cm]

\blk{event-ground} & \textbf{Bodensensoren}

grau: Detektor ignorieren, rot: Boden, white: kein Boden &

\blk{action-colors-down} & \textbf{Farbauswahl Unterseite}

Farbmischung aus rot, grün und blau für die Unterseite.%
%
\\[.6cm]

\blk{event-tap} & \textbf{Klopfen auf den Roboter}

Der Roboter wird erschüttert&

\blk{action-music} & \textbf{Spiele Musik}

 Linien:~Tonhöhe, Farbe:~Tonlänge. %
%
\\[.6cm]

\blk{event-clap} & \textbf{Klatschen}

Der Roboter erkennt laute Geräusche&

\blk{action-timer} & \textbf{Starte den Timer}

Das Timerereignis findet nach der gesetzten Zeit statt.%
%
\\[.6cm]

\blk{event-timer} & \textbf{Abgelaufener Timer}

Timer ist abgelaufen.&

\cellcolor{advancedmode} \blk{action-states} & \cellcolor{advancedmode} \textbf{Zustand des Roboters}

Wähle den 4-bit Zustand des Roboters.\\

\end{tabularx}

\vfill

\begin{tabularx}{\textwidth}{l@{\hspace{.7cm}}L}

\multicolumn{2}{l}{\textbf{Erstelle dein Programm}} \\[.4cm]


\blkbig{event-action-pair-empty} & Ziehe (drag and drop) Ereignisse in das linke und Aktionen in das rechte Quadrat. Findet das Ereignis statt, führt der Roboter die Aktion aus.
\\

\end{tabularx}

\vfill

\begin{tabularx}{\textwidth}{l@{\hspace{.7cm}}Ll}

\multicolumn{3}{l}{\textbf{Sensoren werden mit einem Ereignis kombiniert UND sind Teil eines Ereignisses}} \\[.4cm]

\blk{sensor-and-button} & Wenn zwei Sensoren ausgewählt werden, müssen beide Bedingungen wahr sein, damit das Ereignis stattfindet.

Links \textbf{und} rechts müssen gedrückt werden/ein Objekt erkennen. &

\blk{sensor-and-prox}\\

\end{tabularx}

\vfill

{\normalsize M. Ben-Ari, S. Magnenat, J. Shin - \href{http://creativecommons.org/licenses/by-sa/3.0/}{CC-BY-SA}}\hfill\colorbox{advancedmode}{Funktionen des fortgeschrittenen Modus}

\end{document}
