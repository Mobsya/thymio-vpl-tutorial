%
%    Aseba/VPL Reference Card
%
%  Copyright 2013-14 by Mordechai (Moti) Ben-Ari.
%  This work is licensed under the Creative Commons
%  Attribution-ShareAlike 3.0 License. 
%  To view a copy of this license, 
%  visit http://creativecommons.org/licenses/by-sa/3.0/; 
%  or, (b) send a letter to Creative Commons, 
%  543 Howard Street, 5th Floor, San Francisco, California, 94105, USA.
%
\documentclass[a4paper]{leaflet}
\usepackage{mathptmx}
\usepackage{url}
\usepackage{graphicx}
\graphicspath{{../../images/}}
\renewcommand{\baselinestretch}{1.1}
\renewcommand{\arraystretch}{1.5}
\textheight=190mm

\urlstyle{sf}

\newcommand{\sct}[1]{\subsubsection{#1}\mbox{}\\}

\CutLine*{1}\CutLine*{3}\CutLine*{4}\CutLine*{6}

\newcommand*{\blk}[2][-20]{\raisebox{#1pt}%
{\includegraphics[height=30pt,keepaspectratio=true]{#2}}}

\newcommand*{\blkwide}[2][-50]{\raisebox{#1pt}%
{\includegraphics[width=.4\textwidth,keepaspectratio=true]{#2}}}

\begin{document}
\thispagestyle{empty}
\begin{center}
\begin{bfseries}
\begin{large}
VPL Reference Card (Version 1.5)
\end{large}

\medskip

Moti Ben-Ari, St\'{e}phane Magnenat, Jiwon Shin
\end{bfseries}
\end{center}

\vspace*{-1ex}
{\scriptsize Copyright 2013--14 by Moti Ben-Ari, St\'{e}phane Magnenat and
Jiwon Shin. This work is licensed under the Creative Commons
Attribution-ShareAlike 3.0 License. To view a copy of this license,
visit \url{http://creativecommons.org/licenses/by-sa/3.0/}; or, (b) send
a letter to Creative Commons, 543 Howard Street, 5th Floor, San
Francisco, California, 94105, USA.}


\sct{VPL user interface}

\smallskip

\begin{tabular}{lp{.73\textwidth}}
\blk{new} & Clear the editor; return to simple mode.\\

\blk{open} & Open an existing program.\\

\blk{save} & Save the program.\\

\blk{saveas} & Save the program with a new name.\\

\blk{undo} & Undo the previous operation.\\

\blk{redo} & Redo the previous undo.\\

\blk{run} & Load and run the program on the robot.\\

\blk{stop} & Stop the program on the robot.\\

\blk{advanced} & Change to advanced mode.\\

\blk{info1} & Display the VPL documentation.\\

\blk{export} & Export the program in a graphics format.

\end{tabular}

\newpage

\begin{tabular}{lp{.5\textwidth}}

\blk{x} & Delete this event-actions pair.\\

&\\

\blk{plus} & Add an event-actions pair.\\

&\\

Drag and drop & Move an event-actions pair.\\

&\\

Control\\drag and drop & Copy an event-actions pair.\\
\end{tabular}


\newpage

\sct{Event blocks}

\begin{tabular}{lp{.73\textwidth}}

\blk{forward} & Buttons are touched.\par
Red buttons are active.\\
[-0.5cm]

&\\

\blk[-25]{event-rc-arrow} & (Advanced mode) Remote control arrows.\par
Signal from the remote control, arrows buttons.\\
[-0.5cm]

&\\

\blk[-25]{event-rc-keypad} & (Advanced mode) Remote control keypad.\par
Signal from the remote control, keypad buttons.\\
[-0.5cm]

&\\

\blk{event-prox} & Horizontal sensors detect an object.\par
White = an object is detected.\par Black = No object is detected.\\
[-0.5cm]

&\\

\blk[-25]{event-prox-advanced} & (Advanced mode) As above, but the slides can be
used to set the thresholds.\\
[-0.5cm]

&\\
\blk[-25]{event-prox-ground} & Ground sensors detect light or dark.\par 
White = a lot of reflected light is detected.\par
Black = little reflected light is detected.\\
[-0.5cm]

&\\

\blk[-25]{event-prox-ground-advanced} & (Advanced mode) As above, but the slides can be
used to set the thresholds.\\
[-0.5cm]

&\\

\blk{event-tap} & The robot has been tapped.\\
[-0.5cm]

&\\

\blk{event-tap-advanced} & (Advanced mode) The robot has been tapped.\\
[-0.5cm]

&\\

\blk{event-pitch} & (Advanced mode) The pitch (forwards and backwards) of the robot
is within the red segment.\\
[-0.5cm]

&\\

\blk{event-roll} & (Advanced mode) The roll (left and right) of the robot
is within the red segment.\\
\end{tabular}


\begin{tabular}{lp{.73\textwidth}}

\blk{event-clap} & The robot detects a loud noise.\\

&\\

\blk{event-timer} & (Advanced mode) The timer has counted down to zero.\\

\end{tabular}

\bigskip
\bigskip

\sct{Action blocks}

\begin{tabular}{lp{.73\textwidth}}

\blk[-27]{action-motors} & Set the power of the left and right motors.\par
Move a slider up (forward)\par
or down (backwards).\\

&\\

\blk{action-colors-up} & Set the colour of the top of the robot.\par
Move the sliders to mix red, green and blue.\\

&\\

\blk{action-colors-down} & Set the colour of the bottom of the robot.\par
Move the sliders to mix red, green and blue.\\

&\\

\blk[-27]{action-music} & Play music.\par
Click on a bar to set a note.\par
White notes are longer than black notes.\par
Click on a note to change white $\leftrightarrow$ black.\par
Click again to silence this note.\\

&\\

\blk{action-timer} & (Advanced mode) Start a timer in the range of 0--4 seconds.\par
Click on the clock face to set the time.\\

&\\

\blk[-30]{states2} & (Advanced mode) Set the current state.\par
Grey = do not change the value.\par
White = set to 0.\par
Yellow = set to 1.\\

\end{tabular}


\newpage

\sct{Examples}


\begin{tabular}{lp{.5\textwidth}}

\blkwide[-20]{dont-fall} &
The robot stops when the end of a table is detected.\\

&\\

\blkwide{likes-turns} & 
The robot turns towards you when you
are detected by the rightmost or leftmost sensor.\\

&\\

\blkwide{hates} &
The robot turns away you when you
are detected by the rightmost or leftmost sensor.\\

&\\

\blkwide{line-controller} & The robot moved off a line on the floor and
turns back.\\

\blkwide[-35]{clap-lights} & \mbox{}\par
A clap changes the bottom colour of the robot.\\

\blkwide[-35]{turn-back} & \mbox{}\par
When the timer counts down to zero, turn right.\\


\end{tabular}

\newpage

\sct{Examples with multiple actions}

\vspace*{-4ex}


\begin{tabular}{lp{.5\textwidth}}

\blkwide[-35]{colors-multiple} & \mbox{}\par
An event-actions pair with multiple actions.\\

\blkwide[-35]{tap-on-off1} & \mbox{}\par
An event-actions pair that
depends on the current state and changes the state (advanced mode).\\

\blkwide[-35]{turn-clock} & \mbox{}\par
When the center sensor detects an object, turn left and set a two-second timer.\\

\blkwide[-35]{tap-on-off2} & \mbox{}\par
Tap changes the first part of the state from
1 to 0 and turns off the top lights.\\

\blkwide[-35]{tap-on-off1} & \mbox{}\par
Tap changes the first part of the state
from 0 to 1 and turns the top color magenta.\\

\end{tabular}

\vfill

\sct{References}

\begin{itemize}
\item VPL reference:\\
\url{https://www.thymio.org/en:thymiovpl}.

\item VPL tutorial:\\
\url{https://www.thymio.org/local--files/en:visualprogramming/thymio-vpl-tutorial-en.zip}.

\item Sources of this document in ref-cards at:\\
\url{http://github.com/aseba-community/thymio-vpl-tutorial/}.

\end{itemize}
\end{document}
