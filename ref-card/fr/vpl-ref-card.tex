%
%    Aseba/VPL Reference Card
%
%  Copyright 2013 by Mordechai (Moti) Ben-Ari.
%  This work is licensed under the Creative Commons
%  Attribution-ShareAlike 3.0 License. 
%  To view a copy of this license, 
%  visit http://creativecommons.org/licenses/by-sa/3.0/; 
%  or, (b) send a letter to Creative Commons, 
%  543 Howard Street, 5th Floor, San Francisco, California, 94105, USA.
%
\documentclass[a4paper]{article}
\usepackage[a4paper,margin=.9cm]{geometry}
%\usepackage{mathpazo}
\usepackage[utf8x]{inputenc}
\usepackage[T1]{fontenc}
\usepackage{graphicx,url}
\usepackage{tabularx}
\usepackage[usenames,dvipsnames]{xcolor}
\usepackage{colortbl}
\usepackage[colorlinks=true,linkcolor=Blue,urlcolor=Blue,citecolor=Blue]{hyperref}
\usepackage{libertine}

\graphicspath{{../../images/}}

\renewcommand*\familydefault{\sfdefault}

\renewcommand{\baselinestretch}{1.1}
\setlength{\parskip}{0.3\baselineskip plus 1pt minus 1pt}
\parindent=0pt

\newcommand*{\blk}[1]{\raisebox{-40pt}%
{\includegraphics[height=50pt,keepaspectratio=true]{#1}}}

\newcommand*{\blkbig}[1]{\raisebox{-50pt}%
{\includegraphics[height=60pt,keepaspectratio=true]{#1}}}

\newcolumntype{L}{>{\raggedright\arraybackslash}X}

\colorlet{advancedmode}{orange!20}

\begin{document}
\thispagestyle{empty}

\fontsize{15pt}{18pt}\selectfont

\begin{center}
{\Huge \textbf{Carte de référence VPL}}
\end{center}

\bigskip

\begin{tabularx}{\textwidth}{lL@{\hspace{1cm}}lL}

\multicolumn{2}{l}{\textbf{Événements}} & \multicolumn{2}{l}{\textbf{Actions}} \\[.4cm]

\blk{event-buttons} & \textbf{Boutons touchés}

gris: ignore le bouton, rouge: doit être touché &

\blk{action-motors} & \textbf{Vitesse des moteurs}

Défini la vitesse des moteurs et roues gauches et droites.%
%
\\[.6cm]

\blk{event-prox} & \textbf{Détecteurs d'obstacle}

gris: ignore, rouge: objet proche, blanc: objet loin&

\blk{action-colors-up} & \textbf{Couleur du haut}

Colore le haut avec un mélange de rouge, vert et bleu.%
%
\\[.6cm]

\blk{event-ground} & \textbf{Détecteurs de sol}

gris: ignore, rouge: sol, blanc: pas de sol&

\blk{action-colors-down} & \textbf{Couleur du bas}

Colore le bas avec un mélange de rouge, vert et bleu.%
%
\\[.6cm]

\blk{event-tap} & \textbf{Robot tapé}

Le robot a reçu un choc.&

\blk{action-music} & \textbf{Jouer une musique}

Choisir la hauteur, blanche deux fois la durée de noire.%
%
\\[.6cm]

\blk{event-clap} & \textbf{Claquement de main}

Le roobt a entendu un fort bruit.&

\blk{action-timer} & \textbf{Démarrer un minuteur}

Un événement temps écoulé se produira après un temps.%
%
\\[.6cm]

\blk{event-timer} & \textbf{Temps écoulé}

Le temps du minuteur est écoulé.&

\cellcolor{advancedmode} \blk{action-states} & \cellcolor{advancedmode} \textbf{Défini l'état du robot}

Défini les 4 bits de l'état interne du robot.\\

\end{tabularx}

\vfill

\begin{tabularx}{\textwidth}{l@{\hspace{.7cm}}L}

\multicolumn{2}{l}{\textbf{Construire votre programme}} \\[.4cm]

\blkbig{event-action-pair-empty} & Glissez/déposez des événements dans le carré gauche, des actions dans le carré droite.

Lorsque l'événement se produit, le robot fait l'action.
\\

\end{tabularx}

\vfill

\begin{tabularx}{\textwidth}{l@{\hspace{.7cm}}Ll}

\multicolumn{2}{l}{\textbf{Les capteurs sont combinés avec ET dans un événement}} & \\[.4cm]

\blk{sensor-and-button} & Si deux capteurs sont séléctionnés, les deux conditions doivent être vraies pour que l'événement se produise.

Gauche \textbf{et} droite doivent être touchés/avoir un objet à proximité. &

\blk{sensor-and-prox}\\

\end{tabularx}

\vfill

{\normalsize M. Ben-Ari, S. Magnenat, J. Shin - \href{http://creativecommons.org/licenses/by-sa/3.0/}{CC-BY-SA}}\hfill\colorbox{advancedmode}{fonctionnalité du mode avancé}

\end{document}
