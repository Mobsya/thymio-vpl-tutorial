%
%    Aseba/VPL Reference Card
%
%  Copyright 2013 by Mordechai (Moti) Ben-Ari.
%  This work is licensed under the Creative Commons
%  Attribution-ShareAlike 3.0 License. 
%  To view a copy of this license, 
%  visit http://creativecommons.org/licenses/by-sa/3.0/; 
%  or, (b) send a letter to Creative Commons, 
%  543 Howard Street, 5th Floor, San Francisco, California, 94105, USA.
%
\documentclass[a4paper,italian]{leaflet}
\usepackage{mathptmx}
\usepackage{url}
\usepackage{graphicx,url}
\usepackage[utf8x]{inputenc}
\graphicspath{{../../images/}}
\renewcommand{\baselinestretch}{1.1}
\renewcommand{\arraystretch}{1.5}
\textheight=190mm

\newcommand{\sct}[1]{\subsubsection{#1}\mbox{}\\}

\CutLine*{1}\CutLine*{3}\CutLine*{4}\CutLine*{6}

\newcommand*{\blk}[2][-20]{\raisebox{#1pt}%
{\includegraphics[height=35pt,keepaspectratio=true]{#2}}}

\newcommand*{\blkwide}[2][-50]{\raisebox{#1pt}%
{\includegraphics[width=.4\textwidth,keepaspectratio=true]{#2}}}

\begin{document}
\thispagestyle{empty}
\begin{center}
\begin{bfseries}
\begin{large}
Scheda di riferimento Aseba/VPL
\end{large}

\medskip

Moti Ben-Ari, St\'{e}phane Magnenat, Jiwon Shin
\end{bfseries}
\end{center}

\vspace*{-1ex}
{\scriptsize Copyright 2013 by Moti Ben-Ari, St\'{e}phane Magnenat and
Jiwon Shin. Questo lavoro è rilasciato sotto la licenza Creative Commons
Attribution-ShareAlike 3.0. Per vedere una copia di questa licenza,
visitare \url{http://creativecommons.org/licenses/by-sa/3.0/}; o, (b) inviare
una lettera a Creative Commons, 543 Howard Street, 5th Floor, San
Francisco, California, 94105, USA.}


\sct{Interfaccia utente VPL}

\smallskip

\begin{tabular}{lp{.73\textwidth}}
\blk{new} & Cancella editor per un nuovo programma; ritorna a modalità semplice.\\

\blk{open} & Apri un programma esistente.\\

\blk{save} & Salva il programma.\\

\blk{saveas} & Salva il programma con un nuovo nome.\\

\blk{run} & Carica ed esegui il programma sul robot.\\

\blk{stop} & Arresta il programma in esecuzione sul robot.\\

\blk{scheme} & Cambia i colori dei blocchi.\\

\blk{advanced} & Modalità avanzata.\\

\blk{info1} & Mostra la documentazione del VPL.\\

\blk{quit} & Chiudi VPL (solo se avviato da Aseba Studio).\\

\end{tabular}


\newpage

\sct{Blocchi Evento}

Un evento accade quando:      

\bigskip

\begin{tabular}{lp{.73\textwidth}}

\blk{forward} & I bottoni sono stati toccati.\par
I bottoni rossi sono attivi.\\

&\\

\blk[-25]{horizontal} & I sensori orizzontali avvertono la vicinanza ad un oggetto.\par
Rosso = un oggetto è nelle vicinanze.\par Bianco = non ci sono oggetti nelle vicinanze.\\

&\\

\blk[-25]{ground} & I sensori per il terreno riconosco il buio o la luce.\par 
Rosso = Il terreno è chiaro.\par Bianco = il terreno è scuro o assente.\\

&\\

\blk{event-tap} & Il robot è stato urtato.\\

&\\

\blk{event-clap} & Il robot sente un forte rumore.\\

&\\

\blk{event-timer} & Il timer è arrivato a fine conteggio.\\

\end{tabular}

\bigskip
\bigskip

I bottoni e i sensori che sono di colore grigio non sono attivi.

\newpage

\sct{Blocchi Azione}

\begin{tabular}{lp{.73\textwidth}}

\blk[-27]{action-motors} & Seleziona la velocità dei motori sinistro e destro.\par
Muovere il cursore in su (avanti)\par
o in basso (indietro).\\

&\\

\blk{action-colors-up} & Sceglie il colore della parte superiore del robot.\par
Muovere i cursori per miscelare rosso, verde e blu.\\

&\\

\blk{action-colors-down} & Sceglie il colore della parte inferiore del robot.\par
Muovere i cursori per miscelare rosso, verde e blu.\\

&\\

\blk[-27]{action-music} & Suona la musica.\par
Cliccare su un rigo per scegliere l'altezza della nota.\par
Le note bianche durano il doppio di quelle nere.\par
Cliccare sulla nota per cambiare la durata. $\leftrightarrow$ black.\\

&\\

\blk{clock} & Avvia il timer per un tempo da 0 a 4 secondi.\par
Cliccare sul quadrante dell' orologio per scegliere la durata.\\

&\\

\blk[-30]{states2} & Configura le 4 parti dello stato corrente del robot.\par
Grigio = non cambia il valore.\par
Bianco = mette il valore a 0.\par
Giallo = mette il valore a 1.\\

\end{tabular}

\newpage

\sct{Coppie evento-azione}

\begin{tabular}{lp{.5\textwidth}}

\blkwide[-25]{e-a-pair} & Una coppia evento-azione.\\

&\\

\blkwide[-20]{tap-state-off} & Una coppia evento-azione che dipende dallo stato corrente (solo in modalità avanzata).\\

&\\

\blk{x} & Cancella questa coppia evento-azione.\\

&\\

\blk{plus} & Aggiunge una coppia evento-azione.\\

\end{tabular}


\newpage

\sct{Esempi}

\vspace{-2ex}

\begin{tabular}{lp{.5\textwidth}}

\blkwide[-35]{dont-fall} & \mbox{}\par
Il robot si ferma quando riconosce il bordo del tavolo.\\

&\\

\blkwide{likes-turns} & Il robot viene verso di te.\\

&\\

\blkwide{hates} & il robot si allontana da te.\\

&\\

\blkwide{line-controller} & Il robot segue una linea sul terreno.\\

\blkwide[-35]{clap} & \mbox{}\par
Un battito di mani cambia il colore inferiore del robot.\\

\end{tabular}

\newpage

\vspace*{1ex}

\begin{tabular}{lp{.5\textwidth}}

\blkwide[-35]{turn-clock} & \mbox{}\par
Quando il sensore di destra riconosce un oggetto fa partire un timer di 2 secondi.\\

\blkwide[-35]{turn-back} & \mbox{}\par
Quando il timer è arrivato a zero, gira a destra.\\

&\\

\blkwide{tap-on-off2} & Un urto cambia la prima parte dello stato da 0 a 1 e da 1 a 0.\\

&\\

\blkwide{tap-on-off1} & Un urto cambia il colore superiore in magenta se in stato 0 e
spegne il colore se in stato 1.\\

\end{tabular}

\bigskip
\bigskip
\bigskip
\bigskip

\sct{Riferimenti}

\begin{itemize}
\item Documentazione di riferimento VPL:\\
\url{https://aseba.wikidot.com/en:thymiovpl}.

\item Tutorail VPL\\
\url{https://aseba.wdfiles.com/local--files/it:thymioprogram/thymio-vpl-tutorial-it.zip}.

\end{itemize}
\end{document}
