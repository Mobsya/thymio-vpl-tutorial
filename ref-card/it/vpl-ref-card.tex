%
%    Aseba/VPL Reference Card
%
%  Copyright 2013 by Mordechai (Moti) Ben-Ari.
%  This work is licensed under the Creative Commons
%  Attribution-ShareAlike 3.0 License. 
%  To view a copy of this license, 
%  visit http://creativecommons.org/licenses/by-sa/3.0/; 
%  or, (b) send a letter to Creative Commons, 
%  543 Howard Street, 5th Floor, San Francisco, California, 94105, USA.
%
\documentclass[a4paper,italian]{article}
\usepackage[a4paper,margin=.9cm]{geometry}
%\usepackage{mathpazo}
\usepackage[utf8x]{inputenc}
\usepackage[T1]{fontenc}
\usepackage{graphicx,url}
\usepackage{tabularx}
\usepackage[usenames,dvipsnames]{xcolor}
\usepackage{colortbl}
\usepackage[colorlinks=true,linkcolor=Blue,urlcolor=Blue,citecolor=Blue]{hyperref}
\usepackage{libertine}

\graphicspath{{../../images/}}

\renewcommand*\familydefault{\sfdefault}

\renewcommand{\baselinestretch}{1.1}
\setlength{\parskip}{0.3\baselineskip plus 1pt minus 1pt}
\parindent=0pt

\newcommand*{\blk}[1]{\raisebox{-40pt}%
{\includegraphics[height=50pt,keepaspectratio=true]{#1}}}

\newcommand*{\blkbig}[1]{\raisebox{-50pt}%
{\includegraphics[height=60pt,keepaspectratio=true]{#1}}}

\newcolumntype{L}{>{\raggedright\arraybackslash}X}

\colorlet{advancedmode}{orange!20}

\begin{document}
\thispagestyle{empty}

\fontsize{15pt}{18pt}\selectfont

\begin{center}
{\Huge \textbf{Scheda di riferimento VPL}}
\end{center}

\bigskip

\begin{tabularx}{\textwidth}{lL@{\hspace{1cm}}lL}

\multicolumn{2}{l}{\textbf{Eventi}} & \multicolumn{2}{l}{\textbf{Azioni}} \\[.4cm]

\blk{event-buttons} & \textbf{Bottoni toccati}

grigio: ignora il bottone, rosso: deve essere toccato &

\blk{action-motors} & \textbf{Motori}

Assegna la velocità del motore di destra e di sinistra e delle ruote.%
%
\\[.6cm]

\blk{event-prox} & \textbf{Sensori ostacoli}

grigio: ignora il sensore, rosso: oggetto vicino, bianco: oggetto lontano&

\blk{action-colors-up} & \textbf{Colori superiori}

Miscela di rosso, verde e blu per la parte superiore del robot.%
%
\\[.6cm]

\blk{event-ground} & \textbf{Sensori terreno}

grigio: ignora sensore, rosso: terreno chiaro o presente, bianco: terreno scuro o assente &

\blk{action-colors-down} & \textbf{Colori inferiori}

Miscela di rosso, verde e blu per la parte inferiore del robot.%
%
\\[.6cm]

\blk{event-tap} & \textbf{Robot urtato}

Il Robot ha ricevuto un urto.&

\blk{action-music} & \textbf{Suona la musica}

Scegli l'altezza del suono, nota bianca doppia durata rispetto a nera.%
%
\\[.6cm]

\blk{event-clap} & \textbf{Battito di mani}

Il Robot ha sentito un rumore forte.&

\blk{action-timer} & \textbf{Avvia timer}

Trascorso il tempo impostato genera l'evento Timer scaduto.%
%
\\[.6cm]

\blk{event-timer} & \textbf{Timer scaduto}

Il timer ha completato il conteggio.&

\cellcolor{advancedmode} \blk{action-states} & \cellcolor{advancedmode} \textbf{Definisce lo stato del Robot}

Configura lo stato interno a 4-bit del robot.\\

\end{tabularx}

\vfill

\begin{tabularx}{\textwidth}{l@{\hspace{.7cm}}L}

\multicolumn{2}{l}{\textbf{Costruire il proprio programma}} \\[.4cm]

\blkbig{event-action-pair-empty} & Prendere e trascinare gli eventi nel quadrato di sinistra, le azioni nel quadrato di destra. 

Quando accade l'evento, il robot esegue l'azione.
\\

\end{tabularx}

\vfill

\begin{tabularx}{\textwidth}{l@{\hspace{.7cm}}Ll}

\multicolumn{2}{l}{\textbf{I sensori sono combinati in AND negli eventi}} & \\[.4cm]

\blk{sensor-and-button} & Se due sensori sono selezionati, entrambe le condizioni devono essere vere perchè accada l'evento.

Sinistra \textbf{e} destra devono essere toccati o avere un oggetto vicino. &

\blk{sensor-and-prox}\\

\end{tabularx}

\vfill

{\normalsize M. Ben-Ari, S. Magnenat, J. Shin - \href{http://creativecommons.org/licenses/by-sa/3.0/}{CC-BY-SA}}\hfill\colorbox{advancedmode}{funzionalità avanzata}

\end{document}
