\documentclass[11pt,a4paper,english]{article}
\usepackage{../../vpl}
\graphicspath{{../../images/}}

\begin{document}

\thispagestyle{empty}

\begin{center}
\begin{Huge}
\begin{bfseries}
VPL Tutorial Teachers' Guide
\end{bfseries}
\end{Huge}

\vskip 2cm

\begin{LARGE}
\href{http://www.weizmann.ac.il/sci-tea/benari/}{Moti Ben-Ari}\\% and other contributors\\
\end{LARGE}
%\bigskip
%\begin{Large}
%see \texttt{authors.txt} for details
%\end{Large}

\vskip 1cm

\begin{Large}
Version 1.0
\end{Large}

\end{center}

\vfill

\begin{center}
\copyright{}\  2015 by \href{http://www.weizmann.ac.il/sci-tea/benari/}{Moti Ben-Ari}. %and other contributors.
\end{center}

This work is licensed under the Creative Commons
Attribution-ShareAlike 3.0 Unported License. To view a copy
of this license, visit
\href{http://creativecommons.org/licenses/by-sa/3.0/}{http://creativecommons.org/licenses/by-sa/3.0/}
or send a letter to Creative Commons, 444 Castro Street, Suite 900,
Mountain View, California, 94041, USA.

\begin{center}
\includegraphics[width=.2\textwidth]{by-sa}
\end{center}

%\newpage
%\begin{spacing}{0.7}
%\tableofcontents
%\end{spacing}
%\thispagestyle{empty}

\newpage

%%%%%%%%%%%%%%%%%%%%%%%%%%%%%

\setcounter{page}{1}

\href{https://www.thymio.org/en:visualprogramming}{\emph{First Steps in
Robotics with the Thymio Robot and the Aseba/VPL Environment}} is a
tutorial on the \emph{Visual Programming Environment (VPL)} for the
\emph{Thymio} educational robot. The purpose of this \emph{Guide} is to
assist teachers in using the material in the tutorial in robotics
activities with students. Specifically, it includes:

\begin{itemize}
\item A description of the concepts that can be studied within each activity. 

\item Indications of which parts of an activity explore fundamental
concepts and algorithms of robotics and which present features that can
be explored if time permits.

\item Tips for conducting activities with the Thymio and VPL.

\end{itemize}

\paragraph{Teaching concepts}

Many concepts have real-world analogs that will be familiar to the
students. Nevertheless, it is essential to \emph{explicitly} teach the
concepts so that the students will recognize their importance.

\paragraph{Parsons puzzles}

Parsons puzzles (Chapter~11) are exercises designed so that students
construct programs by filling in missing instructions or re-arranging
out-of-place instructions, rather than by writing a program from its
specification. Parsons puzzles have proved to be very effective in
facilitating learning and you are encouraged to design your own puzzles.

\paragraph{From VPL to AESL}

Once students have mastered programming in the VPL environment they
can study how to write programs in the AESL language in the Studio
environment. This language gives the user full control over the Thymio
robot and enables advanced projects to be constructed. Chapter~19
uses a method called \emph{mediated transfer} to introduce the student
to AESL by examining how VPL programs are translated into AESL.

\paragraph{Tips}
\begin{itemize}

\item Before writing programs with VPL, students can learn about the
Thymio robot by exploring the \emph{pre-programmed behaviors}
at \href{https://www.thymio.org/en:thymiostarting}%
{https://www.thymio.org/en:thymiostarting}.

\item A summary of the VPL user interface is given in Appendix~A and a
summary of the event and action blocks is given in Appendix~B.
These are useful if you or your students forget the meaning of an
interface icon or a VPL block.

\item Appendix~C contains a list of tips for writing VPL programs
and for finding and fixing problems. We suggest that you read this
carefully before teaching with VPL so that you can guide the students
in good practices.

\end{itemize}

%%%%%%%%%%%%%%%%%%%%%%%%%%%%%

\newpage

\section{Introducing the Thymio}

Chapter~1 is an overview of the Thymio robot and the VPL programming
environment.

\paragraph{Concepts}

\begin{itemize}

\item There several general concepts that are presented in this
chapter: program, running a program, opening and saving a program.

\item The fundamental concepts introduced are \emph{events} and
\emph{event handlers}. In VPL, an event handler is called an
\emph{event-actions pair} consisting of an event block and one or more
action blocks. When the event \emph{occurs} the action is \emph{carried
out}.


\end{itemize}

\paragraph{Tips}

\begin{itemize}
\item Fully charge the battery before the activities. This is
especially important when using the pre-programmed behaviors and
when writing programs for a wireless Thymio.

\item Students who routinely use computers (including smartphones and
tablets) will already be familiar with the concepts of running a
program, and opening and saving files.

\item It is not necessary to learn all the features of the VPL
environment initially. Students must, of course, use the Run and Stop
buttons, but since their first programs will be short, New, Open and
Save can be skipped during the first session.

\item The concepts of event and handler will be familiar, although the
terminology may not be known. When playing a game, \emph{click a button}
is an event that causes the action \emph{shoot a laser at an alien} to
take place. On a smartphone, \emph{touch the green telephone icon} is an
event that causes the action \emph{dial a number} to take place. On the
Thymio, events include touching a button on the robot and detecting an
object by a sensor. Actions include turning on the colored lights
and setting the speeds of the motors.

\end{itemize}

%%%%%%%%%%%%%%%%%%%%%%%%%%%%%

\section{Media}

The Thymio robot has five touch-sensitive buttons on its top. There are
large colored lights on both its top and its bottom. Chapter~2 starts
with projects using the buttons and colors, because they can be fun to
play with and they facilitate gaining experience with event-actions
pairs in VPL. Chapter~6 shows how to use sound with the Thymio: playing
notes and sensing a clap or a tap on the robot.

\paragraph{Concepts}

\begin{itemize}

\item An event can have more than one action associated with it.

\item All colors can be produced by combining red, green and blue (RGB).

\end{itemize}

\paragraph{Tips}
\begin{itemize}

\item To keep the first programs simple, it is not necessary to
introduce the concept of multiple actions for an event at an early
stage. However, the need will come up fairly soon, since VPL does not
allow two event-actions pairs to have the same event. For example, if
you want the event of touching the center button to turn on the top red
light and also to start the motors, multiple actions must be used.

\item The production of all colors from RGB is not subsequently used in
this tutorial on robotics. You can either skip the topic, let the
students discover it on their own, or pose projects for the students to
work on. The RGB concept is explored in detail in the Wikipedia article
\href{https://en.wikipedia.org/wiki/RGB_color_model}{RGB color model}.

\item The Thymio's capabilities for detecting and producing sound are
very basic and learning them is optional. The clap and tap events
can be difficult to work with if the motors are running, because the
motors produce sound that can cause unexpected events.

\end{itemize}

%%%%%%%%%%%%%%%%%%%%%%%%%%%%%

\section{Fundamental robotics concepts}

Chapters~3--5 explore the fundamental concepts of autonomous mobile
robotics: sensing the environment, deciding how to respond and carrying
out actions, especially moving within the environment. The most powerful
features of the Thymio are its nine \emph{infrared proximity sensors}
(five on the front, two on the back and two on the bottom) that can be
used to detect objects, and its motors that enable the robot to move
within its environment.

\paragraph{Concepts}
\begin{itemize}

\item A sensor can cause an event when it \emph{detects} an object
(there is a lot of reflected light). Alternatively, a sensor can cause
an event when it \emph{does not detect} an object (there is little
reflected light).

\item The robot uses \emph{differential drive}: the speeds of the left
and right motors can be set independently and the robot will turn if the
motors are set to different speeds.

\item Real robots don't drive straight! Because of slight differences in
the components (both when manufactured and as they age) and because the
table or floor surface will be uneven, no two robots behave alike. Even
running a robot with the same program multiple times will result in
inconsistent behavior. 

\item When an event occurs, the robot must decide what to do. This is
called a \emph{control algorithm}; these algorithms are implemented as
VPL programs composed of event-actions pairs. Following a line and
turning towards an object demonstrate a central concept of robotics:
\emph{feedback control}, where to robot achieves a goal by changing its
behavior in response to measurements of the environment. Feedback
control is essential to obtain correct and consistent behavior.

\item \emph{Conjunction}: An event can be defined to occur only if \emph{more than one}
sensor detects (or doesn't detect) an object at the same time. It is
implemented by selecting more than one sensor \emph{in a single event
block} to be white or black. Conjunction means ``and''; for example, an
event occurs if sensor~1 \emph{and} sensor~2 both detect an object.

\item \emph{Disjunction}: An event can be defined to occur if
\emph{some} sensor detects (or doesn't detect) an object. It is
implemented by selecting sensors in \emph{different} blocks to be white
or black. Disjunction means ``or''; for example, an event occurs if
\emph{either} sensor~1 \emph{or} sensor~2 (\emph{or both}) detects an object.

\end{itemize}

\paragraph{Tips}
\begin{itemize}

\item If you work on a table, always include an event-actions pair to
stop the robot if it reaches an edge of the table.

\item The sensors are very sensitive to how much light is reflected.
Students can experiment with various materials---colored tapes, markers,
metal or plastic objects---to see how the sensors behave.

\item Objects covered with reflecting tape (used as a safety device by
joggers and cyclists) can be detected by the sensors at a significantly
greater range than ordinary objects.

\item Since no two robots are alike programs cannot be copied from
one robot to another without checking that the \emph{parameters}, such
as the motor speeds, still work.

\item The motor speeds can be set in increments of 50 from $-$500 to
500. You can observe the setting in the text program on the right side
of the VPL window.

\end{itemize}

%%%%%%%%%%%%%%%%%%%%%%%%%%%%%

\section{Timers}

Although Chapter~7 on timers is short, the concept is important and
should not be skipped. Timers are familiar: a microwave oven uses a
timer to decide how long to heat the food.

\paragraph{Concepts}
\begin{itemize}

\item A timer is a clock that is \emph{set} to a specific amount of time
(2 minutes to make popcorn). When that amount of time has
\emph{expired}, an event occurs (the microwave bell sounds). There are
two VPL blocks associated with a timer: an action block to set the timer
and an event block used in an event-actions pairs that specifies the
actions to be carried out when the timer expires.

\item All events in a VPL program are checked at about the same
time and all actions associated with each event that occurs are carried
out at about the same time. Timers enable actions to occur in a
\emph{sequence}. For example, the robot's motors can be run for two
seconds and \emph{then} the motors can be stopped.

\end{itemize}

\paragraph{Tips}
\begin{itemize}

\item When VPL is opened initially, it is in \emph{basic mode}. By
clicking an icon, it enters \emph{advanced mode}. The reason for the two
modes is to display a very simple interface to the beginning student and
thus to reduce anxiety; when the student gains more experience, the
additional features of the advanced mode enable more sophisticated
projects to be constructed. Feel free to change the order of topics and
to introduce advanced mode earlier or later.

\item The setting of the timer duration can be observed in the text
program on the right side of the VPL window. The duration is measured in
\emph{milliseconds}---thousandths of a second---from 0 to 4000 (0 to 4
seconds) in increments of 250 milliseconds (1/4 of a second). While the
timer is electronic and accurate, mechanical components like the motors
and wheels are not, so moving forward for 2 seconds will not always
bring the robot to the same spot.

\end{itemize}

%%%%%%%%%%%%%%%%%%%%%%%%%%%%%

\section{States}

\emph{State} is a fundamental concept of computer science that is
supported in the advanced mode of VPL. Chapters~8 and~9 explore the use
of state when writing programs for the Thymio.

\paragraph{Concepts}

\begin{itemize}

\item The concept of \emph{state} is familiar: A light in a room is
\textit{in the state on} or it is \textit{in the state off}. A fan is
\textit{in} one of four states: \textit{off}, \textit{low},
\textit{medium}, \textit{high}. A room can be in one of a large numbers
of states describing its temperature: $20.5^\circ$, $18.31^\circ$,
$25^\circ$. The Thymio robot can be in one of 16 possible states as
shown in Figure~8.2 of the tutorial.

\item In programs without states, if the event occurs, the corresponding
actions are carried out unconditionally. With states, it is possible to
place a condition on an event handler: if the event occurs, the
corresponding actions are carried out only if the event is
\emph{enabled}---the Thymio is in one of a set of states; otherwise, the
event is \emph{disabled}---the Thymio is \emph{not} in one of a set of
states.

Enabling and disabling an event is encountered in the real world, in
particular, in safety measures: You cannot start a car unless the brake
pedal is pushed down (is in the state \textit{down}). Hotel showers
will not allow you to run very hot water unless an additional switch is
pressed.

\item Enabling and disabling events can be used for:

\begin{itemize}

\item \emph{Sequence}. Like timers, state can be used to sequence
actions. A timer causes one event to occur after a period of time has
passed since a previous event. With states, actions can be performed in
a sequence without waiting. Consider the follow event-actions pairs:

\begin{quote}
Event 1 -> Actions 1 and set state S1\\
Event 2 and in State S1 -> Actions 2
\end{quote}

Occurrences of Event 2 will be ignored unless the Thymio is in State 1,
which will be true only if Event 1 occurs. Therefore, Actions 2
associated with Event 2 will occur \emph{after} the Actions 1 associated
with Event 1.

\item \emph{Alternative run}. There can be only one event-actions pair
associated with an event. Therefore, an occurrence of an event will
\emph{always} cause the same set of actions to be carried out. With
states, you can specify that an event will cause different sets of
actions to be carried out depending on the state. For example, the event
of detecting an object by the center sensor can cause the robot to turn
left in one state and to turn right in another.

If you have experience in programming, you will recognize that this is
similar to the construct \textit{if Expression then Statement 1 else
Statement 2}. In terms of states, an \textit{if} statement is similar
to:

\begin{quote}
if Event and in States 1 -> Actions 1\\
if Event and not in States 1 -> Actions 2
\end{quote}

The condition \emph{not in States 1} is implemented by creating
the set of all states that are not in States 1.

\end{itemize}

\item \emph{Memory}. The Thymio \emph{remembers} state it is in; this
can be used to store a value. For example, you can count the number of
lines of tape that the Thymio encounters as it moves on the floor. In
terms of programming languages, the Thymio has four \emph{variables} v1,
v2, v3, v4 that can each store two possible values that we can call 0
and 1. Using state blocks you can write an event-actions pair equivalent
to the statement:

\begin{quote}
if Event and (v2 equals 0) and (v4 equals 1) -> Actions
\end{quote}

\item \emph{Unary and binary arithmetic}. The state of the Thymio is
represented in VPL by four \textit{quarters} in a block that can be
either white or orange (or gray to denote \textit{don't care}). These
quarters can be used to count 0, 1, 2, 3, 4, depending on which quarter
is orange (or none are orange). Alternatively, they can encode the
values 0 through 16 in binary as described in Chapter~9.

\end{itemize}
\paragraph{Tips}
\begin{itemize}

\item Implementing non-trivial robotics tasks will almost certainly
require sequencing and alternative run, so these uses of states are
important and should be taught. The material on memory and arithmetic is
a central topic of computer science, but it is somewhat advanced and
less fundamental for robotics, so it can be considered optional.

\item Pay attention to the circle lights on the top of the Thymio since
they are your only indication of the current state.

\end{itemize}

%%%%%%%%%%%%%%%%%%%%%%%%%%%%%

\section{Accelerometers}

The Thymio has a three-axis accelerometer, of which two axes
(left-right, front-back) can be used in VPL in advanced mode.

\paragraph{Concepts}
\begin{itemize}

\item \emph{Acceleration} is the rate of change of speed. It can be
either positive if the speed increases or negative if the speed
decreases. When you step on the gas pedal of a car, it accelerates
positively. When you step on the brake pedal, the car accelerates
negatively (decelerates). When an airplane takes off, the acceleration
is high enough so that you feel pressed back into your seat.

\item An \emph{accelerometer} measures acceleration.

\item The most familiar acceleration is that of \emph{gravity}. The
force of attraction of the earth causes us to accelerate towards the
center of the earth; otherwise, we would fly off into space as the
earth rotates.

\end{itemize}

\paragraph{Tips}
\begin{itemize}

\item The Thymio's accelerometers are not sensitive and can reliably
measure only the acceleration caused by gravity. They can be used
implement game controllers: changing colors and sounds as the Thymio is
tilted from left to right or from front to back.

\item Projects using accelerometers can be found
\href{https://www.thymio.org/en:creations}%
{https://www.thymio.org/en:creations}: slope avoidance, weighing scale,
pendulum oscillations. The projects were implemented using LEGO bricks
and the AESL textual language, but you might be inspired to develop
similar VPL projects.

\end{itemize}

%%%%%%%%%%%%%%%%%%%%%%%%%%%%%

\section{Projects}

Chapters~12--19 describe additional projects for the Thymio robot and
the VPL environment. Here we give an overview of each project in terms
of its purpose, relative difficulty and concepts used. Feel free to
modify or extend the projects.

\begin{itemize}

\item \textbf{Braitenberg creatures}. These tasks are well-known and are
a lot of fun to implement. Although the projects are very simple, they
explore a wide range of interesting behavior that can be implemented
primarily with the sensors and the motors.

\item \textbf{The rabbit and the fox}. This project is somewhat larger
than previous ones; it uses the sensors and the motors together with the
timer.

\item \textbf{Reading barcodes}. The Thymio has five sensors on its
front that can be used to decode a barcode. Constructing a working
program requires experimenting with the object containing the barcode
and the thresholds of the sensors. Therefore, the solution in the
archive uses advanced mode so that the thresholds can be set as
described in Appendix~D. However, you can try to solve it in basic mode.

\item \textbf{Sweep the floor}. Since real robots don't drive straight,
we emphasized the use of events rather than timers to control the motion
of the Thymio. This project tries to get the robot to follow a fixed
route using just timers and states. Given the unevenness of the floor
and the inconsistencies in the motors, you will have to do a lot of
experimenting to get it to work reasonably well.

\item \textbf{Measuring speed}. The task is to measure the amount of
time it takes the robot to move over a strip of tape of a fixed length.
By dividing the length of the tape by the time, the speed of the robot
can be computed. The project requires extensive use of timers and
states.

\item \textbf{Catch the speeders}. This project is rather cool: use
the Thymio to implement a police speed trap by measuring the speed of
an object that passes in front of the sensors. It requires the use
of timers and states. 

\item \textbf{Finite automata} are an abstraction of computers that have
many applications. This project is rather advanced when compared with the
previous ones. The goal is to read a \emph{tape} consisting of an
arbitrary pattern of reflecting and non-reflecting cells, and to change
state according to rules programed into the robot. It uses the bottom
sensors in an unusual way inspired by the amazing
\href{https://www.thymio.org/en:barcodelightpainting}%
{\textit{light-painting project}}. One sensor is used to track a line
while the other is used to decode the cells.

A sample page with a track and cells is available in the archive.
Instructions on how to print such pages are given in the tutorial,
but they require familiarity with the \LaTeX\ document preparation
system.

\item \textbf{Multiple sensor thresholds}. Sensor squares in the sensor
event blocks can be either gray (not enabled), white (an event occurs
when the value of the sensor is above a threshold), or black (an event
occurs when the value of the sensor is below a threshold). This project
introduces a fourth possibility: dark gray (an event occurs when the
value of the sensor is \emph{between} two thresholds). This feature
enables detection of objects at three distances: far, medium, near.

\end{itemize}

\end{document}

%%%%%%%%%%%%%%%%%%%%%%%%%%%%%

\section{}

\paragraph{Concepts}
\begin{itemize}

\end{itemize}
\paragraph{Tips}
\begin{itemize}

\end{itemize}