\chap{Accelerometers (Advanced Mode)}\label{ch.angles}

We are all familiar with \emph{acceleration}, the rate of change of
speed, for example, when a car speeds up or slows down. An
\emph{accelerometer} is a device for measuring acceleration. An airbag
in a car uses an accelerometer to detect if the speed of the car is
descreasing ``too fast'' because the car has crashed; if so, the airbag
is inflated.

The Thymio robot has three accelerometers, one for each direction:
forward / backwards, left / right, and up / down.

%\informationbox{Advanced mode}{Accelerometers are supported in \emph{advanced
%mode}. Click on \blkmed{advanced} to enter advanced mode.}

It is hard to achieve measurable accelerations, except for the case of
\emph{gravity} which is an acceleration towards the center of the earth.
In this project, we use the accelerometers to measure the angle at which
the robot is tilted.

There are two events that can detect the angle of the robot relative to
the earth: \label{p.accel}

\begin{itemize}

\item \blksm{event-roll}: An event occurs when the left / right angle of
the robot is within the white angle segment in the half-circle. (The
technical term is \emph{roll}.)

\item \blksm{event-pitch}: An event occurs when the forward /
backwards angle of the robot is within the white angle segment of the
half-circle. (The technical term is \emph{pitch}.)

\end{itemize}

Initially, these blocks have the white segment pointing upwards from the
top of the image of the Thymio, so that an event occurs when the robot
is placed on a level surface such as a table or the floor. By dragging
the segment with the mouse, you can select other angles; for
example, the following block causes an event to occur when the robot is
tilted left roughly half-way from vertical to horizontal:
\blkc{roll-left}

%\newpage

\begin{quote}
\textbf{Program}\\
Hold the robot so that it is facing you and tilt it left and right. The
top light of the robot will display a different color for each range
of the angle of the tilt.
\end{quote}

{\raggedleft \hfill \textbf{Program file}: \bu{measure-angles.aesl}}

Construct a set of event-actions pairs where each event is a left-right
accelerometer event and the corresponding action changes the top color:
\blkc{measure-angles}
Make a list relating colors to angles so that you can translate any
color to a specific angle. 

The quarters of the event-state block are gray so the event causes
the action in any state.

\bigskip

\exercisebox{\thechapter.1}{Can two events use the same white segment of
angles?\\
How many events with different angles and you can construct? }
