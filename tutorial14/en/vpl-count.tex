% !TeX root = vpl.tex

\chap{Counting (Advanced Mode)}\label{ch.counting}

In this chapter we show how states of the Thymio robot can be used to
count numbers and even perform simple arithmetic.

The design and implementation of the projects will not be presented
in detail. We assume that you have enough experience by now to develop
them yourself. The source code of working programs is included in the
archive, but don't look at them unless you really have difficulties
solving a problem.

These projects use the clap event \blksm{event-clap} to change states
and the default behaviour of the circle lights to display the state.
Feel free to change either of these behaviours.

\importantbox{By default, the current state of the robot is displayed
in the circle lights on the top of the robot.
\cref{fig.state-leds} shows the state \bu{(on, on, on, on)}.}


\sect{Odd and even}

\begin{quote}
\textbf{Program}\\Choose one of the quarters of the state.
It will be \bu{off} (white) if the number of claps
is even and \bu{on} (orange) if the number of claps is odd.
Touching the center button will reset the count to even
(since zero is an even number).
\end{quote}

{\raggedleft \hfill Program file \bu{count-to-two.aesl}}

Even and odd are terms from \emph{modulo 2 arithmetic}, where we count
from 0 (even) to 1 (odd) and then back to 0. The term \emph{modulo} is
like the term \emph{remainder}: if there have been 7 claps, then
dividing 7 by 2 gives 3 and remainder 1. We only keep the remainder 1.

Another term for the same concept is \emph{cyclic arithmetic}.
Instead of counting from 0 to 1 and then from 1 to 2,
we \emph{cycle} back to the beginning:
0, 1, 0, 1, \ldots.

These concepts are familiar from counting time: minutes and seconds are
computed modulo 60 and hours are computed modulo 12 or 24. The second
after 59 is not 60; instead, we cycle around and start counting from 0
again. Similarly, the hour after 23 is not 24, but 0. If the time is
23:00 and we agree to meet after 3 hours, the time set for the meeting
is (23+3) modulo 24 = 26 modulo 24 = 02:00 in the morning.

\sect{Counting in unary}

\begin{quote}
Modify the program to count modulo 4.
There are four possible remainders, 0, 1, 2, 3.
Choose three quarters, one each to
represent the values 1, 2 and 3; the value 0 will be represented
by setting all quarters to \bu{off}.
\end{quote}

This method of representing numbers is called
\emph{unary representation}
because different elements of a state represent different numbers.
We often use unary representation to keep track of the count
of some objects; for example,
\begin{picture}(35,10)
\multiput(5,0)(5,0){4}{\put(0,0){\line(0,1){10}}}
\put(0,0){\line(3,1){25}}
\put(32,0){\line(0,1){10}}
\end{picture}
represents 6.

{\raggedleft \hfill Program file \bu{count-to-four.aesl}}

\exercisebox{\thechapter.1}{How high can we count on the Thymio using
unary representation?}

\sect{Counting in binary}

We are familiar with \emph{based representation}, in particular base 10
(decimal) representation. The symbols 256 in base 10 don't represent
three unrelated objects. Instead, the 6 represents the number of 1's,
the 5 represents the number of 10's, and the 2 represents the number of
10$\times$10=100's. Adding these factors gives the number two hundred
and fifty-six. Using base 10 representation, we can write very large
numbers in a compact representation. Furthermore, arithmetic on large
numbers is relatively easy using the methods we learned at school.

We use base 10 because we have 10 fingers so arithmetic in base 10 is
easy to learn. Computers, however, have two ``fingers'' (\bu{off} and
\bu{on}) so base 2 arithmetic is used in computation. Base 2 arithmetic
looks strange at first; while we use the familiar symbols 0 and 1 also
used in base 10, the rules for counting are cyclic at 2 instead of
cyclic at 10:

\begin{displaymath}
0, 1, 10, 11, 100, 101, 110, 111, 1000, \ldots
\end{displaymath}

Given a base 2 number such as 1101, we compute its value from right to
left just as in base 10. The rightmost digit represents the number of
1's, the next digit represents the number of 2's, the third digit
represents the number of 2$\times$2=4's, and the leftmost digit
represents the numbers of 2$\times$2$\times$2=8's. Therefore, 1101
represents 1+0+4+8, which is thirteen.

\begin{quote}
\textbf{Program}\\
Modify the program for counting modulo 4 to use binary representation.
\end{quote}

{\raggedleft \hfill Program file \bu{count-to-four-binary.aesl}}

We need only \emph{two} quarters of the state to represent the numbers
0, 1, 2, 3 in base 2. Let the upper-right quarter represent the number
of 1's---\bu{off} (white) for none and \bu{on} (orange) for one---and let
the upper-left quarter represent the number of 2's. For example,
\blksm{state-right} represents the number 1 because the upper-left
quarter is white and the upper-right quarter is orange. If both quarters
are white, the state represents 0, and if both quarters are orange, the
state represents 3. The number 2 is represented by \blksm{state-left},
where the upper-left quarter is orange and the upper-right quarter is
white.

There are four transitions $0\rightarrow 1, 1\rightarrow 2, 2
\rightarrow 3, 3\rightarrow 0$, so four event-actions pairs are needed,
in addition to a pair to reset the program when the center button is
touched.

\bigskip

\informationbox{Ignore unused quarters of the state}{The two bottom
quarters are not used, so they are left gray and are ignored by the
program.}

\bigskip

\exercisebox{\thechapter.2}{Extend the program so that it counts
modulo 8. The lower-left quarter will represent the number of 4's.}

\exercisebox{\thechapter.3}{How high can we count on the Thymio using
binary representation?}

\sect{Adding and subtracting}

Writing the program to count to 8 is quite tedious because you had to
program 8 event-actions pairs, one for each transition from $n$ to $n+1$
(modulo 8). Of course, that is not how we count in a based
representation; instead, we have methods for performing addition by
adding the digits in each place and carrying to the next place. In base
10 representation:

\begin{displaymath}
\begin{array}{r}
387\\
+426\\
\rule[1pt]{1.5em}{1pt}\\
813\\
\end{array}
\end{displaymath}
and similarly in base 2 notation:
\begin{displaymath}
\begin{array}{r}
0011\\
+1011\\
\rule[1pt]{2em}{1pt}\\
1110\\
\end{array}
\end{displaymath}

When adding 1 to 1, instead of 2, we get 10. The 0 is written in the
same column and we carry the 1 to the next column to the left. The
example above shows the addition of 3 (=0011) and 11 (=1011) to obtain
14 (=1110).

\begin{quote}
\textbf{Program}\\
Write a program that starts with a representation of 0.
Each clap adds 1 to the number.
The addition is modulo 16, so adding 1 to 15 results in 0.
\end{quote}

{\raggedleft \hfill Program file \bu{addition.aesl}}

\textbf{Guidance}

\begin{itemize}
\item Starting the in upper-right corner and continuing counter-clockwise,
the quarters will represent the number of 1', 2's, 4's and 8's in the number.
Thus, the lower-right quarter represents the number of 8's.
\item If the upper-right quarter representing the number
of 1's shows 0 (white),
simply change it to 1 (orange). Do this regardless of what the other
quarters show.
\item If the upper-right quarter representing the number of 1's
shows 1 (orange),
change it to 0 (white) and then carry the 1.
There will be three event-actions pairs,
depending on the location of the \emph{next} quarter showing 0
(white).
\item If all quarters show 1 (orange), the value of 15 is represented.
Adding 1 to 15 modulo 16 results in 0, represented by all quarters
showing 0 (white).
\end{itemize}

\bigskip

\exercisebox{\thechapter.4}{Modify the program so that it starts
with the value 15 and subtracts one at each clap down to zero,
and then cyclically back to 15.}

\bigskip

\exercisebox{\thechapter.5}{Place a sequence of short segments of black
tape on a light surface. Write a program that causes the Thymio to move
forward and stop at the fourth tape.}

\bigskip

This exercise is not easy: the strips of tape have to be
sufficiently wide so that the robot detects them,
but not so wide that more than one event occurs per strip.
You will also have to experiment with the speed of the robot.

