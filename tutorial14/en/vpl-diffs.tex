% !TeX root = vpl.tex

\chapter*{Versions 1.3 and 1.4}

The following list is intended for readers with experience in Aseba 1.3;
it points out differences between that version and version 1.4.

\textbf{Changes to the event and action blocks}

\begin{itemize}

\item The graphic design of the buttons for the blocks has been changed,
primarily to support additional features.

\item In 1.3, a \emph{red} box in the event block for a horizontal
sensor caused an event when the sensor detected an object, while a
\emph{white} box caused an event when there was no object in front of
the sensor. In 1.4, a \emph{white} box causes an event when a lot of
reflected light is detected from an object, while a \emph{black} box
causes an event when little or no reflected light is detected because
there is no object in front of the sensor
(pages~\pageref{p.proximity-colors1},~\pageref{p.proximity-colors2}).
The bottom sensors also use white and black, instead of white and red,
but the behavior in 1.4 is the same as in 1.3 except that black is used
instead of red.

\warningbox{This change in VPL means that existing programs will no
longer work correctly until they are modified.}

\item In advanced mode, the thresholds of sensors can be
set (page~\pageref{p.proximity-sensitivity}).

\item In advanced mode, an event can be associated with ranges of values
of the forward/backward and left/right accelerometers
(page~\pageref{p.accel}).


\end{itemize}

\textbf{Changes to the user interface}

\begin{itemize}

\item You can now have multiple actions associated with an event. This
makes VPL programs much more concise (page~\pageref{p.multiple}).

\item Blocks and event-actions pairs can be copied
(page~\pageref{p.copy-pairs}).

\item Screenshots of VPL programs can be exported in several
 graphics formats (page~\pageref{p.export}).

\item Undo/Redo buttons have been added (page~\pageref{p.undo}).

\item The Run button blinks green when the program has been changed
(page~\pageref{p.blink}).

\item Dynamic feedback has been added (page~\pageref{p.feedback}).

\item It is no longer possible to change the color scheme of VPL.

\end{itemize}
