% !TeX root = vpl.tex

\chap{What Next?}\label{ch.next}

This tutorial has introduced the Thymio robot and the Aseba/VPL
environment. The visual programming of the VPL environment is intended
for beginners. To develop more advanced programs for the robot, you will
want to learn how to use the Aseba Studio environment
(\cref{fig.studio}).

\begin{figure}[hbt]
\begin{center}
\gr{studio}{.9}
\caption{Aseba Studio environment}\label{fig.studio}
\end{center}
\end{figure}

Programming in Aseba Studio is also based upon the concepts of events
and actions. Since VPL programs are translated into textual programs,
everything you learned in this tutorial is supported in Studio, but now
you have the flexibility of a full programming language with variables,
expressions, and control statements. Aseba Studio also gives you access
to features of the Thymio that are not available in VPL:

\begin{itemize}
\item You can control all the lights such as the circle of lights
surrounding the buttons.
\item You have more flexibility in synthesizing sound.
\item There is a temperature sensor.
\item A remote control device can be used with the robot.
\end{itemize}

When you are working with Aseba Studio, you can open VPL by clicking on
the button \bu{Launch VPL} in the \emph{Tools} tab at the bottom left of
the window. You can import VPL programs into Aseba Studio simply by
opening its file.

To learn about Aseba Studio, read the \emph{Programming Thymio II} page at:\\
\url{https://aseba.wikidot.com/en:thymioprogram}\\
and follow the link \emph{Text Programming Environment}.

You can find many interesting projects at:\\ \url{https://aseba.wikidot.com/en:thymioexamples}.

\vspace{4em}

\informationbox{Have fun and learn a lot!}{Thank you for reading this tutorial!}

