% !TeX root = vpl.tex

\chapter*{Preface}

\sect{What is a robot?}

You are riding your bicycle and suddenly you see that the street starts
to go uphill. You pedal faster to supply more power to the wheels so
that the bicycle won't slow down. When you reach the top of the hill and
start to go downhill, you squeeze the brake lever. This causes a rubber
pad to be pressed against the wheel and the bicycle slows down. When you
ride a bicycle, your eyes are \textit{sensors} that sense what is going
on in the world. When these sensors---your eyes---detect an
\textit{event} such as a curve in the street, you perform an
\textit{action}, such as moving the handlebar left or right.

In a car, there are sensors that \textit{measure} what is going on in
the world. The speedometer measures how fast the car is going; if you
see it measuring a speed higher than the limit, you might tell the
driver that he is going too fast. In response, he can perform an action,
such as stepping on the brake pedal to slow the car down. The fuel meter
measures how much fuel remains in the car; if you see that it is too
low, you can tell the driver to find a gas station. In response, he can
perform an action: raise the turn-signal lever to indicate a right turn
and turn the steering wheel in order to drive into the station.

The rider of the bicycle and the driver of the car receive data from the
sensors, decide what actions to take and cause the actions to be
performed. A \textit{robot} is a system where this process is carried out by a
computerized system, usually without the participation of a human.

\sect{The Thymio robot and the Aseba VPL environment}

The Thymio is a small robot intended for educational purposes
(\cref{fig.front}). The robot includes sensors that can measure light,
sound and distance, and can detect when buttons are touched and when the
robot's body is tapped. The most important action that it can perform is
to move using two wheels, each powered by its own motor. Other actions
include generating sound and turning lights on and off.

The name Thymio is used in this document to refer to the Thymio~II
robot.

Aseba is a programming environment for small mobile robots such as the
Thymio. VPL is a component of Aseba for \textit{visual programming} that
was designed to program Thymio in an easy way through event and action
blocks. This tutorial assumes that you have Aseba installed on your
computer; if not, download it from 
\url{https://aseba.wikidot.com/en:downloadinstall} and install.

\newpage

\sect{Overview of the tutorial}

\textbf{\cref{ch.intro,ch.colors}} are an essential
introduction to the robot, the VPL environment and its principal
programming construct: the event-actions pair.

\textbf{\cref{ch.moving,ch.pet,ch.line}} present the events,
actions and algorithms for constructing autonomous mobile robots and
should be the core of any activity using Thymio and VPL.

\textbf{\cref{ch.bells}} describes features of the robot that can
be fun to use but are not essential. You can skip the chapter or you can
use it immediately after the introductory chapters for a more gentler
introduction to robotics.


\importantbox[Advanced mode]{VPL has a basic mode which supports
elementary events and actions that are easy for beginners to master. The
advanced mode of VPL supports more events and actions that require
experience to use. Explanations of the features of advanced mode
start in \cref{ch.time}.}

\textbf{\cref{ch.time}} presents timed events. There is an action
to set a timer and when the timer expires, an event occurs.

\textbf{\cref{ch.states}} explains state machines, which are a
fundamental construct used in robotics.

\textbf{\cref{ch.counting}} shows how to use states to implement
arithmetic.

\textbf{\cref{ch.angles}} describes how to use the accelerometers
in the Thymio robot.

\textbf{\cref{ch.brait,ch.rabbit,ch.barcode,ch.sweep,ch.speed,ch.radar,ch.fa}}
differ from the previous chapters: they contains no new material and no
detailed explanation of how to develop the projects. Instead, they
specify VPL projects and invite you to design and implement them
yourself.

\textbf{\cref{ch.next}} points to the next step: using the
textual Studio environment which offers significantly more support for
developing robots than does the VPL environment.

\informationbox{Reference material}{The appendices contains reference
material that you will want to look at from time to time to learn about
new featuers of VPL or to refresh your memory.}

\textbf{\cref{a.toolbar}} contains a description of the user
interface---the buttons on the toolbar.
%It also describes how to display dynamic feedback when a VPL program is run.

\textbf{\cref{a.blocks}} is a list of the event and
action blocks in both basic and advanced modes.

\textbf{\cref{a.tech}} describes techniques for working with the
sliders of the sensor and motor blocks.

\textbf{\cref{a.tips}} provides guidance for teachers and
mentors of students. The first section suggests ways to encourage
exploration and experimentation. The next section focuses on good
programming practices. The final section lists some pitfalls that may be
encountered and offers hints on how to overcome them.

\bigskip

\sect{Chapter summary}

Here is an overview of the chapters of this tutorial. For each chapter,
we give the main topic, as well as lists of the event and action
blocks that are introduced.

{\centering \textbf{\cref{ch.intro}}\\}
\textbf{Topics}: Thymio robot, VPL programming environment.

\textbf{Events}: Buttons \hfill \textbf{Actions}: Top colors

\blkmed{event-buttons} \hfill \blkmed{action-colors-up}

\bigskip

{\centering \textbf{\cref{ch.colors}}\\}
\textbf{Topics}: Event-actions pairs.

\textbf{Events}: Buttons \hfill \textbf{Actions}: Top colors, bottom colors

\blkmed{event-buttons} \hfill \blkmed{action-colors-up} \quad \blkmed{action-colors-down}

\bigskip

{\centering \textbf{\cref{ch.moving}}\\}
\textbf{Topics}: Moving, sensing.

\textbf{Events}: Buttons, bottom sensors \hfill \textbf{Actions}: Motors

\blkmed{event-buttons} \quad \blkmed{event-ground} \hfill  \blkmed{action-motors}

\bigskip

{\centering \textbf{\cref{ch.pet}}\\}
\textbf{Topics}: Feedback control, motor speeds.

\textbf{Events}: Front sensors \hfill \textbf{Actions}: Motors

\blkmed{event-prox} \hfill \blkmed{action-motors}

\bigskip

{\centering \textbf{\cref{ch.line}}\\}
\textbf{Topics}: Line following.

\textbf{Events}: Bottom sensors \hfill \textbf{Actions}: Motors

\blkmed{event-ground} \hfill \blkmed{action-motors}

\bigskip

{\centering \textbf{\cref{ch.bells}}\\}
\textbf{Topics}: Sound, shocks.

\textbf{Events}: Tap, clap \hfill \textbf{Actions}: Music, top colors,
bottom colors

\blkmed{event-tap} \quad \blkmed{event-clap} \hfill \blkmed{action-music}
\quad \blkmed{action-colors-up} \quad \blkmed{action-colors-down}

%\bigskip
\newpage

{\centering \textbf{\cref{ch.time}}\\}
\textbf{Topics}: Timers.

\textbf{Events}: Timer expired \hfill \textbf{Actions}: Set timer

\blkmed{event-timer} \hfill \blkmed{action-timer}

\bigskip

{\centering \textbf{\cref{ch.states}}\\}
\textbf{Topics}: States.

\textbf{Events}: State associated with an event \hfill \textbf{Actions}:
Change state

\blkmed{state-filter} \hfill \blkmed{action-states}

\bigskip

{\centering \textbf{\cref{ch.counting}}\\}
\textbf{Topics}: Counting, binary arithmetic.

\textbf{Events}: State associated with an event \hfill \textbf{Actions}:
Change state

\blkmed{state-filter} \hfill \blkmed{action-states}

\bigskip

{\centering \textbf{\cref{ch.angles}}\\}
\textbf{Topics}: Accelerometers.

\textbf{Events}: Accelerometer events

\blkmed{event-pitch} \quad \blkmed{event-roll}

\bigskip

{\centering
\textbf{\cref{ch.brait,ch.rabbit,ch.barcode,ch.sweep,ch.speed,ch.radar,ch.fa}}\\}
\textbf{Topics}: VPL projects for you to implement with minimal
guidance.

\textbf{Events}: Advanced blocks for sensors and taps

\blkmed{event-prox-advanced} \quad \blkmed{event-prox-ground-advanced}
\quad \blkmed{event-tap-advanced}

\bigskip

{\centering \textbf{\cref{ch.next}}\\}
\textbf{Topics}: Aseba studio environment.

\medskip
%\newpage

{\centering \textbf{\cref{a.toolbar}}\\}
\textbf{Topics}: A description of the buttons on the toolbar.

\medskip

{\centering \textbf{\cref{a.blocks}}\\}
\textbf{Topics}: A summary of the blocks, and
notes on motor, sensor and button blocks.

\medskip
{\centering \textbf{\cref{a.tips}}\\}
\textbf{Topics}: Exploring and experimenting,
constructing a program, troubleshooting.

\medskip

{\centering \textbf{\cref{a.tech}}\\}
\textbf{Topics}: Techniques for using the sliders.

\sect{Reference cards}

You will find it useful to print out one or both of the reference cards,
which are in the same zip file as this document and available
online at \url{https://aseba.wikidot.com/en:thymioprogram}.

\begin{itemize}
\item A single page that summarizes the event and action blocks
(\href{https://aseba.wdfiles.com/local--files/en:thymioprogram/thymio-vpl-ref-card-en.pdf}{click here to download}).
\item A double page that can be folded in three to form a handy card.
It summarizes the VPL interface, the event and action blocks,
and includes example programs
(\href{https://aseba.wdfiles.com/local--files/en:thymioprogram/thymio-vpl-folding-ref-card-en.pdf}{click here to download})
\end{itemize}
