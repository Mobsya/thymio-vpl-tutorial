% !TeX root = vpl.tex

\chap{Project: Catch the Speeders}\label{ch.radar}

\textbf{Specification}

Help the police catch drivers who travel at high speed. Measure the
speed by detecting how far the car travels in a fixed period of time.

Implement the following behavior: The robot detects an object moving
from left to right in front of its sensors. Turn the top LED a different
color to indicate how far the object has moved during one second from
when it is first detected by the leftmost sensor.

\textbf{Guidance}

\begin{itemize}

\item In an initial state, when the leftmost sensor detects the object,
start a one-second timer.

\item When the timer expires, change the state to a new state; let us
call it the \emph{measure} state.

\item Construct four event-actions pairs, one with an event for each of
the other sensors; the event will only occur in the measure state. When
a sensor detects the object, it turns the top LED on to a color
associated with that sensor.

\item It is possible that the object will be detected by more than one
sensor. Ensure that an event-actions pair is run only when the object is
detected by its sensor and not by the next one.

\end{itemize}

\bigskip

{\raggedleft \hfill \textbf{Program file}: \bu{speeders.aesl}}
