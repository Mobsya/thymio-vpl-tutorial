% !TeX root = vpl.tex

\chap{Project: Sweeping the Floor}\label{ch.sweep}

Are you tired of cleaning your house? Now there are
\href{http://en.wikipedia.org/wiki/Robotic_vacuum_cleaner}{robotic
vacuum cleaners} that can do the job for you! The robot systematically
moves over the floor of your apartment, navigating around furniture and
other obstacles, while it vacuums the dirt.

\textbf{Specification}

When the forwards button is touched, the Thymio robot travels from one
side of the room to the opposite side, turns, moves a few steps and then
returns to the first side:
\begin{center}
\begin{picture}(200,30)
%\put(0,0){\framebox(200,30){}}
\put(0,30){\vector(1,0){200}}
\put(200,30){\vector(0,-1){30}}
\put(200,0){\vector(-1,0){200}}
\end{picture}
\end{center}

\textbf{Guidance}

There are three subtasks that the robot must implement: (1) move the
length of the room (to the right or to the left), (2) turn right, (3)
move a few steps down. The subtasks are performed in the following
order:

\begin{center}
\begin{picture}(380,20)
%\put(0,0){\framebox(380,20){}}
\put(30,10){\oval(60,20)}
\put(110,10){\oval(60,20)}
\put(190,10){\oval(60,20)}
\put(270,10){\oval(60,20)}
\put(350,10){\oval(60,20)}
\put(0,0){ \makebox(60,20){right}}
\put(80,0){\makebox(60,20){turn right}}
\put(160,0){\makebox(60,20){down}}
\put(240,0){\makebox(60,20){turn right}}
\put(320,0){\makebox(60,20){left}}
\put( 60,10){\vector(1,0){20}}
\put(140,10){\vector(1,0){20}}
\put(220,10){\vector(1,0){20}}
\put(300,10){\vector(1,0){20}}
\end{picture}
\end{center}

The robot will use states to keep track of which subtask it is
performing. The direction and amount of movement for each subtask is
determined by the speeds of the left and right motors, and by the length
of time that the motors run. Therefore, each subtask will be implemented
by an event-actions pair, where the event is the expiration of the timer
for the previous subtask (except initially, where it is touching the
button), and the (three) actions are to set the following parameters for
the next subtask: (1) the state; (2) the left and right motor speeds
(see page~\pageref{s.sliders} on setting the motor speeds accurately);
(3) the timer period. You will have to experiment with the speeds and
the timer period to cause the robot to follow the required rectangular
path.

\bigskip

{\raggedleft \hfill \textbf{Program file} \bu{sweep.aesl}}
