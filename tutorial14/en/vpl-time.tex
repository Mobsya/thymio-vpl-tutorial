\chap{A Time to Like (Advanced Mode)}\label{ch.time}

In \cref{ch.pet} we programmed a pet robot which either did or did not
like us. Let us consider a more advanced behavior: a shy pet who can't
make up its mind whether it likes us or dislikes us. Initially, the pet
will turn towards our outreached hand, but then it will turn away. After
a while, it will reconsider and turn back in the direction of our hand.

{\raggedleft \hfill Program file \bu{shy.aesl}}

When the right button is touched the robot turns to the right:
\blkc{start-turn} When it detects your hand, it turns to the left:
\blkc{turn-away} The behavior of turning back ``after a while'' can be
divided into two parts:

\begin{itemize}

\item \emph{When} the robot starts to turn away $\rightarrow$
\emph{start a timer} for two seconds.

\item \emph{When} the timer runs down to zero $\rightarrow$ \emph{turn}
to the right.

\end{itemize}

We need a new \emph{action} for the first part and a new \emph{event}
for the second part.

The action to set a \emph{timer} is like an alarm clock
\blksm{action-timer}. Normally, we set an alarm clock to an absolute
time, but when I set the alarm clock in my smartphone to an absolute
time like 07:00, it tells me the relative time: ``Alarm set for 11 hours
and 23 minutes from now.'' You can set the timer action block for a
certain number of seconds; when the timer has \emph{expired}---that is,
when the number of seconds has passed from the setting of the timer---a
timer \emph{event} occurs.

The timer can be set for up to four seconds, where each second is
represented by one quarter of the clock face. Click anywhere within the
white circle; there will be a short animation and then the appropriate
part of the clock face will be colored dark blue.

\informationbox{Advanced mode}{Timers are supported in \emph{advanced
mode}. Click on \blkmed{advanced} to enter advanced mode.\\The icon will
change to \blkmed{basic} and you can click on it to change back to
\emph{basic mode}.}

%\newpage

The event-actions pair for this first part of the behavior is:
\blkc{turn-clock}

When the event of detecting your hand occurs, there will be two actions:
turning the robot to the left and setting the timer to two seconds.

The second part of the behavior uses a ringing-alarm-clock timer event
\blksm{event-timer} that occurs when the amount of time set on the timer
expires.

Here is the event-actions pair to turn the robot to the right when the
timer expires: \blkc{turn-back}

\bigskip

\exercisebox{\thechapter.1}{
Write a program that causes the robot to move forward at top speed for
three seconds when the forward button is touched; then it runs
backwards. Add an event-actions pair to stop the robot by touching the
center button.
}
