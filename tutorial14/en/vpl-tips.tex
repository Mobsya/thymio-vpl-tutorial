% !TeX root = vpl.tex

\chap{Tips for Programming with VPL}\label{a.tips}

\sect{Exploring and experimenting}
\
\begin{description}

\item[Understand each event and action block] For each event block and
each action block, spend some time experimenting until you understand
exactly how it works. To explore the behavior of an action block,
construct a pair whose event is touching a button. This is a simple
event that will let you concentrate on learning what the action block
does. To explore the behavior of an event block, construct a pair whose
action is changing the color of the robot.

\item[Experiment with the sensor event blocks] The small red lights next
to each sensor show when that sensor detects an object. Move your
fingers in front of the sensors and see which lights are turned on.
Construct an event-actions pair consisting of the sensor event and the
top color action and experiment with the settings of the small squares
in the event block (gray, white, black).

\item[Experiment with the motors] Experiment with the settings of the
motors so that you get a general impression how fast the robot moves for
particular settings. Experiment with different settings for the two
motors in order to learn how the robot turns.

\end{description}


\sect{Constructing a program}

\begin{description}

\item[Plan your program] Before beginning to write a program, write down
a description of how the program is supposed to work: a sentence for
each event-actions pair.

\item[Construct one event-actions pair at a time] When you understand
how each event-actions pair works, you can put them together in a
program.

\item[Test each addition to the program] Test your program each time you
add a new event-actions pair so that you can identify which pair causes
an error.

\item[Use ``Save as'' after modifying a program] Before you change your
program, use \blksm{saveas} to save the program under a different name.
If the change makes things worse, it will be easy to go back to the
previous version.

\item[Display what happens] Use colors or sounds to display what the
program is doing. For example, if one sensor causes the robot to turn
left and another to turn right, add an action to the sensor events to
display different top colors. You will be able to see if a problem is
caused by a sensor or if the motors are not responding correctly to a
sensor event.

\end{description}


\sect{Troubleshooting}

\begin{description}

\item[Use a smooth surface] Make sure that the surface on which the
robot moves---a table or the floor---is very clean and smooth.
Otherwise, the motors may not be able to move the robot, or turns may be
uneven.

\item[Use a long cable] Make sure that your cable is long enough. If the
robot moves too far, the cable can cause the robot to slow down or stop.

\item[Sensor events may not be detected] Sensor events happen 10 times
per second. If the robot is moving very fast, an event might not be
detected.

For example, if the robot is supposed to detect the edge of the table
and stop, and if the robot is moving very fast, it might fall off the
table before the sensor event can stop the motor. When you run a
program, start at a low speed and gradually increase it.

For another example, consider the line-following program in
Chapter~\ref{ch.line}. Its algorithm depends on being able to detect
when one ground sensor detects the line and the other doesn't. If the
robot is moving too fast, the position where only one sensor detects
the line does not cause an event.

\item[Event-actions pairs are run sequentially] In theory, the
event-actions pairs are run \emph{concurrently}---at the same time; in
practice, they are run one after the other in the order they appear. As
shown in Exercise~4.2, this can cause a problem, because the second
action might conflict with what was done by the first action.

\item[Problems with the clap event] Do \emph{not} use the clap event
\blksm{event-clap} when the motors are running. The motors make a lot of
noise and can cause unexpected clap events.

Similarly, do \emph{not} use the tap event \blksm{event-tap} in the same
program with the clap event. Tapping on the robot causes noise that can
be interpreted as a clap.

\end{description}
